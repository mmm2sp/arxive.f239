
%\documentstyle[12pt,russian,amsthm,amsmath,amssymb]{article}
\documentclass[a4paper,11pt,twoside]{article}
\usepackage[left=14mm, top=10mm, right=14mm, bottom=10mm, nohead, nofoot]{geometry}
\usepackage{amsmath, amsfonts, amssymb, amsthm} % стандартный набор AMS-пакетов для математ. текстов
\usepackage{mathtext}
\usepackage[utf8]{inputenc} % кодировка utf8
\usepackage[russian]{babel} % русский язык
\usepackage[pdftex]{graphicx} % графика (картинки)
\usepackage{tikz}
\usepackage{fancyhdr,pageslts} % настройка колонтитулов
\usepackage{enumitem} % работа со списками
\usepackage{multicol} % работа с таблицами
%\usepackage{pscyr} % красивый шрифт
\usepackage{pgfornament} % красивые рюшечки и вензеля
\usepackage{ltxgrid} % управление написанием текста в две колонки
\usepackage{lipsum} % стандартный текст
\usepackage{tcolorbox} % рамка вокруг текста
\tcbuselibrary{skins}
% ----------------------------------------

\newcommand\ProblemName{Атмосферное электричество}

\newcommand\Source{X24}

\newcommand\Type{Условие задачи}

\newcommand\MyTextLeft{Президентский ФМЛ 239, г.~Санкт-Петербург}
\newcommand\MyTextRight{Использованы материалы сайта pho.rs}
\newcommand\MyHeading{Учебно-тренировочные сборы по физике}
% ----------------------------------------

% настройки полей
\geometry{
	left=12mm,
	top=21mm,
	right=15mm,
	bottom=26mm,
	marginparsep=0mm,
	marginparwidth=0mm,
	headheight=22pt,
	headsep=2mm,
	footskip=7mm}
% ----------------------------------------

% настройки колонтитулов
\pagestyle{fancy}

\fancypagestyle{style}{
	\fancyhf{}
	\fancyhead[L]{{\Large{\FancyTitle}}\\\vskip -5pt \dotfill}
	\fancyhead[R]{{\Large{\textbf{\Type}}}\\\vskip -5pt \dotfill}
	\renewcommand{\headrulewidth}{0pt}
	\renewcommand{\footrulewidth}{0pt}
	\fancyfoot[C]{\pgfornament[width=2em,anchor=south]{72}\hspace{1mm}
		{Страница \textbf{\thepage} из \textbf{\pageref{VeryLastPage}}}\hspace{2mm}
		\pgfornament[width=2em,symmetry=v,anchor=south]{72}\\ \vskip2mm
		{\small{\textit{\MyTextLeft\hfill\MyTextRight}}}}
}

\fancypagestyle{plain}{
	\fancyhf{}
	\renewcommand{\headrulewidth}{0pt}
	\renewcommand{\footrulewidth}{0pt}
	\fancyhead[C]{{\Large{\textit{\MyHeading}}}\\\vskip -5pt \dotfill}
	\fancyfoot[C]{\pgfornament[width=2em,anchor=south]{72}\hspace{1mm}
		{Страница \textbf{\thepage} из \textbf{\pageref{VeryLastPage}}}\hspace{2mm}
		\pgfornament[width=2em,symmetry=v,anchor=south]{72}\\ \vskip2mm
		{\small{\textit{\MyTextLeft\hfill\MyTextRight}}}}
}
% ----------------------------------------

% другие настройки
\pagenumbering{arabic}
\setlist[enumerate,itemize]{leftmargin=0pt,itemindent=2.7em,itemsep=0cm}
% ----------------------------------------

% собственные команды
\newcommand{\FancyTitle}{\textbf{\Source} --- \ProblemName}
\newcommand{\Title}{\begin{center}{\huge{\textbf{\Source} --- \ProblemName}}\end{center}}
\newcommand{\Chapter}[1]{\vskip5pt{\Large{\textbf{#1}}}\vskip5pt}
\newcommand{\QText}[1]{#1}
\newcommand{\QBlock}[3]{
	\begin{tcolorbox}[left=2mm,top=2mm,bottom=1mm,right=2mm,colback=white]
		\begin{tcolorbox}[enhanced,colframe=ProcessBlue,colback=ProcessBlue!30!white,
			frame style={opacity=0.7},interior style={opacity=1.0},
			nobeforeafter,tcbox raise base,shrink tight,extrude by=1.7mm,width=1.5cm]
			\textbf{#1\textsuperscript{#2}}
		\end{tcolorbox}\hspace{3mm}#3
	\end{tcolorbox}
}
\newcommand{\QPicture}[4]{
	\begin{figure}[H]
		\centering
		\includegraphics[width=0.35\linewidth]{#1}
		\caption{#3}
	\end{figure}
	
	#4
}
\newcommand{\ABlock}[1]{
	\vskip2mm
	\begin{tcolorbox}[enhanced,colframe=Magenta,colback=Magenta!15!white,
		frame style={opacity=0.5},interior style={opacity=1.0},
		nobeforeafter,tcbox raise base,shrink tight,extrude by=1.7mm,width=1.6cm]
		\textbf{Ответ:}
	\end{tcolorbox}\hspace{3mm}#1
}
\newcommand{\MBlock}[2]{
	\begin{tcolorbox}[enhanced,colframe=Yellow,colback=Yellow!15!white,
		frame style={opacity=0.5},interior style={opacity=1.0},
		nobeforeafter,tcbox raise base,shrink tight,extrude by=1.7mm,width=1.1cm]
		\textbf{#1}
	\end{tcolorbox}\hspace{3mm}#2
}
\newcommand{\MMBlock}[3]{
	\begin{tcolorbox}[enhanced,colframe=Yellow,colback=Yellow!15!white,
		frame style={opacity=0.5},interior style={opacity=1.0},
		nobeforeafter,tcbox raise base,shrink tight,extrude by=1.7mm,width=1.1cm]
		\textbf{#1}
	\end{tcolorbox}\hspace{3mm}
	\begin{tcolorbox}[enhanced,colframe=Orange,colback=Orange!15!white,
		frame style={opacity=0.5},interior style={opacity=1.0},
		nobeforeafter,tcbox raise base,shrink tight,extrude by=1.7mm,width=0.8cm]
		\textbf{#2}
	\end{tcolorbox}\hspace{3mm}#3
}
% ----------------------------------------


\begin{document}
	
	% настройки
	\pagestyle{style}\thispagestyle{plain}
	\Title
	% ----------------------------------------
	
	%\vskip5mm
	%\centering{\pgfornament[width=5cm,anchor=south]{89}}
	
	% смысловая часть


\QText{Явления атмосферного электричества можно лицезреть любому человеку в повседневной жизни. Каждый из нас хоть и побаивается, а всё же восхищается красотой возникающих в небе молний, или же с восторгом наблюдает за искрами в воздухе.
В рамках данной задачи вам предлагается поближе познакомиться с возникающими в атмосфере электрическими явлениями.}

\Chapter{Часть A. Перезарядка дождевых капель (4.6 балла)}

\QText{Первая возникающая мысль о заряде дождевых капель состоит в том, что они при падении сохраняют тот же заряд, который несли внутри облаков, т.е в среднем равный нулю. Как показывает опыт, данное предположение неверно, и средний заряд падающих на поверхность Земли капель положительный. Данная часть задачи посвящена объяснению и количественному анализу этого явления.}

\QText{Вам наверняка известен закон Ома в дифференциальной форме:
$$\vec{j}=\sigma\vec{E}{,}
$$
где $\vec{j}$ и $\vec{E}$ - плотность тока и напряжённость электростатического поля в среде, а $\sigma$ - величина, обратная удельному сопротивлению и называющаяся проводимостью.
В атмосферном воздухе величина плотности тока обусловлена движением положительно и отрицательно заряженных ионов, движение которых происходит независимо друг от друга, в силу чего можно написать:
$$\vec{j}=\vec{j}_++\vec{j}_-{,}
$$
где $\vec{j}_+$ и $\vec{j}_-$ - плотности токов, обусловленные движением положительно и отрицательно заряженных ионов соответственно.
Каждая из плотностей тока оказывается пропорциональной напряжённости электростатического поля $\vec{E}$, поэтому закон Ома в дифференциальной форме для атмосферы записывается следующим образом:
$$\vec{j}=\sigma_+\vec{E}+\sigma_-\vec{E}{.}
$$
Здесь $\sigma_+$ характеризует плотность тока, обусловленную движением положительно заряженных ионов, а $\sigma_-$ - плотность тока, обусловленную движением отрицательно заряженных ионов. При этом $\sigma_+$ и $\sigma_-$ являются строго положительными величинами. 
Из положительности проводимостей $\sigma_+$ и $\sigma_-$ следует, что положительно заряженные ионы движутся вдоль направления электрического поля, а отрицательно заряженные ионы - противоположно ему.
Разница между положительно и отрицательно заряженными ионами в атмосфере заключается не только в отличии знаков их зарядов, но и в их концентрациях. В Земной атмосфере преобладают положительные ионы, поэтому для неё всегда будем считать выполненным соотношение:
$$\sigma_+{>}\sigma_-{.}
$$}

\QPicture{X24 Атмосферное электричество_files/29521.jpeg}{"width:100\%; max-width: 100\%;display:block;margin-left: auto;margin-right: auto; cursor:pointer;" data-toggle="modal" data-target="#kt_modal_29521"}{}{Перейдём непосредственно к постановке задачи. Водяная капля, которую можно считать проводящим шаром радиусом $R$, расположена в направленном вертикально вниз электростатическом поле напряжённостью $\vec{E}_0$. Капля движется в атмосфере с положительными и отрицательными ионами, проводимости которых равны $\sigma_+$ и $\sigma_-$ соответственно. Глобальной целью данной части задачи является нахождение стационарного заряда капли $Q_0$. При достаточно быстром падении капли по отношению к окружающему воздуху изменение заряда шара обусловлено только движением ионов к поверхности капли.}

\QText{При решении используйте следующую модель:}

\QText{\begin{itemize} 
\item При падении капля всегда сохраняет форму шара радиусом $R$;
\item Величина и направление электростатического поля $\vec{E}_0$ остаются постоянными в процессе всего падения капли;
\item Заряд капли изменяется только при её контакте с положительно и отрицательно заряженными ионами, а заряды, уже находящиеся на поверхности капли, её никогда не покидают. 
\end{itemize}}

\QBlock{A1}{0.80}{Пусть проводящий шар радиусом $R$, несущий заряд $Q$, помещён в однородное электростатическое поле напряжённостью $\vec{E}_0$. Определите полную напряжённость $\vec{E}$ электрического поля в точке с радиус-вектором $\vec{r}$ относительно центра шара, находящейся вне шара. Ответ выразите через $Q$, $\vec{E}_0$, $R$, $\varepsilon_0$ и $\vec{r}$.}

\QText{Пусть $\vec{n}$ - вектор нормали к поверхности шара, направленный наружу.}

\QBlock{A2}{0.40}{Пусть $\theta$ - угол между направлением вектора электростатического поля $\vec{E}_0$ и радиус-вектором $\vec{r}$ некоторой точки поверхности шара относительно его центра. 
Определите проекцию напряжённости электрического поля $E_n(\theta)$ на направление нормали. Ответ выразите через $Q$, $E_0$, $R$, $\varepsilon_0$ и $\theta$.}

\QBlock{A3}{0.20}{При каких значениях заряда $Q$ шара величина $E_n$ может обращаться в ноль? Ответ выразите через $E_0$, $\varepsilon_0$ и $R$.
Далее во всех пунктах считайте, что заряд шара $Q$ попадает в найденный вами диапазон.}

\QBlock{A4}{0.50}{Пусть заряд шара равен $Q$. При каком значении угла $\theta_0$ компонента напряжённости $E_n(\theta_0)$ обращается в ноль? Ответ выразите через $Q$, $E_0$, $\varepsilon_0$ и $R$.
Определите также, при каких значениях угла $\theta$ на поверхность шара попадают отрицательные ионы, а при каких - положительные. Ответы выразите через $\theta_0$.}

\QBlock{A5}{1.00}{Определите полную производную по времени заряда шара $dQ/dt$. Ответ выразите через $Q$, $E_0$, $\varepsilon_0$, $R$, $\sigma_+$, $\sigma_-$ и, если понадобится, $\theta_0$.}

\QBlock{A6}{0.80}{Определите стационарный заряд шара $Q_0$, при котором он остаётся постоянным во времени. Ответ выразите через $E_0$, $\varepsilon_0$, $R$, $\sigma_+$ и $\sigma_-$.}

\QText{Пусть $\Delta{Q}=Q-Q_0$ - отклонение заряда шара от стационарного. В момент времени $t=0$ отклонение заряда шара от стационарного составляло $\Delta{Q}_0=Q(0)-Q_0$, причём $|\Delta{Q}_0|\ll{Q}_0$.}

\QBlock{A7}{0.60}{При малых значениях $\Delta{Q}$ зависимость $\dot{Q}(\Delta{Q})$ можно представить в виде:
$$\dot{Q}\approx A\Delta{Q}{.}
$$
Выразите $A$ через $\sigma_+$, $\sigma_-$ и $\varepsilon_0$.
Является ли найденное значение заряда $Q_0$ устойчивым? Ответ обоснуйте.}

\QBlock{A8}{0.30}{Получите зависимость отклонения заряда шара $\Delta{Q}$ от времени $t$. Ответ выразите через $\Delta{Q}_0$, $\varepsilon_0$, $\sigma_+$, $\sigma_-$ и $t$. Покажите также, что вид временной зависимости определяется только произведением $\sigma_+\sigma_-$.}

\QText{Далее мы изучим комплекс явлений, называемых грозовыми. Он охватывает целый ряд вопросов относящихся и к грозовым разрядам, то есть молниям, и к тлеющим разрядам, связанным с коронированием остроконечных предметов.}

\Chapter{Часть B. Возникновение коронного разряда на острие (2.4 балла)}

\QText{Рассмотрим вопрос об аномальном увеличении напряжённости электростатического поля вблизи остроконечных предметов. 
Пусть проводником является вытянутый эллипсоид вращения, заданным уравнением:
$$\cfrac{z^2}{a^2}+\cfrac{x^2+y^2}{b^2}=1{.}
$$
Здесь $a$ и $b$ обозначают большую и малую полуоси эллипсоида соответственно.
Ось $z$ эллипсоида ориентирована параллельно внешнему однородному электростатическому полю напряжённостью $\vec{E}_0$, направленному вертикально вниз.
Можно показать, что напряжённость $\vec{E}$ электрического поля внутри изолированного равномерно заряженного по объёму эллипсоида вращения зависит линейно от координат $x{,}y{,}z$:
$$\vec{E}=\cfrac{\rho}{\varepsilon_0}\left(Az\vec{e}_z+B(x\vec{e}_x+y\vec{e}_y)\right){.}
$$
Здесь $A$ и $B$
$$A=\left(\cfrac{b}{c}\right)^2\left(\cfrac{a}{c}\cdot\ln\cfrac{a+c}{b}-1\right)\qquad B=\cfrac{a}{2c}\left(\cfrac{a}{c}-\left(\cfrac{b}{c}\right)^2\ln\cfrac{a+c}{b}\right){,}
$$
где $c=\sqrt{a^2-b^2}$ - половина расстояния между фокусами эллипсоида.
Если $b\ll{a}$, то указанные выражения можно приблизить следующими:
$$A\approx \left(\cfrac{b}{a}\right)^2\left(\ln\cfrac{2a}{b}-1\right)\qquad B\approx\cfrac{1}{2}\left(1-\left(\cfrac{b}{a}\right)^2\ln\cfrac{2a}{b}\right)
$$
Далее во всех пунктах, требующих подстановки $A$, используйте приближённое выражение.}

\QPicture{X24 Атмосферное электричество_files/29532.jpeg}{"width:100\%; max-width: 100\%;display:block;margin-left: auto;margin-right: auto; cursor:pointer;" data-toggle="modal" data-target="#kt_modal_29532"}{}{Задача о помещении проводящего эллипсоида в однородное электрическое поле, направленное параллельно его оси $z$, может быть решена аналогично задаче для проводящего шара: 
Рассмотрим два эллипсоида с совпадающей осью $z$, несущих заряды с постоянными плотностями $\rho$ и $-\rho$. Центр положительного эллипсоида имеет радиус-вектор $\vec{l}$ относительно центра отрицательно заряженного эллипсоида.}

\QBlock{B1}{0.50}{Определите вектор $\vec{E}$ напряжённости электростатического поля в области пересечения изолированных эллипсоидов. Ответ выразите через $\rho$, $\vec{l}$, $\varepsilon_0$ и $A$.}

\QText{Если расстояние $l$ между центрами эллипсоидов устремить к нулю, то их комбинацию можно рассматривать как один эллипсоид с постоянным вектором поляризации $\vec{P}$.}

\QBlock{B2}{0.30}{Рассмотрим изолированный равномерно поляризованный вдоль оси $z$ эллипсоид.
Пусть напряжённость электростатического поля внутри эллипсоида равняется $\vec{E}$. Определите вектор поляризации эллипсоида $\vec{P}$. Ответ выразите через $\vec{E}$, $\varepsilon_0$ и $A$.}

\QText{Рассмотрим незаряженный проводящий эллипсоид, помещённый в однородное электростатическое поле напряжённостью $\vec{E}_0$, направленное вдоль оси $z$ эллипсоида.}

\QBlock{B3}{0.30}{Определите вектор поляризации $\vec{P}$ эллипсоида. Ответ выразите через $\vec{E}_0$, $\varepsilon_0$ и $A$.
Определите также максимальную величину поверхностной плотности заряда $\sigma_{max}$ на поверхности эллипсоида. Ответ выразите через $P$.}

\QBlock{B4}{0.40}{Выразите полную компоненту напряжённости электростатического поля $E_n$ на поверхности проводника через поверхностную плотность заряда $\sigma$ и $\varepsilon_0$.
Определите максимальную величину напряжённости $E_{max}$ электростатического поля на поверхности эллипсоида. Ответ выразите через $E_0$ и $A$.}

\QText{Рассмотрим половину равномерно поляризованного вдоль оси $z$ эллипсоида так, как показано на рисунке ниже. Своим экваториальным сечением он контактирует с бесконечной проводящей плоскостью.}

\QPicture{X24 Атмосферное электричество_files/29536.jpeg}{"max-width:500px;px;display:block;margin-left: auto;margin-right: auto;"}{}{}

\QBlock{B5}{0.40}{В листах ответов приведён рисунок, на котором над бесконечной проводящей плоскостью расположен точечный диполь, дипольный момент которого направлен перпендикулярно плоскости. В листах ответов приведите электростатическое изображение диполя в проводящей плоскости.
Используя полученный результат, приведите в листах ответов электростатическое изображение половины равномерно поляризованного эллипсоида вращения, контактирующего с проводящей плоскостью экваториальным сечением.}

\QText{Приведённые рассуждения могут быть использованы при изучении реальной конструкции: Рассмотрим проводник в форме половины вытянутого эллипсоида вращения с полуосями $a = 1\text{км}$ и $b = 1\text{мм}$, покоящегося своим экваториальным сечением на горизонтальной поверхности Земли, которую также можно считать проводящей. В атмосфере, на большом удалении от эллипсоида, присутствует направленное вертикально вниз электростатическое поле напряжённостью $E_0 = 100 \text{В}/\text{м}$.}

\QBlock{B6}{0.50}{Покажите, что выражение для максимальной напряжённости электростатического поля $E_{max}$ совпадает с выражением, найденным в пункте $\mathrm{B4}$, и найдите его численное значение. Достаточно ли величины напряжённости электростатического поля $E_0$ для пробоя воздуха в какой-либо точке пространства, если он происходит при напряжённости, равной $E_\text{пр}=30 \text{кВ}/\text{см}$?}

\Chapter{Часть C. Извлечение электричества из атмосферы (1.6 балла)}

\QText{С помощью конструкции, похожей на рассмотренную в части $\mathrm{B}$ задачи, можно добиться извлечения электрического заряда из атмосферы.
В данной части задачи проводимость атмосферного воздуха при нормальных условиях равна $\sigma_0$, а эффектами, описанными в части $\mathrm{A}$ задачи, можно пренебречь.
Рассмотрим проводящий шар радиусом $R_0$, соединённый с Землёй длинным тонким прямолинейным проводом длиной $h \gg R_0$. Радиус провода считайте малым по сравнению с $R_0$.
Электростатическое поле в атмосфере Земли можно принять постоянным, направленным вертикально вниз и равным $E_0$.}

\QPicture{X24 Атмосферное электричество_files/29589.jpeg}{"max-width:500px;px;display:block;margin-left: auto;margin-right: auto;"}{}{}

\QBlock{C1}{0.30}{Принимая потенциал шара равным потенциалу на поверхности Земли, т.е нулю, определите величину заряда $q_0$ шара. Ответ выразите через $\varepsilon_0$, $E_0$, $R_0$ и $h$.
Влиянием электростатического поля шара на электростатическое поле Земли можно пренебречь. Влиянием электростатического поля зарядов, расположенных на проводе, можно пренебречь во всём пространстве.}

\QBlock{C2}{0.40}{Определите величину силы тока $I$, перетекающего из атмосферы в шар, если проводимость воздуха во всей атмосфере можно принять равной $\sigma_0$. Ответ выразите через $E_0$, $R_0$, $h$, $\varepsilon_0$ и $\sigma_0$.}

\QPicture{X24 Атмосферное электричество_files/29538.jpeg}{"width:100\%; max-width: 100\%;display:block;margin-left: auto;margin-right: auto; cursor:pointer;" data-toggle="modal" data-target="#kt_modal_29538"}{}{Из результатов, полученных в предыдущих пунктах частей $\mathrm{B}$ и $\mathrm{C}$ видно, что величина напряжённости электростатического поля вблизи поверхности проводника может достигать достаточно больших величин, что приводит к ионизации ближайших к нему слоёв воздуха и сильному увеличению их проводимости. Для учёта данного эффекта будем считать, что проводящий шар находится внутри сферической оболочки с внутренним и внешним радиусами $R_0$ и $R$ соответственно, в которой расположен ионизированный воздух с проводимостью $\sigma$. На расстояниях $r{>}R$ от центра шара проводимость остаётся равной $\sigma_0$.
Считайте распределение зарядов в среде стационарным, так что в любой точке пространства плотность заряда остаётся постоянной.
В данной системе заряд располагается только на сферических поверхностях радиусами $R_0$ и $R$, причём на поверхности радиусом $R_0$ суммарный заряд равен $q_0$, а на поверхности радиусом $R$ - $q$ (см.рис).}

\QBlock{C3}{0.30}{Запишите выражение для условия равенства нулю потенциала шара. В уравнение могут войти $E_0$, $h$, $q_0$, $q$, $R_0$ и $R$.}

\QBlock{C4}{0.30}{Из условия равенства силы текущего тока $I$, пересекающего сферическую поверхность внутри и вне ионизированного слоя, получите уравнение, связывающее заряды $q_0$ и $q$. В уравнение также могут войти проводимости $\sigma_0$ и $\sigma$.}

\QBlock{C5}{0.20}{Определите величину силы тока $I$, перетекающего из атмосферы в шар в этом случае. Ответ выразите через $E_0$, $R_0$, $R$, $h$, $\varepsilon_0$, $\sigma_0$ и $\sigma$.}

\QBlock{C6}{0.10}{Покажите, что при $R\approx R_0$ приближённое выражение для силы тока $I$ переходит в выражение, соответствующее отсутствию ионизированного слоя.}

\Chapter{Часть D. Канал распространения молнии (2.4 балла)}

\QText{Исследуем образование и развитие грозового разряда. Для примера рассмотрим механизм образования молнии из острия, рассматриваемого в части $\mathrm{B}$ задачи. Проводник может создавать рядом с собой сильное электростатическое поле, вызывающее ионизацию воздуха. Ионизированный воздух ведёт себя как проводник относительно долгое время, так как процесс рекомбинации ионов в нём происходит медленно. В результате этого проводящая область пространства будет разрастаться, пока величины электростатического поля будет хватать для ионизации. Если образующаяся таким образом проводящая область (искровой канал) соединит облака и поверхность Земли, то по нему сможет течь электрический ток очень большой величины. 
Данная часть задачи посвящена оценке скорости образования искрового канала. Экспериментально получено, что она примерно равна скорости движения свободных электронов на границе проводящей области, где напряжённость поля равна критическому значению $E_\text{пр}$, при котором происходит ионизация.}

\QText{Предполагаемая модель явления выглядит следующим образом:
В атмосферной области находятся электроны с зарядом $-e$, масса которых равна $m$. Также в атмосфере находятся нейтральные молекулы воздуха, масса $M$ которых во много раз превышает массу электронов, т.е $M\gg{m}$. 
Под воздействием электростатического поля электроны приобретают ускорение и сталкиваются с молекулами воздуха, при этом среднее расстояние, проходимое электронами между двумя последовательными столкновениями с молекулами воздуха, равняется $\lambda$. Величина $\lambda$ называется длиной свободного пробега. Молекулы воздуха при этом можно всегда считать неподвижными.
Средняя кинетическая энергия теплового движения электронов равна $\overline{W}$ и настолько велика, что между двумя последовательными столкновениями практически не успевает измениться. Считайте, что величина $\overline{W}$ связана со средним значением скорости теплового движения электронов $\overline{v}_\text{т}$ соотношением:
$$\overline{W}=\cfrac{m\overline{v}^2_\text{т}}{2}{.}
$$
Вся данная конструкция расположена в однородном электрическом поле $\vec{E}$, направленном противоположно скорости распространения искрового канала.}

\QText{Средний вектор тепловой скорости теплового движения электронов равен нулю:
$$\langle\vec{v}_\text{т}\rangle=0{.}
$$
Тогда среднее перемещение электронов в направлении электростатического поля за время $\tau=\lambda/\overline{v}_\text{т}$, соответствующее характерному времени движения электрона между двумя последовательными столкновениями с молекулами воздуха, составляет:
$$\langle\vec{S}\rangle=\langle\vec{v}_\text{т}\rangle\tau+\vec{a}\tau^2/2=\vec{a}\tau^2/2\Rightarrow \vec{u}=\cfrac{\langle\vec{S}\rangle}{\tau}=\cfrac{\vec{a}\tau}{2}{.}
$$
Вектор $\vec{u}$ называется средней скоростью дрейфа электронов в электростатическом поле. Данная величина равна скорости распространения искрового канала.}

\QBlock{D1}{0.20}{Определите величину дрейфовой скорости $u$ движения электронов. Ответ выразите через $e$, $E$, $\lambda$, $m$ и $\overline{v}_\text{т}$.}

\QText{Когда среднее значение кинетической энергии теплового движения электронов достигает стационарного значения $\overline{W}$ - средняя мощность потерь кинетической энергии электронов при столкновении с молекулами воздуха компенсируется работой электрического поля по перемещению электронов.}

\QBlock{D2}{0.40}{Из баланса энергии определите среднюю величину $\overline{\Delta{W}}$ потери кинетической энергии электрона при столкновении с молекулами воздуха. Ответ выразите через $e$, $E$, $\lambda$, $m$ и $\overline{W}$.}

\QText{Величина $\overline{\Delta{W}}$ также может быть найдена непосредственно из анализа столкновений.}

\QText{Для определения величины $\overline{\Delta W}$ примем следующую модель:}

\QText{\begin{itemize} 
\item Все столкновения, происходящие между электронами и молекулами воздуха, являются упругими;
\item Изначально молекула воздуха массой $M$ покоится, а кинетическая энергия налетающего на него электрона массой $m$ всегда равна $\overline{W}$;
\item В системе отсчёта центра масс рассматриваемой системы из двух частиц все направления скорости молекулы воздуха сразу после столкновения с электроном равновероятны.
\end{itemize}}

\QText{Пусть $\varphi$ - угол между вектором скорости налетающего электрона и скоростью молекулы воздуха в системе отсчёта центра масс сразу после столкновения. Тогда, поскольку все направления скорости молекулы воздуха в системе отсчёта центра масс равновероятны, среднее значение потерь кинетической энергии электрона определяется выражением:}

\QText{где $E_k(\varphi)$ – кинетическая энергия в лабораторной системе отсчёта молекулы воздуха, сразу после столкновения движущейся в направлении угла $\varphi$ в системе отсчёта центра масс, а $d\Omega=2\pi\sin\varphi d\varphi$ – телесный угол, под которым виден сегмент поверхности сферы между углами $\varphi$ и $\varphi+d\varphi$.}

\QBlock{D3}{1.00}{Получите точное выражение для величины $\overline{\Delta{W}}/\overline{W}$. Ответ выразите через $m$ и $M$. Упростите ваш ответ с учётом $m\ll{M}$.
Если вы не смогли решить этот пункт - в дальнейшем считайте, что $\overline{\Delta{W}}/\overline{W}=m/M$.}

\QText{Считайте известными следующие данные:}

\QText{\begin{itemize} 
\item Заряд электрона $-e=-1{.}6\cdot 10^{-19} \text{Кл}$;
\item Масса электрона $m=9{.}1\cdot 10^{-31} \text{кг}$;
\item Масса молекулы воздуха $M\approx 6\cdot 10^4 m$;
\item Критическое напряжённость электростатического поля, при которой возникает пробой, равна $E_\text{пр}=30 \text{кВ}/\text{см}$;
\item Средняя длина свободного пробега $\lambda\approx 1 \text{мкм}$.
\end{itemize}}

\QBlock{D4}{0.80}{Определите стационарное значение кинетической энергии теплового движения электронов $\overline{W}$ и скорости образования искрового канала $u$. Ответы выразите через $m$, $M$, $e$, $E$ и $\lambda$. Рассчитайте полученные значения.}

\end{document}