
%This file is part of Get pho.rs!

%Get pho.rs! is free software: you can redistribute it and/or modify it under the terms of the GNU General Public License as published by the Free Software Foundation, either version 3 of the License, or (at your option) any later version.

%Get pho.rs! is distributed in the hope that it will be useful, but WITHOUT ANY WARRANTY; without even the implied warranty of MERCHANTABILITY or FITNESS FOR A PARTICULAR PURPOSE. See the GNU General Public License for more details.

%You should have received a copy of the GNU General Public License along with Foobar. If not, see <https://www.gnu.org/licenses/>.

%\documentstyle[12pt,russian,amsthm,amsmath,amssymb]{article}
\documentclass[a4paper,11pt,twoside]{article}
\usepackage[left=14mm, top=10mm, right=14mm, bottom=10mm, nohead, nofoot]{geometry}
\usepackage{amsmath, amsfonts, amssymb, amsthm} % стандартный набор AMS-пакетов для математ. текстов
\usepackage{mathtext}
\usepackage[utf8]{inputenc} % кодировка utf8
\usepackage[russian]{babel} % русский язык
\usepackage[pdftex,dvipsnames]{xcolor} % работа с цветами
\usepackage[pdftex]{graphicx} % графика (картинки)
\usepackage{tikz} % рисунки
\usepackage{fancyhdr,pageslts} % настройка колонтитулов
\usepackage{enumitem} % работа со списками
\usepackage{multicol} % работа с таблицами
%\usepackage{pscyr} % красивый шрифт
\usepackage{pgfornament} % красивые рюшечки и вензеля
\usepackage{ltxgrid} % управление написанием текста в две колонки
\usepackage{lipsum} % стандартный текст
\usepackage{tcolorbox} % рамка вокруг текста
\usepackage{float} % для корректного размещения картинок
\tcbuselibrary{skins}
% ----------------------------------------

\newcommand\ProblemName{Зонные пластинки}

\newcommand\Source{Cn}

\newcommand\Type{Условие задачи}

% настройки полей
\geometry{
	left=12mm,
	top=21mm,
	right=15mm,
	bottom=26mm,
	marginparsep=0mm,
	marginparwidth=0mm,
	headheight=22pt,
	headsep=2mm,
	footskip=7mm}
% ----------------------------------------

% настройки колонтитулов
\pagestyle{fancy}

\fancypagestyle{style}{
	\fancyhf{}
	\fancyhead[L]{{\Large{\FancyTitle}}\\\vskip -5pt \dotfill}
	\fancyhead[R]{{\Large{\textbf{\Type}}}\\\vskip -5pt \dotfill}
	\renewcommand{\headrulewidth}{0pt}
	\renewcommand{\footrulewidth}{0pt}
	\fancyfoot[C]{\pgfornament[width=2em,anchor=south]{72}\hspace{1mm}
		{Страница \textbf{\thepage} из \textbf{\pageref{VeryLastPage}}}\hspace{2mm}
		\pgfornament[width=2em,symmetry=v,anchor=south]{72}\\ \vskip2mm
		{\small{\textit{Условие собрано и подготовлено в Президентском ФМЛ №239 г.~Санкт-Петербурга}}}}
}

\fancypagestyle{plain}{
	\fancyhf{}
	\renewcommand{\headrulewidth}{0pt}
	\renewcommand{\footrulewidth}{0pt}
	\fancyhead[C]{{\Large{\textit{Учебно-тренировочные сборы к X23}}}\\\vskip -5pt \dotfill}
	\fancyfoot[C]{\pgfornament[width=2em,anchor=south]{72}\hspace{1mm}
		{Страница \textbf{\thepage} из \textbf{\pageref{VeryLastPage}}}\hspace{2mm}
		\pgfornament[width=2em,symmetry=v,anchor=south]{72}\\ \vskip2mm
		{\small{\textit{Условие собрано и подготовлено в Президентском ФМЛ №239 г.~Санкт-Петербурга}}}}
}
% ----------------------------------------

% другие настройки
\pagenumbering{arabic}
\setlist[enumerate,itemize]{leftmargin=0pt,itemindent=2.7em,itemsep=0cm}
% ----------------------------------------

% собственные команды
\newcommand{\FancyTitle}{\textbf{\Source} --- \ProblemName}
\newcommand{\Title}{\begin{center}{\huge{\textbf{\Source} --- \ProblemName}}\end{center}}
\newcommand{\Chapter}[1]{\vskip5pt{\Large{\textbf{#1}}}\vskip5pt}
\newcommand{\QText}[1]{#1}
\newcommand{\QBlock}[3]{
	\begin{tcolorbox}[left=4mm,top=3mm,bottom=2mm,right=4mm,colback=white]
		\begin{tcolorbox}[enhanced,colframe=blue,colback=blue!10!white,
			frame style={opacity=0.3},interior style={opacity=1.0},
			nobeforeafter,tcbox raise base,shrink tight,extrude by=1.7mm,width=1.5cm]
			\textbf{#1\textsuperscript{#2}}
		\end{tcolorbox}\hspace{3mm}#3
	\end{tcolorbox}
}
\newcommand{\QPicture}[4]{\QText{#4}  \includegraphics{#1}}
\newcommand{\ABlock}[1]{#1}
\newcommand{\MBlock}[2]{#1 #2}
\newcommand{\MMBlock}[3]{#1 #2 #3}
% ----------------------------------------


\begin{document}
	
	% настройки
	\pagestyle{style}\thispagestyle{plain}
	\Title
	% ----------------------------------------
	
	%\vskip5mm
	%\centering{\pgfornament[width=5cm,anchor=south]{89}}

\QText{На рисунке (a) показана непрозрачная перегородка, в которой проделано небольшое круглое отверстие радиусом $OM=1.00 \text{мм}$. В качестве источника света используется водородно-неоновый лазер с длиной волны $\lambda=632.8 \text{нм}$, параллельный пучок света которого падает на отверстие слева. Справа на оси симметрии отверстия находится точка $P$. Волну в этой точке можно рассматривать как комбинацию волн от полуволновых зон. Обозначим $r_0=PO$, тогда сферы с центром в точке $P$ радиусами $r_0+\frac{\lambda}2$, $r_0+2\frac{\lambda}2$, $r_0+3\frac{\lambda}2$, $\ldots$ разбивают отверстие на $N\in\mathbb N$ колец. Расстояние от точки $P$ до края отверстия $M$ равно $r_0+N\frac\lambda2$, а кольцо с наименьшим радиусом представляет собой круг. Каждое такое кольцо называется полуволновой зоной, поскольку разность оптический путей от его краёв до точки $P$ равна $\frac\lambda2$. Ясно, что количество зон $N$ определяется положением точки $P$.}

\QPicture{Cn Зонные пластинки_files/15017.jpeg}{"max-width:500px;display:block;margin-left: auto;margin-right: auto;"}{рис. (a)}{}

\QBlock{1}{}{Если $N=2n+1$, найдите расстояние $r_0$ до точки $P_0$ ($P_0$ - крайняя справа яркая точка, называемая главным фокусом) и расстояние $r_1$ до точки $P_1$ ($P_1$ - тоже яркая точка, располагающаяся левее $P_0$, называемая вторичным фокусом).}

\QText{Пусть теперь $N=4$, и в первой и третьей волновых зонах помещён прозрачный материал, при прохождении через который оптический путь света увеличивается на $\frac\lambda2$ (см. рисунок (b)).}

\QPicture{Cn Зонные пластинки_files/15019.jpeg}{"max-width:500px;display:block;margin-left: auto;margin-right: auto;"}{рис. (b)}{}

\QBlock{2.1}{}{Найдите расстояние $r'_0$ до главного фокуса $P'_0$ такой пластинки.}

\QBlock{2.2}{}{Найдите расстояние $r'_{-1}$ до вторичного фокуса $P'_{-1}$, находящегося непосредственно слева от главного.}

\QBlock{2.3}{}{Найдите расстояние $r'_{+1}$ до вторичного фокуса $P'_{+1}$, находящегося непосредственно справа от главного.}

\QText{Зонную пластинку можно использовать не только для фокусировки света, но и для формирования изображения. Рассмотренный выше процесс фокусировки параллельного пучка эквивалентен ситуации, когда предмет находится на бесконечности, а расстояние до изображения равно фокусному. Пусть теперь точечный источник света расположен слева от $O$ на расстоянии $s=3 \text{м}$ в точке $S$ на оси симметрии. Как показано на рисунке (c), его изображение обозначим $S'$.}

\QPicture{Cn Зонные пластинки_files/15024.jpeg}{"max-width:500px;display:block;margin-left: auto;margin-right: auto;"}{рис. (c)}{}

\QBlock{3.1}{}{Найдите $OS'$, соответствующее главному фокусу зонной пластинки. Справедлива ли формула тонкой линзы?}

\QBlock{3.2}{}{Если пластинка формирует несколько изображений, на каком расстоянии $s'$ от $O$ формируется изображение, ближайшее к рассмотренному в предыдущем пункте? Чему равно фокусное расстояние $f'$ соответствующего вторичного фокуса (формула тонкой линзы неприменима)?}

\QBlock{3.3}{}{Если предмет расположен слева от зонной пластинки на расстоянии $\frac{OP'_0}2$ от точки $O$, найдите расстояния $s''$ и $s'''$ до главного и вторичного изображений, соответствующих фокусам, рассмотренным в предыдущем пункте (формула тонкой линзы также неприменима). Действительные они или мнимые?}

\end{document}