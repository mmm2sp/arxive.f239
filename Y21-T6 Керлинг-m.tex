
%\documentstyle[12pt,russian,amsthm,amsmath,amssymb]{article}
\documentclass[a4paper,11pt,twoside]{article}
\usepackage[left=14mm, top=10mm, right=14mm, bottom=10mm, nohead, nofoot]{geometry}
\usepackage{amsmath, amsfonts, amssymb, amsthm} % стандартный набор AMS-пакетов для математ. текстов
\usepackage{mathtext}
\usepackage[utf8]{inputenc} % кодировка utf8
\usepackage[russian]{babel} % русский язык
\usepackage[pdftex]{graphicx} % графика (картинки)
\usepackage{tikz}
\usepackage{fancyhdr,pageslts} % настройка колонтитулов
\usepackage{enumitem} % работа со списками
\usepackage{multicol} % работа с таблицами
%\usepackage{pscyr} % красивый шрифт
\usepackage{pgfornament} % красивые рюшечки и вензеля
\usepackage{ltxgrid} % управление написанием текста в две колонки
\usepackage{lipsum} % стандартный текст
\usepackage{tcolorbox} % рамка вокруг текста
\tcbuselibrary{skins}
% ----------------------------------------

\newcommand\ProblemName{Керлинг}

\newcommand\Source{Y21-T6}

\newcommand\Type{Разбалловка}

\newcommand\MyTextLeft{Президентский ФМЛ 239, г.~Санкт-Петербург}
\newcommand\MyTextRight{Использованы материалы сайта pho.rs}
\newcommand\MyHeading{Учебно-тренировочные сборы по физике}
% ----------------------------------------

% настройки полей
\geometry{
	left=12mm,
	top=21mm,
	right=15mm,
	bottom=26mm,
	marginparsep=0mm,
	marginparwidth=0mm,
	headheight=22pt,
	headsep=2mm,
	footskip=7mm}
% ----------------------------------------

% настройки колонтитулов
\pagestyle{fancy}

\fancypagestyle{style}{
	\fancyhf{}
	\fancyhead[L]{{\Large{\FancyTitle}}\\\vskip -5pt \dotfill}
	\fancyhead[R]{{\Large{\textbf{\Type}}}\\\vskip -5pt \dotfill}
	\renewcommand{\headrulewidth}{0pt}
	\renewcommand{\footrulewidth}{0pt}
	\fancyfoot[C]{\pgfornament[width=2em,anchor=south]{72}\hspace{1mm}
		{Страница \textbf{\thepage} из \textbf{\pageref{VeryLastPage}}}\hspace{2mm}
		\pgfornament[width=2em,symmetry=v,anchor=south]{72}\\ \vskip2mm
		{\small{\textit{\MyTextLeft\hfill\MyTextRight}}}}
}

\fancypagestyle{plain}{
	\fancyhf{}
	\renewcommand{\headrulewidth}{0pt}
	\renewcommand{\footrulewidth}{0pt}
	\fancyhead[C]{{\Large{\textit{\MyHeading}}}\\\vskip -5pt \dotfill}
	\fancyfoot[C]{\pgfornament[width=2em,anchor=south]{72}\hspace{1mm}
		{Страница \textbf{\thepage} из \textbf{\pageref{VeryLastPage}}}\hspace{2mm}
		\pgfornament[width=2em,symmetry=v,anchor=south]{72}\\ \vskip2mm
		{\small{\textit{\MyTextLeft\hfill\MyTextRight}}}}
}
% ----------------------------------------

% другие настройки
\pagenumbering{arabic}
\setlist[enumerate,itemize]{leftmargin=0pt,itemindent=2.7em,itemsep=0cm}
% ----------------------------------------

% собственные команды
\newcommand{\FancyTitle}{\textbf{\Source} --- \ProblemName}
\newcommand{\Title}{\begin{center}{\huge{\textbf{\Source} --- \ProblemName}}\end{center}}
\newcommand{\Chapter}[1]{\vskip5pt{\Large{\textbf{#1}}}\vskip5pt}
\newcommand{\QText}[1]{#1}
\newcommand{\QBlock}[3]{
	\begin{tcolorbox}[left=2mm,top=2mm,bottom=1mm,right=2mm,colback=white]
		\begin{tcolorbox}[enhanced,colframe=ProcessBlue,colback=ProcessBlue!30!white,
			frame style={opacity=0.7},interior style={opacity=1.0},
			nobeforeafter,tcbox raise base,shrink tight,extrude by=1.7mm,width=1.5cm]
			\textbf{#1\textsuperscript{#2}}
		\end{tcolorbox}\hspace{3mm}#3
	\end{tcolorbox}
}
\newcommand{\QPicture}[4]{
	\begin{figure}[H]
		\centering
		\includegraphics[width=0.35\linewidth]{#1}
		\caption{#3}
	\end{figure}
	
	#4
}
\newcommand{\ABlock}[1]{
	\vskip2mm
	\begin{tcolorbox}[enhanced,colframe=Magenta,colback=Magenta!15!white,
		frame style={opacity=0.5},interior style={opacity=1.0},
		nobeforeafter,tcbox raise base,shrink tight,extrude by=1.7mm,width=1.6cm]
		\textbf{Ответ:}
	\end{tcolorbox}\hspace{3mm}#1
}
\newcommand{\MBlock}[2]{
	\begin{tcolorbox}[enhanced,colframe=Yellow,colback=Yellow!15!white,
		frame style={opacity=0.5},interior style={opacity=1.0},
		nobeforeafter,tcbox raise base,shrink tight,extrude by=1.7mm,width=1.1cm]
		\textbf{#1}
	\end{tcolorbox}\hspace{3mm}#2
}
\newcommand{\MMBlock}[3]{
	\begin{tcolorbox}[enhanced,colframe=Yellow,colback=Yellow!15!white,
		frame style={opacity=0.5},interior style={opacity=1.0},
		nobeforeafter,tcbox raise base,shrink tight,extrude by=1.7mm,width=1.1cm]
		\textbf{#1}
	\end{tcolorbox}\hspace{3mm}
	\begin{tcolorbox}[enhanced,colframe=Orange,colback=Orange!15!white,
		frame style={opacity=0.5},interior style={opacity=1.0},
		nobeforeafter,tcbox raise base,shrink tight,extrude by=1.7mm,width=0.8cm]
		\textbf{#2}
	\end{tcolorbox}\hspace{3mm}#3
}
% ----------------------------------------


\begin{document}
	
	% настройки
	\pagestyle{style}\thispagestyle{plain}
	\Title
	% ----------------------------------------
	
	%\vskip5mm
	%\centering{\pgfornament[width=5cm,anchor=south]{89}}
	
	% смысловая часть


\QBlock{A1}{0.50}{Покажите, что суммарная сила, действующая на кольцо, определяется выражением:
$$
\vec F_{tot} = - \mu mg \cdot f \left( \frac{v(t)}{\omega (t)\ r} \right) \hat x,
$$
где 
$$
f(a) = \frac{1}{2 \pi} \int \limits_0^{2 \pi} \frac{a - \sin \theta}{\sqrt{ 1 + a^2 - 2 a \sin \theta}} d \theta
$$}

\MBlock{0.10}{Кольцо корректно разбито на малые кусочки и используется интегрирование.}

\MBlock{0.20}{Доказано, что $F_y = 0$}

\MBlock{0.20}{$$
\vec F_{tot} = - \mu mg \cdot \frac{1}{2 \pi} \int \limits_0^{2 \pi} \frac{v - \omega r\sin \theta}{\sqrt{ v^2 + (\omega r)^2 - 2 v \:  \omega r \sin \theta}} d \theta
$$}

\QBlock{A2}{0.50}{Покажите, что суммарный момент, действующий на кольцо, равен:
$$
\tau_{tot} = - \mu mg r f \left(\frac{\omega(t) r}{v(t)} \right)
$$}

\MBlock{0.10}{Кольцо корректно разбито на малые кусочки и используется интегрирование.}

\MBlock{0.40}{$$
\vec \tau = - \mu m g r\int \limits_0^{2 \pi} \frac{\omega r-v \sin \theta }{\sqrt{v^2 + (\omega r)^2 - 2 v \omega r \sin \theta}} \: \frac{d \theta}{2 \pi},
$$}

\QBlock{A3}{0.10}{Докажите, что уравнения движения имеют вид:
$$
\dot v = - \mu g\cdot  f \left( \frac{v}{\omega r} \right) \\
\dot \omega r = - \mu g \cdot f \left( \frac{ \omega r}{v} \right)
$$}

\MBlock{0.05}{Записан второй закон Ньютона}

\MBlock{0.05}{Записано уравнение моментов для вращательного движения}

\QBlock{B1}{0.50}{Докажите:
a) $ f(0) = 0, \: f(1) = \dfrac{2}{\pi}, \: f(\infty) = 1$
b) $ f(a) $ строго возрастает при $a \geqslant 0 $}

\MBlock{3 $\times$ 0.10}{Взяты соответствующие интегралы $f(0)$, $f(1)$ и $f(\infty)$.}

\MBlock{0.20}{Корректно доказано возрастание (например, через производную)}

\QBlock{B2}{0.30}{Рассмотрим поведение параметра $a(t)  = \dfrac{v(t)}{\omega (t) r} $. Покажите, что происходит с  $a(t)$ (растёт/уменьшается/остаётся неизменным) в каждом из следующих случаев:

a) в некоторый момент $a(t) = 1$

b) в некоторый момент $a(t) < 1$

c) в некоторый момент $a(t) > 1$}

\MBlock{0.15}{Взята производная $\dot a$ или другим способом доказано, что система асимптотически стремится к $a \rightarrow 1$}

\MBlock{3 $\times$ 0.05}{Получены верные ответы}

\QBlock{B3}{0.60}{Нарисуйте качественно на графике, осями которого являются $v$ и $\omega r$, траектории, отображающие разное движение кольца, то есть при заданных  $v_0$  и $\omega_0 r$ нарисуйте, как они будут изменяться с течением времени. 

Необходимо нарисовать хотя бы одну траекторию на каждый пункт предыдущего задания. Кроме того, нарисуйте траекторию, проходящую через точку $(v_0, 0)$ и еще одну, начинающуюся в точке $(0, \omega_0 r)$

Подпишите оси графика и укажите направления движения системы для каждой нарисованной траектории}

\MBlock{0.20}{На графике присутствует прямая $a =1$}

\MBlock{0.20}{Траектории асимптотически стремятся к $a = 1$}

\MBlock{0.20}{Указаны верные направления движения}

\QBlock{B4}{0.10}{Вычислите мгновенную мощность, которая расходуется, когда есть только угловая скорость $\omega$ $(v = 0)$, и отдельно,  когда присутствует только линейная $v$ $(\omega = 0)$.}

\MBlock{2 $\times$ 0.05}{Получены верные ответы
$$
P_v = - \mu  mg \:  v \\
P_\omega =  - \mu mg \: \omega r
$$}

\QBlock{B5}{0.60}{Для заданных $v$ и $\omega$ вычислите мгновенную мощность $P$, которая расходуется на трение в данный момент времени. Дайте ответ в виде интеграла с безразмерной переменной.}

\MBlock{0.10}{Корректно используется идея разбиения системы на малые кусочки и выполняется интегрирование мощности по ним}

\MBlock{0.50}{Получен верный ответ
$$
P = - \mu mg \: \omega r \int \limits_0^{2 \pi} \sqrt{1+\left(\frac{v}{\omega r} \right)^2 - 2 \left(\frac{v}{\omega r} \right) \sin (\theta)} \: \:\frac{d \theta }{2 \pi}
$$}

\QBlock{B6}{1.20}{Предположим, что кольцу придали определённую начальную кинетическую энергию $E_0$. Каково должно быть соотношение $a_0 = \dfrac{v_0}{\omega_0 r}$, при котором кольцо будет двигаться максимальное время?

Подсказка: Постарайтесь дать ответ на предыдущий пункт при помощи только $E_0$ и $a_0$ (и других данных из этого пункта), исключив из уравнения $v$ и $\omega$}

\MBlock{0.50}{Показано, что в точке $a_0 = 1$ достигается экстремум мощности}

\MBlock{0.50}{Доказано, что минимальная мощность при заданной энергии достигается при
$$
P_{min}(E_0) = P(1, E_0)
$$}

\MBlock{0.20}{Указан тот факт, что при $a_0 = 1$ в дальнейшем движении $a = a_0$}

\QBlock{B7}{0.50}{Каково максимальное время движения при начальной энергии $E_0$?}

\MBlock{0.10}{В уравнении для мощности разделены переменные и проведено интегрирование}

\MBlock{0.40}{Получен верный ответ
$$
\tau = \frac{\pi}{2 \mu g} \sqrt{\frac{E_0}{m}}
$$}

\QBlock{C1}{0.60}{Напишите заново уравнения движения из пункта А3 таким образом, чтобы они подходили под новое условие.}

\MBlock{2 $\times$ 0.30}{Получены следующие уравнения движения (или аналогичные):
$$
\dot v = -\mu g \: f\left( \frac{v}{\omega r}\right) \cos{\alpha} + g \sin{\alpha} \cos{\varphi} \\
\dot \omega r = - \mu g \: f \left(\frac{\omega r}{v} \right) \cos \alpha
$$}

\MBlock{-0.20}{В законе Ньютона забыт $\cos \varphi$}

\QBlock{C2}{2.00}{При заданных начальных $\omega_0$ и $v_0 = 0$ нарисуйте все возможные семейства траекторий движения кольца  в координатах $(v, \omega r)$ (для каждого типа кривых нарисуйте свой график).  Укажите следующие составляющие:

a) cоответствующие значения параметров;

b) конечные точки (в которые траектории приходят за конечное или бесконечное время) в плоскости $(v, \omega r)$ . Здесь достаточно написать для каждой составляющей, что она стремится к нулю/ стремится к бесконечности/ равна или стремится к какой-то положительной величине.

Подпишите оси графика и укажите направления движения системы для каждой нарисованной траектории}

\MBlock{0.50}{Верно проанализированы $\dot v$ и $\dot \omega r$ для различных $\tan \alpha$.}

\MBlock{3 $\times$ 0.50}{Для всех случаев верно указаны конечные точки траекторий и направления движения}

\end{document}