
%This file is part of Get pho.rs!

%Get pho.rs! is free software: you can redistribute it and/or modify it under the terms of the GNU General Public License as published by the Free Software Foundation, either version 3 of the License, or (at your option) any later version.

%Get pho.rs! is distributed in the hope that it will be useful, but WITHOUT ANY WARRANTY; without even the implied warranty of MERCHANTABILITY or FITNESS FOR A PARTICULAR PURPOSE. See the GNU General Public License for more details.

%You should have received a copy of the GNU General Public License along with Foobar. If not, see <https://www.gnu.org/licenses/>.

%\documentstyle[12pt,russian,amsthm,amsmath,amssymb]{article}
\documentclass[a4paper,11pt,twoside]{article}
\usepackage[left=14mm, top=10mm, right=14mm, bottom=10mm, nohead, nofoot]{geometry}
\usepackage{amsmath, amsfonts, amssymb, amsthm} % стандартный набор AMS-пакетов для математ. текстов
\usepackage{mathtext}
\usepackage[utf8]{inputenc} % кодировка utf8
\usepackage[russian]{babel} % русский язык
\usepackage[pdftex,dvipsnames]{xcolor} % работа с цветами
\usepackage[pdftex]{graphicx} % графика (картинки)
\usepackage{tikz} % рисунки
\usepackage{fancyhdr,pageslts} % настройка колонтитулов
\usepackage{enumitem} % работа со списками
\usepackage{multicol} % работа с таблицами
%\usepackage{pscyr} % красивый шрифт
\usepackage{pgfornament} % красивые рюшечки и вензеля
\usepackage{ltxgrid} % управление написанием текста в две колонки
\usepackage{lipsum} % стандартный текст
\usepackage{tcolorbox} % рамка вокруг текста
\usepackage{float} % для корректного размещения картинок
\tcbuselibrary{skins}
% ----------------------------------------

\newcommand\ProblemName{Снижение орбиты МКС}

\newcommand\Source{A23}

\newcommand\Type{Условие задачи}

% настройки полей
\geometry{
	left=12mm,
	top=21mm,
	right=15mm,
	bottom=26mm,
	marginparsep=0mm,
	marginparwidth=0mm,
	headheight=22pt,
	headsep=2mm,
	footskip=7mm}
% ----------------------------------------

% настройки колонтитулов
\pagestyle{fancy}

\fancypagestyle{style}{
	\fancyhf{}
	\fancyhead[L]{{\Large{\FancyTitle}}\\\vskip -5pt \dotfill}
	\fancyhead[R]{{\Large{\textbf{\Type}}}\\\vskip -5pt \dotfill}
	\renewcommand{\headrulewidth}{0pt}
	\renewcommand{\footrulewidth}{0pt}
	\fancyfoot[C]{\pgfornament[width=2em,anchor=south]{72}\hspace{1mm}
		{Страница \textbf{\thepage} из \textbf{\pageref{VeryLastPage}}}\hspace{2mm}
		\pgfornament[width=2em,symmetry=v,anchor=south]{72}\\ \vskip2mm
		{\small{\textit{Условие собрано и подготовлено в Президентском ФМЛ №239 г.~Санкт-Петербурга}}}}
}

\fancypagestyle{plain}{
	\fancyhf{}
	\renewcommand{\headrulewidth}{0pt}
	\renewcommand{\footrulewidth}{0pt}
	\fancyhead[C]{{\Large{\textit{Учебно-тренировочные сборы к X23}}}\\\vskip -5pt \dotfill}
	\fancyfoot[C]{\pgfornament[width=2em,anchor=south]{72}\hspace{1mm}
		{Страница \textbf{\thepage} из \textbf{\pageref{VeryLastPage}}}\hspace{2mm}
		\pgfornament[width=2em,symmetry=v,anchor=south]{72}\\ \vskip2mm
		{\small{\textit{Условие собрано и подготовлено в Президентском ФМЛ №239 г.~Санкт-Петербурга}}}}
}
% ----------------------------------------

% другие настройки
\pagenumbering{arabic}
\setlist[enumerate,itemize]{leftmargin=0pt,itemindent=2.7em,itemsep=0cm}
% ----------------------------------------

% собственные команды
\newcommand{\FancyTitle}{\textbf{\Source} --- \ProblemName}
\newcommand{\Title}{\begin{center}{\huge{\textbf{\Source} --- \ProblemName}}\end{center}}
\newcommand{\Chapter}[1]{\vskip5pt{\Large{\textbf{#1}}}\vskip5pt}
\newcommand{\QText}[1]{#1}
\newcommand{\QBlock}[3]{
	\begin{tcolorbox}[left=4mm,top=3mm,bottom=2mm,right=4mm,colback=white]
		\begin{tcolorbox}[enhanced,colframe=blue,colback=blue!10!white,
			frame style={opacity=0.3},interior style={opacity=1.0},
			nobeforeafter,tcbox raise base,shrink tight,extrude by=1.7mm,width=1.5cm]
			\textbf{#1\textsuperscript{#2}}
		\end{tcolorbox}\hspace{3mm}#3
	\end{tcolorbox}
}
\newcommand{\QPicture}[4]{\QText{#4}  \includegraphics{#1}}
\newcommand{\ABlock}[1]{#1}
\newcommand{\MBlock}[2]{#1 #2}
\newcommand{\MMBlock}[3]{#1 #2 #3}
% ----------------------------------------


\begin{document}
	
	% настройки
	\pagestyle{style}\thispagestyle{plain}
	\Title
	% ----------------------------------------
	
	%\vskip5mm
	%\centering{\pgfornament[width=5cm,anchor=south]{89}}

\Chapter{Введение}

\QPicture{A23 Снижение орбиты МКС_files/28061.jpeg}{"max-width:500px;px;display:block;margin-left: auto;margin-right: auto;"}{Рис. 1: Международная космическая станция на орбите.}{}

\QText{В настоящий момент Международная Космическая Станция (сокращённо МКС, англ. ISS - International Space Station) двигается по орбите, близкой к круговой. Минимальное среднее расстояние до Земли при её движении составляет $370 км$, максимальное - $460 км$. Станция двигается в термосфере. Плоскость орбиты составляет угол $\theta = 51.6^\circ$ с плоскостью экватора. Траектория станции похожа на спираль с медленно меняющимся расстоянием от неё до поверхности Земли. Изменение данного расстояния за один оборот вокруг Земли незначительно.}

\QText{Масса МКС составляет $M_s = 4.5 \times 10^5 кг$, общая длина $L_s = 109 м$. Огромные солнечные панели шириной $W_S = 73 м$ обеспечивают МКС электрической энергией [Официальный Отчёт NASA (2023)].}

\QText{С учётом всех батарей и других частей, эффективная площадь поперечного сечения станции составляет приблизительно $S \approx 2.5 \times 10^3 м^2$ [Европейское Космическое Агентство, SDC6-23].}

\QText{Снижение орбиты МКС связано с несколькими явлениями, уменьшающими энергию орбитального движения. Основными из них являются:}

\QText{\begin{itemize} 
\item сопротивление атмосферы, вызванное частыми столкновениями молекул газа со станцией;
\item сила Ампера, возникающая при движении проводящих частей МКС в магнитном поле Земли;
\item взаимодействие с ионами атомарного кислорода.
\end{itemize}}

\QText{"... В мае 2008 высота орбиты МКС составляла 350 километров, станция потеряла 4.5 км высоты, а затем с помощью транспортного грузового космического корабля Прогресс М-64 была поднята на 5.5 км ..." [https://mod.jsc.nasa.gov]}

\QPicture{A23 Снижение орбиты МКС_files/28065.jpeg}{"max-width:500px;px;display:block;margin-left: auto;margin-right: auto;"}{Рис. 2: Высота орбиты МКС (км) в период с 1998 по 2019 год.}{}

\QPicture{A23 Снижение орбиты МКС_files/28066.jpeg}{"max-width:500px;px;display:block;margin-left: auto;margin-right: auto;"}{Рис. 3: Средняя высота орбиты МКС (км) в 2022-2023 годах.}{}

\QText{"... МКС теряет до 100 м высоты каждый день... " [Данные NASA (2021)]. В 2023 году МКС движется на высоте 410 км, и снижается приблизительно на 70 м в день ($\sim$ 2 км в месяц). Во время магнитных бурь снижение достигает 300 м. МКС корректирует свою орбиту с помощью своих двигателей, а также двигателей посещающих её аппаратов [Отчёт о МКС (2022)].}

\QPicture{A23 Снижение орбиты МКС_files/28069.jpeg}{"max-width:500px;px;display:block;margin-left: auto;margin-right: auto;"}{Рис. 4: Площадь сечения МКС при взгляде с различных углов (дм$^2$).}{}

\Chapter{Обозначения и физические постоянные:}

\QText{\begin{tabular}{|c|c|c|c|} \hline 
Универсальная газовая постоянная & $R$ & $=$ & $8.31~$Дж$\cdot$К$^{-1} \cdot $моль$^{-1}$\\ 
 \hline 
Число Авогадро & $N_A$ & $=$ & $6.022 \cdot 10^{23}~$моль$^{-1}$\\ 
 \hline 
Молярная масса воздуха & $\mu$ & $=$ & $0.029~$кг$\cdot$моль$^{-1}$\\ 
 \hline 
Масса Земли & $M_E$ & $=$ & $5.97 \cdot 10^{24}~$кг\\ 
 \hline 
Радиус Земли & $R_E$ & $=$ & $6.38 \cdot 10^6~$м\\ 
 \hline 
Гравитационная постоянная & $G$ & $=$ & $6.67 \cdot 10^{-11}~$м$^3 \cdot$с$^{-2} \cdot$кг$^{-1}$\\ 
 \hline 
Плотность воздуха на поверхности Земли & $\rho_0$ & $=$ & $1.29~$кг$/$м$^3$\\ 
 \hline 
Ускорение свободного падения на поверхности Земли & $g_0$ & $=$ & $9.81~$м$\cdot$с$^{-2}$\\ 
 \hline 
Средняя величина магнитного поля Земли & $B$ & $=$ & $5.0 \cdot 10^{-5}~$Тл\\ 
 \hline 
Модуль заряда электрона & $e$ & $=$ & $1.60 \cdot 10^{-19}~$Кл\\ 
 \hline 
\end{tabular}}

\Chapter{Часть A. Уточнённая барометрическая формула (2 балла)}

\QText{Воздух в атмосфере состоит по большей части из нейтральных молекул $O_2$ и $N_2$. Воздух подчиняется уравнению Менделеева-Клапейрона: $pV = \frac{M}{\mu} RT$, где $p$, $V$, $T$, $M$ и $\mu$ это давление, объём, температура, масса и молярная масса порции газа соответственно, $R$ - универсальная газовая постоянная.}

\QText{Есть два уравнения для вычисления зависимости давления воздуха от высоты. Первое уравнение применимо к стандартной модели \textbf{тропосферы }(высоты $h<100 $км). Оно предполагает изменение температуры с высотой.}

\QText{Второе уравнение относится к стандартной модели \textbf{термосферы }($h>250 $км). В нём температура почти не зависит от высоты. Это уравнение применимо к исследованию движения МКС.}

\QText{В этой задаче можно считать давление гидростатическим и изотропным (то есть оно действует одинаково по всем направлениям).}

\QBlock{A1}{0.50}{Найдите зависимость давления $p_h$ от высоты $h$. Зависимость может содержать интегральное выражение. Это уравнение называется основной барометрической формулой.}

\QText{\textit{Примечание 1. }Температура термосферы Земли на высотах $300 - 600 $км меняется незначительно и в среднем составляет $800 - 900 $К на солнечной стороне [Данные NASA]. Следовательно, при исследовании полёта МКС мы можем считать $T_h = T = const$. Так как космический корабль проводит почти половину времени полёта с теневой стороны Земли, где температура резко снижается, можно считать, что в среднем на этих высотах температура равна $T = 425 $К.}

\QText{Эта температура также соответствует значению плотности воздуха $\rho_h \sim 10^{-12} $кг$/$м$^3$ на высоте $h = 400 $км [Модель Верхних Слоёв Атмосферы Земли MSISE-90].}

\QPicture{A23 Снижение орбиты МКС_files/28076.jpeg}{"max-width:500px;px;display:block;margin-left: auto;margin-right: auto;"}{Рис. 5: Термосфера Земли.}{}

\QBlock{A2}{0.30}{Получите стандартную барометрическую формулу: зависимость давления от высоты $p_h^{sta}$, считая, что температура и ускорение свободного падения не зависят от $h$.

Рассчитайте величину $h_0 = \frac{RT}{\mu g_0}$ при $T = 425 $К.}

\QBlock{A3}{0.60}{Получите уточнённую барометрическую формулу: зависимость давления от высоты $p_h^{imp}$, считая, что температура постоянна, а ускорение свободного падения зависит от высоты $h$.}

\QBlock{A4}{0.40}{Рассчитайте отношение значений давлений, вычисленных по стандартной и по уточнённой барометрическим формулам при $h = 4.0 \times 10^5 $м. Далее используйте уточнённую формулу.}

\QBlock{A5}{0.20}{Найдите плотность воздуха $\rho_h$ и концентрацию нейтральных молекул воздуха $n_h$ на высоте $h$, используя линейное приближение.}

\Chapter{Часть B. Орбитальное замедление и скорость снижения станции (3 балла)}

\QText{Пусть на станцию массой $M_S$ действует постоянная тормозящая сила $\vec{F}_{drag}$. В этой части задачи оценивается скорость уменьшения высоты орбиты МКС. Считайте, что изменение высоты $dh$ значительно меньше высоты полёта $h$ ($dh \ll h$).}

\QBlock{B1}{0.50}{Найдите скорость станции $v_h$ и период обращения $\tau_h$, если станция движется по орбите высотой $h$.}

\QBlock{B2}{0.50}{Найдите полную энергию $E_S$ станции, двигающейся по круговой орбите радиусом $R_E + h$.}

\QBlock{B3}{1.00}{На станцию действует некоторая суммарная тормозящая сила $\vec{F}_{drag}$. В результате МКС замедляется, и высота её орбиты уменьшается на $dh$ за малое время $dt$. Запишите закон изменения энергии МКС, считая известным значение $F_{drag}$.}

\QBlock{B4}{0.50}{Найдите скорость снижения станции $u_h$.}

\QBlock{B5}{0.50}{Найдите изменение высоты $H_h$ станции за один оборот вокруг Земли и полное время $T_h$, за которое станция упадёт на поверхность Земли с начальной высоты $h$.}

\Chapter{Часть С. Сопротивление атмосферы (1 балл)}

\QText{Скорость станции $v$ во много раз больше чем средние скорости (сотни м$/$с) теплового движения молекул в атмосфере на высоте $h \approx 300 - 400 $км, то есть можно считать, что молекулы покоятся перед столкновением с МКС. Для грубой оценки считайте, что после столкновения молекулы приобретают такую же скорость, что и станция.}

\QBlock{C1}{0.50}{Найдите силу сопротивления воздуха $F_{air}$, скорость уменьшения высоты орбиты $u_h^{air}$ и изменение высоты за один оборот $H^{air}_h$ в этом случае.}

\QBlock{C2}{0.50}{Найдите полное время $T_h^{air}$,  за которое станция упадёт на поверхность Земли с начальной высоты $h$ из-за сопротивления атмосферы.}

\Chapter{Часть D. Учёт атомарного кислорода (1 балл)}

\QText{В термосфере под действием излучения воздух ионизируется (из-за чего, например, возникает "северное сияние"). В отличие от кислорода $O_2$, азот $N_2$ не диссоциирует под действием солнечного излучения, поэтому в верхних слоях атмосферы атомарного азота $N$ гораздо меньше, чем атомарного кислорода $O$. На высотах больше $250 $км преобладает атомарный кислород $O$. Слои, содержащие электроны и ионы атомов кислорода, возникают на освещенной стороне атмосферы. В этом случае концентрация ионов атомарного кислорода составляет $n_{ion} \sim 10^{12} м^{-3}$.}

\QBlock{D1}{0.30}{Найдите среднюю (за 24 часа) тормозящую силу $F_{ion}$, обусловленную столкновениями с этими частицами. Ночью ионизацией молекул можно пренебречь.

Найдите также плотность ионизированных молекул кислорода $\rho_{ion}$.}

\QBlock{D2}{0.70}{Найдите скорость уменьшения высоты орбиты станции $u_h^{ion}$, связанную со взаимодействием с ионами атомарного кислорода. Найдите также изменение высоты за один оборот $H_h^{ion}$ в этом случае.}

\Chapter{Часть E. Торможение магнитным полем Земли (2 балла)}

\QText{Рассмотрим влияние магнитного поля Земли на движение станции. Магнитное поля Земли вблизи поверхности изменяется в диапазоне $(3.5 - 6.5) \cdot 10^{-5} $Тл , среднее значение составляет $B = 5 \cdot 10^{-5} $Тл.

Когда станция движется с большой скоростью в магнитном поле, в её проводящих элементах возникает электрический ток. Электродвижущая сила вызывает перераспределение электрических зарядов в проводящих элементах станции. Электрическое поле вблизи станции приводит к движению заряженных частиц в окружающем её пространстве. Электроны притягиваются к частям станции, у которых положительный потенциал (относительно средней части станции), а ионы притягиваются к частям с отрицательным потенциалом. Электроны и ионы, сталкиваясь с поверхностью станции, стремятся собираться в нейтральные атомы кислорода. Эти электроны, которые движутся по проводящим частям станции, и создают электрический ток. Станция, двигаясь в космосе, "собирает" электроны и ионы из окружающего пространства, сталкиваясь с ними.

Для грубой оценки тока, возникающего в проводящих частях станции, считайте, что она "собирает" частицы только с площади, эквивалентной площади поперечного сечения станции $S$. Также считайте, что все ионы и электроны участвуют в создании тока.}

\QBlock{E1}{0.60}{Оцените величину возникающего в проводящих частях станции тока $I_{ind}$.}

\QBlock{E2}{0.60}{Получите приближённое выражение для тормозящей силы Ампера $F_{ind}$ в направлении, противоположном направлению движению станции.

Пусть $\phi$ - угол между магнитным полем Земли $\vec{B}$, направленным вдоль меридианов, и скоростью МКС $\vec{v}$. Для простоты считайте, что длина станции $L$ равна корню квадратному из её площади $S$. Кроме того, вместо подсчёта среднего значения $\sin(\phi)$ вы можете аппроксимировать его значением $\sin(\pi/2 - \theta)$. Вы можете использовать дискретное число точек для подсчёта среднего значения.}

\QBlock{E3}{0.80}{Найдите скорость снижения станции из-за её взаимодействия с магнитным полем Земли. Найдите также изменение высоты за один оборот $H_h^{ind}$ в этом случае.}

\Chapter{Часть F. Численные расчёты и выводы (1 балл)}

\QBlock{F1}{0.40}{}

\QText{Рассчитайте необходимые величины и заполните Таблицу 1 в листе ответов.}

\QText{\begin{tabular}{|c|c|c|c|c|c|c|} \hline 
$h, км$ & $T_h^{air}, дней$ & $u_{air}, м/день$ & $u_{ion}, м/день$ & $u_{ind}, м/день$ & $\sum, м/день$ & $u_{ISS}, м/день$\\ 
 \hline 
350 &   &   &   &   &   &  \\ 
 \hline 
375 &   &   &   &   &   &  \\ 
 \hline 
400 &   &   &   &   &   &  \\ 
 \hline 
410 &   &   &   &   &   &  \\ 
 \hline 
\end{tabular}}

\QBlock{F2}{0.40}{}

\QText{Рассчитайте необходимые величины и заполните Таблицу 2 в листе ответов.}

\QText{\begin{tabular}{|c|c|c|c|} \hline 
$h, км$ & $H_h^{air}, м$ & $H_h^{ion}, м$ & $H_h^{ind}, м$\\ 
 \hline 
350 &   &   &  \\ 
 \hline 
375 &   &   &  \\ 
 \hline 
400 &   &   &  \\ 
 \hline 
410 &   &   &  \\ 
 \hline 
\end{tabular}}

\QBlock{F3}{0.20}{МКС обращается по орбите на высотах выше 380 км. Расположите три рассмотренных эффекта торможения станции в порядке убывания их влияния.}

\end{document}