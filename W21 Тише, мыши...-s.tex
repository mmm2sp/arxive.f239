
%\documentstyle[12pt,russian,amsthm,amsmath,amssymb]{article}
\documentclass[a4paper,11pt,twoside]{article}
\usepackage[left=14mm, top=10mm, right=14mm, bottom=10mm, nohead, nofoot]{geometry}
\usepackage{amsmath, amsfonts, amssymb, amsthm} % стандартный набор AMS-пакетов для математ. текстов
\usepackage{mathtext}
\usepackage[utf8]{inputenc} % кодировка utf8
\usepackage[russian]{babel} % русский язык
\usepackage[pdftex]{graphicx} % графика (картинки)
\usepackage{tikz}
\usepackage{fancyhdr,pageslts} % настройка колонтитулов
\usepackage{enumitem} % работа со списками
\usepackage{multicol} % работа с таблицами
%\usepackage{pscyr} % красивый шрифт
\usepackage{pgfornament} % красивые рюшечки и вензеля
\usepackage{ltxgrid} % управление написанием текста в две колонки
\usepackage{lipsum} % стандартный текст
\usepackage{tcolorbox} % рамка вокруг текста
\tcbuselibrary{skins}
% ----------------------------------------

\newcommand\ProblemName{Тише, мыши...}

\newcommand\Source{W21}

\newcommand\Type{Решение}

\newcommand\MyTextLeft{Президентский ФМЛ 239, г.~Санкт-Петербург}
\newcommand\MyTextRight{Использованы материалы сайта pho.rs}
\newcommand\MyHeading{Учебно-тренировочные сборы по физике}
% ----------------------------------------

% настройки полей
\geometry{
	left=12mm,
	top=21mm,
	right=15mm,
	bottom=26mm,
	marginparsep=0mm,
	marginparwidth=0mm,
	headheight=22pt,
	headsep=2mm,
	footskip=7mm}
% ----------------------------------------

% настройки колонтитулов
\pagestyle{fancy}

\fancypagestyle{style}{
	\fancyhf{}
	\fancyhead[L]{{\Large{\FancyTitle}}\\\vskip -5pt \dotfill}
	\fancyhead[R]{{\Large{\textbf{\Type}}}\\\vskip -5pt \dotfill}
	\renewcommand{\headrulewidth}{0pt}
	\renewcommand{\footrulewidth}{0pt}
	\fancyfoot[C]{\pgfornament[width=2em,anchor=south]{72}\hspace{1mm}
		{Страница \textbf{\thepage} из \textbf{\pageref{VeryLastPage}}}\hspace{2mm}
		\pgfornament[width=2em,symmetry=v,anchor=south]{72}\\ \vskip2mm
		{\small{\textit{\MyTextLeft\hfill\MyTextRight}}}}
}

\fancypagestyle{plain}{
	\fancyhf{}
	\renewcommand{\headrulewidth}{0pt}
	\renewcommand{\footrulewidth}{0pt}
	\fancyhead[C]{{\Large{\textit{\MyHeading}}}\\\vskip -5pt \dotfill}
	\fancyfoot[C]{\pgfornament[width=2em,anchor=south]{72}\hspace{1mm}
		{Страница \textbf{\thepage} из \textbf{\pageref{VeryLastPage}}}\hspace{2mm}
		\pgfornament[width=2em,symmetry=v,anchor=south]{72}\\ \vskip2mm
		{\small{\textit{\MyTextLeft\hfill\MyTextRight}}}}
}
% ----------------------------------------

% другие настройки
\pagenumbering{arabic}
\setlist[enumerate,itemize]{leftmargin=0pt,itemindent=2.7em,itemsep=0cm}
% ----------------------------------------

% собственные команды
\newcommand{\FancyTitle}{\textbf{\Source} --- \ProblemName}
\newcommand{\Title}{\begin{center}{\huge{\textbf{\Source} --- \ProblemName}}\end{center}}
\newcommand{\Chapter}[1]{\vskip5pt{\Large{\textbf{#1}}}\vskip5pt}
\newcommand{\QText}[1]{#1}
\newcommand{\QBlock}[3]{
	\begin{tcolorbox}[left=2mm,top=2mm,bottom=1mm,right=2mm,colback=white]
		\begin{tcolorbox}[enhanced,colframe=ProcessBlue,colback=ProcessBlue!30!white,
			frame style={opacity=0.7},interior style={opacity=1.0},
			nobeforeafter,tcbox raise base,shrink tight,extrude by=1.7mm,width=1.5cm]
			\textbf{#1\textsuperscript{#2}}
		\end{tcolorbox}\hspace{3mm}#3
	\end{tcolorbox}
}
\newcommand{\QPicture}[4]{
	\begin{figure}[H]
		\centering
		\includegraphics[width=0.35\linewidth]{#1}
		\caption{#3}
	\end{figure}
	
	#4
}
\newcommand{\ABlock}[1]{
	\vskip2mm
	\begin{tcolorbox}[enhanced,colframe=Magenta,colback=Magenta!15!white,
		frame style={opacity=0.5},interior style={opacity=1.0},
		nobeforeafter,tcbox raise base,shrink tight,extrude by=1.7mm,width=1.6cm]
		\textbf{Ответ:}
	\end{tcolorbox}\hspace{3mm}#1
}
\newcommand{\MBlock}[2]{
	\begin{tcolorbox}[enhanced,colframe=Yellow,colback=Yellow!15!white,
		frame style={opacity=0.5},interior style={opacity=1.0},
		nobeforeafter,tcbox raise base,shrink tight,extrude by=1.7mm,width=1.1cm]
		\textbf{#1}
	\end{tcolorbox}\hspace{3mm}#2
}
\newcommand{\MMBlock}[3]{
	\begin{tcolorbox}[enhanced,colframe=Yellow,colback=Yellow!15!white,
		frame style={opacity=0.5},interior style={opacity=1.0},
		nobeforeafter,tcbox raise base,shrink tight,extrude by=1.7mm,width=1.1cm]
		\textbf{#1}
	\end{tcolorbox}\hspace{3mm}
	\begin{tcolorbox}[enhanced,colframe=Orange,colback=Orange!15!white,
		frame style={opacity=0.5},interior style={opacity=1.0},
		nobeforeafter,tcbox raise base,shrink tight,extrude by=1.7mm,width=0.8cm]
		\textbf{#2}
	\end{tcolorbox}\hspace{3mm}#3
}
% ----------------------------------------


\begin{document}
	
	% настройки
	\pagestyle{style}\thispagestyle{plain}
	\Title
	% ----------------------------------------
	
	%\vskip5mm
	%\centering{\pgfornament[width=5cm,anchor=south]{89}}
	
	% смысловая часть


\QBlock{A1}{5.00}{Найдите максимально возможное начальное расстояние $L_{max}$ между мышами и Леопольдом.}

\QText{Предисловие: Последняя “перестрелка” Леопольда и мышей произошла в 2002 году, с тех пор мыши кота не беспокоили. Спустя 18 лет, в 2020 году, нашлись новые подводные камушки для легендарной рогатки и мыши решили “тряхнуть стариной”.
Примечание: в ходе перестрелки никто из животных не пострадал.}

\QPicture{W21 Тише, мыши...-s_files/407.jpeg}{"width:100\%; max-width: 100\%;display:block;margin-left: auto;margin-right: auto; cursor:pointer;" data-toggle="modal" data-target="#kt_modal_407"}{}{В системе отсчёта Леопольда камень движется прямолинейно с постоянной скоростью $v_{отн}$. Расстояние от Леопольда до камня минимально, когда соединяющий их отрезок перпендикулярен вектору относительной скорости (см. рис.). Из подобия треугольников определим перемещение камня относительно Леопольда:}

\QText{$$S_{отн}=L \frac{v_1}{v_{отн}} =v_{отн}t$$
откуда
$$L=\frac{v^2_{отн}t}{v_1}$$}

\QPicture{W21 Тише, мыши...-s_files/405.jpeg}{"width:100\%; max-width: 100\%;display:block;margin-left: auto;margin-right: auto; cursor:pointer;" data-toggle="modal" data-target="#kt_modal_405"}{}{Найдем максимально возможное время, через которое скорости камня и Леопольда вновь станут перпендикулярны. Построим треугольники скоростей камня и Леопольда для максимального сближения, объединив их в один четырёхугольник. Заметим, что этот четырёхугольник можно вписать в окружность, поскольку сумма противоположных углов равна 180°. Диаметр окружности фиксирован и равен $v_{отн}$. Поскольку $L$ максимально при максимальном значении $t$, необходимо, чтобы вторая диагональ четырёхугольника, равная $gt$, также была максимальна. Это достигается если $gt$ – диаметр данной окружности. Таким образом,}

\QText{$$t_{max}=\frac{v_{отн}}{g}$$
и окончательно:
$$L_{max}=\frac{v^3_{отн}}{gv_1}=\frac{\left({v^2_1}+{v^2_2}\right)^\frac{3}{2}}{gv_1}$$}

\Chapter{Второе решение}

\QPicture{W21 Тише, мыши...-s_files/402.jpeg}{"width:100\%; max-width: 100\%;display:block;margin-left: auto;margin-right: auto; cursor:pointer;" data-toggle="modal" data-target="#kt_modal_402"}{}{Введём систему координат, как показано на рисунке. Пусть угол, под которым брошен камень к горизонту, равен $\alpha$. Получим зависимости координат, проекций скоростей и расстояния  между камнем и Леопольдом от времени.}

\QText{$$x_к=v_1t\cos\alpha; y_к=v_1t \sin \alpha-\frac{gt^2}{2}$$
$$v_{yк}=v_1\sin\alpha -gt; v_{xк}=v_1\cos\alpha $$
$$x_л=L \cos \alpha-v_2 t \sin\alpha; y_л=L \sin\alpha+v_2t \cos\alpha-\frac{gt^2}{2}$$
$$v_{yл}=v_1\cos\alpha-gt; v_{xл}=-v_1\sin\alpha $$
Расстояние между Леопольдом и камнем назовём $l$.
$$l^2=(x_к-x_л )^2+(y_к-y_л )^2$$
$$l^2=(L \cos\alpha -(v_1\cos\alpha +v_2\sin\alpha  )t)^2+(L \sin \alpha-(v_1\sin\alpha -v_2\cos\alpha  )t)^2$$
$$l^2=L^2-2v_1Lt+(v^2_1+v^2_2 ) t^2$$
Минимальное значение $l$ достигается в вершине параболы в момент времени
$$\tau=Lv_1/(v^2_1+v^2_2 )$$ 
Для нахождения максимально возможного значения $L$ необходимо найти максимально возможное значение $\tau$.
Пусть в момент времени $\tau$ скорость камня направлена под углом $\beta$ к горизонту. Поскольку в данный момент времени скорости Леопольда и камня перпендикулярны, скорость Леопольда направлена под углом $90-\beta$ к горизонту. Значит, для этого момента времени можно записать
$$v_{yк}/v_{xк} =-v_{xл}/v_{yл}$$
$$(v_1\sin\alpha -g\tau )(v_2\cos\alpha-g\tau)-v_1v_2\sin\alpha \cos\alpha =0$$
Из последнего соотношения находим $\tau$
$$\tau = \frac{v_1 \sin \alpha + v_2 \cos \alpha}{g}$$
Комбинируя два способа получения $\tau$, получим зависимость расстояния от угла броска камня
$$L=\frac{(v^2_1+v^2_2 )(v_1 \sin\alpha+v_2\cos\alpha)}{gv_1}$$
Для нахождения максимума $L$ необходимо найти максимум величины $v'=(v_1\sin\alpha +v_2\cos\alpha )$
$$(v'-v_1\sin\alpha)^2=v^2_2(1-\sin^2\alpha )
$$
$$(v^2_1+v^2_2)\sin^2\alpha -2v' v_1\sin\alpha+{v^2}'-v^2_2=0
$$
Найдём дискриминант квадратного уравнения относительно $\sin \alpha$
$$D=4{v^2}' v^2_1-4(v^2_1+v^2_2 )({v^2}'-v^2_2 )
$$
$$D=4v^2_2 (v^2_1+v^2_2 )-{v^2}')
$$
Максимальное значение $v'$ достигается при нулевом дискриминанте и равно
$$v'=\sqrt{v^2_1+v^2_2}
$$
Таким образом
}

\ABlock{$$L_{max}=\frac{(v^2_1+v^2_2 )^\frac{3}{2}}{gv_1}$$}

\QBlock{A2}{5.00}{Найдите значения скоростей $v_1$ и $v_2$. Ускорение свободного падения $g=9,8 \text{м}/{\text{с}^2}$.}

\QPicture{W21 Тише, мыши...-s_files/413.jpeg}{"width:100\%; max-width: 100\%;display:block;margin-left: auto;margin-right: auto; cursor:pointer;" data-toggle="modal" data-target="#kt_modal_413"}{}{Заметим, что вектор средней скорости камня в момент фотографии равен $\frac{\vec{v_{отн}}}{2}$, в вектор средней скорости Леопольда $-\frac{\vec{v_{отн}}}{2}$. Это значит, что перемещения камня и Леопольда равны по модулю и противоположны по направлению. Пусть $D$ - середина отрезка $AC$. Тогда из условия равенства перемещений по модулю следует, что $BD=\frac{L}{2}$, а также, что треугольник $DCB$ - прямоугольный, причём $\frac{v_2}{v_1}=\frac{CD}{BC}$. Измеряя $BC$, $BD$ и $CD$, получим}

\QText{$$v_2=\frac{CD}{BC}v_1$$ $$BD=\sqrt{BC^2+CD^2}=\frac{(v_1^2+v_2^2)^\frac{3}{2}}{2gv_1}=\frac{(BC^2+CD^2)^\frac{3}{2}}{BC^3}\frac{v_1^2}{2g}$$}

\QPicture{W21 Тише, мыши...-s_files/411.jpeg}{"width:100 ; max-width: 100;display:block;margin-left: auto;margin-right: auto; cursor:pointer;" data-toggle="modal" data-target="#kt_modal_411"}{}{$$v_1=\sqrt{\frac{2gBC}{BC^2+CD^2}}BC=10,\!8 \text{м/с}$$
$$v_2=\sqrt{\frac{2gBC}{BC^2+CD^2}}CD=2,\!7 \text{м/с}$$}

\ABlock{$$v_1=\sqrt{\frac{2gBC}{BC^2+CD^2}}BC=10,\!8 \text{м/с}$$
$$v_2=\sqrt{\frac{2gBC}{BC^2+CD^2}}CD=2,\!7 \text{м/с}$$}

\Chapter{Второе решение}

\QText{Выражение для $\sin\alpha$ из решения предыдущего пункта
$$\sin\alpha=\frac{v_1}{\sqrt{v^2_1+v^2_2}}
$$
Найдём координаты камня и расстояние между ним и Леопольдом в момент времени $\tau$
$$x_к=v_1\tau\cos\alpha=\frac{v_1v_2}{g}
$$
$$y_к=v_1\tau\sin\alpha-\frac{g{\tau}^2}{2}=\frac{v^2_1}{g}-\frac{v^2_1+v^2_2}{2g}=\frac{v^2_1-v^2_2}{2g}
$$
$$l=\frac{(v^2_1+v^2_2 )v_2}{gv_1}
$$
Таким образом, перемещение камня
$$S_к=\frac{v^2_1+v^2_2}{2g}
$$
Измеряя перемещение камня, а также расстояние между ним и Леопольдом, мы получим систему из двух уравнений с двумя неизвестными.
Выразим начальные скорости через $l$ и $S_k$
$$\frac{S_k}{l}=\frac{v_1}{2v_2}
$$
$$v_2=\frac{l}{2S_k}v_1
$$
$$S_k=\frac{1+\frac{l^2}{4S_k^2}}{2g}v_1^2
$$
$$v_1=\sqrt{\frac{8gS_k}{l^2+4S_k^2}}S_k=10,\!8 \text{м/с}
$$
$$v_2=\sqrt{\frac{2gS_k}{l^2+4S_k^2}}l=2,\!7 \text{м/с}
$$}

\end{document}