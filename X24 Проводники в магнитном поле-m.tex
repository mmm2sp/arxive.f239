
%This file is part of Get pho.rs!

%Get pho.rs! is free software: you can redistribute it and/or modify it under the terms of the GNU General Public License as published by the Free Software Foundation, either version 3 of the License, or (at your option) any later version.

%Get pho.rs! is distributed in the hope that it will be useful, but WITHOUT ANY WARRANTY; without even the implied warranty of MERCHANTABILITY or FITNESS FOR A PARTICULAR PURPOSE. See the GNU General Public License for more details.

%You should have received a copy of the GNU General Public License along with Foobar. If not, see <https://www.gnu.org/licenses/>.

%\documentstyle[12pt,russian,amsthm,amsmath,amssymb]{article}
\documentclass[a4paper,11pt,twoside]{article}
\usepackage[left=14mm, top=10mm, right=14mm, bottom=10mm, nohead, nofoot]{geometry}
\usepackage{amsmath, amsfonts, amssymb, amsthm} % стандартный набор AMS-пакетов для математ. текстов
\usepackage{mathtext}
\usepackage[utf8]{inputenc} % кодировка utf8
\usepackage[russian]{babel} % русский язык
\usepackage[pdftex,dvipsnames]{xcolor} % работа с цветами
\usepackage[pdftex]{graphicx} % графика (картинки)
\usepackage{tikz} % рисунки
\usepackage{fancyhdr,pageslts} % настройка колонтитулов
\usepackage{enumitem} % работа со списками
\usepackage{multicol} % работа с таблицами
%\usepackage{pscyr} % красивый шрифт
\usepackage{pgfornament} % красивые рюшечки и вензеля
\usepackage{ltxgrid} % управление написанием текста в две колонки
\usepackage{lipsum} % стандартный текст
\usepackage{tcolorbox} % рамка вокруг текста
\usepackage{float} % для корректного размещения картинок
\tcbuselibrary{skins}
% ----------------------------------------

\newcommand\ProblemName{Проводники в магнитном поле}

\newcommand\Source{X24}

\newcommand\Type{Разбалловка}

% настройки полей
\geometry{
	left=12mm,
	top=21mm,
	right=15mm,
	bottom=26mm,
	marginparsep=0mm,
	marginparwidth=0mm,
	headheight=22pt,
	headsep=2mm,
	footskip=7mm}
% ----------------------------------------

% настройки колонтитулов
\pagestyle{fancy}

\fancypagestyle{style}{
	\fancyhf{}
	\fancyhead[L]{{\Large{\FancyTitle}}\\\vskip -5pt \dotfill}
	\fancyhead[R]{{\Large{\textbf{\Type}}}\\\vskip -5pt \dotfill}
	\renewcommand{\headrulewidth}{0pt}
	\renewcommand{\footrulewidth}{0pt}
	\fancyfoot[C]{\pgfornament[width=2em,anchor=south]{72}\hspace{1mm}
		{Страница \textbf{\thepage} из \textbf{\pageref{VeryLastPage}}}\hspace{2mm}
		\pgfornament[width=2em,symmetry=v,anchor=south]{72}\\ \vskip2mm
		{\small{\textit{Условие собрано и подготовлено в Президентском ФМЛ №239 г.~Санкт-Петербурга}}}}
}

\fancypagestyle{plain}{
	\fancyhf{}
	\renewcommand{\headrulewidth}{0pt}
	\renewcommand{\footrulewidth}{0pt}
	\fancyhead[C]{{\Large{\textit{Учебно-тренировочные сборы к X23}}}\\\vskip -5pt \dotfill}
	\fancyfoot[C]{\pgfornament[width=2em,anchor=south]{72}\hspace{1mm}
		{Страница \textbf{\thepage} из \textbf{\pageref{VeryLastPage}}}\hspace{2mm}
		\pgfornament[width=2em,symmetry=v,anchor=south]{72}\\ \vskip2mm
		{\small{\textit{Условие собрано и подготовлено в Президентском ФМЛ №239 г.~Санкт-Петербурга}}}}
}
% ----------------------------------------

% другие настройки
\pagenumbering{arabic}
\setlist[enumerate,itemize]{leftmargin=0pt,itemindent=2.7em,itemsep=0cm}
% ----------------------------------------

% собственные команды
\newcommand{\FancyTitle}{\textbf{\Source} --- \ProblemName}
\newcommand{\Title}{\begin{center}{\huge{\textbf{\Source} --- \ProblemName}}\end{center}}
\newcommand{\Chapter}[1]{\vskip5pt{\Large{\textbf{#1}}}\vskip5pt}
\newcommand{\QText}[1]{#1}
\newcommand{\QBlock}[3]{
	\begin{tcolorbox}[left=4mm,top=3mm,bottom=2mm,right=4mm,colback=white]
		\begin{tcolorbox}[enhanced,colframe=blue,colback=blue!10!white,
			frame style={opacity=0.3},interior style={opacity=1.0},
			nobeforeafter,tcbox raise base,shrink tight,extrude by=1.7mm,width=1.5cm]
			\textbf{#1\textsuperscript{#2}}
		\end{tcolorbox}\hspace{3mm}#3
	\end{tcolorbox}
}
\newcommand{\QPicture}[4]{\QText{#4}  \includegraphics{#1}}
\newcommand{\ABlock}[1]{#1}
\newcommand{\MBlock}[2]{#1 #2}
\newcommand{\MMBlock}[3]{#1 #2 #3}
% ----------------------------------------


\begin{document}
	
	% настройки
	\pagestyle{style}\thispagestyle{plain}
	\Title
	% ----------------------------------------
	
	%\vskip5mm
	%\centering{\pgfornament[width=5cm,anchor=south]{89}}

\QBlock{A1}{0.60}{Пусть момент времени $t_0=0$ груз находится в начале координат, а проекция его скорости на ось $x$ равна $v_0$. Определите зависимости координаты $x(t)$ и скорости $v_x(t)$ груза от времени $t$. Ответ выразите через $v_0$, $\gamma$, $\omega_0$ и $t$.}

\MBlock{0.20}{Решение для $x(t)$ ищется в следующем виде:
$$x(t)=Ce^{-\gamma t}\sin\left(\sqrt{\omega^2_0-\gamma^2}t+\varphi_0\right){.}
$$}

\MBlock{0.10}{Записана система начальных условий:
$$
\begin{cases}
x(0)=0\\
v_x(0)=v_0
\end{cases}
$$}

\MBlock{0.10}{Записана выражения для $x(0)$ и $v_x(0)$:
$$\begin{cases}
x(0)=C\sin\varphi_0\\
v_x(0)=C\left(\sqrt{\omega^2_0-\gamma^2}\cos\varphi_0-\gamma\sin\varphi_0\right)
\end{cases}
$$}

\MBlock{0.10}{Получен правильная зависимость $x(t)$:
$$x(t)=\cfrac{v_0}{\sqrt{\omega^2_0-\gamma^2}}e^{-\gamma t}\sin\left(\sqrt{\omega^2_0-\gamma^2}t\right){.}
$$}

\MBlock{0.10}{Получен правильная зависимость $v_x(t)$:
$$v_x(t)=\cfrac{v_0\omega_0}{\sqrt{\omega^2_0-\gamma^2}}e^{-\gamma t}\cos\left(\sqrt{\omega^2_0-\gamma^2}t+\arcsin\cfrac{\gamma}{\omega_0}\right){.}
$$}

\QBlock{A2}{0.40}{Получите точное выражение для $Q$. Ответ выразите через $\omega_0$ и $\gamma$.}

\MBlock{0.10}{Для добротности записано:
$$Q=\cfrac{2\pi}{1-\left(\cfrac{v_1}{v_0}\right)^2}
$$}

\MBlock{0.10}{Определено время $T$, за которое величина скорости изменяется от значения $v_0$ до значения $v_1$:
$$T=\cfrac{2\pi}{\sqrt{\omega^2_0-\gamma^2}}{.}
$$}

\MBlock{0.10}{Записано выражение для $v_1/v_0$:
$$\cfrac{v_1}{v_0}=e^{-\gamma T}{.}
$$}

\MBlock{0.10}{Получено выражение для добротности $Q$:
$$Q=\cfrac{2\pi}{1-e^{-4\pi\gamma/\sqrt{\omega^2_0-\gamma^2}}}{.}
$$}

\QBlock{A3}{0.20}{Получите приближённое выражение для добротности $Q$ при слабом затухании ($\gamma\ll\omega_0$).
Ответ выразите через $m$, $k$ и $\beta$.}

\MBlock{0.10}{Получено приближённое выражение для добротности $Q$:
$$Q\approx \cfrac{\omega_0}{2\gamma}{.}
$$}

\MBlock{0.10}{Добротность $Q$ выражена через требуемое величины:
$$Q\approx \cfrac{\sqrt{mk}}{\beta}{.}
$$}

\QBlock{B1}{0.60}{Отклонение $x$ груза от положения зависит от времени $t$ следующим образом:
$$x(t)=A\sin\left(\Omega t+\varphi_0\right)
$$
Найдите $A$ и $\varphi_0$. Ответы выразите через $A_0$, $\Omega$, $\omega_0$ и $\gamma$.}

\MBlock{0.10}{Записано уравнение движения груза:
$$\ddot{x}+2\gamma\dot{x}+\omega^2_0\Delta{x}=\omega^2_0A_0\sin\Omega t{.}
$$}

\MBlock{0.20}{Получена комплексная амплитуда $\hat{A}$:
$$\hat{A}=\cfrac{\omega^2_0A_0\left((\omega^2_0-\Omega^2)-2i\Omega\gamma\right)e^{-i\pi/2}}{\left((\omega^2_0-\Omega^2)^2+4\gamma^2\Omega^2\right)}{.}
$$}

\MBlock{0.10}{Получено выражение для $A$:
$$A=\cfrac{A_0\omega^2_0}{\sqrt{(\omega^2_0-\Omega^2)^2+4\gamma^2\Omega^2}}{.}
$$}

\MBlock{0.20}{Получено выражение для $\varphi_0$:
$$\varphi_0=\begin{cases}
-\arctan\cfrac{2\gamma\Omega}{\omega^2_0-\Omega^2}\quad\text{при}\quad \Omega{<}\omega_0\\
-\cfrac{\pi}{2}\quad\text{при}\quad \Omega=\omega_0\\
-\pi-\arctan\cfrac{2\gamma\Omega}{\omega^2_0-\Omega^2}\quad\text{при}\quad \Omega{>}\omega_0
\end{cases}
$$}

\MBlock{-0.10}{Пункт оценивается, если рассмотрен только случай, соответствующий $\Omega{<}\omega_0$.}

\QBlock{B2}{0.30}{Получите точные выражения для резонансной циклической частоты $\Omega_\text{рез}$ и соответствующей ей амплитуды колебаний $A_\text{рез}$. Ответы выразите через $\omega_0$, $\gamma$ и $A_0$. Считайте, что $\gamma\sqrt{2}{<}\omega_0$.}

\MBlock{0.20}{Получено выражение для $\Omega_\text{рез}$:
$$\Omega_\text{рез}=\sqrt{\omega^2_0-2\gamma^2}{.}
$$}

\MBlock{0.10}{Получено выражение для $A_\text{рез}$:
$$A_\text{рез}=\cfrac{A_0\omega^2_0}{2\gamma\sqrt{\omega^2_0-\gamma^2}}{.}
$$}

\QBlock{B3}{0.30}{Получите приближённые выражения для $\Omega_\text{рез}$, $A_\text{рез}$ и $\Delta{\omega}$ при слабом затухании ($\gamma\ll{\omega_0}$).
Ответы выразите через $A_0$, $\omega_0$ и $\gamma$.}

\MBlock{0.05}{Получено приближённое выражение для $A_\text{рез}$:
$$A_\text{рез}\approx{\cfrac{A_0\omega_0}{2\gamma}}{.}
$$}

\MBlock{0.05}{Получено приближённое выражение для $\Omega_\text{рез}$:
$$\Omega_\text{рез}\approx{\omega_0}{.}
$$}

\MBlock{0.10}{Подкоренное выражение приведено к виду:
$$\left(\omega^2_0-\Omega^2\right)^2+4\gamma^2\Omega^2\approx 4\omega^2_0\Delta\Omega^2+4\gamma^2\omega^2_0{.}
$$}

\MBlock{0.10}{Получено выражение для ширины резонансной кривой $\Delta{\omega}$:
$$\Delta{\omega}=2\gamma{.}
$$}

\QBlock{C1}{0.30}{Найдите индукцию $B_x$ магнитного поля кольца на его оси в точке с координатой $x$.
Ответ выразите через $x$, $R$, $I$ и магнитную постоянную $\mu_0$.}

\MBlock{0.10}{Записан закон Био-Савара-Лапласа:
$$d\vec{B}=\cfrac{\mu_0}{4\pi}\cfrac{\left[\vec{r}\times d\vec{r}\right]}{r^3}{,}
$$
где $\vec{r}$ - радиус-вектор элемента кольца относительно точки с координатой $x$.}

\MBlock{0.20}{Получено выражение для $B_x$:
$$B_x=\cfrac{\mu_0IR^2}{2(R^2+x^2)^{3/2}}{.}
$$}

\QBlock{C2}{1.00}{Определите магнитный момент $\vec{m}$ диска.
Ответ выразите через $\vec{e}_x$, $r_0$, $h$, $\rho$ и $\dot{B}$.}

\MBlock{0.10}{Записан закон электромагнитной индукции Фарадея:
$$\int_S\cfrac{\partial\vec{B}}{\partial t}\cdot d\vec{S}=-\oint_L\vec{E}\cdot d\vec{l}{.}
$$}

\MBlock{0.30}{Определена величина вихревого электрического поля $E(r)$ в направлении против часовой стрелки:
$$E=-\cfrac{r\dot{B}}{2}{.}
$$
Пункт оценивается, даже если знак неверный.}

\MBlock{0.10}{Записан закон Ома в дифференциальной форме:
$$\vec{j}=\cfrac{\vec{E}}{\rho}{.}
$$}

\MBlock{0.10}{Для элементарного магнитного момента записано:
$$d\vec{m}=\vec{S}dI{.}
$$}

\MBlock{0.20}{Для магнитного момента диска, обусловленного течением тока в кольце с внутренним и внешним радиусом $r$ и $r+dr$ соответственно записано:
$$dm_x=-\cfrac{\pi\dot{B}h}{2\rho}r^3dr{,}
$$
Пункт оценивается, даже если знак неверный.}

\MBlock{2 $\times$ 0.10}{Получен правильный ответ (по $0{.}1$ балла за величину и знак, полученный без чётного числа ошибок):
$$\vec{m}=-\vec{e}_x\cdot\cfrac{\pi hr^4_0\dot{B}}{8\rho}{.}
$$}

\QBlock{C3}{0.50}{Определите магнитный момент $\vec{m}$ шара.
Ответ выразите через $\vec{e}_x$, $R_0$, $\rho$ и $\dot{B}$.}

\MBlock{0.20}{После перехода к сферическим координатам для магнитного момента шара получено:
$$m_x=-\cfrac{\pi{R}^5_0\dot{B}}{8\rho}\int\limits_{0}^{\pi}\sin^5\theta d\theta{.}
$$}

\MBlock{0.20}{Вычислен интеграл от $\sin^5\theta$:
$$\int\limits_{0}^{\pi}\sin^5\theta d\theta=\cfrac{16}{15}{.}
$$}

\MBlock{0.10}{Получен правильный ответ:
$$\vec{m}=-\vec{e}_x\cdot\cfrac{2\pi{R}^5_0\dot{B}}{15\rho}{.}
$$}

\QBlock{C4}{0.40}{Получите производную по времени индукции магнитного поля кольца в центре шара $dB_x/dt$, эквивалентную величине $\dot{B}$.
Ответ выразите через $v$, $I$, $R$, $x$ и магнитную постоянную $\mu_0$.}

\MBlock{0.20}{Величина $dB_x/dt$ представлена в виде производной сложной функции и получено:
$$\cfrac{dB_x}{dt}=v\cfrac{dB_x}{dx}{.}
$$}

\MBlock{2 $\times$ 0.10}{Определена производная $dB_x/dx$ и получен правильный ответ (по $0{.}1$ балла за величину и знак, полученный без чётного числа ошибок):
$$\dot{B}=-\cfrac{3\mu_0IR^2xv}{2(R^2+x^2)^{\frac{5}{2}}}{.}
$$}

\QBlock{C5}{0.50}{Найдите коэффициент пропорциональности $\beta(x)$.
Ответ выразите через $I$, $R$, $x$, $R_0$, $\rho$ и магнитную постоянную $\mu_0$.}

\MBlock{0.30}{Для силы, действующей на шар, записано:
$$\vec{F}=\vec{e}_x\cdot m_x\cfrac{dB_x}{dx}{.}
$$}

\MBlock{0.10}{Для магнитного момента шара записано:
$$m_x=-\cfrac{2\pi{R}^5_0v}{15\rho}\cdot\cfrac{dB_x}{dx}{,}
$$}

\MBlock{0.10}{Получена правильная зависимость $\beta(x)$:
$$\beta(x)=\cfrac{3\pi\mu^2_0I^2R^4R^5_0x^2}{10\rho(R^2+x^2)^5}{.}
$$}

\QBlock{C6}{0.80}{Определите удельное сопротивление $\rho$ шара, используемого в первом эксперименте.
Ответ выразите через $m$, $k$, $R_0$, $R$, $H$, $I$ и магнитную постоянную $\mu_0$.}

\MBlock{0.30}{Для отношения амплитуд $A_{i+N}/A_i$, где $N$ - число прошедших колебаний, записано:
$$\cfrac{A_{i+N}}{A_{i}}=e^{-2\pi\gamma/\omega}{.}
$$}

\MBlock{0.30}{Получено отношение $\gamma/\omega_0$:
$$\cfrac{\gamma}{\omega_0}\approx 0{.}03{.}
$$}

\MBlock{2 $\times$ 0.10}{Получено правильный ответ для $\rho$ (по $0{.}1$ балла за попадание в узкие и широкие ворота):
$$\rho=(15{.}7\pm 0{.}5)\cdot\cfrac{\mu^2_0I^2R^5_0R^4H^2}{\sqrt{mk}(R^2+H^2)^5}
$$
$$\rho=(15{.}7\pm 0{.}7)\cdot\cfrac{\mu^2_0I^2R^5_0R^4H^2}{\sqrt{mk}(R^2+H^2)^5}
$$}

\QBlock{C7}{0.70}{Определите удельное сопротивление $\rho$ шара, используемого во втором эксперименте.
Ответ выразите через $m$, $k$, $R_0$, $R$, $H$, $I$ и магнитную постоянную $\mu_0$.}

\MMBlock{0.40}{M1}{Записано соотношение:
$$A_\text{рез}=\cfrac{A_0\omega_0}{2\gamma}{.}
$$}

\MMBlock{0.10}{M1}{Определено соотношение между $\omega_0$ и $\gamma$:
$$\cfrac{\omega_0}{\gamma}=50{.}
$$}

\MMBlock{0.20}{M1}{Получен правильный ответ для $\rho$:
$$\rho=23{.}6\cfrac{\mu^2_0I^2R^5_0R^4H^2}{\sqrt{mk}(R^2+H^2)^5}
$$}

\MMBlock{0.10}{M2}{Записано выражение для ширины резонансной кривой:
$$\Delta{\omega}=2\gamma{.}
$$}

\MMBlock{2 $\times$ 0.05}{M2}{Получено соотношение между $\omega_0$ и $\gamma$ по $0{.}1$ балла за попадание в узкие и широкие ворота)::
$$\cfrac{\omega_0}{2\gamma}\approx 27{.}5\pm 2{.}5
$$
$$\cfrac{\omega_0}{2\gamma}=30\pm 5
$$}

\MMBlock{0.10}{M2}{Получен правильный ответ для $\rho$:
$$\rho=(28{.}5\pm 4{.}5)\cfrac{\mu^2_0I^2R^5_0R^4H^2}{\sqrt{mk}(R^2+H^2)^5}
$$}

\QBlock{D1}{0.60}{Определите индукцию $B_z$ магнитного поля соленоида, а также её производную $dB_z/dz$ в точке с координатой $z$. Ответ выразите через $\mu_0$, $n$, $I$, $R$ и $z$.}

\MBlock{0.20}{Использована теорема о телесном угле для магнитного поля:
$$B_z=\cfrac{\mu_0i\Omega_\text{бок}}{4\pi}{.}
$$}

\MBlock{0.20}{Определён телесный угол $\Omega_\text{бок}$:
$$\Omega_\text{бок}=2\pi\left(1-\cfrac{z}{\sqrt{z^2+R^2}}\right){.}
$$}

\MBlock{0.10}{Получена правильная зависимость $B_z(z)$:
$$B_z(z)=\cfrac{\mu_0nI}{2}\left(1-\cfrac{z}{\sqrt{z^2+R^2}}\right){.}
$$}

\MBlock{0.10}{Получена правильная зависимость $dB_z(z)/dz$:
$$\cfrac{dB_z(z)}{dz}=-\cfrac{\mu_0nIR^2}{2(R^2+z^2)^{3/2}}{.}
$$}

\QBlock{D2}{1.00}{Определите линейную плотность тока $i$ на поверхности цилиндра в точке с координатой $z$. Ответ выразите через $\mu_0$, $x$ и $dB_z(z)/dz$.}

\MBlock{0.20}{Записаны выражения для индукции магнитного поля внутри и снаружи стержня
$$B_{z(in)}=B_z(z-x)\qquad B_{z(out)}=B(z){.}
$$}

\MBlock{0.50}{Предложен метод, позволяющий определить линейную плотность тока $i$, например, теорема о циркуляции.}

\MBlock{0.10}{Записана теорема о циркуляции:
$$(B_{z(in)}-B_{z(out)})=\mu_0ix{.}
$$}

\MBlock{2 $\times$ 0.10}{Получено выражение для $i$ (по $0{.}1$ балла за величину и знак, полученный без чётного числа ошибок):
$$i(z)=-\cfrac{x}{\mu_0}\cfrac{dB_z}{dz}{.}
$$}

\QBlock{D3}{1.50}{Определите силу $F_x$, действующую на цилиндр со стороны магнитного поля соленоида. Ответ выразите через $\mu_0$, $r$, $R$, $n$, $I$ и $x$.}

\MBlock{0.10}{Для магнитного момента элемента цилиндра высотой $dz$ записано:
$$dm_z=i(z)\pi r^2dz{.}
$$}

\MBlock{0.20}{Записано выражение для силы $dF_{z}$, действующей на рассмотренный магнитный момент:
$$dF_x=dm_z\cfrac{dB_z}{dz}{.}
$$}

\MBlock{0.40}{Получено выражение для $F_x$:
$$F_x\approx -\cfrac{\mu_0\pi{r}^2n^2I^2R^4x}{4}\int\limits_{-\infty}^{\infty}\cfrac{dz}{(R^2+z^2)^3}{.}
$$}

\MBlock{0.20}{Интеграл преобразован следующим образом:
$$\int\limits_{-\infty}^{\infty}\cfrac{dz}{(R^2+z^2)^3}=\cfrac{1}{R^5}\int\limits_{-\pi/2}^{\pi/2}\cos^4\varphi d\varphi
$$}

\MBlock{0.40}{Для интеграла от $\cos^4\varphi$ получено:
$$\int\limits_{-\pi/2}^{\pi/2}\cos^4\varphi d\varphi=\cfrac{3\pi}{8}{.}
$$}

\MBlock{2 $\times$ 0.10}{Получен правильный ответ (по $0{.}1$ балла за величину и знак, полученный без чётного числа ошибок):
$$F_x=-\cfrac{3\pi^2\mu_0\pi r^2n^2I^2}{32R}x{.}
$$}

\QBlock{D4}{0.30}{Получите зависимость перемещения стержня $x$ от времени $t$. Ответ выразите через $\mu_0$, $r$, $R$, $n$, $I$ и $m$.}

\MBlock{0.10}{Определена циклическая частота гармонических колебаний $\omega_0$:
$$\omega_0=\sqrt{\cfrac{3\mu_0\pi^2 r^2n^2I^2}{32mR}}{.}
$$}

\MBlock{0.10}{Получена зависимость $x(t)$:
$$x(t)=\cfrac{v_0\sin\omega_0t}{\omega_0}{.}
$$}

\MBlock{0.10}{Получена правильная зависимость $x(t)$:
$$x(t)=v_0\sqrt{\cfrac{32mR}{3\mu_0\pi^2{r}^2n^2I^2}}\sin\sqrt{\cfrac{3\mu_0\pi^2 r^2n^2I^2}{32mR}}t{.}
$$}

\end{document}