
%This file is part of Get pho.rs!

%Get pho.rs! is free software: you can redistribute it and/or modify it under the terms of the GNU General Public License as published by the Free Software Foundation, either version 3 of the License, or (at your option) any later version.

%Get pho.rs! is distributed in the hope that it will be useful, but WITHOUT ANY WARRANTY; without even the implied warranty of MERCHANTABILITY or FITNESS FOR A PARTICULAR PURPOSE. See the GNU General Public License for more details.

%You should have received a copy of the GNU General Public License along with Foobar. If not, see <https://www.gnu.org/licenses/>.

%\documentstyle[12pt,russian,amsthm,amsmath,amssymb]{article}
\documentclass[a4paper,11pt,twoside]{article}
\usepackage[left=14mm, top=10mm, right=14mm, bottom=10mm, nohead, nofoot]{geometry}
\usepackage{amsmath, amsfonts, amssymb, amsthm} % стандартный набор AMS-пакетов для математ. текстов
\usepackage{mathtext}
\usepackage[utf8]{inputenc} % кодировка utf8
\usepackage[russian]{babel} % русский язык
\usepackage[pdftex,dvipsnames]{xcolor} % работа с цветами
\usepackage[pdftex]{graphicx} % графика (картинки)
\usepackage{tikz} % рисунки
\usepackage{fancyhdr,pageslts} % настройка колонтитулов
\usepackage{enumitem} % работа со списками
\usepackage{multicol} % работа с таблицами
%\usepackage{pscyr} % красивый шрифт
\usepackage{pgfornament} % красивые рюшечки и вензеля
\usepackage{ltxgrid} % управление написанием текста в две колонки
\usepackage{lipsum} % стандартный текст
\usepackage{tcolorbox} % рамка вокруг текста
\usepackage{float} % для корректного размещения картинок
\tcbuselibrary{skins}
% ----------------------------------------

\newcommand\ProblemName{Добыча нефти}

\newcommand\Source{X24}

\newcommand\Type{Условие задачи}

% настройки полей
\geometry{
	left=12mm,
	top=21mm,
	right=15mm,
	bottom=26mm,
	marginparsep=0mm,
	marginparwidth=0mm,
	headheight=22pt,
	headsep=2mm,
	footskip=7mm}
% ----------------------------------------

% настройки колонтитулов
\pagestyle{fancy}

\fancypagestyle{style}{
	\fancyhf{}
	\fancyhead[L]{{\Large{\FancyTitle}}\\\vskip -5pt \dotfill}
	\fancyhead[R]{{\Large{\textbf{\Type}}}\\\vskip -5pt \dotfill}
	\renewcommand{\headrulewidth}{0pt}
	\renewcommand{\footrulewidth}{0pt}
	\fancyfoot[C]{\pgfornament[width=2em,anchor=south]{72}\hspace{1mm}
		{Страница \textbf{\thepage} из \textbf{\pageref{VeryLastPage}}}\hspace{2mm}
		\pgfornament[width=2em,symmetry=v,anchor=south]{72}\\ \vskip2mm
		{\small{\textit{Условие собрано и подготовлено в Президентском ФМЛ №239 г.~Санкт-Петербурга}}}}
}

\fancypagestyle{plain}{
	\fancyhf{}
	\renewcommand{\headrulewidth}{0pt}
	\renewcommand{\footrulewidth}{0pt}
	\fancyhead[C]{{\Large{\textit{Учебно-тренировочные сборы к X23}}}\\\vskip -5pt \dotfill}
	\fancyfoot[C]{\pgfornament[width=2em,anchor=south]{72}\hspace{1mm}
		{Страница \textbf{\thepage} из \textbf{\pageref{VeryLastPage}}}\hspace{2mm}
		\pgfornament[width=2em,symmetry=v,anchor=south]{72}\\ \vskip2mm
		{\small{\textit{Условие собрано и подготовлено в Президентском ФМЛ №239 г.~Санкт-Петербурга}}}}
}
% ----------------------------------------

% другие настройки
\pagenumbering{arabic}
\setlist[enumerate,itemize]{leftmargin=0pt,itemindent=2.7em,itemsep=0cm}
% ----------------------------------------

% собственные команды
\newcommand{\FancyTitle}{\textbf{\Source} --- \ProblemName}
\newcommand{\Title}{\begin{center}{\huge{\textbf{\Source} --- \ProblemName}}\end{center}}
\newcommand{\Chapter}[1]{\vskip5pt{\Large{\textbf{#1}}}\vskip5pt}
\newcommand{\QText}[1]{#1}
\newcommand{\QBlock}[3]{
	\begin{tcolorbox}[left=4mm,top=3mm,bottom=2mm,right=4mm,colback=white]
		\begin{tcolorbox}[enhanced,colframe=blue,colback=blue!10!white,
			frame style={opacity=0.3},interior style={opacity=1.0},
			nobeforeafter,tcbox raise base,shrink tight,extrude by=1.7mm,width=1.5cm]
			\textbf{#1\textsuperscript{#2}}
		\end{tcolorbox}\hspace{3mm}#3
	\end{tcolorbox}
}
\newcommand{\QPicture}[4]{\QText{#4}  \includegraphics{#1}}
\newcommand{\ABlock}[1]{#1}
\newcommand{\MBlock}[2]{#1 #2}
\newcommand{\MMBlock}[3]{#1 #2 #3}
% ----------------------------------------


\begin{document}
	
	% настройки
	\pagestyle{style}\thispagestyle{plain}
	\Title
	% ----------------------------------------
	
	%\vskip5mm
	%\centering{\pgfornament[width=5cm,anchor=south]{89}}

\QText{Месторождения нефти  представляют собой пласт пористой проводящей жидкость среды, которую сверху перекрывает покрышка плохо проводящей жидкость горной породы. Под проводящей пористой средой должна находиться нефтематеринская порода. В течение длительного (порядка миллионов лет) времени, под действием высоких температур и давлений в практически бескислородной среды, органическое вещество нефтематеринской породы (кероген) распадается на углеводороды. Поры среды заполняются смесью этих углеводородов и воды, которой в среднем всегда больше. Эта смесь называется флюидом.  Из-за увеличения удельного объема, а также из-за уплотнения горной породы с течением времени под силой тяжести, в поровой жидкости, флюиде, начинает увеличиваться давление. Это приводит к движению флюида. Непроводящая глинистая покрышка является естественной преградой для флюида. За миллионы лет он накапливается у поверхности раздела в "ловушках" - неровностях горной породы (см. рис.).}

\QPicture{X24 Добыча нефти_files/29872.jpeg}{"max-width:500px;px;display:block;margin-left: auto;margin-right: auto;"}{Схематичное изображение месторождения нефти}{}

\QText{Считайте известными следующие численные данные:}

\QText{\begin{itemize} 
\item Плотность нефти $\rho=800 \text{кг}/\text{м}^3$;
\item Атмосферное давление $p_0=1 \text{атм}=10^5 \text{Па}$;
\item Ускорение свободного падения $g=10 \text{м}/\text{с}^2$. Во всех частях задачи, кроме $\mathrm{A}$, полностью им пренебрегайте.
\end{itemize}}

\Chapter{Часть A. Оценка запасов нефти (1.5 балла)}

\QText{Как было указано в предисловии, весь нефтяной флюид расположен в порах среды. Важной характеристикой среды является пористость -  величина 
$$\varphi=\cfrac{V_\text{пор}}{V_\text{ср}}{,}
$$ 
где $V_\text{пор}$ – объём, занимаемый порами в выделенном объёме среды $V_\text{ср}$.}

\QBlock{A1}{0.30}{Пусть залежь нефти представляет собой участок древних речных отложений песчаника в форме параллелепипеда высотой $h = 10 \text{м}$, шириной $b = 100 \text{м}$ и длинной $L = 2000 \text{м}$. Пористость породы $\varphi = 0.1$. Оцените запасы нефти $m_\text{н}$ в данном месторождении. Выразите ответ через $L$, $b$, $h$, $\rho$ и $\varphi$, а также приведите его численное значение в тоннах. Считайте, что нефтяной флюид целиком заполняет объём пор.}

\QText{Одной из наиболее важных величин в отрасли, связанной с нефтью, является пластовое давление. Пластовым давлением $p_\text{пл}$ называют величину давления жидкостей в порах в том случае, когда поры соединены между собой. Данный случай чаще всего и реализуется в реальности.

Пластовое давление обусловлено тем, что флюид в порах сжат относительно нормальных условий. Сжатие флюида характеризуется сжимаемостью  вещества  $\beta$, которая определяется соотношением:
$$\beta=-\cfrac{1}{V}\cfrac{dV}{dp}{,}
$$
где $p$ и $V$ - давление и объём вещества, а производная $dV/dp$ берётся при постоянном его количестве.}

\QText{Далее во всех пунктах данной части задачи считайте, что все поры соединены между собой, поэтому распределение давления в них определяется по законам гидростатики.}

\QBlock{A2}{0.30}{Пусть пластовое давление нефти на дне залежей составляет $p_\text{пл}=250 \text{атм}$. Найдите, при какой максимальной глубине залегания $H_{max}$ месторождение будет фонтанирующим, т.е. нефть будет вытекать на поверхность под действием собственного давления. Выразите ответ через $\rho$, $g$ и $p_{\text{пл}}$, а также приведите его численное значение. Сжимаемостью нефти можно пренебречь.}

\QText{Из месторождения можно добыть далеко не всю нефть, а только её часть. Доля извлеченной нефти от общих запасов называется коэффициентом извлечения нефти (КИН) $\alpha$. Как правило он невысок и почти никогда не достигает половины.}

\QBlock{A3}{0.60}{Оцените максимально возможный КИН $\alpha_{max}$ в режиме фонтанирования при пластовом давлении $p_\text{пл}=250 \text{атм}$, если сжимаемость нефти $\beta = 5\cdot 10^{-10} \text{ Па}$. Выразите ответ через $\beta$ и $p_\text{пл}$, а также приведите его численное значение. Считайте, что отложения русла рек изолированы непроницаемыми глинами с малой пористостью. Глубина залежей $H$ может быть выбрана произвольным образом.}

\QBlock{A4}{0.30}{При тех же самых данных оцените максимально возможный КИН $\alpha_{max}$ в режиме фонтанирования, если снизу в пластовых отложениях находится вода объемом $kV_0$ ($k = 9$) при начальных запасах нефти $V_0$. Сжимаемость воды считайте равной сжимаемости нефти. Выразите ответ через $\beta$ и $p_\text{пл}$, а также приведите его численное значение. Считайте что забор жидкости происходит сверху, т.е. забирается только нефть. Глубина залежей $H$ может быть выбрана произвольным образом.}

\Chapter{Часть B. Гидроразрыв пласта (3.2 балла)}

\QText{Распространенный метод повышения отдачи нефти на месторождении заключается в том, что перед добычей в скважину под большим давлением закачивается специальная жидкость – расклинивающий агент. Это приводит к образованию трещины плоской формы порядка $100 \text{м}$ в длину и не более $1 \text{см}$ в ширину. Созданная в нефтесодержащих слоях трещина заметно упрощает приток нефти в скважину при том же давлении, что существенно ускоряет добычу и уменьшает затраты. Этот метод называется гидроразрывом пласта.}

\QText{При описании трещины воспользуемся следующей моделью:}

\QText{\begin{itemize} 
\item Трещина состоит из двух одинаковых симметричных половин, лежащих в одной вертикальной плоскости (рис. а);
\item Скорость жидкости, текущей в трещине, считайте направленной горизонтально вдоль трещины (рис. б);
\item Скорость элементов жидкости, лежащих на одной вертикали, совпадают;
\item Поток $Q=6 \text{м}^3/\text{мин}$ расклинивающей жидкости делится поровну между половинами рассматриваемой трещины, поступает в них при $x=0$ и остаётся постоянным во всех сечениях $wh$ (рис. б);
\item Высота $h=10 \text{м}$ рассматриваемой трещины одинакова в любой её точке;
\item Ширина трещины $w$ (рис. а) зависит от избыточного по сравнению с горным давления $p'=p-\sigma_0$, где $\sigma_0=const$, по закону $$w=\cfrac{p'h}{E}{,}$$ где $E=10^{10} \text{Па}$ – модуль плоской деформации;
\item Вязкость расклинивающей жидкости равна $\eta=1{.}00 \text{Па}\cdot\text{с}$.
\end{itemize}}

\QText{На (рис. а) показан вид сверху на трещину, а на (рис. б) приведено поперечное сечение трещины $wh$, лежащее в вертикальной плоскости и перпендикулярное оси $x$. через которое течёт жидкость.}

\QPicture{X24 Добыча нефти_files/29857.jpeg}{"max-width:500px;px;display:block;margin-left: auto;margin-right: auto;"}{}{}

\QText{В пунктах $\mathrm{B1}$ и $\mathrm{B2}$ рассматривается половина трещины, соответствующая $x{>}0$, в которой скорость расклинивающей жидкости направлена вдоль оси $x$.}

\QBlock{B1}{1.00}{Рассмотрим горизонтальное течение жидкости вдоль оси $x$ между двумя параллельными плоскостями высотой $h$. Расстояние между плоскостями $w\ll{h}$. Определите объёмный расход (далее во всех пунктах задачи - поток) жидкости $Q$ через поперечное сечение $wh$. Ответ выразите через $\eta$, $w$, $h$ и градиент давления $dp(x)/dx$.}

\QText{Поскольку расстояние между плоскостями уменьшается медленно по длине трещины, всегда считайте применимым результат, полученный в пункте $\mathrm{B1}$.}

\QBlock{B2}{1.00}{В центре щели создается избыточное давление $\Delta p$. Найдите зависимость избыточного давления $p'$ в щели от координаты $x$. Ответ выразите через $\Delta{p}$, $Q$, $E$, $h$, $\eta$ и $x$.}

\QBlock{B3}{0.20}{Трещина заканчивается в положении, соответствующем равному нулю избыточному давлению. Определите длину трещины $L$. Ответ выразите через $\Delta{p}$, $E$, $h$, $\eta$ и $Q$.}

\QBlock{B4}{0.70}{Определите объем трещины $V$. Ответ выразите через $\Delta{p}$, $h$, $\eta$, $Q$ и $E$.}

\QText{Критическое избыточное давление, выдерживаемое барьерами, составляет $\Delta{p}=100 \text{атм}$.}

\QBlock{B5}{0.30}{Рассчитайте максимально возможные значения длины трещины $L_{max}$ и её объёма $V_{max}$.}

\Chapter{Часть С. Время добычи нефти (3.2 балла)}

\QText{Пусть по краям нефтяного месторождения пробурено по скважине, каждая из которых создает трещину.  Трещины параллельны боковым  граням месторождения и полностью их перекрывают. Нагнетающая скважина создает повышенное давление, а добывающая - пониженное давление. 
Распространение жидкости в пласте описывается законом Дарси 
$$
\vec{v} = - \frac{k\nabla p}{\eta},
$$ 
где $\vec{v}$ - скорость течения жидкости, $\eta$ - вязкость жидкости, а $k$ - величина, называющаяся проницаемостью пласта для данной жидкости.
В рамках данной задачи движение является одномерным, поэтому величину $\nabla p$ можно записать следующим образом:
$$\nabla p=\vec{e}_x\cfrac{dp}{dx}{.}
$$}

\QPicture{X24 Добыча нефти_files/29509.jpeg}{"max-width:500px;px;display:block;margin-left: auto;margin-right: auto;"}{}{}

\QText{Рассмотрим следующую модель течения жидкости:}

\QText{ }

\QText{\begin{itemize} 
\item Жидкости можно считать несжимаемыми;
\item Жидкости текут по трубе постоянного сечения не перемешиваясь друг с другом;
\item При течении жидкостей область их контакта (далее – фронт) всегда сохраняет плоскую форму, перпендикулярную направлению скорости жидкостей;
\item Нагнетающая жидкость с проницаемостью $k_2$ и вязкостью $\eta_2=1\cdot 10^{-3} \text{Па}\cdot\text{c}$ поступает в трубку при давлении $p_2=350 \text{атм}$, а вытекающая жидкость с проницаемостью $k_1$ и вязкостью $\eta_1=5\cdot 10^{-2} \text{Па}\cdot\text{c}$ вытекает из трубки при давлении $p_1=100 \text{атм}$;
\item Длина трубки равна $L = 2 \text{км}$, а величина $S$ обозначает координату $x$ фронта;
\item В начальный момент времени жидкость $1$ полностью заполняет трубку, так что $S(0)=0$;
\item Считайте, что в каждой из жидкостей давление меняется вдоль пласта линейно.
\end{itemize}}

\QBlock{С1}{1.00}{Определите скорость $v$ движения границы жидкостей при перемещении фронта на величину $S$. Ответ выразите через $p_1$, $p_2$, $L$, $\eta_1$, $\eta_2$, $k_1$ и $k_2$.}

\QText{В пунктах $\mathrm{C2}$ и $\mathrm{C3}$ считайте, что $k_1=k_2=k=5\cdot 10^{-12} \text{м}^2$.}

\QBlock{C2}{0.90}{Определите зависимость перемещения $S$ фронта от времени $t$. Ответ выразите через $p_1$, $p_2$, $L$, $\eta_1$, $\eta_2$, $k$ и $t$}

\QBlock{C3}{0.50}{Определите полное время $\tau$ вытеснения нефти из месторождения. Выразите ответ через $p_1$, $p_2$, $L$, $\eta_1$, $\eta_2$ и $k$ и рассчитайте его.}

\QBlock{С4}{0.80}{При каком условии на параметры системы движение границы будет устойчивым, то есть при малом отклонении формы границы от плоской это отклонение не будет возрастать? Запишите условие устойчивости через $\eta_1$, $\eta_2$, $k_1$ и $k_2$.
Устойчиво ли течение жидкости, рассмотренное в пунктах $\mathrm{C2}$ и $\mathrm{C3}$?}

\Chapter{Часть D. Течение нефти в забое скважины (2.1 балла)}

\QText{Забой скважины - цилиндрический участок скважины с проницаемыми стенками, через который может проникать нефть. В рамках данной части задачи вам предлагается изучить распределение поля скоростей жидкости внутри данного цилиндра.}

\QText{Для начала рассмотрим классическое течение жидкости с вязкостью $\eta$ в трубе длиной $L$ радиусом $R$, к концам которой приложена разность давлений $\Delta p$. В пунктах $\mathrm{D1}$ и $\mathrm{D2}$ боковая поверхность цилиндра непроницаема, т.е жидкость через стенки не выходит, а движение каждого её элемента является одномерным.}

\QBlock{D1}{0.80}{Найдите зависимость скорости течения жидкости в такой трубе от расстояния до оси трубы $v(r)$, максимальное значение скорости $v_{max}$ и полный поток $Q$ жидкости через сечение цилиндра. Ответы выразите через $\Delta{p}$, $\eta$, $L$, $R$ и $r$.}

\QBlock{D2}{0.20}{Выразите распределение скорости течения жидкости $v(r)$ через полный поток $Q$, $R$ и $r$.}

\QPicture{X24 Добыча нефти_files/29579.jpeg}{"width:100\%; max-width: 100\%;display:block;margin-left: auto;margin-right: auto; cursor:pointer;" data-toggle="modal" data-target="#kt_modal_29579"}{}{Далее перейдём непосредственно к анализу течения жидкости в забое скважины. Он представляет собой цилиндр радиусом $R$ и высотой $H$, через стенки которого поступает полный поток нефти $Q_0$.}

\QText{При решении задачи используйте следующие модель и обозначения:}

\QText{\begin{itemize} 
\item Величина $h$ является расстоянием, отсчитываемым от нижнего края забоя;
\item Жидкость поступает в забой равномерно по всей площади боковой поверхности цилиндра, а её поток через нижний края забоя равен нулю;
\item Величина $u_r$ обозначает радиальную компоненту скорости течения жидкости, направленную от оси цилиндра;
\item Несмотря на наличие радиальной компоненты скорости течения жидкости, она является малой, поэтому при нахождении распределения осевой компоненты скорости $v$ течения жидкости используйте результаты, полученные в пунктах $\mathrm{D1}$ и $\mathrm{D2}$.
\end{itemize}}

\QBlock{D3}{0.20}{Найдите поток $Q$ в сечении забоя на расстоянии $h$ от его нижнего края и соответствующее выражение для вертикальной скорости $v(r{,}h)$ в зависимости от расстояния до оси $r$ и высоты $h$. Ответы выразите через $Q_0$, $H$, $R$, $r$ и $h$.}

\QBlock{D4}{0.30}{Рассмотрим кольцо высотой $dh$ с внутренним и внешним радиусами $r$ и $r+dr$ соответственно. Используя тот факт, что жидкость несжимаема, покажите, что из условия постоянства объёма жидкости внутри выделенного кольца следует соотношение:
$$\cfrac{\partial v}{\partial h}=-\cfrac{1}{r}\cfrac{\partial (u_rr)}{\partial r}{.}
$$
Вы можете использовать это соотношение, даже если не смогли его доказать.}

\QBlock{D5}{0.50}{Найдите радиальную скорость течения жидкости $u_r(r{,}h)$ в зависимости от расстояния до оси $r$ и высоты $h$, а также максимальную величину её модуля $u_{r(max)}$. Ответы выразите через $Q_0$, $R$, $H$, $h$ и $r$.}

\QBlock{D6}{0.10}{Чему равно отношение $u_{r(max)}/v_{max}$? Ответ выразите через $R$ и $H$.}

\end{document}