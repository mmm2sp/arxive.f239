
%This file is part of Get pho.rs!

%Get pho.rs! is free software: you can redistribute it and/or modify it under the terms of the GNU General Public License as published by the Free Software Foundation, either version 3 of the License, or (at your option) any later version.

%Get pho.rs! is distributed in the hope that it will be useful, but WITHOUT ANY WARRANTY; without even the implied warranty of MERCHANTABILITY or FITNESS FOR A PARTICULAR PURPOSE. See the GNU General Public License for more details.

%You should have received a copy of the GNU General Public License along with Foobar. If not, see <https://www.gnu.org/licenses/>.

%\documentstyle[12pt,russian,amsthm,amsmath,amssymb]{article}
\documentclass[a4paper,11pt,twoside]{article}
\usepackage[left=14mm, top=10mm, right=14mm, bottom=10mm, nohead, nofoot]{geometry}
\usepackage{amsmath, amsfonts, amssymb, amsthm} % стандартный набор AMS-пакетов для математ. текстов
\usepackage{mathtext}
\usepackage[utf8]{inputenc} % кодировка utf8
\usepackage[russian]{babel} % русский язык
\usepackage[pdftex,dvipsnames]{xcolor} % работа с цветами
\usepackage[pdftex]{graphicx} % графика (картинки)
\usepackage{tikz} % рисунки
\usepackage{fancyhdr,pageslts} % настройка колонтитулов
\usepackage{enumitem} % работа со списками
\usepackage{multicol} % работа с таблицами
%\usepackage{pscyr} % красивый шрифт
\usepackage{pgfornament} % красивые рюшечки и вензеля
\usepackage{ltxgrid} % управление написанием текста в две колонки
\usepackage{lipsum} % стандартный текст
\usepackage{tcolorbox} % рамка вокруг текста
\usepackage{float} % для корректного размещения картинок
\tcbuselibrary{skins}
% ----------------------------------------

\newcommand\ProblemName{Patch-clamp}

\newcommand\Source{X20-PE}

\newcommand\Type{Условие задачи}

% настройки полей
\geometry{
	left=12mm,
	top=21mm,
	right=15mm,
	bottom=26mm,
	marginparsep=0mm,
	marginparwidth=0mm,
	headheight=22pt,
	headsep=2mm,
	footskip=7mm}
% ----------------------------------------

% настройки колонтитулов
\pagestyle{fancy}

\fancypagestyle{style}{
	\fancyhf{}
	\fancyhead[L]{{\Large{\FancyTitle}}\\\vskip -5pt \dotfill}
	\fancyhead[R]{{\Large{\textbf{\Type}}}\\\vskip -5pt \dotfill}
	\renewcommand{\headrulewidth}{0pt}
	\renewcommand{\footrulewidth}{0pt}
	\fancyfoot[C]{\pgfornament[width=2em,anchor=south]{72}\hspace{1mm}
		{Страница \textbf{\thepage} из \textbf{\pageref{VeryLastPage}}}\hspace{2mm}
		\pgfornament[width=2em,symmetry=v,anchor=south]{72}\\ \vskip2mm
		{\small{\textit{Условие собрано и подготовлено в Президентском ФМЛ №239 г.~Санкт-Петербурга}}}}
}

\fancypagestyle{plain}{
	\fancyhf{}
	\renewcommand{\headrulewidth}{0pt}
	\renewcommand{\footrulewidth}{0pt}
	\fancyhead[C]{{\Large{\textit{Учебно-тренировочные сборы к X23}}}\\\vskip -5pt \dotfill}
	\fancyfoot[C]{\pgfornament[width=2em,anchor=south]{72}\hspace{1mm}
		{Страница \textbf{\thepage} из \textbf{\pageref{VeryLastPage}}}\hspace{2mm}
		\pgfornament[width=2em,symmetry=v,anchor=south]{72}\\ \vskip2mm
		{\small{\textit{Условие собрано и подготовлено в Президентском ФМЛ №239 г.~Санкт-Петербурга}}}}
}
% ----------------------------------------

% другие настройки
\pagenumbering{arabic}
\setlist[enumerate,itemize]{leftmargin=0pt,itemindent=2.7em,itemsep=0cm}
% ----------------------------------------

% собственные команды
\newcommand{\FancyTitle}{\textbf{\Source} --- \ProblemName}
\newcommand{\Title}{\begin{center}{\huge{\textbf{\Source} --- \ProblemName}}\end{center}}
\newcommand{\Chapter}[1]{\vskip5pt{\Large{\textbf{#1}}}\vskip5pt}
\newcommand{\QText}[1]{#1}
\newcommand{\QBlock}[3]{
	\begin{tcolorbox}[left=4mm,top=3mm,bottom=2mm,right=4mm,colback=white]
		\begin{tcolorbox}[enhanced,colframe=blue,colback=blue!10!white,
			frame style={opacity=0.3},interior style={opacity=1.0},
			nobeforeafter,tcbox raise base,shrink tight,extrude by=1.7mm,width=1.5cm]
			\textbf{#1\textsuperscript{#2}}
		\end{tcolorbox}\hspace{3mm}#3
	\end{tcolorbox}
}
\newcommand{\QPicture}[4]{\QText{#4}  \includegraphics{#1}}
\newcommand{\ABlock}[1]{#1}
\newcommand{\MBlock}[2]{#1 #2}
\newcommand{\MMBlock}[3]{#1 #2 #3}
% ----------------------------------------


\begin{document}
	
	% настройки
	\pagestyle{style}\thispagestyle{plain}
	\Title
	% ----------------------------------------
	
	%\vskip5mm
	%\centering{\pgfornament[width=5cm,anchor=south]{89}}

\Chapter{Введение}

\Chapter{Клеточные ионные каналы}

\QPicture{X20-PE Patch-clamp_files/2205.jpeg}{"width:100\%; max-width: 100\%;display:block;margin-left: auto;margin-right: auto; cursor:pointer;" data-toggle="modal" data-target="#kt_modal_2205"}{Рис. 1}{Живые клетки покрыты мембраной, структурную основу которой составляет двойной слой липидов, слабо проницаемый для воды и практически непроницаемый для ионов. Каждая клетка должна обмениваться с внешней средой различными веществами и, в частности, ионами. Перенос ионов через мембрану играет важную роль в процессах возбуждения клетки и передачи сигналов. Ионы проникают в клетку и выходят из нее через встроенные в мембрану белки — каналы. Каналы — это белки, которые выполняют функцию мембранных пор, так как формируют отверстия, сквозь которые могут проходить ионы. Мембранные каналы селективны—проницаемы только для определенных веществ. Селективность обусловлена радиусом пор и распределением заряженных функциональных групп в них. Существуют каналы, селективно пропускающие ионы натрия (натриевые каналы), а также калиевые, кальциевые и хлорные каналы. (см. рис. 1)}

\QPicture{X20-PE Patch-clamp_files/2206.jpeg}{"width:100\%; max-width: 100\%;display:block;margin-left: auto;margin-right: auto; cursor:pointer;" data-toggle="modal" data-target="#kt_modal_2206"}{Рис. 2: Закрытое и открытое состояние канала}{Бывают светочувствительные ионные каналы - канальные родопсины. Эти каналы переходят из закрытого состояния, в котором они непроницаемы для ионов, в открытое, поглощая фотоны определенной длины волны (см. рис 2). Далее белок проходит через несколько промежуточных состояний, чтобы вернуться из открытого состояния в закрытое (в дальнейшем мы будем считать, что
в промежуточных состояниях канал также непроницаем для любых ионов).}

\Chapter{Биофизические методы изучения каналов: Patch-clamp}

\QPicture{X20-PE Patch-clamp_files/2208.jpeg}{"width:100\%; max-width: 100\%;display:block;margin-left: auto;margin-right: auto; cursor:pointer;" data-toggle="modal" data-target="#kt_modal_2208"}{Рис. 3: Whole-cell patch-clamp}{Благодаря свободному пропусканию заряженных частиц (ионов) каналы в открытом состоянии эффективно увеличивают электрическую проводимость клеточной мембраны. Для изучения электрических свойств мембраны и изучения свойств ионных каналов существует метод локальной фиксации потенциала (Patch-clamp). Метод заключается в том, что стеклянная пипетка образует с клеточной мембраной контакт с сопротивление в несколько гигаом – это так называемый гигаомный контакт. В пипетку, заполненную электролитом, помещается электрод, второй электрод помещается внеклеточно, в омывающей жидкости. Для того, чтобы производить измерения тока, протека ющего через полную клеточную мембрану, ее кусочек, заключенный внутри пипетки, пробивается избыточным давлением. Такой метод измерения называется Whole-cell patch-clamp (дословно «полноклеточный пэтч-клэмп», см. рис. 3). Эффективная электрическая схема такая: один из электродов снаружи клетки, а второй — внутри.}

\Chapter{Часть А. Вольтамперные характеристики каналов}

\QText{Одной из характеристик, которую можно измерить для каналов, является их ВАХ. ВАХ канального родопсина—это зависимость стационарного тока от приложенного к клеточной мембране напряжения посредством электродов. ВАХ зависит от состава растворов, в которых производятся измерения.
В этой части вы будете обрабатывать экспериментальные данные снятые для канального родопсина, который пропускает положительные одновалентные ионы: натрий, калий и водород.
В приложении даны зависимости силы тока, проходящего через мембранные канальные родопсины при включении света, от времени для трех разных растворов, омывающих измеряемую клетку. Концентрации ионов во внеклеточной жидкости в трех случаях:}

\QText{1. $\mathrm{[H]}$ = $10^{-7.5}$ моль/л (pH=7.5), $\mathrm{[Na]}$ = 140 ммоль/л, $\mathrm{[K]}$ = 0 ммоль/л, $\mathrm{[Cl]}$ = 140 ммоль/л
2. $\mathrm{[H]}$ = $10^{-7.5}$ моль/л (pH=7.5), $\mathrm{[Na]}$ = 0 ммоль/л, $\mathrm{[K]}$ = 140 ммоль/л, $\mathrm{[Cl]}$ = 140 ммоль/л
3. $\mathrm{[H]}$ = $10^{-6.0}$ моль/л (pH=6.0), $\mathrm{[Na]}$ = 0 ммоль/л, $\mathrm{[K]}$ = 0 ммоль/л, $\mathrm{[Cl]}$ = 140 ммоль/л}

\QText{Внутриклеточный раствор задается раствором, который наливается в пипетку. Он во всех трех экспериментах одинаковый:
$\mathrm{[H]}$ = $10^{-7.5}$ моль/л (pH=7.5), $\mathrm{[Na]}$ = 110 ммоль/л, $\mathrm{[K]}$ = 0 ммоль/л, $\mathrm{[Cl]}$ = 110 ммоль/л}

\QText{\textbf{Напряжение, при котором плюс находится внутри клетки, считается положительным. Сила тока, при котором положительно заряженные частицы текут изнутри клетки наружу, считается положительной.}}

\QBlock{A1}{1.50}{Постройте ВАХи канального родопсина в трех экспериментах.}

\QText{Другой характеристикой, описывающей канальные родопсины, которую можно измерить, является проницаемость для разных ионов. Проницаемости для $\mathrm{Na}, \mathrm{K}$ и протонов будем обозначать $P_\mathrm{Na}; P_\mathrm{K}; P_\mathrm{H}$. Проницаемость является характеристикой мембраны с каналами: она зависит от количества каналов, но не зависит от омывающих мембрану растворов. Отношение же проницаемостей для двух разных ионов не зависит от количества каналов и является наиболее фундаментальной характеристикой канальных родопсинов.
Проницаемость входит в уравнение Гольдмана-Ходжкина-Катца, которое связывает ток через мембрану с напряжением на ней:}

\QText{$$
J(u)=\cfrac{z^2e^2u}{kT} \cfrac{P_{out}c_{out}-P_{in}c_{in}\,\mathrm{exp}\left(\cfrac{zeu}{kT}\right)}{1-\mathrm{exp}\left(\cfrac{zeu}{kT}\right)}
$$}

\QText{где $J$ - сила тока, $u$ - приложенное к мембране напряжение, $e$ - элементарный заряд, $z$ - зарядовое число пропускаемого иона, $k$ — постоянная Больцмана, $T = 24 {}^\circ\mathrm{C}$ - температура во время проведения эксперимента, $P_{out}$ и $c_{out}$ — проницаемость и концентрация для иона, который находится во внеклеточном растворе, $P_{in}$ и $c_{in}$ —проницаемость и концентрация для иона, который находится внутри клетки.
Считайте, что за время эксперимента, составы растворов снаружи и внутри клетки
остаются неизменными.}

\QBlock{A2}{4.00}{Постройте графики и с помощью них определите отношения $\cfrac{P_\mathrm{K}}{P_\mathrm{Na}}$ и $\cfrac{P_\mathrm{H}}{P_\mathrm{Na}}$.}

\Chapter{Часть B. Спектр действия канальных родопсинов}

\QText{Другой фундаментальной характеристикой канальных родопсинов является их спектр действия. Спектр действия — это зависимость заряда проходящего через клеточную мембрану с каналами от длины волны возбуждающего света.
Для измерения спектра действия канального родопсина, исследуемую клетку облучают ультракороткими вспышками лазера (продолжительностью 5 нс; импульсы содержат одинаковое количество фотонов на вспышку для разных длин волн) и измеряют ток (см. рис 4).
Заряд q, проходящий через мембрану, считается за время полузатухания тока. Полученная зависимость $q(\lambda)$ нормируется на максимальное значение заряда $q_{\max} = q (\lambda_{\max})$.}

\QPicture{X20-PE Patch-clamp_files/2221.jpeg}{"max-width:500px;display:block;margin-left: auto;margin-right: auto;"}{Рис. 4: Зависимости тока (нА) от времени (мс) при облучении ультракороткими вспышками разной длины волны.}{}

\QBlock{B1}{2.50}{Постройте спектр действия канального родопсина. Определите положение максимума спектра $\lambda_{\max}$.}

\Chapter{Часть C. Параметры фотоцикла}

\QPicture{X20-PE Patch-clamp_files/2225.jpeg}{"width:100\%; max-width: 100\%;display:block;margin-left: auto;margin-right: auto; cursor:pointer;" data-toggle="modal" data-target="#kt_modal_2225"}{Рис. 5: Фотоцикл канального родопсина}{Во введении говорилось, что канал при поглощении фотонов проходит несколько промежуточных состояний и возвращаются в начальное закрытое состояние. Все эти состояния вместе называются фотоциклом (см. рис 5, на котором показан фотоцикл, в котором одно промежуточное состояние $I, C$ и $O$ – закрытое и открытое состояния соответственно). Переходы между этими состояниями происходят с некоторой вероятностью. Переход из закрытого состояния в открытое невозможен без поглощения света. При наличии света этот переход также происходит с некоторой вероятностью. Вероятность перехода между двумя состояниями $A$ и $B$ описывается величиной $\tau$ - характерное время перехода из $A$ в $B$. Это время определяется, как обратная производная вероятности перехода по времени: $\tau_{AB}=\left(\cfrac{dp}{dt}\right)^{-1}$ (то есть вероятность перехода за время $dt$ равна $dp$).}

\QBlock{C1}{1.00}{Используя все экспериментальные данные, полученные выше, определите параметры фотоцикла канального родопсина $\tau_{OI}$ и $\tau_{IC}$.}

\Chapter{Часть D. Селективность канального родопсина}

\QText{Одной из важнейших характеристик канальных родопсинов является их селективность. Селективность — это способность пропускать ионы только определенного типа. Селективность может быть разной: могут через канал пропускаться только положительные или только отрицательные ионы, или может пропускаться, например, только натрий (или калий, или любой другой ион). Селективность определяется внутренним устройством белка.
Рассмотрим эксперимент с другим канальным родопсинов (см. рис. 6, графики, данные в пунктах А-С, не имею к нему отношения).
В этом эксперименте составы растворов следующие:
Омывающий: pH=7.5, $\mathrm{[Na]}$ = 200 ммоль/л, $\mathrm{[K]}$ = 0 ммоль/л, $\mathrm{[Cl]}$ = 200 ммоль/л
Внутриклеточный: pH=7.5, $\mathrm{[Na]}$ = 100 ммоль/л, $\mathrm{[K]}$ = 0 ммоль/л, $\mathrm{[Cl]}$ = 100 ммоль/л}

\QPicture{X20-PE Patch-clamp_files/2230.jpeg}{"max-width:500px;display:block;margin-left: auto;margin-right: auto;"}{Рис. 6: ВАХ канального родопсина}{}

\QBlock{D1}{1.00}{По ВАХ определите положительные или отрицательные ионы пропускает этот канальный родопсин. При помощи схем и рисунков объясните, как вы определили селективность.}

\Chapter{Приложение}

\QPicture{X20-PE Patch-clamp_files/2232.jpeg}{"max-width:500px;display:block;margin-left: auto;margin-right: auto;"}{Условие 1. Снизу вверх напряжение меняется от -100 мВ до 80 мВ с шагом 20 мВ. Линия 0 мВ выделена жирным. Свет включен между 250 мс и 750 мс.}{}

\QPicture{X20-PE Patch-clamp_files/2233.jpeg}{"max-width:500px;display:block;margin-left: auto;margin-right: auto;"}{Условие 2. Снизу вверх напряжение меняется от -100 мВ до 80 мВ с шагом 20 мВ. Линия 0 мВ выделена жирным. Свет включен между 250 мс и 750 мс.}{}

\QPicture{X20-PE Patch-clamp_files/2234.jpeg}{"max-width:500px;display:block;margin-left: auto;margin-right: auto;"}{Условие 3. Снизу вверх напряжение меняется от -100 мВ до 80 мВ с шагом 20 мВ. Линия 0 мВ выделена жирным. Свет включен между 250 мс и 750 мс.}{}

\end{document}