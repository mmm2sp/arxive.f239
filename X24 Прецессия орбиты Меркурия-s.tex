
%\documentstyle[12pt,russian,amsthm,amsmath,amssymb]{article}
\documentclass[a4paper,11pt,twoside]{article}
\usepackage[left=14mm, top=10mm, right=14mm, bottom=10mm, nohead, nofoot]{geometry}
\usepackage{amsmath, amsfonts, amssymb, amsthm} % стандартный набор AMS-пакетов для математ. текстов
\usepackage{mathtext}
\usepackage[utf8]{inputenc} % кодировка utf8
\usepackage[russian]{babel} % русский язык
\usepackage[pdftex]{graphicx} % графика (картинки)
\usepackage{tikz}
\usepackage{fancyhdr,pageslts} % настройка колонтитулов
\usepackage{enumitem} % работа со списками
\usepackage{multicol} % работа с таблицами
%\usepackage{pscyr} % красивый шрифт
\usepackage{pgfornament} % красивые рюшечки и вензеля
\usepackage{ltxgrid} % управление написанием текста в две колонки
\usepackage{lipsum} % стандартный текст
\usepackage{tcolorbox} % рамка вокруг текста
\tcbuselibrary{skins}
% ----------------------------------------

\newcommand\ProblemName{Прецессия орбиты Меркурия}

\newcommand\Source{X24}

\newcommand\Type{Решение}

\newcommand\MyTextLeft{Президентский ФМЛ 239, г.~Санкт-Петербург}
\newcommand\MyTextRight{Использованы материалы сайта pho.rs}
\newcommand\MyHeading{Учебно-тренировочные сборы по физике}
% ----------------------------------------

% настройки полей
\geometry{
	left=12mm,
	top=21mm,
	right=15mm,
	bottom=26mm,
	marginparsep=0mm,
	marginparwidth=0mm,
	headheight=22pt,
	headsep=2mm,
	footskip=7mm}
% ----------------------------------------

% настройки колонтитулов
\pagestyle{fancy}

\fancypagestyle{style}{
	\fancyhf{}
	\fancyhead[L]{{\Large{\FancyTitle}}\\\vskip -5pt \dotfill}
	\fancyhead[R]{{\Large{\textbf{\Type}}}\\\vskip -5pt \dotfill}
	\renewcommand{\headrulewidth}{0pt}
	\renewcommand{\footrulewidth}{0pt}
	\fancyfoot[C]{\pgfornament[width=2em,anchor=south]{72}\hspace{1mm}
		{Страница \textbf{\thepage} из \textbf{\pageref{VeryLastPage}}}\hspace{2mm}
		\pgfornament[width=2em,symmetry=v,anchor=south]{72}\\ \vskip2mm
		{\small{\textit{\MyTextLeft\hfill\MyTextRight}}}}
}

\fancypagestyle{plain}{
	\fancyhf{}
	\renewcommand{\headrulewidth}{0pt}
	\renewcommand{\footrulewidth}{0pt}
	\fancyhead[C]{{\Large{\textit{\MyHeading}}}\\\vskip -5pt \dotfill}
	\fancyfoot[C]{\pgfornament[width=2em,anchor=south]{72}\hspace{1mm}
		{Страница \textbf{\thepage} из \textbf{\pageref{VeryLastPage}}}\hspace{2mm}
		\pgfornament[width=2em,symmetry=v,anchor=south]{72}\\ \vskip2mm
		{\small{\textit{\MyTextLeft\hfill\MyTextRight}}}}
}
% ----------------------------------------

% другие настройки
\pagenumbering{arabic}
\setlist[enumerate,itemize]{leftmargin=0pt,itemindent=2.7em,itemsep=0cm}
% ----------------------------------------

% собственные команды
\newcommand{\FancyTitle}{\textbf{\Source} --- \ProblemName}
\newcommand{\Title}{\begin{center}{\huge{\textbf{\Source} --- \ProblemName}}\end{center}}
\newcommand{\Chapter}[1]{\vskip5pt{\Large{\textbf{#1}}}\vskip5pt}
\newcommand{\QText}[1]{#1}
\newcommand{\QBlock}[3]{
	\begin{tcolorbox}[left=2mm,top=2mm,bottom=1mm,right=2mm,colback=white]
		\begin{tcolorbox}[enhanced,colframe=ProcessBlue,colback=ProcessBlue!30!white,
			frame style={opacity=0.7},interior style={opacity=1.0},
			nobeforeafter,tcbox raise base,shrink tight,extrude by=1.7mm,width=1.5cm]
			\textbf{#1\textsuperscript{#2}}
		\end{tcolorbox}\hspace{3mm}#3
	\end{tcolorbox}
}
\newcommand{\QPicture}[4]{
	\begin{figure}[H]
		\centering
		\includegraphics[width=0.35\linewidth]{#1}
		\caption{#3}
	\end{figure}
	
	#4
}
\newcommand{\ABlock}[1]{
	\vskip2mm
	\begin{tcolorbox}[enhanced,colframe=Magenta,colback=Magenta!15!white,
		frame style={opacity=0.5},interior style={opacity=1.0},
		nobeforeafter,tcbox raise base,shrink tight,extrude by=1.7mm,width=1.6cm]
		\textbf{Ответ:}
	\end{tcolorbox}\hspace{3mm}#1
}
\newcommand{\MBlock}[2]{
	\begin{tcolorbox}[enhanced,colframe=Yellow,colback=Yellow!15!white,
		frame style={opacity=0.5},interior style={opacity=1.0},
		nobeforeafter,tcbox raise base,shrink tight,extrude by=1.7mm,width=1.1cm]
		\textbf{#1}
	\end{tcolorbox}\hspace{3mm}#2
}
\newcommand{\MMBlock}[3]{
	\begin{tcolorbox}[enhanced,colframe=Yellow,colback=Yellow!15!white,
		frame style={opacity=0.5},interior style={opacity=1.0},
		nobeforeafter,tcbox raise base,shrink tight,extrude by=1.7mm,width=1.1cm]
		\textbf{#1}
	\end{tcolorbox}\hspace{3mm}
	\begin{tcolorbox}[enhanced,colframe=Orange,colback=Orange!15!white,
		frame style={opacity=0.5},interior style={opacity=1.0},
		nobeforeafter,tcbox raise base,shrink tight,extrude by=1.7mm,width=0.8cm]
		\textbf{#2}
	\end{tcolorbox}\hspace{3mm}#3
}
% ----------------------------------------


\begin{document}
	
	% настройки
	\pagestyle{style}\thispagestyle{plain}
	\Title
	% ----------------------------------------
	
	%\vskip5mm
	%\centering{\pgfornament[width=5cm,anchor=south]{89}}
	
	% смысловая часть


\QBlock{A1}{0.40}{В этом пункте рассмотрим движение Меркурия без учета других планет. Орбиту считайте круговой. Найдите период его обращения вокруг Солнца $T$ (формулу и численное значение в годах) и величину момента импульса  $L$ (только формулу). Выразите ответ через $G$, $a$, $M_S$, $m$.}

\QText{При движении по окружности 
$$
\frac{m v^2}{a} = \frac{GM_S m}{a^2}, \quad  v = \sqrt{\frac{G M_S}{a}},
$$
откуда период движения
$$
T = \frac{2 \pi a}{v} = 2\pi \sqrt{\frac{a^3}{G M_S}},
$$
а момент импульса
$$
L= m v a = m \sqrt{G M_S a}.
$$}

\ABlock{$$
T = \frac{2 \pi a}{v} = 2\pi \sqrt{\frac{a^3}{G M_S}} = 0.24  \text{года}, \quad
L= m v a = m \sqrt{G M_S a}
$$}

\QBlock{A2}{0.20}{Найдите проекцию ускорения падения $g_z$, создаваемого кольцом на его оси симметрии на расстоянии $z \ll R$ от центра. Выразите ответ через $G$, $M$, $R$, $z$.}

\QText{Из соображений симметрии компоненты ускорения свободного падения, не направленные вдоль оси $z$, равны 0. Рассмотрим участок кольца массы $dM$. Он создает ускорение $dg = G dM/(R^2 + z^2) \approx GdM/R^2$, а его проекция на ось $z$ равна $dg_z = - dg /\sqrt{R^2 + z^2} \approx dg z/R =  -G dM z/R^3$ (знак - означает, что поле направлено к центру кольца), поэтому}

\ABlock{$$
g_z = -  \frac{G M z }{R^3}
$$}

\QBlock{A3}{0.50}{Найдите радиальную проекцию ускорения свободного падения $g_r$, создаваемого кольцом в точке в плоскости кольца на расстоянии $ r \ll R$ от его центра. Используйте теорему Гаусса для гравитационного поля. Запишите ответ с точностью до слагаемых порядка $r$. Выразите ответ через $G$, $M$, $R$, $r$. Положительным считается направление от оси кольца.}

\QText{Внутри конца нет массивных тел, создающих гравитационное поле, поэтому поток вектора $\vec{g}$ через любую замкнутую поверхность равен нулю. В качестве такой поверхности выберем цилиндр радиуса $r$, ось которого совпадает с осью кольца, нижнее  основание лежит в плоскости кольца, координата верхнего  основания равна $z$. Поскольку $r \ll R$, можно считать, что во всех точках верхнего основания проекция $g_z$ совпадает со своим значением в центре. На нижнем основании $g_z = 0$, поскольку все массы находятся в плоскости кольца. Поскольку $z \ll r$, на всех точках боковой поверхности цилиндра радиальная составляющая равна своему значению $g_r$ в плоскости кольца. Окончательно, поток через цилиндр
$$
\Phi_g = \pi r^2 g_z + 2\pi r z g_r = 0,
$$
откуда
$$
g_r = - \frac{r}{2z} g_z = +\frac{G M r }{ 2 R^3}.
$$}

\ABlock{$$
g_r  = +\frac{G M r }{ 2 R^3}
$$}

\QBlock{A4}{0.30}{Найдите потенциальную энергию $V(r)$ Меркурия, если он находится на расстоянии $r$ от  Солнца. Выразите ответ через $r$, $R$, $G$, $M_S$, $M$, $m$. Потенциальная энергия взаимодействия с кольцом равна нулю, когда Меркурий находится в его центре, а потенциальная энергия взаимодействия с Солнцем - когда Меркурий находится на бесконечности.}

\QText{Сила, действующая на Меркурий со стороны кольца, равна
$$
F_r = m g_r = \frac{GMm}{2 R^3} r, 
$$
интегрируя найдем потенциальную энергию
$$
V_1 = - \frac{GM m}{4 R^3}r^2.
$$
К ней нужно добавить потенциальную энергию взаимодействия с Солнцем.}

\ABlock{$$
V = - \frac{GM m}{4 R^3}r^2 - \frac{GM_S m}{r}
$$}

\QBlock{B1}{0.40}{Пусть момент импульса Меркурия равен $L$. Выразите производную расстояния до Солнца по времени $\dot{r} = dr/dt$ через производную $u$ по углу $u' = du/d\theta$, а также через $L$, $u$, $m$.}

\QText{Выразим момент импульса через производную угла по времени
$$
L = m r v_\theta = m r^2 \frac{d\theta}{dt}.
$$
Тогда
$$
\frac{dr}{dt} = \frac{d r}{d \theta} \frac{d \theta}{dt} = \frac{L}{m r^2} \frac{d}{d\theta} \left( \frac{1}{u}\right) = - \frac{L u^2}{m} \frac{1}{u^2} \frac{du}{d\theta} = - \frac{L}{m} u'.
$$}

\ABlock{$$
\frac{dr}{dt} = - \frac{L}{m} u'
$$}

\QBlock{B2}{0.40}{Запишите выражение для полной механической энергии Меркурия. Выразите ответ через $u$, $u'$, $m$, $L$, $V(r)$.}

\QText{Кинетическая энергия
$$
T = \frac{m v^2}{2} = \frac{m \dot{r}^2}{2 } + \frac{m v_\theta^2}{2} = \frac{L^2}{2 m} u^{\prime 2} + \frac{L^2}{2 m r^2},
$$
$$
T = \frac{L^2}{2 m} (u^{\prime 2} + u^2).
$$
К ней нужно добавить потенциальную энергию $V(r) = V(1/u)$.}

\ABlock{$$
E =  \frac{L^2}{2 m} (u^{\prime 2} + u^2) + V \left( \frac{1}{u}\right)
$$}

\QBlock{B3}{0.60}{Продифференцировав закон сохранения (например по углу), получите выражение для $u''(\theta)$. Выразите ответ через $u$, $G$, $m$, $M_S$, $M$, $R$, $L$.}

\QText{Подставим выражение для потенциальной энергии:
$$
E = \frac{L^2}{2 m} (u^{\prime 2} + u^2) - \frac{GM m}{4 R^3}\frac{1}{u^2} - GM_S m u.
$$
Дифференцируя, получим
$$
E' = 0 = \frac{L^2}{m} (u' u'' + u u') + \frac{G M m}{2 R^3} \frac{u'}{u^3} - G M_S m u'.
$$
Сокращая на $u'$ (поскольку в общем случае $u'  = 0$ не является решением уравнений движения), найдем}

\ABlock{$$
u'' = - u + \frac{G M_S m^2 }{L^2} - \frac{GM m^2}{2 R^3 L^2} \frac{1}{u^3}
$$}

\QBlock{C1}{0.30}{Используя уравнение из $B3$, получите уравнение, из которого можно найти радиус круговой орбиты $a$ (решать его не нужно). В уравнение могут входить $m$, $M_s$, $M$,  $R$, $G$, $L$, $a$.}

\QText{При движении по круговой орбите $u'' = 0$. Подставляя $u = 1/a$, получим уравнение на радиус орбиты.}

\ABlock{$$
- \frac{1}{a} + \frac{G M_S m^2 }{L^2}- \frac{GM m^2}{2 R^3 L^2} a^3 = 0
$$}

\QBlock{C2}{0.60}{Получите линеаризованное уравнение движения для отклонения $\delta u$ формы орбиты от круговой, то есть выражение для $\delta u''$ в первом порядке по $\delta u$. В ответ могут входить величины, использованные в C1.}

\QText{Подставим в уравнение $u = u_0 + \delta u$ и разложим до первого порядка по $\delta u$:
$$
\delta u '' = - u_0 - \delta u+  \frac{G M_S m^2 }{L^2} - \frac{GM m^2}{2 R^3 L^2} \left(\frac{1}{u_0^3}  - \frac{3 \delta u}{u_0^4} \right).
$$
Не зависящие от $\delta u$ слагаемые сокращаются в силу уравнения из предыдущего пункта. Подставляя $u_0 = 1/a$, получим}

\ABlock{$$
\delta u'' = - \left( 1 - \frac{3 G M m^2}{2 R^3 L^2} a^4\right) \delta u
$$}

\QBlock{C3}{0.50}{Пренебрежем слагаемыми, описывающими поле кольца. Запишите в таком приближении решение для $\delta u(\theta)$. Считайте, что $\theta = 0$ отвечает максимальному значению $u$. Выразите ответ через радиус круговой орбиты $a$ и эксцентриситет $e \ll 1$. Покажите, что решение описывает замкнутую орбиту.}

\QText{Полю кольца отвечает слагаемое, содержащее $M$. Отбросив его, получим
$$
\delta u ''  = - \delta u.
$$
Отметим, что это уравнение является точным, поскольку если пренебречь вкладом кольца, уравнение на $u$ будет линейным. 
Его решение, максимальное при $\theta = 0$, имеет вид 
$$
\delta u = A \cos \theta, \quad A > 0.
$$
Тогда полное решение
$$
 u = \frac{1}{a} \left(1 + a A \cos \theta \right) = \frac{1}{a} \left(1 + e \cos \theta \right),
$$
поэтому коэффициент перед косинусом
$$
A = \frac{e}{a}.
$$
Поскольку $u(\theta) = u(\theta  + 2\pi)$, орбита замкнута.}

\ABlock{$$
\delta u = \frac{e}{a } \cos \theta.
$$}

\QBlock{C4}{0.50}{Найдите решение уравнения для $\delta u(\theta)$ из $C2$  с теми же начальными условиями, что и в $C3$. Теперь учитывайте вклад кольца. Выразите ответ через радиус орбиты $a$, ее эксцентриситет $e\ll 1$ и $M$, $M_S$, $R$. Значение момента импульса равно найденному в A1.}

\QText{Подставим в уравнение из C2 выражение для момента импульса
$$
\delta u'' = - \left(1 -\frac{3 G M m^2 a^4}{2 R^3 m^2 G M_S a}  \right) \delta u = - \left(1 -\frac{3  M a^3 }{2 M_S R^3 } \ \right) \delta u.
$$
Его решение имеет вид}

\ABlock{$$
\delta u = \frac{e}{a} \cos \left( \theta\sqrt{1 - \frac{3  M a^3 }{2 M_S R^3 } }\right)
$$}

\QBlock{C5}{0.50}{Из найденного решения следует, что за один оборот орбиты положение перигелия Меркурия меняется на некоторую величину $\delta \theta$. Получите точное выражение для $\delta  \theta$, следующее из решения в $C4$, а также приближенное выражение для $\delta  \theta$ в первом порядке по $M$. Укажите направление смещения перигелия (по направлению вращения Меркурия или против). Выразите ответ через $M$, $M_S$, $R$, $a$.}

\QText{Меркурий оказывается в перигелии (т.е. его расстояние до Солнца минимально), когда величина $u$ максимальна, то есть
$$
\theta = \frac{2\pi n}{\sqrt{1 - \dfrac{3  M a^3 }{2 M_S R^3 }} }.
$$
В частности, изменение положения перигелия за один оборот
$$
\delta \theta  = \frac{2\pi }{\sqrt{1 - \dfrac{3  M a^3 }{2 M_S R^3 }} } - 2\pi \approx
2\pi \left( 1 + \dfrac{3  M a^3 }{4 M_S R^3 }\right) - 2\pi = \frac{3 \pi }{2} \frac{M a^3}{M_S R^3}.
$$
Поскольку $\delta \theta > 0$, то есть перигелий отстоит от предыдущего на угол больше $2\pi$, направление поворота перигелия совпадает с направлением движения Меркурия.}

\ABlock{$$
\delta \theta  = \frac{3 \pi }{2} \frac{M a^3}{M_S R^3}
$$}

\QBlock{С6}{0.70}{В таблице указаны радиусы орбит (считаем их приближенно равным большим полуосям) и массы для планет Солнечной системы. Для каждой из них вычислите сдвиг перигелия Меркурия за один оборот вокруг Солнца. Во всех случаях можете считать, что радиус орбиты планеты много больше радиуса орбиты Меркурия. Выразите ответ в угловых секундах. Укажите две планеты, дающие наибольший вклад в прецессию. \textit{Примечание: }Угловая секунда – единица измерения углов, которая составляет 1/3600 градуса.}

\ABlock{\begin{table}[h!] \centering \begin{tabular}{|c|c|} \hline 
Планета  & Вклад в $\delta \theta$,   ‘’\\ 
 \hline 
Венера & $0.3669$\\ 
 \hline 
Земля & $0.1677$\\ 
 \hline 
Марс & $0.0051$\\ 
 \hline 
Юпитер & $0.3806$\\ 
 \hline 
Сатурн & $0.0183$\\ 
 \hline 
Уран  & $3.47\cdot 10^{-4} $\\ 
 \hline 
Нептун & $1.06 \cdot 10^{-4}$\\ 
 \hline 
\end{tabular} \end{table}}

\ABlock{Наибольший вклад дают Венера (поскольку расположена ближе всего) из Юпитер (поскольку его масса максимальна).}

\QBlock{C7}{0.30}{Найдите угол, на который перигелий Меркурия смещается за одно столетие, с учетом вкладов всех планет, приведенных в таблице. Выразите ответ в угловых секундах.}

\QText{В первом приближении можно считать, что вклады всех планет складываются. Для того, чтобы получить смещение за столетие, умножим суммарный вклад всех планет на $100/T$, где $T$ - период Меркурия в годах, получим}

\ABlock{$$
\Delta \theta = 391'' \text{ за столетие}
$$}

\QText{Отметим, что полученное значение $\Delta \theta$ отличается от точного предсказания Ньютоновской механики для прецессии орбиты. Это связано с тем, что мы не учитывали следующие порядки в разложении потенциала планет (радиус Венеры всего в 2 раза больше орбиты Меркурия, поэтому первого члена разложения явно недостаточно). Кроме этого, нужно учитывать эллиптичность орбит и то, что они не лежат в одной плоскости. Точное предсказание классической механики $\Delta \theta = 532 ''$ за столетие.}

\QBlock{D1}{0.80}{Пусть Меркурий движется в центральном поле,  действующая на него сила равна $\vec{F}  = F_r \vec{e}_r$. Покажите, что производная равна
$$
\frac{d}{dt} \left( \vec{v} \times \vec{L}\right) = B \frac{d}{d t} \vec{e}_r.
$$
Найдите коэффициент $B$, выразите его через $r$, $m$, $F_r$.}

\QText{Используем второй закон Ньютона 
$$
m \vec{a} = m \frac{d}{dt} \vec{v}  = \vec{F} = F_r \vec{e}_r,
$$
а также выражение для момента импульса через угловую скорость Меркурия
$$
\vec{L}  = m r^2 \vec{\omega}.
$$
Тогда требуемая производная
$$
\frac{d}{dt} \left( \vec{v} \times \vec{L}\right) = \vec{a} \times \vec{L} = \frac{F_r}{m} \vec{e}_r \times m r^2 \vec{\omega}= - r^2F_r \vec{\omega} \times \vec{e}_r = - r^2 F_r \frac{d \vec{e}_r}{dt}.
$$
Тот же результат можно получить и более прямолинейным вычислением
$$
\frac{d}{dt} \left( \vec{v} \times \vec{L}\right) = \frac{F_r}{m} \vec{e}_r \times \vec{L}  = \frac{F_r}{m} \vec{e}_r \times m \left[\vec{r} \times \vec{v} \right].
$$
Раскрывая двойное векторное произведение, получим
$$
\frac{d}{dt} \left( \vec{v} \times \vec{L}\right) = F_r \left(\vec{r} (\vec{e}_r \vec{v}) - \vec{v} (\vec{r} \vec{e}_r) \right)=
F_r \left( \frac{(\vec{r} \vec{v}) \vec{r}}{r} - r \vec{v}\right) = F_r r^2\left( \frac{(\vec{r} \vec{v}) \vec{r}}{r^3} - \frac{\vec{v}}{r}\right) .
$$
Найдем теперь производную $\vec{e}_r$:
$$
\frac{d}{dt} \vec{e}_r = \frac{d}{dt} \left( \frac{\vec{r}}{r}\right) = \frac{\vec{v}}{r} - \vec{r} \frac{1}{r^2} \frac{d r}{dt} = \frac{\vec{v}}{r} - \frac{\vec{r} (\vec{v} \vec{r})}{r^3}.
$$
Сравнивая эти выражения, возвращаемся к тому же результату}

\ABlock{$$
\frac{d}{dt} \left( \vec{v} \times \vec{L}\right)  = - r^2 F_r \frac{d \vec{e}_r}{dt}, \quad B = - r^2 F_r
$$}

\QBlock{D2}{0.50}{Найдите производную по времени вектора Лапласа  для Меркурия, который движется в поле Солнца и кольца. Выразите ответ через $m$, $M$, $R$,  $r$, $G$, $\dfrac{d \vec{e}_r}{dt}$.}

\QText{Используя соотношение из предыдущего пункта, продифференцируем вектор Лапласа и получим
$$
\frac{d \vec{A}}{dt} = \frac{d}{dt} \left( \vec{v} \times \vec{L}\right) - G M_S m \frac{d \vec{e}_r}{dt} = \left( -r^2 F_r - G M_S m\right) \frac{d \vec{e}_r}{dt}.
$$
Сила, действующая на Меркурий, равна
$$
F_r = - \frac{GM_S m}{r^2 } + \frac{GMm}{2 R^3} r.
$$
Первое слагаемое в силе сокращается с вторым слагаемым из определения вектора Лапласа. Поэтому при движении под действием притяжения к Солнцу вектор Лапласа сохраняется. С учетом кольца получаем}

\ABlock{$$
\frac{d \vec{A}}{dt} = - \frac{GMm}{2 R^3} r^3\frac{d \vec{e}_r}{dt}
$$}

\QBlock{D3}{1.00}{Пусть $\theta$ - полярный угол между радиус-вектором Меркурия $\vec{r}$ и направлением оси $x$. Угол $\theta$ возрастает при движении Меркурия. Вычислите производные компонент вектора Лапласа по $\theta$, $A_x'$ и $A_y'$. Выразите ответ через $m$, $M$, $R$,  $r$, $G$, $\theta$.}

\QText{Из предыдущего результата следует
$$
d \vec{A} = - \frac{GMm}{2 R^3} r^3 d \vec{e}_r,
$$
дифференцируя по углу, получим
$$
\frac{d \vec{A}}{d\theta} = - \frac{GMm}{2 R^3} r^3\frac{d \vec{e}_r}{d\theta}.
$$
Проекции единичного вектора $\vec{e}_r$ на координатные оси
$$
e_{rx} = \cos \theta, \quad e_{r y} = \sin \theta,
$$
а их производные
$$
\frac{d e_{rx}}{d\theta}  = - \sin\theta, \quad \frac{d e_{ry}}{d\theta} = \cos \theta.
$$
Тогда для вектора Лапласа}

\ABlock{$$
\frac{d A_x}{d\theta} =  \frac{GMm}{2 R^3} r^3 \sin \theta, \quad \frac{d A_y}{d\theta} =   - \frac{GMm}{2 R^3} r^3 \cos \theta
$$}

\QBlock{D4}{1.00}{Найдите изменение вектора Лапласа $\Delta \vec{A}$ за период. Укажите проекции этого изменения на оси, указанные в предыдущем пункте. При вычислении считайте, что Меркурий движется по эллиптической орбите. Выразите ответ через большую полуось орбиты $a$, эксцентриситет $e$, $G$, $m$, $M$, $M_S$.}

\QText{При движении по эллиптической орбите 
$$
r = \frac{p} {1 + e \cos \theta},
$$
где $p = a(1 - e^2)$ - параметр эллипса.
Тогда производные компонент вектора Лапласа имеют вид
$$
\frac{d A_x}{d\theta} =  \frac{GMm}{2 R^3} \frac{p^3}{(1 + e\cos \theta)^3} \sin \theta, \quad \frac{d A_y}{d\theta} =   - \frac{GMm}{2 R^3} \frac{p^3}{(1 + e\cos \theta)^3} \cos \theta.
$$
Интегрируя по углу от $0$ до $2 \pi $ (или что то же самое, от $-\pi$ до $\pi$, поскольку функции периодичные), получим изменение вектора Лапласа за период:
$$
\Delta A_x = \frac{GM mp^3}{2 R^3} \int_{-\pi}^{\pi} \frac{\sin \theta \, d \theta}{(1 + e \cos \theta)^3} = 0, 
$$
поскольку функция под интегралом нечетная,
$$
\Delta A_y =- \frac{GM mp^3}{2 R^3} \int_{0}^{2 \pi} \frac{\cos \theta \, d \theta}{(1 + e \cos \theta)^3} = \frac{GM mp^3}{2 R^3} \frac{3 \pi e}{(1- e^2)^{5/2}}.
$$
Здесь использовался интеграл из условия. Подставляя формулу для параметра, получим}

\ABlock{$$
\Delta A_x = 0, \quad \Delta A_y = \frac{3 \pi GM m a^3}{2 R^3}  e(1- e^2)^{1/2}.
$$}

\QBlock{D5}{0.50}{Найдите поворот направления на перигелий Меркурия за период $\Delta \theta$. Выразите ответ через $G$, $M_S$, $M$, $R$, $a$, $e$. На сколько процентов изменится результат пункта C7 для смещения перигелия за счет учета эксцентриситета орбиты Меркурия?}

\QText{Вектор Лапласа направлен вдоль оси $x$, поэтому изменение вектора Лапласа перпендикулярно ему. Значит в первом приближении его длина не меняется, а он сам поворачивается на угол 
$$
\Delta \theta = \frac{\Delta A_y}{A} = \frac{3\pi }{2} \frac{M a^3}{M_S R^3} (1 - e^2)^{1/2}.
$$
Поскольку вектор Лапласа направлен к перигелию, его угол поворота равен смещению перигелия. Точный ответ отличается от приближенного ответа из части C только множителем $(1 - e^2)^{1/2} \approx 1 - e^2/2$. Из-за этой поправки ответ уменьшится  на $e^2/2 \approx 0.02 = 2\text{\%}$.}

\ABlock{$$
\Delta \theta =  \frac{3\pi }{2} \frac{M a^3}{M_S R^3} (1 - e^2)^{1/2}.
$$ 
Угол поворота перигелия уменьшится на $2 \text{\%}$.}

\end{document}