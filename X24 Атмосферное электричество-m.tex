
%\documentstyle[12pt,russian,amsthm,amsmath,amssymb]{article}
\documentclass[a4paper,11pt,twoside]{article}
\usepackage[left=14mm, top=10mm, right=14mm, bottom=10mm, nohead, nofoot]{geometry}
\usepackage{amsmath, amsfonts, amssymb, amsthm} % стандартный набор AMS-пакетов для математ. текстов
\usepackage{mathtext}
\usepackage[utf8]{inputenc} % кодировка utf8
\usepackage[russian]{babel} % русский язык
\usepackage[pdftex]{graphicx} % графика (картинки)
\usepackage{tikz}
\usepackage{fancyhdr,pageslts} % настройка колонтитулов
\usepackage{enumitem} % работа со списками
\usepackage{multicol} % работа с таблицами
%\usepackage{pscyr} % красивый шрифт
\usepackage{pgfornament} % красивые рюшечки и вензеля
\usepackage{ltxgrid} % управление написанием текста в две колонки
\usepackage{lipsum} % стандартный текст
\usepackage{tcolorbox} % рамка вокруг текста
\tcbuselibrary{skins}
% ----------------------------------------

\newcommand\ProblemName{Атмосферное электричество}

\newcommand\Source{X24}

\newcommand\Type{Разбалловка}

\newcommand\MyTextLeft{Президентский ФМЛ 239, г.~Санкт-Петербург}
\newcommand\MyTextRight{Использованы материалы сайта pho.rs}
\newcommand\MyHeading{Учебно-тренировочные сборы по физике}
% ----------------------------------------

% настройки полей
\geometry{
	left=12mm,
	top=21mm,
	right=15mm,
	bottom=26mm,
	marginparsep=0mm,
	marginparwidth=0mm,
	headheight=22pt,
	headsep=2mm,
	footskip=7mm}
% ----------------------------------------

% настройки колонтитулов
\pagestyle{fancy}

\fancypagestyle{style}{
	\fancyhf{}
	\fancyhead[L]{{\Large{\FancyTitle}}\\\vskip -5pt \dotfill}
	\fancyhead[R]{{\Large{\textbf{\Type}}}\\\vskip -5pt \dotfill}
	\renewcommand{\headrulewidth}{0pt}
	\renewcommand{\footrulewidth}{0pt}
	\fancyfoot[C]{\pgfornament[width=2em,anchor=south]{72}\hspace{1mm}
		{Страница \textbf{\thepage} из \textbf{\pageref{VeryLastPage}}}\hspace{2mm}
		\pgfornament[width=2em,symmetry=v,anchor=south]{72}\\ \vskip2mm
		{\small{\textit{\MyTextLeft\hfill\MyTextRight}}}}
}

\fancypagestyle{plain}{
	\fancyhf{}
	\renewcommand{\headrulewidth}{0pt}
	\renewcommand{\footrulewidth}{0pt}
	\fancyhead[C]{{\Large{\textit{\MyHeading}}}\\\vskip -5pt \dotfill}
	\fancyfoot[C]{\pgfornament[width=2em,anchor=south]{72}\hspace{1mm}
		{Страница \textbf{\thepage} из \textbf{\pageref{VeryLastPage}}}\hspace{2mm}
		\pgfornament[width=2em,symmetry=v,anchor=south]{72}\\ \vskip2mm
		{\small{\textit{\MyTextLeft\hfill\MyTextRight}}}}
}
% ----------------------------------------

% другие настройки
\pagenumbering{arabic}
\setlist[enumerate,itemize]{leftmargin=0pt,itemindent=2.7em,itemsep=0cm}
% ----------------------------------------

% собственные команды
\newcommand{\FancyTitle}{\textbf{\Source} --- \ProblemName}
\newcommand{\Title}{\begin{center}{\huge{\textbf{\Source} --- \ProblemName}}\end{center}}
\newcommand{\Chapter}[1]{\vskip5pt{\Large{\textbf{#1}}}\vskip5pt}
\newcommand{\QText}[1]{#1}
\newcommand{\QBlock}[3]{
	\begin{tcolorbox}[left=2mm,top=2mm,bottom=1mm,right=2mm,colback=white]
		\begin{tcolorbox}[enhanced,colframe=ProcessBlue,colback=ProcessBlue!30!white,
			frame style={opacity=0.7},interior style={opacity=1.0},
			nobeforeafter,tcbox raise base,shrink tight,extrude by=1.7mm,width=1.5cm]
			\textbf{#1\textsuperscript{#2}}
		\end{tcolorbox}\hspace{3mm}#3
	\end{tcolorbox}
}
\newcommand{\QPicture}[4]{
	\begin{figure}[H]
		\centering
		\includegraphics[width=0.35\linewidth]{#1}
		\caption{#3}
	\end{figure}
	
	#4
}
\newcommand{\ABlock}[1]{
	\vskip2mm
	\begin{tcolorbox}[enhanced,colframe=Magenta,colback=Magenta!15!white,
		frame style={opacity=0.5},interior style={opacity=1.0},
		nobeforeafter,tcbox raise base,shrink tight,extrude by=1.7mm,width=1.6cm]
		\textbf{Ответ:}
	\end{tcolorbox}\hspace{3mm}#1
}
\newcommand{\MBlock}[2]{
	\begin{tcolorbox}[enhanced,colframe=Yellow,colback=Yellow!15!white,
		frame style={opacity=0.5},interior style={opacity=1.0},
		nobeforeafter,tcbox raise base,shrink tight,extrude by=1.7mm,width=1.1cm]
		\textbf{#1}
	\end{tcolorbox}\hspace{3mm}#2
}
\newcommand{\MMBlock}[3]{
	\begin{tcolorbox}[enhanced,colframe=Yellow,colback=Yellow!15!white,
		frame style={opacity=0.5},interior style={opacity=1.0},
		nobeforeafter,tcbox raise base,shrink tight,extrude by=1.7mm,width=1.1cm]
		\textbf{#1}
	\end{tcolorbox}\hspace{3mm}
	\begin{tcolorbox}[enhanced,colframe=Orange,colback=Orange!15!white,
		frame style={opacity=0.5},interior style={opacity=1.0},
		nobeforeafter,tcbox raise base,shrink tight,extrude by=1.7mm,width=0.8cm]
		\textbf{#2}
	\end{tcolorbox}\hspace{3mm}#3
}
% ----------------------------------------


\begin{document}
	
	% настройки
	\pagestyle{style}\thispagestyle{plain}
	\Title
	% ----------------------------------------
	
	%\vskip5mm
	%\centering{\pgfornament[width=5cm,anchor=south]{89}}
	
	% смысловая часть


\QBlock{A1}{0.80}{Пусть проводящий шар радиусом $R$, несущий заряд $Q$, помещён в однородное электростатическое поле напряжённостью $\vec{E}_0$. Определите полную напряжённость $\vec{E}$ электрического поля в точке с радиус-вектором $\vec{r}$ относительно центра шара, находящейся вне шара. Ответ выразите через $Q$, $\vec{E}_0$, $R$, $\varepsilon_0$ и $\vec{r}$.}

\MBlock{0.20}{Указано, что напряжённость электростатического поля вне шара представляет собой суперпозицию однородного поля, поля точечного заряда и поля диполя, расположенного в центре шара.}

\MBlock{0.10}{Записано выражение для напряжённости поля диполя с дипольным моментом $\vec{p}$:
$$\vec{E}=\cfrac{1}{4\pi\varepsilon_0}\left(\cfrac{3\bigl(\vec{p}\cdot\vec{r}\bigr)\vec{r}}{r^5}-\cfrac{\vec{p}}{r^3}\right){.}
$$}

\MBlock{0.20}{Предложен метод определения дипольного момента шара.}

\MBlock{0.20}{Определён дипольный момент шара:
$$\vec{p}=4\pi R^3\varepsilon_0\vec{E}_0
$$}

\MBlock{0.10}{Получен ответ:
$$\vec{E}=\vec{E}_0\left(1-\cfrac{R^3}{r^3}\right)+\cfrac{3\bigl(\vec{E}_0\cdot\vec{r}\bigr)\vec{r}}{r^2}\cfrac{R^3}{r^3}+\cfrac{Q\vec{r}}{4\pi\varepsilon_0r^3}{.}
$$}

\QBlock{A2}{0.40}{Пусть $\theta$ - угол между направлением вектора электростатического поля $\vec{E}_0$ и радиус-вектором $\vec{r}$ некоторой точки поверхности шара относительно его центра. 
Определите проекцию напряжённости электрического поля $E_n(\theta)$ на направление нормали. Ответ выразите через $Q$, $E_0$, $R$, $\varepsilon_0$ и $\theta$.}

\MBlock{0.40}{Получено выражение для $E_n$:
$$E_n=\cfrac{Q}{4\pi\varepsilon_0R^2}+3E_0\cos\theta
$$}

\QBlock{A3}{0.20}{При каких значениях заряда $Q$ шара величина $E_n$ может обращаться в ноль? Ответ выразите через $E_0$, $\varepsilon_0$ и $R$.
Далее во всех пунктах считайте, что заряд шара $Q$ попадает в найденный вами диапазон.}

\MBlock{0.10}{Записано условие равенства нулю компоненты напряжённости $E_n$:
$$Q=-12\pi\varepsilon_0R^2E_0\cos\theta{.}
$$}

\MBlock{0.10}{Определён искомый диапазон $Q$:
$$Q\in\bigl[-12\pi R^2\varepsilon_0E_0{;}12\pi R^2\varepsilon_0E_0\bigr]{.}
$$}

\QBlock{A4}{0.50}{Пусть заряд шара равен $Q$. При каком значении угла $\theta_0$ компонента напряжённости $E_n(\theta_0)$ обращается в ноль? Ответ выразите через $Q$, $E_0$, $\varepsilon_0$ и $R$.
Определите также, при каких значениях угла $\theta$ на поверхность шара попадают отрицательные ионы, а при каких - положительные. Ответы выразите через $\theta_0$.}

\MBlock{0.10}{Определено значение угла $\theta_0$:
$$\theta_0=\arccos\left(-\cfrac{Q}{12\pi R^2\varepsilon_0E_0}\right){.}
$$}

\MBlock{4 $\times$ 0.10}{Определён диапазон углов, соответствующий попаданию ионов (по $0{.}1$ балла за границу):
Отрицательные ионы попадают на поверхность шара при $\theta\in[0{,}\theta_0]$;
Положительные ионы попадают на поверхность шара при $\theta\in[\theta_0{,}\pi]$.}

\QBlock{A5}{1.00}{Определите полную производную по времени заряда шара $dQ/dt$. Ответ выразите через $Q$, $E_0$, $\varepsilon_0$, $R$, $\sigma_+$, $\sigma_-$ и, если понадобится, $\theta_0$.}

\MBlock{0.20}{Записано выражение для компоненты $dQ_-/dt$, обусловленной попаданием отрицательных ионов:
$$\cfrac{dQ_-}{dt}=-\sigma_-\int\limits_{0}^{\theta_0}\left(\cfrac{Q}{2\varepsilon_0}+6\pi R^2R_0\cos\theta\right)\sin\theta d\theta{.}
$$}

\MBlock{0.20}{Вычислен интеграл для $dQ_-/dt$:
$$\cfrac{dQ_-}{dt}=-\sigma_-\left(\cfrac{Q(1-\cos\theta_0)}{2\varepsilon_0}+3\pi R^2E_0\sin^2\theta_0\right){.}
$$}

\MBlock{0.20}{Записано выражение для компоненты $dQ_+/dt$, обусловленной попаданием положительных ионов:
$$\cfrac{dQ_+}{dt}=-\sigma_+\int\limits_{\theta_0}^{\pi}\left(\cfrac{Q}{2\varepsilon_0}+6\pi R^2E_0\cos\theta\right)\sin\theta d\theta{,}
$$}

\MBlock{0.20}{Вычислен интеграл для $dQ_+/dt$:
$$\cfrac{dQ_+}{dt}=-\sigma_+\left(\cfrac{Q(1+\cos\theta_0)}{2\varepsilon_0}-3\pi R^2E_0\sin^2\theta_0\right){.}
$$}

\MBlock{0.20}{Получен ответ для $dQ/dt$:
$$\cfrac{dQ}{dt}=-\cfrac{Q(\sigma_++\sigma_-)}{2\varepsilon_0}+(\sigma_+-\sigma_-)\left(3\pi R^2E_0\sin^2\theta_0-\cfrac{Q\cos\theta_0}{2\varepsilon_0}\right){.}
$$}

\QBlock{A6}{0.80}{Определите стационарный заряд шара $Q_0$, при котором он остаётся постоянным во времени. Ответ выразите через $E_0$, $\varepsilon_0$, $R$, $\sigma_+$ и $\sigma_-$.}

\MBlock{0.40}{Подставлено значение $\theta_0$ и получено квадратное уравнение относительно $Q_0$:
$$Q^2_0-Q_0\cfrac{24\pi R^2\varepsilon_0E_0(\sigma_++\sigma_-)}{(\sigma_+-\sigma_-)}+144\pi^2R^4\varepsilon^2_0E^2_0=0{.}
$$}

\MBlock{0.20}{Правильно решено квадратное уравнение:
$$Q_0=12\pi R^2\varepsilon_0E_0\left(\cfrac{\sqrt{\sigma_+}+\sqrt{\sigma_-}}{\sqrt{\sigma_+}-\sqrt{\sigma_-}}\right)^{\pm1}{.}
$$}

\MBlock{0.20}{Выбран нужный корень и получен ответ для $Q_0$:
$$Q_0=12\pi R^2\varepsilon_0E_0\left(\cfrac{\sqrt{\sigma_+}-\sqrt{\sigma_-}}{\sqrt{\sigma_+}+\sqrt{\sigma_-}}\right){.}
$$}

\QBlock{A7}{0.60}{При малых значениях $\Delta{Q}$ зависимость $\dot{Q}(\Delta{Q})$ можно представить в виде:
$$\dot{Q}\approx A\Delta{Q}{.}
$$
Выразите $A$ через $\sigma_+$, $\sigma_-$ и $\varepsilon_0$.
Является ли найденное значение заряда $Q_0$ устойчивым? Ответ обоснуйте.}

\MBlock{0.30}{Для зависимости $\dot{Q}(Q)$ в виде:
$$\dot{Q}=aQ^2-bQ+c
$$
записано приближение:
$$\dot{Q}=-(b-2aQ_0)\Delta{Q}
$$}

\MBlock{0.20}{Определено значение $A$:
$$A=-\cfrac{\sqrt{\sigma_+\sigma_-}}{\varepsilon_0}{.}
$$}

\MBlock{0.10}{Для своего знака $A$ сделан вывод об устойчивости значения заряда $Q_0$.}

\QBlock{A8}{0.30}{Получите зависимость отклонения заряда шара $\Delta{Q}$ от времени $t$. Ответ выразите через $\Delta{Q}_0$, $\varepsilon_0$, $\sigma_+$, $\sigma_-$ и $t$. Покажите также, что вид временной зависимости определяется только произведением $\sigma_+\sigma_-$.}

\MBlock{0.20}{Для своего значения $A$ получено:
$$\Delta{Q}(t)=\Delta{Q}_0e^{At}
$$}

\MBlock{0.10}{Получено выражение для $\Delta{Q}(t)$ в следующем виде:
$$\Delta{Q}(t)=\Delta{Q}_0\exp\left(-\cfrac{\sqrt{\sigma_+\sigma_-}t}{\varepsilon_0}\right){.}
$$}

\QBlock{B1}{0.50}{Определите вектор $\vec{E}$ напряжённости электростатического поля в области пересечения изолированных эллипсоидов. Ответ выразите через $\rho$, $\vec{l}$, $\varepsilon_0$ и $A$.}

\MBlock{0.20}{Показано, что компоненты напряжённости $E_x$ и $E_y$ в области пересечения равны нулю.}

\MBlock{0.30}{Получено выражение для $\vec{E}$:
$$\vec{E}=-\cfrac{\rho A\vec{l}}{\varepsilon_0}{.}
$$}

\MBlock{-0.10}{Пункт оценивается при неправильном знаке в выражении для $\vec{E}$.}

\QBlock{B2}{0.30}{Рассмотрим изолированный равномерно поляризованный вдоль оси $z$ эллипсоид.
Пусть напряжённость электростатического поля внутри эллипсоида равняется $\vec{E}$. Определите вектор поляризации эллипсоида $\vec{P}$. Ответ выразите через $\vec{E}$, $\varepsilon_0$ и $A$.}

\MBlock{0.10}{Записана связь величины $\rho\vec{l}$ с вектором поляризации $\vec{P}$:
$$\vec{P}=\rho\vec{l}{.}
$$}

\MBlock{0.10}{Выражение для $\vec{E}$ записано в виде:
$$\vec{E}=-\cfrac{A\vec{P}}{\varepsilon_0}{.}
$$}

\MBlock{0.10}{Получен ответ для $\vec{P}$:
$$\vec{P}=-\cfrac{\varepsilon_0\vec{E}}{A}{.}
$$}

\QBlock{B3}{0.30}{Определите вектор поляризации $\vec{P}$ эллипсоида. Ответ выразите через $\vec{E}_0$, $\varepsilon_0$ и $A$.
Определите также максимальную величину поверхностной плотности заряда $\sigma_{max}$ на поверхности эллипсоида. Ответ выразите через $P$.}

\MBlock{0.10}{Получен ответ для $\vec{P}$:
$$\vec{P}=\cfrac{\varepsilon_0\vec{E}_0}{A}{.}
$$}

\MBlock{0.20}{Получен ответ для $\sigma_{max}$:
$$\sigma_{max}=P{.}
$$}

\QBlock{B4}{0.40}{Выразите полную компоненту напряжённости электростатического поля $E_n$ на поверхности проводника через поверхностную плотность заряда $\sigma$ и $\varepsilon_0$.
Определите максимальную величину напряжённости $E_{max}$ электростатического поля на поверхности эллипсоида. Ответ выразите через $E_0$ и $A$.}

\MBlock{0.20}{Из теоремы Гаусса получено:
$$E_n=\cfrac{\sigma}{\varepsilon_0}{.}
$$}

\MBlock{0.10}{Указано, что величина $E_n$ максимальна при максимальном значении $\sigma$.}

\MBlock{0.10}{Получен ответ для $E_{max}$:
$$E_{max}=\cfrac{E_0}{A}{.}
$$}

\QBlock{B5}{0.40}{В листах ответов приведён рисунок, на котором над бесконечной проводящей плоскостью расположен точечный диполь, дипольный момент которого направлен перпендикулярно плоскости. В листах ответов приведите электростатическое изображение диполя в проводящей плоскости.
Используя полученный результат, приведите в листах ответов электростатическое изображение половины равномерно поляризованного эллипсоида вращения, контактирующего с проводящей плоскостью экваториальным сечением.}

\MBlock{0.10}{Указано, что изображение точечного диполя с дипольным моментом $\vec{p}$ представляет собой диполь с тем же дипольным моментом $\vec{p}$ и расположено на том же расстоянии от плоскости.}

\MBlock{0.10}{В листах ответов приведено электростатическое изображение точечного диполя.}

\MBlock{0.10}{Указано, что электростатическое изображение половины равномерно поляризованного эллипсоида дополняет его до целого.}

\MBlock{0.10}{В листах ответов приведено электростатическое изображение половины равномерно поляризованного эллипсоида.}

\QBlock{B6}{0.50}{Покажите, что выражение для максимальной напряжённости электростатического поля $E_{max}$ совпадает с выражением, найденным в пункте $\mathrm{B4}$, и найдите его численное значение. Достаточно ли величины напряжённости электростатического поля $E_0$ для пробоя воздуха в какой-либо точке пространства, если он происходит при напряжённости, равной $E_\text{пр}=30 \text{кВ}/\text{см}$?}

\MBlock{0.30}{Обосновано, что если половина эллипсоида поляризована равномерно, на плоской поверхности выполняются граничные условия.}

\MBlock{0.10}{Рассчитана величина $E_{max}$:
$$E_{max}\approx 7{.}4\cdot 10^{12} \text{В}/\text{м}{.}
$$}

\MBlock{0.10}{Сделан вывод, что величины $E_0$ достаточно для пробоя воздуха.}

\QBlock{C1}{0.30}{Принимая потенциал шара равным потенциалу на поверхности Земли, т.е нулю, определите величину заряда $q_0$ шара. Ответ выразите через $\varepsilon_0$, $E_0$, $R_0$ и $h$.
Влиянием электростатического поля шара на электростатическое поле Земли можно пренебречь. Влиянием электростатического поля зарядов, расположенных на проводе, можно пренебречь во всём пространстве.}

\MBlock{0.20}{Записано условие равенства потенциала шара нулю:
$$E_0h+\cfrac{q_0}{4\pi\varepsilon_0R_0}=0{.}
$$}

\MBlock{0.10}{Получен ответ для $q_0$:
$$q_0=-4\pi\varepsilon_0R_0hE_0{.}
$$}

\QBlock{C2}{0.40}{Определите величину силы тока $I$, перетекающего из атмосферы в шар, если проводимость воздуха во всей атмосфере можно принять равной $\sigma_0$. Ответ выразите через $E_0$, $R_0$, $h$, $\varepsilon_0$ и $\sigma_0$.}

\MBlock{0.20}{Записано выражение для силы тока $I$:
$$I=-4\pi{R}^2_0\sigma_0E{.}
$$}

\MBlock{0.10}{Получено выражение для $E$:
$$E=-\cfrac{E_0h}{R_0}{.}
$$}

\MBlock{0.10}{Получен ответ для силы тока $I$:
$$I=4\pi R_0h\sigma_0E_0{.}
$$}

\QBlock{C3}{0.30}{Запишите выражение для условия равенства нулю потенциала шара. В уравнение могут войти $E_0$, $h$, $q_0$, $q$, $R_0$ и $R$.}

\MBlock{0.20}{Потенциал электростатического поля сферических поверхностей в центре шара составляет:
$$\varphi_q+\varphi_{q_0}=\cfrac{1}{4\pi\varepsilon_0}\left(\cfrac{q_0}{R_0}+\cfrac{q}{R}\right){.}
$$}

\MBlock{0.10}{Получено правильное уравнение:
$$E_0h+\cfrac{1}{4\pi\varepsilon_0}\left(\cfrac{q_0}{R_0}+\cfrac{q}{R}\right)=0{.}
$$}

\QBlock{C4}{0.30}{Из условия равенства силы текущего тока $I$, пересекающего сферическую поверхность внутри и вне ионизированного слоя, получите уравнение, связывающее заряды $q_0$ и $q$. В уравнение также могут войти проводимости $\sigma_0$ и $\sigma$.}

\MBlock{0.10}{Записано выражение для $I_{in}$:
$$I_{in}=-\cfrac{q_0\sigma}{\varepsilon_0}{.}
$$}

\MBlock{0.10}{Записано выражение для $I_{out}$:
$$I_{out}=-\cfrac{(q_0+q)\sigma_0}{\varepsilon_0}{.}
$$}

\MBlock{0.10}{Получено правильное уравнение:
$$(q_0+q)\sigma_0=q_0\sigma{.}
$$}

\QBlock{C5}{0.20}{Определите величину силы тока $I$, перетекающего из атмосферы в шар в этом случае. Ответ выразите через $E_0$, $R_0$, $R$, $h$, $\varepsilon_0$, $\sigma_0$ и $\sigma$.}

\MBlock{0.10}{Получено выражение для $q_0$:
$$q_0=-\cfrac{4\pi\varepsilon_0R_0hE_0}{1+\cfrac{R_0}{R}\left(\cfrac{\sigma}{\sigma_0}-1\right)}{.}
$$}

\MBlock{0.10}{Получено выражение для $I$:
$$I=\cfrac{4\pi R_0h\sigma E_0}{1+\cfrac{R_0}{R}\left(\cfrac{\sigma}{\sigma_0}-1\right)}{.}
$$}

\QBlock{C6}{0.10}{Покажите, что при $R\approx R_0$ приближённое выражение для силы тока $I$ переходит в выражение, соответствующее отсутствию ионизированного слоя.}

\MBlock{0.10}{Выражение для $I$ приведено к виду:
$$I\approx 4\pi Rh\sigma_0E_0{.}
$$}

\QBlock{D1}{0.20}{Определите величину дрейфовой скорости $u$ движения электронов. Ответ выразите через $e$, $E$, $\lambda$, $m$ и $\overline{v}_\text{т}$.}

\MBlock{0.10}{Записано выражение для $a$:
$$a=\cfrac{eE}{m}{.}
$$}

\MBlock{0.10}{Получено выражение для $u$:
$$u=\cfrac{eE\lambda}{2m\overline{v}_\text{т}}{.}
$$}

\QBlock{D2}{0.40}{Из баланса энергии определите среднюю величину $\overline{\Delta{W}}$ потери кинетической энергии электрона при столкновении с молекулами воздуха. Ответ выразите через $e$, $E$, $\lambda$, $m$ и $\overline{W}$.}

\MBlock{0.20}{Записано выражение для средней мощность электростатического поля по перемещения одного электрона:
$$\overline{P}=eEu
$$}

\MBlock{0.10}{Записано выражение для $\Delta{W}$:
$$\overline{\Delta{W}}=eEu\tau{.}
$$}

\MBlock{0.10}{Получен ответ для $\overline{\Delta{W}}$:
$$\overline{\Delta{W}}=\cfrac{(eE\lambda)^2}{4W}{.}
$$
К данному пункту применяется PEP от пункта $\mathrm{D1}$.}

\QBlock{D3}{1.00}{Получите точное выражение для величины $\overline{\Delta{W}}/\overline{W}$. Ответ выразите через $m$ и $M$. Упростите ваш ответ с учётом $m\ll{M}$.
Если вы не смогли решить этот пункт - в дальнейшем считайте, что $\overline{\Delta{W}}/\overline{W}=m/M$.}

\MBlock{0.20}{Определена скорость молекулы воздуха в системе отсчёта центра масс:
$$v'=v_C=\cfrac{mv}{m+M}{.}
$$}

\MBlock{0.30}{Определена скорость молекулы воздуха в лабораторной системе отсчёта:
$$v_M=2v_C\cos(\varphi/2){.}
$$}

\MBlock{0.20}{Записано выражение для $\overline{\Delta{W}}$:
$$\Delta W=\cfrac{Mm^2v^2}{4\pi(m+M)^2}\int\limits_{0}^{\pi}(1+\cos\varphi)\cdot 2\pi\sin\varphi d\varphi{.}
$$}

\MBlock{0.10}{Вычислен интеграл для $\overline{\Delta{W}}$:
$$\overline{\Delta{W}}=\cfrac{Mm^2v^2}{(m+M)^2}{.}
$$}

\MBlock{0.10}{Получено выражение для $\overline{\Delta{W}}/\overline{W}$:
$$\cfrac{\overline{\Delta{W}}}{\overline{W}}=\cfrac{2Mm}{(M+m)^2}{.}
$$}

\MBlock{0.10}{Правильное приближение для $\overline{\Delta{W}}/\overline{W}$:
$$\cfrac{\overline{\Delta{W}}}{\overline{W}}\approx\cfrac{2m}{M}{.}
$$}

\QBlock{D4}{0.80}{Определите стационарное значение кинетической энергии теплового движения электронов $\overline{W}$ и скорости образования искрового канала $u$. Ответы выразите через $m$, $M$, $e$, $E$ и $\lambda$. Рассчитайте полученные значения.}

\MBlock{2 $\times$ 0.20}{Получен ответ для $\overline{W}$ (по $0{.}2$ балла за формулу и численное значение):
$$\overline{W}=\cfrac{1}{2}\sqrt{\cfrac{M}{2m}}eE\lambda\approx 4{.}2\cdot 10^{-17} \text{Дж}{.}
$$
К данному пункту применяется PEP от пунктов $\mathrm{D1}-\mathrm{D3}$.}

\MBlock{2 $\times$ 0.20}{Получен ответ для $u$ (по $0{.}2$ балла за формулу и численное значение):
$$u=\cfrac{1}{2}\sqrt{\cfrac{eE\lambda\sqrt{2}}{\sqrt{Mm}}}\approx 27{.}6 \text{км}/\text{с}{.}
$$
К данному пункту применяется PEP от пунктов $\mathrm{D1}-\mathrm{D3}$.}

\end{document}