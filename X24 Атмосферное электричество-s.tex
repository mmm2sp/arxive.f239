
%\documentstyle[12pt,russian,amsthm,amsmath,amssymb]{article}
\documentclass[a4paper,11pt,twoside]{article}
\usepackage[left=14mm, top=10mm, right=14mm, bottom=10mm, nohead, nofoot]{geometry}
\usepackage{amsmath, amsfonts, amssymb, amsthm} % стандартный набор AMS-пакетов для математ. текстов
\usepackage{mathtext}
\usepackage[utf8]{inputenc} % кодировка utf8
\usepackage[russian]{babel} % русский язык
\usepackage[pdftex]{graphicx} % графика (картинки)
\usepackage{tikz}
\usepackage{fancyhdr,pageslts} % настройка колонтитулов
\usepackage{enumitem} % работа со списками
\usepackage{multicol} % работа с таблицами
%\usepackage{pscyr} % красивый шрифт
\usepackage{pgfornament} % красивые рюшечки и вензеля
\usepackage{ltxgrid} % управление написанием текста в две колонки
\usepackage{lipsum} % стандартный текст
\usepackage{tcolorbox} % рамка вокруг текста
\tcbuselibrary{skins}
% ----------------------------------------

\newcommand\ProblemName{Атмосферное электричество}

\newcommand\Source{X24}

\newcommand\Type{Решение}

\newcommand\MyTextLeft{Президентский ФМЛ 239, г.~Санкт-Петербург}
\newcommand\MyTextRight{Использованы материалы сайта pho.rs}
\newcommand\MyHeading{Учебно-тренировочные сборы по физике}
% ----------------------------------------

% настройки полей
\geometry{
	left=12mm,
	top=21mm,
	right=15mm,
	bottom=26mm,
	marginparsep=0mm,
	marginparwidth=0mm,
	headheight=22pt,
	headsep=2mm,
	footskip=7mm}
% ----------------------------------------

% настройки колонтитулов
\pagestyle{fancy}

\fancypagestyle{style}{
	\fancyhf{}
	\fancyhead[L]{{\Large{\FancyTitle}}\\\vskip -5pt \dotfill}
	\fancyhead[R]{{\Large{\textbf{\Type}}}\\\vskip -5pt \dotfill}
	\renewcommand{\headrulewidth}{0pt}
	\renewcommand{\footrulewidth}{0pt}
	\fancyfoot[C]{\pgfornament[width=2em,anchor=south]{72}\hspace{1mm}
		{Страница \textbf{\thepage} из \textbf{\pageref{VeryLastPage}}}\hspace{2mm}
		\pgfornament[width=2em,symmetry=v,anchor=south]{72}\\ \vskip2mm
		{\small{\textit{\MyTextLeft\hfill\MyTextRight}}}}
}

\fancypagestyle{plain}{
	\fancyhf{}
	\renewcommand{\headrulewidth}{0pt}
	\renewcommand{\footrulewidth}{0pt}
	\fancyhead[C]{{\Large{\textit{\MyHeading}}}\\\vskip -5pt \dotfill}
	\fancyfoot[C]{\pgfornament[width=2em,anchor=south]{72}\hspace{1mm}
		{Страница \textbf{\thepage} из \textbf{\pageref{VeryLastPage}}}\hspace{2mm}
		\pgfornament[width=2em,symmetry=v,anchor=south]{72}\\ \vskip2mm
		{\small{\textit{\MyTextLeft\hfill\MyTextRight}}}}
}
% ----------------------------------------

% другие настройки
\pagenumbering{arabic}
\setlist[enumerate,itemize]{leftmargin=0pt,itemindent=2.7em,itemsep=0cm}
% ----------------------------------------

% собственные команды
\newcommand{\FancyTitle}{\textbf{\Source} --- \ProblemName}
\newcommand{\Title}{\begin{center}{\huge{\textbf{\Source} --- \ProblemName}}\end{center}}
\newcommand{\Chapter}[1]{\vskip5pt{\Large{\textbf{#1}}}\vskip5pt}
\newcommand{\QText}[1]{#1}
\newcommand{\QBlock}[3]{
	\begin{tcolorbox}[left=2mm,top=2mm,bottom=1mm,right=2mm,colback=white]
		\begin{tcolorbox}[enhanced,colframe=ProcessBlue,colback=ProcessBlue!30!white,
			frame style={opacity=0.7},interior style={opacity=1.0},
			nobeforeafter,tcbox raise base,shrink tight,extrude by=1.7mm,width=1.5cm]
			\textbf{#1\textsuperscript{#2}}
		\end{tcolorbox}\hspace{3mm}#3
	\end{tcolorbox}
}
\newcommand{\QPicture}[4]{
	\begin{figure}[H]
		\centering
		\includegraphics[width=0.35\linewidth]{#1}
		\caption{#3}
	\end{figure}
	
	#4
}
\newcommand{\ABlock}[1]{
	\vskip2mm
	\begin{tcolorbox}[enhanced,colframe=Magenta,colback=Magenta!15!white,
		frame style={opacity=0.5},interior style={opacity=1.0},
		nobeforeafter,tcbox raise base,shrink tight,extrude by=1.7mm,width=1.6cm]
		\textbf{Ответ:}
	\end{tcolorbox}\hspace{3mm}#1
}
\newcommand{\MBlock}[2]{
	\begin{tcolorbox}[enhanced,colframe=Yellow,colback=Yellow!15!white,
		frame style={opacity=0.5},interior style={opacity=1.0},
		nobeforeafter,tcbox raise base,shrink tight,extrude by=1.7mm,width=1.1cm]
		\textbf{#1}
	\end{tcolorbox}\hspace{3mm}#2
}
\newcommand{\MMBlock}[3]{
	\begin{tcolorbox}[enhanced,colframe=Yellow,colback=Yellow!15!white,
		frame style={opacity=0.5},interior style={opacity=1.0},
		nobeforeafter,tcbox raise base,shrink tight,extrude by=1.7mm,width=1.1cm]
		\textbf{#1}
	\end{tcolorbox}\hspace{3mm}
	\begin{tcolorbox}[enhanced,colframe=Orange,colback=Orange!15!white,
		frame style={opacity=0.5},interior style={opacity=1.0},
		nobeforeafter,tcbox raise base,shrink tight,extrude by=1.7mm,width=0.8cm]
		\textbf{#2}
	\end{tcolorbox}\hspace{3mm}#3
}
% ----------------------------------------


\begin{document}
	
	% настройки
	\pagestyle{style}\thispagestyle{plain}
	\Title
	% ----------------------------------------
	
	%\vskip5mm
	%\centering{\pgfornament[width=5cm,anchor=south]{89}}
	
	% смысловая часть


\QBlock{A1}{0.80}{Пусть проводящий шар радиусом $R$, несущий заряд $Q$, помещён в однородное электростатическое поле напряжённостью $\vec{E}_0$. Определите полную напряжённость $\vec{E}$ электрического поля в точке с радиус-вектором $\vec{r}$ относительно центра шара, находящейся вне шара. Ответ выразите через $Q$, $\vec{E}_0$, $R$, $\varepsilon_0$ и $\vec{r}$.}

\QText{Рассмотрим для начала незаряженный шар, помещённый в однородное поле напряжённостью $\vec{E}_0$. Внутри шара напряжённость электрического поля равняется нулю, а снаружи представляет собой суперпозицию электрического поля с напряжённостью $\vec{E}_0$ и поля электрического диполя с дипольным моментом $\vec{p}$ в его центре:
$$\vec{E}=\vec{E}_0+\cfrac{1}{4\pi\varepsilon_0}\left(\cfrac{3\bigl(\vec{p}\cdot\vec{r}\bigr)\vec{r}}{r^5}-\cfrac{\vec{p}}{r^3}\right){.}
$$
Величину дипольного момента можно определить из условия равенства нулю тангенциальной компоненты напряжённости электрического поля в любой точке поверхности шара:
$$\bigl[\vec{E}\times\vec{r}\bigr]=0\Rightarrow \vec{p}=4\pi\varepsilon_0R^3\vec{E}_0{.}
$$
Таким образом, в данной конструкции:
$$\vec{E}=\vec{E}_0\left(1-\cfrac{R^3}{r^3}\right)+\cfrac{3\bigl(\vec{E}_0\cdot\vec{r}\bigr)\vec{r}}{r^2}\cfrac{R^3}{r^3}{.}
$$
Если шар несёт заряд $Q$, если к найденному выше распределению добавить распределение этого заряда $Q$ равномерно по поверхности, то электрическое поле внутри шара останется равным нулю, как и его тангенциальная компонента на поверхности шара.
Прибавляя электрическое поле равномерно заряженного по поверхности шара, получим:}

\ABlock{$$\vec{E}=\vec{E}_0\left(1-\cfrac{R^3}{r^3}\right)+\cfrac{3\bigl(\vec{E}_0\cdot\vec{r}\bigr)\vec{r}}{r^2}\cfrac{R^3}{r^3}+\cfrac{Q\vec{r}}{4\pi\varepsilon_0r^3}{.}
$$}

\QBlock{A2}{0.40}{Пусть $\theta$ - угол между направлением вектора электростатического поля $\vec{E}_0$ и радиус-вектором $\vec{r}$ некоторой точки поверхности шара относительно его центра. 
Определите проекцию напряжённости электрического поля $E_n(\theta)$ на направление нормали. Ответ выразите через $Q$, $E_0$, $R$, $\varepsilon_0$ и $\theta$.}

\QText{При $R=r$ получим:
$$\vec{E}=\cfrac{Q\vec{r}}{4\pi\varepsilon_0R^3}+\cfrac{3\bigl(\vec{E}_0\cdot\vec{r}\bigr)\vec{r}}{R^2}{.}
$$
Проецируя на направление вектора нормали, получим:}

\ABlock{$$E_n=\cfrac{Q}{4\pi\varepsilon_0R^2}+3E_0\cos\theta
$$}

\QBlock{A3}{0.20}{При каких значениях заряда $Q$ шара величина $E_n$ может обращаться в ноль? Ответ выразите через $E_0$, $\varepsilon_0$ и $R$.
Далее во всех пунктах считайте, что заряд шара $Q$ попадает в найденный вами диапазон.}

\QText{Условие равенства нулю компоненты электростатического поля $E_n$ можно записать в виде:
$$Q=-12\pi\varepsilon_0R^2E_0\cos\theta{.}
$$
Поскольку $\cos\theta\in[-1{;}1]$, имеем:}

\ABlock{$$Q\in\bigl[-12\pi R^2\varepsilon_0E_0{;}12\pi R^2\varepsilon_0E_0\bigr]{.}
$$}

\QBlock{A4}{0.50}{Пусть заряд шара равен $Q$. При каком значении угла $\theta_0$ компонента напряжённости $E_n(\theta_0)$ обращается в ноль? Ответ выразите через $Q$, $E_0$, $\varepsilon_0$ и $R$.
Определите также, при каких значениях угла $\theta$ на поверхность шара попадают отрицательные ионы, а при каких - положительные. Ответы выразите через $\theta_0$.}

\QText{Величина $E_n$ обращается в ноль при угле $\theta_0$, равном:}

\ABlock{$$\theta_0=\arccos\left(-\cfrac{Q}{12\pi R^2\varepsilon_0E_0}\right){.}
$$}

\QBlock{A5}{1.00}{Определите полную производную по времени заряда шара $dQ/dt$. Ответ выразите через $Q$, $E_0$, $\varepsilon_0$, $R$, $\sigma_+$, $\sigma_-$ и, если понадобится, $\theta_0$.}

\QText{С учётом определённых в предыдущем пункте областей, величина $dQ/dt$ определяется выражением:
$$\cfrac{dQ}{dt}=\cfrac{dQ_-}{dt}+\cfrac{dQ_+}{dt}{,}
$$
где $dQ_+/dt$ и $dQ_-/dt$ обозначают силы тока, текущего на поверхность шара со стороны положительных и отрицательных ионов соответственно.
Для $dQ_+/dt$ и $dQ_-/dt$ имеем:
$$\cfrac{dQ_-}{dt}=-\sigma_-\int\limits_{0}^{\theta_0}E_n(\theta)\cdot2\pi R^2\sin\theta d\theta\qquad \cfrac{dQ_+}{dt}=-\sigma_+\int\limits_{\theta_0}^{\pi}E_n(\theta)\cdot 2\pi R^2\sin\theta d\theta{.}
$$
Определим $dQ_-/dt$:
$$\cfrac{dQ_-}{dt}=-\sigma_-\int\limits_{0}^{\theta_0}E_n(\theta)\cdot2\pi R^2\sin\theta d\theta=-\sigma_-\int\limits_{0}^{\theta_0}\left(\cfrac{Q}{2\varepsilon_0}+6\pi R^2R_0\cos\theta\right)\sin\theta d\theta{,}
$$
или же:
$$\cfrac{dQ_-}{dt}=-\sigma_-\left(\cfrac{Q(1-\cos\theta_0)}{2\varepsilon_0}+3\pi R^2E_0\sin^2\theta_0\right){.}
$$
Аналогично для $dQ_+/dt$:
$$\cfrac{dQ_+}{dt}=-\sigma_+\int\limits_{\theta_0}^{\pi}E_n(\theta)\cdot 2\pi R^2\sin\theta d\theta=-\sigma_+\int\limits_{\theta_0}^{\pi}\left(\cfrac{Q}{2\varepsilon_0}+6\pi R^2E_0\cos\theta\right)\sin\theta d\theta{,}
$$
или же:
$$\cfrac{dQ_+}{dt}=-\sigma_+\left(\cfrac{Q(1+\cos\theta_0)}{2\varepsilon_0}-3\pi R^2E_0\sin^2\theta_0\right){.}
$$
Складывая, находим:}

\ABlock{$$\cfrac{dQ}{dt}=-\cfrac{Q(\sigma_++\sigma_-)}{2\varepsilon_0}+(\sigma_+-\sigma_-)\left(3\pi R^2E_0\sin^2\theta_0-\cfrac{Q\cos\theta_0}{2\varepsilon_0}\right){.}
$$}

\QBlock{A6}{0.80}{Определите стационарный заряд шара $Q_0$, при котором он остаётся постоянным во времени. Ответ выразите через $E_0$, $\varepsilon_0$, $R$, $\sigma_+$ и $\sigma_-$.}

\ABlock{$$Q_0=12\pi R^2\varepsilon_0E_0\left(\cfrac{\sqrt{\sigma_+}-\sqrt{\sigma_-}}{\sqrt{\sigma_+}+\sqrt{\sigma_-}}\right){.}
$$}

\QBlock{A7}{0.60}{При малых значениях $\Delta{Q}$ зависимость $\dot{Q}(\Delta{Q})$ можно представить в виде:
$$\dot{Q}\approx A\Delta{Q}{.}
$$
Выразите $A$ через $\sigma_+$, $\sigma_-$ и $\varepsilon_0$.
Является ли найденное значение заряда $Q_0$ устойчивым? Ответ обоснуйте.}

\QText{Выражение для $\dot{Q}$ можно записать в следующей форме:
$$\dot{Q}=aQ^2-bQ+c{.}
$$
Раскладывая вблизи значения $Q=Q_0$, получим:
$$\dot{Q}=a(Q_0+\Delta{Q})^2-b(Q_0+\Delta{Q})+c\approx -(b-2aQ_0)\Delta{Q}{.}
$$
Подставим $a$, $b$ и $Q_0$:
$$\dot{Q}\approx -\left(\cfrac{\sigma_++\sigma_-}{2\varepsilon_0}-\cfrac{(\sigma_+-\sigma_-)Q_0}{24\pi R^2\varepsilon^2_0E_0}\right)\Delta{Q}{,}
$$
или же:
$$\dot{Q}=-\cfrac{1}{2\varepsilon_0}\left(\sigma_++\sigma_-\left(\sqrt{\sigma_+}-\sqrt{\sigma_-}\right)^2\right)\Delta{Q}=-\cfrac{\sqrt{\sigma_+\sigma_-}}{\varepsilon_0}\Delta{Q}{.}
$$
Таким образом:}

\ABlock{$$A=-\cfrac{\sqrt{\sigma_+\sigma_-}}{\varepsilon_0}{.}
$$}

\QBlock{A8}{0.30}{Получите зависимость отклонения заряда шара $\Delta{Q}$ от времени $t$. Ответ выразите через $\Delta{Q}_0$, $\varepsilon_0$, $\sigma_+$, $\sigma_-$ и $t$. Покажите также, что вид временной зависимости определяется только произведением $\sigma_+\sigma_-$.}

\QText{Для нахождения зависимости $\Delta{Q}(t)$ разделим переменные:
$$\cfrac{dQ}{\Delta{Q}}=-\cfrac{\sqrt{\sigma_+\sigma_-}dt}{\varepsilon_0}{.}
$$
Интегрируя, находим:}

\ABlock{$$\Delta{Q}(t)=\Delta{Q}_0\exp\left(-\cfrac{\sqrt{\sigma_+\sigma_-}t}{\varepsilon_0}\right){.}
$$}

\QBlock{B1}{0.50}{Определите вектор $\vec{E}$ напряжённости электростатического поля в области пересечения изолированных эллипсоидов. Ответ выразите через $\rho$, $\vec{l}$, $\varepsilon_0$ и $A$.}

\QText{Координата $z_+$ относительно центра положительно заряженного эллипсоида равна $z-l$, а координаты $y_+$ и $z_+$ равны координатам $y$ и $z$ соответственно.
Отсюда:
$$\vec{E}=\cfrac{\rho}{\varepsilon_0}\left(A(z_+-z)\vec{e}_z+B(y_+-y)\vec{e}_y+B(x_+-x)\vec{e}_x\right){.}
$$
Подставляя соотношения между координатами, получим:
$$\vec{E}=-\cfrac{\rho Al\vec{e}_z}{\varepsilon_0}{,}
$$
или же:}

\ABlock{$$\vec{E}=-\cfrac{\rho A\vec{l}}{\varepsilon_0}{.}
$$}

\QBlock{B2}{0.30}{Рассмотрим изолированный равномерно поляризованный вдоль оси $z$ эллипсоид.
Пусть напряжённость электростатического поля внутри эллипсоида равняется $\vec{E}$. Определите вектор поляризации эллипсоида $\vec{P}$. Ответ выразите через $\vec{E}$, $\varepsilon_0$ и $A$.}

\QText{В предельном переходе величина $\rho\vec{l}$ по определению представляет собой вектор поляризации $\vec{P}$. Таким образом:
$$\vec{E}=-\cfrac{A\vec{P}}{\varepsilon_0}{.}
$$
Таким образом:}

\ABlock{$$\vec{P}=-\cfrac{\varepsilon_0\vec{E}}{A}{.}
$$}

\QBlock{B3}{0.30}{Определите вектор поляризации $\vec{P}$ эллипсоида. Ответ выразите через $\vec{E}_0$, $\varepsilon_0$ и $A$.
Определите также максимальную величину поверхностной плотности заряда $\sigma_{max}$ на поверхности эллипсоида. Ответ выразите через $P$.}

\QText{При помещении проводящего эллипсоида в однородное электрическое поле, внутри эллипсоида напряжённость электрического поля должна оказаться равной нулю. Это реализуется, если напряжённость собственного электрического поля эллипсоида $\vec{E}=-\vec{E}_0$, при этом все граничные условия автоматически будут выполнены.
Таким образом:}

\QText{Величина поверхностной плотности заряда $\sigma$ определяется выражением:
$$\sigma=\vec{P}\cdot\vec{n}{,}
$$
где $\vec{n}$ - вектор нормали к поверхности проводника, направленный наружу.
Максимальное значение достигается, его вектор поляризации оказывается направлен вдоль вектора нормали, т.е в вершине эллипсоида.
Таким образом:}

\ABlock{$$\vec{P}=\cfrac{\varepsilon_0\vec{E}_0}{A}{.}
$$}

\QBlock{B4}{0.40}{Выразите полную компоненту напряжённости электростатического поля $E_n$ на поверхности проводника через поверхностную плотность заряда $\sigma$ и $\varepsilon_0$.
Определите максимальную величину напряжённости $E_{max}$ электростатического поля на поверхности эллипсоида. Ответ выразите через $E_0$ и $A$.}

\QText{Воспользуемся теоремой Гаусса для напряжённости электростатического поля. Выберем в качестве Гауссовой поверхности цилиндрическую, основания которой параллельны поверхности проводника. Основания расположены бесконечно близко к поверхности проводника, при этом одно из них расположено внутри проводника, а другое - снаружи.
Получим:
$$\oint_S\vec{E}\cdot d\vec{S}=(\vec{E}_n-0)dS=\cfrac{q}{\varepsilon_0}=\cfrac{\sigma dS}{\varepsilon_0}\Rightarrow E_n=\cfrac{\sigma}{\varepsilon_0}{.}
$$
Таким образом, величина электрического поля на поверхности проводника прямо пропорциональна величине поверхностной плотности зарядов, и, соответственно, максимальна в точке со значением $\sigma_{max}$:
$$E_{max}=\cfrac{\sigma_{max}}{\varepsilon_0}=\cfrac{P}{\varepsilon_0}{.}
$$
Подставляя $P$, находим:}

\ABlock{$$E_{max}=\cfrac{E_0}{A}{.}
$$}

\QBlock{B5}{0.40}{В листах ответов приведён рисунок, на котором над бесконечной проводящей плоскостью расположен точечный диполь, дипольный момент которого направлен перпендикулярно плоскости. В листах ответов приведите электростатическое изображение диполя в проводящей плоскости.
Используя полученный результат, приведите в листах ответов электростатическое изображение половины равномерно поляризованного эллипсоида вращения, контактирующего с проводящей плоскостью экваториальным сечением.}

\QText{Известно, что электростатическое изображение точечного заряда $q$ в плоскости расположено на том же расстоянии от плоскости, что и заряд $q$, и несёт заряд, равный $-q$. Воспользуемся этим и получим электростатическое изображение точечного диполя в плоскости.
Поскольку изображение более удалённого от плоскости заряда также является более удалённым и меняет знак, дипольный момент сохраняет не только величину, но и направление:}

\QPicture{X24 Атмосферное электричество-s_files/15124.jpeg}{"max-width:500px;px;display:block;margin-left: auto;margin-right: auto;"}{}{}

\QText{Используя данный факт, становится понятно, что электростатическое изображение половины равномерно поляризованного эллипсоида в проводящей плоскости дополняет его до целого:}

\QPicture{X24 Атмосферное электричество-s_files/15126.jpeg}{"max-width:500px;px;display:block;margin-left: auto;margin-right: auto;"}{}{}

\ABlock{}

\QBlock{B6}{0.50}{Покажите, что выражение для максимальной напряжённости электростатического поля $E_{max}$ совпадает с выражением, найденным в пункте $\mathrm{B4}$, и найдите его численное значение. Достаточно ли величины напряжённости электростатического поля $E_0$ для пробоя воздуха в какой-либо точке пространства, если он происходит при напряжённости, равной $E_\text{пр}=30 \text{кВ}/\text{см}$?}

\QText{Если электростатическое поле плоскости равняется $\vec{E}_0$, то такое распределение заряда, что исходная половина эллипсоида поляризована равномерно, а электростатическое изображение эллипсоида дополняет его до целого, удовлетворяет всем граничным условиям, поскольку напряжённость электростатического поля эллипсоида направлена перпендикулярно плоской поверхности в любой её точке. По теореме о единственности решения электростатических задач данное распределение заряда является единственным верным.
Таким образом:}

\ABlock{$$E_{max}=\left(\cfrac{a}{b}\right)^2\cfrac{E_0}{\ln\cfrac{2a}{b}-1}\approx 7{.}4\cdot 10^{12} \text{В}/\text{м}{.}
$$}

\QBlock{C1}{0.30}{Принимая потенциал шара равным потенциалу на поверхности Земли, т.е нулю, определите величину заряда $q_0$ шара. Ответ выразите через $\varepsilon_0$, $E_0$, $R_0$ и $h$.
Влиянием электростатического поля шара на электростатическое поле Земли можно пренебречь. Влиянием электростатического поля зарядов, расположенных на проводе, можно пренебречь во всём пространстве.}

\QText{Потенциал на поверхности шара должен быть равен нулю.
Поскольку влиянием провода можно пренебречь - получим:
$$\Delta\varphi=E_0h+\cfrac{kq}{R_0}=0{,}
$$
откуда:}

\ABlock{$$q=-4\pi\varepsilon_0R_0hE_0{.}
$$}

\QBlock{C2}{0.40}{Определите величину силы тока $I$, перетекающего из атмосферы в шар, если проводимость воздуха во всей атмосфере можно принять равной $\sigma_0$. Ответ выразите через $E_0$, $R_0$, $h$, $\varepsilon_0$ и $\sigma_0$.}

\QText{Воспользуемся законом Ома в дифференциальной форме:
$$\vec{j}=\sigma_0\vec{E}{.}
$$
Поскольку поверхность сферическая, имеем:
$$I=-4\pi R^2_0j=-4\pi R^2_0\sigma_0E{.}
$$
Для электрического поля имеем:
$$E=\cfrac{q}{4\pi\varepsilon_0R^2_0}=-\cfrac{E_0h}{R_0}{,}
$$
откуда:}

\ABlock{$$I=4\pi R_0h\sigma_0E_0{.}
$$}

\QBlock{C3}{0.30}{Запишите выражение для условия равенства нулю потенциала шара. В уравнение могут войти $E_0$, $h$, $q_0$, $q$, $R_0$ и $R$.}

\QText{Потенциал в центре шара складывается из потенциала электростатического поля Земли, а также потенциала сферических поверхностей:
$$\varphi(h)-\varphi(0)=E_0h+\cfrac{kq_0}{R_0}+\cfrac{kq}{R}=0{.}
$$
Таким образом:}

\ABlock{$$E_0h+\cfrac{1}{4\pi\varepsilon_0}\left(\cfrac{q_0}{R_0}+\cfrac{q}{R}\right)=0{.}
$$}

\QBlock{C4}{0.30}{Из условия равенства силы текущего тока $I$, пересекающего сферическую поверхность внутри и вне ионизированного слоя, получите уравнение, связывающее заряды $q_0$ и $q$. В уравнение также могут войти проводимости $\sigma_0$ и $\sigma$.}

\QText{В ионизированной области имеем:
$$I=-4\pi R^2\sigma E_{in}=-\cfrac{q_0\sigma}{\varepsilon_0}{.}
$$
Вне ионизированного слоя имеем:
$$I=-4\pi R^2\sigma_0E_{out}=-\cfrac{(q_0+q)\sigma_0}{\varepsilon_0}{.}
$$
Приравнивая, получим:}

\ABlock{$$(q_0+q)\sigma_0=q_0\sigma{.}
$$}

\QBlock{C5}{0.20}{Определите величину силы тока $I$, перетекающего из атмосферы в шар в этом случае. Ответ выразите через $E_0$, $R_0$, $R$, $h$, $\varepsilon_0$, $\sigma_0$ и $\sigma$.}

\QText{Определим величину заряда $q_0$:
$$E_0h=-\cfrac{1}{4\pi\varepsilon_0}\left(\cfrac{q_0}{R_0}+\cfrac{q}{R}\right)=-\cfrac{q_0}{4\pi\varepsilon_0R_0}\left(1+\cfrac{R_0}{R}\left(\cfrac{\sigma}{\sigma_0}-1\right)\right){,}
$$
откуда:
$$q_0=-\cfrac{4\pi\varepsilon_0R_0hE_0}{1+\cfrac{R_0}{R}\left(\cfrac{\sigma}{\sigma_0}-1\right)}{.}
$$
Подставляя в выражение для $I$, находим:}

\ABlock{$$I=\cfrac{4\pi R_0h\sigma E_0}{1+\cfrac{R_0}{R}\left(\cfrac{\sigma}{\sigma_0}-1\right)}{.}
$$}

\QBlock{C6}{0.10}{Покажите, что при $R\approx R_0$ приближённое выражение для силы тока $I$ переходит в выражение, соответствующее отсутствию ионизированного слоя.}

\QText{При $R\approx R_0$ и $\sigma\gg\sigma_0$ получим:}

\ABlock{$$I\approx 4\pi Rh\sigma_0E_0{.}
$$}

\QBlock{D1}{0.20}{Определите величину дрейфовой скорости $u$ движения электронов. Ответ выразите через $e$, $E$, $\lambda$, $m$ и $\overline{v}_\text{т}$.}

\QText{В рамках указанной модели:
$$u=\cfrac{a\tau}{2}{.}
$$
Для $a$ и $\tau$ имеем:
$$a=\cfrac{eE}{m}\qquad \tau=\cfrac{\lambda}{\overline{v}_\text{т}}{,}
$$
откуда:}

\ABlock{$$u=\cfrac{eE\lambda}{2m\overline{v}_\text{т}}{.}
$$}

\QBlock{D2}{0.40}{Из баланса энергии определите среднюю величину $\overline{\Delta{W}}$ потери кинетической энергии электрона при столкновении с молекулами воздуха. Ответ выразите через $e$, $E$, $\lambda$, $m$ и $\overline{W}$.}

\QText{В среднем за одно столкновение частица теряет энергию $\Delta{W}$, причём за время $\tau$. Тогда средняя мощность потерь энергии частицы должна компенсироваться работой электростатического поля:
$$\overline{\Delta{W}}=eEu\tau{.}
$$
Подставляя $u$ и $\tau$, получим:
$$\overline{\Delta{W}}=\cfrac{e^2E^2\lambda^2}{2m\overline{v}^2_\text{т}}{,}
$$
или же:}

\ABlock{$$\overline{\Delta{W}}=\cfrac{(eE\lambda)^2}{4\overline{W}}{.}
$$}

\QBlock{D3}{1.00}{Получите точное выражение для величины $\overline{\Delta{W}}/\overline{W}$. Ответ выразите через $m$ и $M$. Упростите ваш ответ с учётом $m\ll{M}$.
Если вы не смогли решить этот пункт - в дальнейшем считайте, что $\overline{\Delta{W}}/\overline{W}=m/M$.}

\QText{Воспользуемся методом векторных диаграмм скоростей. Поскольку изначально тяжёлый ион массой $M$ неподвижен - скорость движения центра масс системы $v_c=mv/(m+M)$ равна скорости движения тяжёлого иона относительно центра масс.
Тогда, если сразу после соударения скорость тяжёлого иона относительно центра масс образует угол $\varphi$ со скоростью налетающего электрона, в лабораторной системе отсчёта скорость тяжёлого иона составляет:
$$v_M=2v_C\cos(\varphi/2){,}
$$
откуда для кинетической энергии $K_M$ тяжёлого иона в лабораторной системе отсчёта находим:
$$K_M=2Mv^2_C\cos^2(\varphi/2)=\cfrac{Mm^2v^2(1+\cos\varphi)}{(m+M)^2}{.}
$$
Произведём усреднение, как указано в условии задачи:
$$\overline{\Delta W}=\cfrac{Mm^2v^2}{4\pi(m+M)^2}\int\limits_{0}^{\pi}(1+\cos\varphi)\cdot 2\pi\sin\varphi d\varphi=\cfrac{Mm^2v^2}{(m+M)^2}{.}
$$
Учитывая, что $W=mv^2/2$, находим:}

\ABlock{$$\cfrac{\overline{\Delta{W}}}{\overline{W}}=\cfrac{2Mm}{(M+m)^2}\approx\cfrac{2m}{M}{.}
$$}

\QBlock{D4}{0.80}{Определите стационарное значение кинетической энергии теплового движения электронов $\overline{W}$ и скорости образования искрового канала $u$. Ответы выразите через $m$, $M$, $e$, $E$ и $\lambda$. Рассчитайте полученные значения.}

\QText{Воспользуемся выражениями для $\Delta{W}$, полученными в пунктах $\mathrm{B2}$ и $\mathrm{B3}$:
$$\overline{\Delta{W}}=\cfrac{2m\overline{W}}{M}=\cfrac{(eE\lambda)^2}{4\overline{W}}{,}
$$
откуда:}

\QText{Воспользуемся выражением для $u$:
$$u=\cfrac{eE\lambda}{2m\overline{v}_\text{т}}=\cfrac{eE\lambda}{2\sqrt{2Wm}}{.}
$$
Подставляя выражение для $W$, находим:}

\ABlock{$$\overline{W}=\cfrac{1}{2}\sqrt{\cfrac{M}{2m}}eE\lambda\approx 4{.}2\cdot 10^{-17} \text{Дж}{.}
$$}

\end{document}