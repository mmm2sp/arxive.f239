
%\documentstyle[12pt,russian,amsthm,amsmath,amssymb]{article}
\documentclass[a4paper,11pt,twoside]{article}
\usepackage[left=14mm, top=10mm, right=14mm, bottom=10mm, nohead, nofoot]{geometry}
\usepackage{amsmath, amsfonts, amssymb, amsthm} % стандартный набор AMS-пакетов для математ. текстов
\usepackage{mathtext}
\usepackage[utf8]{inputenc} % кодировка utf8
\usepackage[russian]{babel} % русский язык
\usepackage[pdftex]{graphicx} % графика (картинки)
\usepackage{tikz}
\usepackage{fancyhdr,pageslts} % настройка колонтитулов
\usepackage{enumitem} % работа со списками
\usepackage{multicol} % работа с таблицами
%\usepackage{pscyr} % красивый шрифт
\usepackage{pgfornament} % красивые рюшечки и вензеля
\usepackage{ltxgrid} % управление написанием текста в две колонки
\usepackage{lipsum} % стандартный текст
\usepackage{tcolorbox} % рамка вокруг текста
\tcbuselibrary{skins}
% ----------------------------------------

\newcommand\ProblemName{Нецентральные движения шара}

\newcommand\Source{X24}

\newcommand\Type{Разбалловка}

\newcommand\MyTextLeft{Президентский ФМЛ 239, г.~Санкт-Петербург}
\newcommand\MyTextRight{Использованы материалы сайта pho.rs}
\newcommand\MyHeading{Учебно-тренировочные сборы по физике}
% ----------------------------------------

% настройки полей
\geometry{
	left=12mm,
	top=21mm,
	right=15mm,
	bottom=26mm,
	marginparsep=0mm,
	marginparwidth=0mm,
	headheight=22pt,
	headsep=2mm,
	footskip=7mm}
% ----------------------------------------

% настройки колонтитулов
\pagestyle{fancy}

\fancypagestyle{style}{
	\fancyhf{}
	\fancyhead[L]{{\Large{\FancyTitle}}\\\vskip -5pt \dotfill}
	\fancyhead[R]{{\Large{\textbf{\Type}}}\\\vskip -5pt \dotfill}
	\renewcommand{\headrulewidth}{0pt}
	\renewcommand{\footrulewidth}{0pt}
	\fancyfoot[C]{\pgfornament[width=2em,anchor=south]{72}\hspace{1mm}
		{Страница \textbf{\thepage} из \textbf{\pageref{VeryLastPage}}}\hspace{2mm}
		\pgfornament[width=2em,symmetry=v,anchor=south]{72}\\ \vskip2mm
		{\small{\textit{\MyTextLeft\hfill\MyTextRight}}}}
}

\fancypagestyle{plain}{
	\fancyhf{}
	\renewcommand{\headrulewidth}{0pt}
	\renewcommand{\footrulewidth}{0pt}
	\fancyhead[C]{{\Large{\textit{\MyHeading}}}\\\vskip -5pt \dotfill}
	\fancyfoot[C]{\pgfornament[width=2em,anchor=south]{72}\hspace{1mm}
		{Страница \textbf{\thepage} из \textbf{\pageref{VeryLastPage}}}\hspace{2mm}
		\pgfornament[width=2em,symmetry=v,anchor=south]{72}\\ \vskip2mm
		{\small{\textit{\MyTextLeft\hfill\MyTextRight}}}}
}
% ----------------------------------------

% другие настройки
\pagenumbering{arabic}
\setlist[enumerate,itemize]{leftmargin=0pt,itemindent=2.7em,itemsep=0cm}
% ----------------------------------------

% собственные команды
\newcommand{\FancyTitle}{\textbf{\Source} --- \ProblemName}
\newcommand{\Title}{\begin{center}{\huge{\textbf{\Source} --- \ProblemName}}\end{center}}
\newcommand{\Chapter}[1]{\vskip5pt{\Large{\textbf{#1}}}\vskip5pt}
\newcommand{\QText}[1]{#1}
\newcommand{\QBlock}[3]{
	\begin{tcolorbox}[left=2mm,top=2mm,bottom=1mm,right=2mm,colback=white]
		\begin{tcolorbox}[enhanced,colframe=ProcessBlue,colback=ProcessBlue!30!white,
			frame style={opacity=0.7},interior style={opacity=1.0},
			nobeforeafter,tcbox raise base,shrink tight,extrude by=1.7mm,width=1.5cm]
			\textbf{#1\textsuperscript{#2}}
		\end{tcolorbox}\hspace{3mm}#3
	\end{tcolorbox}
}
\newcommand{\QPicture}[4]{
	\begin{figure}[H]
		\centering
		\includegraphics[width=0.35\linewidth]{#1}
		\caption{#3}
	\end{figure}
	
	#4
}
\newcommand{\ABlock}[1]{
	\vskip2mm
	\begin{tcolorbox}[enhanced,colframe=Magenta,colback=Magenta!15!white,
		frame style={opacity=0.5},interior style={opacity=1.0},
		nobeforeafter,tcbox raise base,shrink tight,extrude by=1.7mm,width=1.6cm]
		\textbf{Ответ:}
	\end{tcolorbox}\hspace{3mm}#1
}
\newcommand{\MBlock}[2]{
	\begin{tcolorbox}[enhanced,colframe=Yellow,colback=Yellow!15!white,
		frame style={opacity=0.5},interior style={opacity=1.0},
		nobeforeafter,tcbox raise base,shrink tight,extrude by=1.7mm,width=1.1cm]
		\textbf{#1}
	\end{tcolorbox}\hspace{3mm}#2
}
\newcommand{\MMBlock}[3]{
	\begin{tcolorbox}[enhanced,colframe=Yellow,colback=Yellow!15!white,
		frame style={opacity=0.5},interior style={opacity=1.0},
		nobeforeafter,tcbox raise base,shrink tight,extrude by=1.7mm,width=1.1cm]
		\textbf{#1}
	\end{tcolorbox}\hspace{3mm}
	\begin{tcolorbox}[enhanced,colframe=Orange,colback=Orange!15!white,
		frame style={opacity=0.5},interior style={opacity=1.0},
		nobeforeafter,tcbox raise base,shrink tight,extrude by=1.7mm,width=0.8cm]
		\textbf{#2}
	\end{tcolorbox}\hspace{3mm}#3
}
% ----------------------------------------


\begin{document}
	
	% настройки
	\pagestyle{style}\thispagestyle{plain}
	\Title
	% ----------------------------------------
	
	%\vskip5mm
	%\centering{\pgfornament[width=5cm,anchor=south]{89}}
	
	% смысловая часть


\QBlock{A1}{0.40}{Выразите компоненту скорости $\vec{u}_{A}$ точки $A$ через компоненту скорости $\vec{u}_C$ центра шара, его угловую скорость $\vec{\omega}$, а также радиус-вектор $\vec{r}$ в произвольный момент.
Получите также производную по времени $\dot{\vec{u}}_A$ вектора $\vec{u}_A$. Ответ выразите через $\dot{\vec{u}}_C$, $\dot{\vec{\omega}}$ и $\vec{r}$.}

\MBlock{0.10}{Записано выражение для полной скорости точки $A$:
$$\vec{v}_A=\vec{v}_C+\bigl[\vec{\omega}\times\vec{r}\bigr]{.}
$$}

\MBlock{0.10}{Получен правильный ответ для $\vec{u}_A$:
$$\vec{u}_A=\vec{u}_C+\bigl[\vec{\omega}\times\vec{r}\bigr]{.}
$$}

\MBlock{0.10}{Указано или используется, что вектор $\vec{r}$ остаётся постоянным в процессе всего соударения.}

\MBlock{0.10}{Получен правильный ответ для $\dot{\vec{u}}_C$:
$$\dot{\vec{u}}_A=\dot{\vec{u}}_C+\bigl[\dot{\vec{\omega}}\times\vec{r}\bigr]{.}
$$}

\QBlock{A2}{0.60}{Определите силу трения $\vec{F}_0$, действующую на шар в начальный момент контакта со стеной. Ответ выразите через $\vec{e}_x$, $\vec{e}_z$, $\alpha$, $\mu$ и силу нормальной реакции стены $N_0$ в начальный момент.}

\MBlock{0.10}{Правильно применён закон Кулона-Амонтона:
$$\vec{F}=-\mu N\cfrac{\vec{u}_A}{u_A}{.}
$$}

\MBlock{2 $\times$ 0.10}{Записаны выражения для компонент скорости $u_{Ax(0)}$ и $u_{Az(0)}$ точки $A$:
$$u_{Ax(0)}=u_{Cx(0)}\qquad u_{Az(0)}=-\omega_{y(0)}r{.}
$$}

\MBlock{2 $\times$ 0.10}{Определены компоненты скорости $u_{Ax(0)}$ и $u_{Az(0)}$ точки $A$ в начальный момент:
$$u_{Ax(0)}=v\sin\alpha\qquad u_{Az(0)}=-v\cos\alpha{.}
$$}

\MBlock{0.10}{Получено выражение для силы трения $\vec{F}$:
$$\vec{F}_0=\mu N_0\left(\vec{e}_z\cos\alpha-\vec{e}_x\sin\alpha\right){.}
$$}

\QBlock{A3}{1.00}{Докажите, что производная по времени $\dot{\vec{u}}_A$ компоненты скорости $\vec{u}_A$ связана с силой трения $\vec{F}$ соотношением:
$$\dot{\vec{u}}_A=\cfrac{7\vec{F}}{2m}{.}
$$
Данный факт можно использовать далее, даже если вы не смогли его доказать.}

\MBlock{0.10}{Записана теорема о движении центра масс для шара:
$$m\dot{\vec{u}}_C=\vec{F}{.}
$$}

\MBlock{0.20}{Указано, что вектор момента импульса шара относительно его центра определяется выражением:
$$\vec{L}_C=I_C\vec{\omega}{.}
$$
Пункт оценивается, даже если $I_C\neq 2mr^2/5$.}

\MBlock{0.20}{Записано уравнение динамики вращательного движения относительно центра шара:
$$I_C\dot{\vec{\omega}}=\bigl[\vec{r}\times\vec{F}\bigr]{.}
$$
Пункт оценивается, даже если $I_C\neq 2mr^2/5$.}

\MBlock{0.30}{Получено выражение для силы трения $\vec{F}$:
$$\vec{F}=\cfrac{I_C\bigl[\dot{\vec{\omega}}\times\vec{r}\bigr]}{r^2}{.}
$$
Альтернативно: Выражения для $\dot{\vec{u}}_C$ и $\dot{\vec{\omega}}$ подставлены в уравнение, полученное для $\dot{\vec{u}}_A$:
$$\dot{\vec{u}}_A=\cfrac{\vec{F}}{m}+\cfrac{\bigl[\bigl[\vec{r}\times\vec{F}\bigr]\times\vec{r}\bigr]}{I_C}{.}
$$
Пункт оценивается, даже если $I_C\neq 2mr^2/5$.}

\MBlock{0.20}{Выражение приведено к правильному виду:
$$\dot{\vec{u}}_A=\cfrac{\vec{F}}{m}\left(1+\cfrac{mr^2}{I_C}\right)=\cfrac{7\vec{F}}{2m}{.}
$$}

\QBlock{A4}{0.50}{Определите компоненту скорости $\vec{u}_{A\text{к}}$ сразу после соударения, считая, что шар проскальзывает по стенке в течение всего времени соударения. Ответ выразите через $v$, $\alpha$, $\mu$, $\vec{e}_x$ и $\vec{e}_z$. 
При каком максимальном значении коэффициента трения $\mu_{max}$ проскальзывание не прекращается в течение всего времени соударения? Ответ выразите через $\alpha$.}

\MBlock{0.10}{Сделан вывод, что в процессе соударения направление компоненты скорости $\vec{u}_A$ остаётся постоянным.}

\MBlock{0.10}{Записано выражение для компоненты скорости $u_A$ при условии постоянного проскальзывания:
$$u_A=u_{A(0)}-\cfrac{7\mu}{2}\int\limits_{0}^tNdt{.}
$$}

\MBlock{0.10}{Определено значение импульса силы реакции $N$:
$$\int\limits_0^tNdt=2mv\cos\alpha{.}
$$}

\MBlock{0.10}{Получено выражение для конечной скорости точки $A$ при условии постоянного проскальзывания:
$$u_{A}=u_{A(0)}-7\mu v\cos\alpha{.}
$$}

\MBlock{0.10}{Получено выражение для $\mu_{max}$:
$$\mu_{max}=\cfrac{1}{7\cos\alpha}{.}
$$
Пункт оценивается только при наличии полного балла за пункт $\mathrm{A2}$.}

\QBlock{A5}{0.60}{При $\mu{<}\mu_{max}$ определите скорость центра шара $\vec{v}_{C\text{к}}$, а также под каким углом $\beta$ к горизонту она направлена  сразу после соударения. Ответы выразите через $v$, $\alpha$, $\mu$, $\vec{e}_x$, $\vec{e}_y$ и $\vec{e}_z$.}

\MBlock{0.20}{Записано выражение для угла $\beta$:
$$\beta=\operatorname{arctg}\cfrac{v_{Cz}}{\sqrt{v^2_{Cx}+v^2_{Cy}}}{.}
$$}

\MBlock{0.20}{Записано выражение для конечной компоненты скорости $\vec{u}_C$:
$$\vec{u}_C=v\sin\alpha\vec{e}_x+\cfrac{2(\vec{u}_A-\vec{u}_A(0))}{7}
$$}

\MBlock{0.10}{Получено выражение для конечной компоненты скорости центра шара $\vec{u}_C$:
$$\vec{u}_C=v\sin\alpha(1-2\mu\cos\alpha)\vec{e}_x+v\cos\alpha\vec{e}_y+2\mu v\cos^2\alpha\vec{e}_z{.}
$$}

\MBlock{0.10}{Получен правильный ответ для $\beta$:
$$\beta=\operatorname{arctg}\cfrac{2\mu\cos^2\alpha}{\sqrt{\cos^2\alpha+\sin^2\alpha(1-2\mu\cos\alpha)^2}}{.}
$$}

\QBlock{A6}{0.40}{При $\mu{<}\mu_{max}$ определите координаты $x_C$, $y_C$ центра шара в момент его падения на стол. Ответы выразите через $v$, $g$, $\mu$ и $\alpha$.}

\MBlock{2 $\times$ 0.10}{Записаны выражения для координат точки падения шара (по $0{.}1$ балла за каждое):
$$x_C=\cfrac{2v_{Cx}v_{Cz}}{g}\qquad y_C=\cfrac{2v_{Cy}v_{Cz}}{g}{.}
$$}

\MBlock{0.10}{Определена координата $x_C$ точки падения шара:
$$x_C=\cfrac{4\mu v^2\cos^2\alpha\sin\alpha(1-2\mu\cos\alpha)}{g}{.}
$$}

\MBlock{0.10}{Определена координата $y_C$ точки падения шара:
$$y_C=\cfrac{4\mu v^2\cos^3\alpha}{g}{.}
$$}

\QBlock{A7}{1.00}{При произвольных значениях $\mu$ определите количество теплоты $Q$, выделившееся в процессе соударения шара со стенкой. Ответ выразите через $m$, $v$, $\mu$ и $\alpha$. </p><p>\textit{Примечание: }явное вычисление работы силы трения существенно упростит решение задачи.}

\MBlock{0.10}{Записано выражение для мощности силы трения:
$$P_\text{тр}=\vec{F}\cdot\vec{u}_A{.}
$$}

\MBlock{0.10}{Получено выражение для количества выделившейся теплоты:
$$Q=-\int\limits_{0}^tP_\text{тр}dt{.}
$$}

\MBlock{0.20}{Записано выражение для элементарного импульса силы трения:
$$\vec{F}dt=\cfrac{2md\vec{u}_A}{7}{.}
$$}

\MBlock{0.20}{Выражение для количества выделившейся теплоты $Q$ приведено к виду:
$$Q=\int\limits_{\vec{u}_A}^{\vec{u}_A(0)}\cfrac{2m\vec{u}_A\cdot d\vec{u}_A}{7}{.}
$$}

\MBlock{0.20}{Получено выражение для количества выделившейся теплоты $Q$:
$$Q=\cfrac{m(u^2_A(0)-u^2_A)}{7}{.}
$$}

\MBlock{2 $\times$ 0.10}{Получен ответ для количества выделившейся теплоты $Q$ (по $0{.}1$ балла за каждый случай):
$$Q=\begin{cases}
mv^2(2\mu\cos\alpha-7\mu^2\cos^2\alpha)\quad\text{при}\quad \mu\leq\cfrac{1}{7\cos\alpha}\\
\cfrac{mv^2}{7}\quad\text{при}\quad \mu\geq\cfrac{1}{7\cos\alpha}
\end{cases}
$$}

\QBlock{B1}{0.20}{Определите компоненты вектора скорости центра шара $v_\varphi$ и $v_z$ в цилиндрической системе координат. Ответы выразите через $r$, $\dot\varphi$ и $\dot{z}$.}

\MBlock{0.10}{Получен ответ для $v_\varphi$:
$$v_\varphi=r\dot{\varphi}{.}
$$}

\MBlock{0.10}{Получен ответ для $v_z$:
$$v_z=\dot{z}{.}
$$}

\QBlock{B2}{0.30}{Определите компоненты вектора ускорения центра шара $a_r$, $a_\varphi$ и $a_z$ в цилиндрической системе координат. Ответы выразите через $r$, $v_\varphi$, $\dot{v}_\varphi$ и $\dot{v}_z$.}

\MBlock{0.10}{Получен ответ для $a_\varphi$:
$$a_\varphi=\dot{v}_\varphi{.}
$$}

\MBlock{0.10}{Получен ответ для $a_z$:
$$a_z=\dot{v}_z{.}
$$}

\MBlock{0.10}{Получен ответ для $a_r$:
$$a_r=-\cfrac{v^2_\varphi}{r}{.}
$$}

\QBlock{B3}{0.40}{Из условия отсутствия проскальзывания определите компоненты угловой скорости шара $\omega_\varphi$ и $\omega_z$ в цилиндрической системе координат. Ответы выразите через $r$, $v_\varphi$ и $v_z$.}

\MBlock{0.20}{Получен ответ для $\omega_z$:
$$\omega_z=\cfrac{v_\varphi}{r}{.}
$$}

\MBlock{0.20}{Получен ответ для $\omega_\varphi$:
$$\omega_\varphi=-\cfrac{v_z}{r}{.}
$$}

\QBlock{C1}{0.80}{Определите компоненту силу трения $F_\varphi(\varphi)$, действующую на шар, а также компоненту ускорения $a_\varphi(\varphi)$ его центра. Ответы выразите через массу шара $m$, $g$ и $\varphi$.}

\MBlock{}{\textbf{Применимость уравнения моментов относительно оси, проходящей вдоль края стола, треб дополнительного обоснования. Если обоснование отсутствует, все ответы данного пункта, полученные с помощью данного уравнения, оцениваются в 0 баллов}}

\MBlock{0.10}{Записана теорема о движении центра масс в проекции на ось $\varphi$:
$$ma_\varphi=mg\sin\varphi+F_\varphi{.}
$$}

\MBlock{0.30}{Записано уравнение динамики вращательного движения шара относительно оси $z$:
$$I\dot{\omega}_z=-F_\varphi r{.}
$$}

\MBlock{0.20}{Для силы трения $F_\varphi$ получено:
$$F_\varphi=-\cfrac{2mg\sin\varphi}{7}{.}
$$}

\MBlock{0.20}{Для компоненты ускорения $a_\varphi$ центра шара получено:
$$a_\varphi=\cfrac{5g\sin\varphi}{7}{.}
$$}

\QBlock{C2}{0.50}{Получите зависимость $v_\varphi(\varphi)$. Ответ выразите через $v$, $g$, $r$, $\alpha$ и $\varphi$.}

\MBlock{0.30}{Получено выражение:
$$a_\varphi v_\varphi=\cfrac{5gr\sin\varphi\dot{\varphi}}{7}{.}
$$}

\MBlock{0.20}{Получена зависимость $v_\varphi(\varphi)$:
$$v_\varphi=\sqrt{v^2\cos^2\alpha+\cfrac{10gr(1-\cos\varphi)}{7}}{.}
$$}

\QBlock{C3}{0.20}{При каком условии шар не отрывается от стола в момент, когда нижняя точка шара достигает его края? Запишите это условие через $v$, $g$, $r$ и $\alpha$. Во всех дальнейших пунктах считайте, что это условие выполняется.}

\MBlock{0.10}{Записано выражение для силы нормальной реакции в начальный момент:
$$N=mg-\cfrac{mv^2\cos^2\alpha}{r}{.}
$$}

\MBlock{0.10}{Получено ограничение для $v$:
$$v\cos\alpha\leq\sqrt{gr}{.}
$$}

\QBlock{C4}{0.50}{Определите угол $\varphi_1$ в момент отрыва шара от стола. Ответ выразите через $v$, $g$, $r$ и $\alpha$.}

\MBlock{0.30}{Определена сила реакции $N$ в произвольный момент:
$$N=mg\cos\varphi-\cfrac{mv^2_\varphi}{r}{.}
$$}

\MBlock{0.20}{Получен ответ для $\varphi_1$:
$$\varphi_1=\arccos\left(\cfrac{10}{17}+\cfrac{7v^2\cos^2\alpha}{17gr}\right){.}
$$}

\QBlock{D1}{0.50}{Выразите кинетическую энергию шара $E_k$ через $m$, $v_\varphi$, $v_z$, $\omega_r$ и $r$.}

\MBlock{0.20}{Записана теорема Кёнига:
$$E_k=\cfrac{mv^2_C}{2}+\cfrac{I\omega^2}{2}{.}
$$}

\MBlock{0.30}{Получен ответ для $E_k$:
$$E_k=\cfrac{7m(v^2_\varphi+v^2_z)}{10}+\cfrac{m\omega^2_rr^2}{5}{.}
$$}

\QBlock{D2}{0.60}{Запишите для шара закон сохранения механической энергии. Комбинируя его с результатом пункта $\mathrm{C2}$, покажите, что величины $\omega_r$ и $v_z$ связаны соотношением:
$$1=\cfrac{\omega^2_r}{A^2}+\cfrac{v^2_z}{B^2}{,}
$$
где $A{,}B>0$ - постоянные коэффициенты.
Определите $A$ и $B$. Ответы выразите через $v$, $r$ и $\alpha$.}

\MBlock{0.10}{Записан закон сохранения механической энергии:
$$E_k=E_{k(0)}+mgr(1-\cos\varphi){.}
$$}

\MBlock{0.10}{Правильное выражение для начальной кинетической энергии шара:
$$E_{k(0)}=\cfrac{7mv^2}{10}{.}
$$}

\MBlock{0.20}{Получено соотношение, эквивалентное написанному ниже:
$$v^2\sin^2\alpha=v^2_z+\cfrac{2\omega^2_rr^2}{7}{.}
$$}

\MBlock{2 $\times$ 0.10}{Получены ответы для $A$ и $B$ (по $0{.}1$ балла за каждый):
$$A=\sqrt{\cfrac{7}{2}}\cfrac{v\sin\alpha}{r}\qquad B=v\sin\alpha
$$}

\QBlock{D3}{0.50}{Вектор углового ускорения $\vec{\varepsilon}$ шара может быть представлен в виде:
$$\vec{\varepsilon}=\varepsilon_r\vec{e}_r+\varepsilon_\varphi\vec{e}_\varphi+\varepsilon_z\vec{e}_z{.}
$$
Используя уравнение динамики вращательного движения относительно центра шара, покажите, что $\varepsilon_r=0$. Используя полученное равенство, выразите $\dot{\omega}_r$ через $\dot{\varphi}$, $v_z$ и $r$.}

\MBlock{0.20}{Записано уравнение динамики вращательного движения относительно центра шара:
$$I\vec{\varepsilon}=\bigl[\vec{r}\times\vec{F}\bigr]{.}
$$}

\MBlock{0.10}{Указано, что $\varepsilon_r=0$, поскольку $\vec{\varepsilon}\perp\vec{r}$.}

\MBlock{0.10}{Использовано выражение для компоненты производной $\bigl(\dot{\vec{\omega}}\bigr)_r$ в цилиндрической системе координат и получено:
$$\bigl(\dot{\vec{\omega}}\bigr)_r=\dot{\omega}_r-\dot{\varphi}\omega_\varphi{.}
$$}

\MBlock{0.10}{Выражение приведено к нужному виду:
$$\dot{\omega}_r=-\cfrac{\dot{\varphi}v_z}{r}{.}
$$}

\QBlock{D4}{1.20}{Комбинируя результаты пунктов $\mathrm{D2}$ и $\mathrm{D3}$, получите зависимости $\omega_r(\varphi)$ и $v_z(\varphi)$. Ответы выразите через $v$, $\alpha$, $r$ и $\varphi$.}

\MBlock{0.20}{Комбинация пунктов $\mathrm{D2}$ и $\mathrm{D3}$ приведена к виду:
$$1=\cfrac{\omega^2_r}{A^2}+\cfrac{r^2\dot{\omega}^2_r}{B^2\dot{\varphi}^2}{.}
$$}

\MBlock{0.20}{Проведено разделение переменных:
$$d\varphi=-\cfrac{r}{B}\cfrac{d\omega_r}{\sqrt{1-\cfrac{\omega^2_r}{A^2}}}{.}
$$
Балл ставится даже при неправильном знаке.}

\MBlock{0.40}{Правильно проведено интегрирование и получено:
$$\omega_r(\varphi)=-A\sin\left(\cfrac{B\varphi}{Ar}\right)
$$}

\MBlock{2 $\times$ 0.10}{Получен ответ для $\omega_r(\varphi)$ (по $0{.}1$ балла за знак и верные коэффициенты):
$$\omega_r=-\sqrt{\cfrac{7}{2}}\cfrac{v\sin\alpha}{r}\sin\left(\sqrt{\cfrac{2}{7}}\varphi\right){.}
$$}

\MBlock{0.20}{Получен ответ для $v_z(\varphi)$:
$$v_z(\varphi)=v\sin\alpha\cos\left(\sqrt{\cfrac{2}{7}}\varphi\right){.}
$$}

\QBlock{D5}{0.80}{Рассмотрим предельный переход, когда угол $\alpha\to\pi/2$, т.е движение шара до контакта с краем стола происходит практически параллельно ему.
Определите проекцию скорости $v_z$ центра шара, а также проекцию его угловой скорости $\omega_y$ на ось $y$, направленную вертикально вниз, в момент отрыва шара от стола. Ответы выразите через $v$ и $r$. Все численные коэффициенты в ответе должны быть аналитическими, а не приближёнными!}

\MBlock{0.10}{Определено значение угла $\varphi_1$ для указанных начальных условий:
$$\varphi_1=\arccos\left(\cfrac{10}{17}\right){.}
$$}

\MBlock{0.10}{Получен ответ для $v_z(\varphi_1)$:
$$v_z=v\cos\left(\sqrt{\cfrac{2}{7}}\arccos\left(\cfrac{10}{17}\right)\right){.}
$$}

\MBlock{0.20}{Для проекции угловой скорости $\omega_y$ записано:
$$\omega_y=-\omega_r\cos\varphi_1+\omega_\varphi\sin\varphi_1{.}
$$}

\MBlock{0.20}{После подстановки $\omega_r$ и $\omega_\varphi$ получено:
$$\omega_y=\cfrac{v_0}{r}\left(\sqrt{\cfrac{7}{2}}\cos\varphi_1\sin\left(\sqrt{\cfrac{2}{7}}\varphi_1\right)-\sin\varphi_1\cos\left(\sqrt{\cfrac{2}{7}}\varphi_1\right)\right){.}
$$}

\MBlock{0.20}{Получен ответ для $\omega_y$:
$$\omega_y=\cfrac{v_0}{r}\left(\sqrt{\cfrac{7}{2}}\cfrac{10}{17}\sin\left(\sqrt{\cfrac{2}{7}}\arccos\left(\cfrac{10}{17}\right)\right)-\cfrac{\sqrt{189}}{17}\cos\left(\sqrt{\cfrac{2}{7}}\arccos\left(\cfrac{10}{17}\right)\right)\right){.}
$$}

\end{document}