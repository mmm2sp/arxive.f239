
%This file is part of Get pho.rs!

%Get pho.rs! is free software: you can redistribute it and/or modify it under the terms of the GNU General Public License as published by the Free Software Foundation, either version 3 of the License, or (at your option) any later version.

%Get pho.rs! is distributed in the hope that it will be useful, but WITHOUT ANY WARRANTY; without even the implied warranty of MERCHANTABILITY or FITNESS FOR A PARTICULAR PURPOSE. See the GNU General Public License for more details.

%You should have received a copy of the GNU General Public License along with Foobar. If not, see <https://www.gnu.org/licenses/>.

%\documentstyle[12pt,russian,amsthm,amsmath,amssymb]{article}
\documentclass[a4paper,11pt,twoside]{article}
\usepackage[left=14mm, top=10mm, right=14mm, bottom=10mm, nohead, nofoot]{geometry}
\usepackage{amsmath, amsfonts, amssymb, amsthm} % стандартный набор AMS-пакетов для математ. текстов
\usepackage{mathtext}
\usepackage[utf8]{inputenc} % кодировка utf8
\usepackage[russian]{babel} % русский язык
\usepackage[pdftex,dvipsnames]{xcolor} % работа с цветами
\usepackage[pdftex]{graphicx} % графика (картинки)
\usepackage{tikz} % рисунки
\usepackage{fancyhdr,pageslts} % настройка колонтитулов
\usepackage{enumitem} % работа со списками
\usepackage{multicol} % работа с таблицами
%\usepackage{pscyr} % красивый шрифт
\usepackage{pgfornament} % красивые рюшечки и вензеля
\usepackage{ltxgrid} % управление написанием текста в две колонки
\usepackage{lipsum} % стандартный текст
\usepackage{tcolorbox} % рамка вокруг текста
\usepackage{float} % для корректного размещения картинок
\tcbuselibrary{skins}
% ----------------------------------------

\newcommand\ProblemName{Снижение орбиты МКС}

\newcommand\Source{A23}

\newcommand\Type{Решение}

% настройки полей
\geometry{
	left=12mm,
	top=21mm,
	right=15mm,
	bottom=26mm,
	marginparsep=0mm,
	marginparwidth=0mm,
	headheight=22pt,
	headsep=2mm,
	footskip=7mm}
% ----------------------------------------

% настройки колонтитулов
\pagestyle{fancy}

\fancypagestyle{style}{
	\fancyhf{}
	\fancyhead[L]{{\Large{\FancyTitle}}\\\vskip -5pt \dotfill}
	\fancyhead[R]{{\Large{\textbf{\Type}}}\\\vskip -5pt \dotfill}
	\renewcommand{\headrulewidth}{0pt}
	\renewcommand{\footrulewidth}{0pt}
	\fancyfoot[C]{\pgfornament[width=2em,anchor=south]{72}\hspace{1mm}
		{Страница \textbf{\thepage} из \textbf{\pageref{VeryLastPage}}}\hspace{2mm}
		\pgfornament[width=2em,symmetry=v,anchor=south]{72}\\ \vskip2mm
		{\small{\textit{Условие собрано и подготовлено в Президентском ФМЛ №239 г.~Санкт-Петербурга}}}}
}

\fancypagestyle{plain}{
	\fancyhf{}
	\renewcommand{\headrulewidth}{0pt}
	\renewcommand{\footrulewidth}{0pt}
	\fancyhead[C]{{\Large{\textit{Учебно-тренировочные сборы к X23}}}\\\vskip -5pt \dotfill}
	\fancyfoot[C]{\pgfornament[width=2em,anchor=south]{72}\hspace{1mm}
		{Страница \textbf{\thepage} из \textbf{\pageref{VeryLastPage}}}\hspace{2mm}
		\pgfornament[width=2em,symmetry=v,anchor=south]{72}\\ \vskip2mm
		{\small{\textit{Условие собрано и подготовлено в Президентском ФМЛ №239 г.~Санкт-Петербурга}}}}
}
% ----------------------------------------

% другие настройки
\pagenumbering{arabic}
\setlist[enumerate,itemize]{leftmargin=0pt,itemindent=2.7em,itemsep=0cm}
% ----------------------------------------

% собственные команды
\newcommand{\FancyTitle}{\textbf{\Source} --- \ProblemName}
\newcommand{\Title}{\begin{center}{\huge{\textbf{\Source} --- \ProblemName}}\end{center}}
\newcommand{\Chapter}[1]{\vskip5pt{\Large{\textbf{#1}}}\vskip5pt}
\newcommand{\QText}[1]{#1}
\newcommand{\QBlock}[3]{
	\begin{tcolorbox}[left=4mm,top=3mm,bottom=2mm,right=4mm,colback=white]
		\begin{tcolorbox}[enhanced,colframe=blue,colback=blue!10!white,
			frame style={opacity=0.3},interior style={opacity=1.0},
			nobeforeafter,tcbox raise base,shrink tight,extrude by=1.7mm,width=1.5cm]
			\textbf{#1\textsuperscript{#2}}
		\end{tcolorbox}\hspace{3mm}#3
	\end{tcolorbox}
}
\newcommand{\QPicture}[4]{\QText{#4}  \includegraphics{#1}}
\newcommand{\ABlock}[1]{#1}
\newcommand{\MBlock}[2]{#1 #2}
\newcommand{\MMBlock}[3]{#1 #2 #3}
% ----------------------------------------


\begin{document}
	
	% настройки
	\pagestyle{style}\thispagestyle{plain}
	\Title
	% ----------------------------------------
	
	%\vskip5mm
	%\centering{\pgfornament[width=5cm,anchor=south]{89}}

\QBlock{A1}{0.50}{Найдите зависимость давления $p_h$ от высоты $h$. Зависимость может содержать интегральное выражение. Это уравнение называется основной барометрической формулой.\textit{Подсказка}: считайте, что температура и ускорение свободного падения являются функциями $h$.}

\QText{Для изменения давления можем записать:
$$dp_h = -g_h \rho_h dh.$$
С учётом $\rho_h = \frac{\mu P}{RT}$, получаем
$$\frac{dp_h}{p_h} = -\frac{g_h \mu}{RT_h} dh,$$
откуда
$$p_h = p_0 \exp \left(-\frac{\mu}{R} \int\limits_0^h \frac{g_h}{T_h}dh \right),$$
где $p_0$ - давление при $h = 0$.}

\ABlock{$$p_h = p_0 \exp \left(-\frac{\mu}{R} \int\limits_0^h \frac{g_h}{T_h}dh \right)$$}

\QBlock{A2}{0.30}{Получите стандартную барометрическую формулу: зависимость давления от высоты $p_h^{sta}$, считая, что температура и ускорение свободного падения не зависят от $h$.

Рассчитайте величину $h_0 = \frac{RT}{\mu g_0}$ при $T = 425 $К.}

\QText{Подставляя в основную барометрическую формулу $g_h = g_0$, $T_h = T$, получаем стандартную барометрическую формулу:
$$p_h^{sta} = p_0 \exp \left(-\frac{h}{h_0} \right), \qquad h_0 = \frac{RT}{\mu g_0} \approx 12.4 км.$$}

\ABlock{$$p_h^{sta} = p_0 \exp \left(-\frac{h}{h_0} \right) \\ h_0 = \frac{RT}{\mu g_0} \approx 12.4 км$$}

\QBlock{A3}{0.60}{Получите уточнённую барометрическую формулу: зависимость давления от высоты $p_h^{imp}$, считая, что температура постоянна, а ускорение свободного падения зависит от высоты $h$.\textit{Подсказка}: для последнего используйте линейное приближение, считая $z_h = h/R_E \ll 1$.}

\QText{\textit{\textbf{Далее в решении для удобства используется величина $z_h \equiv \frac{h}{R_E}$.}}}

\QText{Ускорение свободного падения задаётся формулой:
$$g_h = \frac{G M_E}{R_E^2 (1+z_h)^2} = \frac{g_0}{(1+z_h)^2}.$$
В линейном приближении
$$g_h \approx g_0(1-2z_h).$$
С учётом этого получаем
$$p_h^{imp} = p_0 \exp \left(-\frac{\mu}{RT} \int\limits_0^h g_0(1-2z_h) dh \right) =  p_0 \exp \left(-\frac{h(1-z_h)}{h_0} \right).$$}

\ABlock{$$p_h^{imp} = p_0 \exp \left(-\frac{h(1-z_h)}{h_0} \right)$$}

\QBlock{A4}{0.40}{Рассчитайте отношение значений давлений, вычисленных по стандартной и по уточнённой барометрическим формулам при $h = 4.0 \times 10^5 $м. Далее используйте уточнённую формулу.}

\QText{Найдём отношение полученных барометрических формул:
$$\frac{p_h^{imp}}{p_h^{sta}} = \frac{\exp \left(-\frac{h(1-z_h)}{h_0} \right)}{ \exp \left(-\frac{h}{h_0} \right)} = e^{\frac{h^2}{h_0 R_E}}.$$
Для $h = 4.0 \times 10^5 м$ получим
$$\frac{p_h^{imp}}{p_h^{sta}} \approx 7.54.$$
Отношение получилось значительно большим единицы, что оправдывает использование улучшенной формулы.}

\ABlock{$$\frac{p_h^{imp}}{p_h^{sta}} = e^{\frac{h^2}{h_0 R_E}} \approx 7.54$$}

\QBlock{A5}{0.20}{Найдите плотность воздуха $\rho_h$ и концентрацию нейтральных молекул воздуха $n_h$ на высоте $h$, используя линейное приближение.}

\QText{В силу соотношений $\rho = \mu P/RT$ и $n = \rho N_A/\mu$ получим
$$\rho_h = \rho_0 \exp \left(-\frac{h(1-z_h)}{h_0} \right), \\ n_h = N_A \frac{\rho_0}{\mu} \exp \left(-\frac{h(1-z_h)}{h_0} \right).$$}

\ABlock{$$\rho_h = \rho_0 \exp \left(-\frac{h(1-z_h)}{h_0} \right) \\ n_h = N_A \frac{\rho_0}{\mu} \exp \left(-\frac{h(1-z_h)}{h_0} \right)$$}

\QBlock{B1}{0.50}{Найдите скорость станции $v_h$ и период обращения $\tau_h$, если станция движется по орбите высотой $h$.}

\QText{Центробежная сила компенсируется гравитационной. Запишем второй закон Ньютона:
$$g_h = \frac{v_h}{R_E (1+z_h)}, \quad где \quad g_h = \frac{g_0}{(1+z_h)^2}.$$
Отсюда получаем искомые величины:
$$v_h = \sqrt{\frac{g_0 R_E}{1+z_h}}, \\ \tau_h = 2\pi \frac{R_E+h}{v_h} = 2\pi \sqrt{\frac{R_E}{g_0}} (1+z_h)^{3/2}.$$}

\ABlock{$$v_h = \sqrt{\frac{g_0 R_E}{1+z_h}} \\ \tau_h = 2\pi \sqrt{\frac{R_E}{g_0}} (1+z_h)^{3/2}$$}

\QBlock{B2}{0.50}{Найдите полную энергию $E_S$ станции, двигающейся по круговой орбите радиусом $R_E + h$.}

\QText{Полная энергия складывается из кинетической
$$E_{K} = \frac{M_S \cdot v_h^2}{2}$$
и потенциальной
$$E_{P} = -M_S g_h R_E (1+z_h).$$
Подставляя $v_h$ из прошлого пункта получаем выражение для $E_S$:
$$E_S = E_K + E_P = -\frac{M_S g_0 R_E}{2(1+z_h)}.$$}

\ABlock{$$E_S = -\frac{M_S g_0 R_E}{2(1+z_h)}$$}

\QBlock{B3}{1.00}{На станцию действует некоторая суммарная тормозящая сила $\vec{F}_{drag}$. В результате МКС замедляется, и высота её орбиты уменьшается на $dh$ за малое время $dt$. Запишите закон изменения энергии МКС, считая известным значение $F_{drag}$.}

\QText{Работа тормозящей силы $F_{drag}$ за время $dt$:}

\QText{$$dA_{drag} = -F_{drag} \cdot v_h \cdot dt.$$}

\QText{При малом \textbf{уменьшении} высоты полёта станции $dh$ изменение полной энергии составит:}

\QText{$$dE_S = -\frac{M_S g_0}{(1+z_h)^2}dh.$$}

\QText{Тогда закон изменения энергии запишется в виде:}

\QText{$$dE_S = dA_{drag}, \\ \frac{M_S g_0}{(1+z_h)^2}dh = F_{drag} v_h dt.$$}

\ABlock{$$\frac{M_S g_0}{2(1+z_h)^2} dh = F_{drag} v_h dt$$}

\QBlock{B4}{0.50}{Найдите скорость снижения станции $u_h$.\textit{Подсказка}: скорость снижения зависит от силы трения, от высоты станции и от её массы.}

\QText{С учетом закона сохранения энергии из прошлого пункта получаем:
$$u_h = \frac{dh}{dt} = \frac{2F_{drag}}{M_S g_0} v_h (1+z_h)^2 = \frac{2F_{drag}}{M_S} \sqrt{\frac{R_E}{g_0}}(1+z_h)^{3/2}.$$}

\ABlock{$$u_h =  \frac{2 F_{drag}}{M_S} \sqrt{\frac{R_E}{g_0}} (1+z_h)^{3/2}$$}

\QBlock{B5}{0.50}{Найдите изменение высоты $H_h$ станции за один оборот вокруг Земли и полное время $T_h$, за которое станция упадёт на поверхность Земли с начальной высоты $h$.\textit{Подсказка}: используйте соотношения $h_0 \ll h \ll R_E$.}

\QText{Выражение для $H_h$ получаем, используя выражения для $u_h$ и $\tau_h$:
$$H_h = u_h \tau_h = \frac{4 \pi R_E}{M_S g_0} F_{drag} (1+z_h)^3.$$
Для нахождения $T_h$ запишем:
$$dh = u_h dt = \frac{2 F_{drag}}{M_S} \sqrt{\frac{R_E}{g_0}} (1+z_h)^{3/2} dt,$$
откуда 
$$dt = \frac{M_S}{2F_{drag} } \sqrt{\frac{g_0}{R_E}} \frac{dh}{(1+z_h)^{3/2}}, \\
T_h = \frac{M_S}{2F_{drag} } \sqrt{\frac{g_0}{R_E}} \int\limits_0^h \frac{1}{(1+z_h)^{3/2}} dh. $$
Интегрируя, получаем:
$$T_h = \frac{M_S R_E}{F_{drag}} \sqrt{\frac{g_0}{R_E}} \left(1-\frac{1}{\sqrt{1+z_h}}\right).$$
С учётом $z_h \ll 1$, можем использовать приближение
$$\frac{1}{\sqrt{1+z_h}} \approx 1-\frac{z_h}{2}.$$
Тогда выражение для $T_h$ принимает вид:
$$T_h = \frac{M_S h}{2F_{drag}} \sqrt{\frac{g_0}{R_E}}.$$}

\ABlock{$$T_h = \frac{M_S R_E}{F_{drag}} \sqrt{\frac{g_0}{R_E}} \left(1-\frac{1}{\sqrt{1+z_h}}\right) \approx  \frac{M_S h}{2F_{drag}} \sqrt{\frac{g_0}{R_E}}$$}

\QBlock{C1}{0.50}{Найдите силу сопротивления воздуха $F_{air}$, скорость уменьшения высоты орбиты $u_h^{air}$ и изменение высоты за один оборот $H^{air}_h$ в этом случае.}

\QText{Считая, что молекулы до столкновения со станцией покоятся, запишем закон сохранения импульса при столкновении МКС с молекулами общей массой $dm$:
$$M_s v_h = M_s (v_h + dv_h) + dm \cdot  v_h, \\
M_s \cdot dv_h = -dm \cdot v_h.$$
За время $dt$ станция сталкивается с молекулами массой $dm = \rho_h v_h S dt$. Подставим это в полученное ранее выражение:
$$M_s \cdot dv_h =  -\rho_h v_h^2 S dt, \\ F_{air} = \bigg|M_S \frac{dv_h} {dt} \bigg| = \rho_h v_h^2 S.$$
Подставляя $F_{air}$ в выражения для $u_h$ и $H_h$, получаем:
$$u_h^{air} = \frac{2 \rho_0 S \sqrt{g_0 R_E^3}}{M_S}(1+z_h)^{1/2} \cdot \exp \left(-\frac{h(1-z_h)}{h_0} \right), \\
H_h^{air}  = \frac{4 \pi S R_E^2}{M_S} \rho_0 \cdot (1+z_h)^2 \cdot \exp  \left(-\frac{h(1-z_h)}{h_0} \right).$$}

\ABlock{$$F_{air} = \rho_h v_h^2 S \\
u_h^{air} = \frac{2 \rho_0 S \sqrt{g_0 R_E^3}}{M_S}(1+z_h)^{1/2} \cdot \exp \left(-\frac{h(1-z_h)}{h_0} \right) \\
H_h^{air}  = \frac{4 \pi S R_E^2}{M_S} \rho_0 \cdot (1+z_h)^2 \cdot \exp  \left(-\frac{h(1-z_h)}{h_0} \right)
$$}

\QBlock{C2}{0.50}{Найдите полное время $T_h^{air}$,  за которое станция упадёт на поверхность Земли с начальной высоты $h$ из-за сопротивления атмосферы.\textit{Подсказка}: используйте соотношения $h_0 \ll h \ll R_E$.}

\QText{Аналогично пункту B5 получаем интегральное выражение для $T_h^{air}$:}

\QText{$$T_h^{air} = \frac{M_S}{2 \rho_0 S \sqrt{g_0 R_E^3}} \int\limits_0^h \left(1-\frac{h}{2R_E} \right) e^{h/h_0} dh.$$}

\QText{Интегрирование дает:}

\QText{$$T_h^{air} = \frac{M_S h_0}{2 \rho_0 S \sqrt{g_0 R_E^3}} \left(1-\frac{h-h_0}{2R_E} \right) \cdot e^{h/h_0}.$$}

\QText{С учётом всех приближений ответ упрощается:}

\QText{$$T_h^{air} = \frac{M_S h_0}{2 \rho_0 S \sqrt{g_0 R_E^3}} \cdot e^{h/h_0}.$$}

\QText{\textit{Примечание. }Приближения можно было использовать уже в интегральной формуле, что существенно упростило бы вычисление интеграла.}

\ABlock{$$T_h^{air} = \frac{M_S h_0}{2 \rho_0 S \sqrt{g_0 R_E^3}} \left(1-\frac{h-h_0}{2R_E} \right) \cdot e^{h/h_0} \approx \frac{M_S h_0}{2 \rho_0 S \sqrt{g_0 R_E^3}} \cdot e^{h/h_0}$$}

\QBlock{D1}{0.30}{Найдите среднюю (за 24 часа) тормозящую силу $F_{ion}$, обусловленную столкновениями с этими частицами. Ночью ионизацией молекул можно пренебречь.

Найдите также плотность ионизированных молекул кислорода $\rho_{ion}$.}

\QText{Выражение для тормозящей силы со стороны ионов аналогично выражению из пункта С1. Так как станция проводит примерно половину времени с неосвещённой (ночной) стороны Земли, а ночью ионизацией можно пренебречь, выражение для средней силы следующее:
$$F_{ion} = \frac{1}{2} \rho_{ion} \cdot S \cdot v_h^2.$$
Плотность ионов выражается через из концентрацию как
$$\rho_{ion} = \frac{\mu_{ion}}{N_A} \cdot n_{ion},$$
где $\mu_{ion} = \frac{1}{2} \mu_{O_2}.$}

\ABlock{$$F_{ion} = \frac{1}{2} \rho_{ion} S v_h^2 \\
\rho_{ion} = \frac{\mu_{ion}}{N_A} n_{ion}$$}

\QBlock{D2}{0.70}{Найдите скорость уменьшения высоты орбиты станции $u_h^{ion}$, связанную со взаимодействием с ионами атомарного кислорода. Найдите также изменение высоты за один оборот $H_h^{ion}$ в этом случае.\textit{Подсказка}: используйте соотношения $h_0 \ll h \ll R_E$.}

\QText{Подставляя $F_{ion}$ в выражения для $u_h$ и $H_h$, получаем:
$$u_h^{ion} = \rho_{ion} \cdot \frac{S \sqrt{g_0 R_E^3}}{M_S}(1+z_h)^{1/2}, \\
H_h^{ion} = u_h^{ion} \tau_h = \frac{2 \pi S R_E^2 \rho_{ion}}{M_S}(1+z_h)^2.$$}

\ABlock{$$u_h^{ion} = \rho_{ion} \frac{S \sqrt{g_0 R_E^3}}{M_S}(1+z_h)^{1/2} \\
H_h^{ion} = u_h^{ion} \tau_h = \frac{2 \pi S R_E^2 \rho_{ion}}{M_S}(1+z_h)^2$$}

\QBlock{E1}{0.60}{Оцените величину возникающего в проводящих частях станции тока $I_{ind}$.}

\QText{За время $dt$ на станцию попадает $dN$ ионов:
$$dN = n_{ion} \cdot v_h \cdot S \cdot dt.$$
Возникающий ток:
$$I_{ion} \approx e \frac{dN}{dt} = e \cdot S \cdot n_{ion} \cdot \sqrt{\frac{g_0 R_E}{1+z_h}}.$$}

\ABlock{$$I_{ion} \approx e \cdot S \cdot n_{ion} \cdot \sqrt{\frac{g_0 R_E}{1+z_h}}$$}

\QBlock{E2}{0.60}{Получите приближённое выражение для тормозящей силы Ампера $F_{ind}$ в направлении, противоположном направлению движению станции.

Пусть $\phi$ - угол между магнитным полем Земли $\vec{B}$, направленным вдоль меридианов, и скоростью МКС $\vec{v}$. Для простоты считайте, что длина станции $L$ равна корню квадратному из её площади $S$. Кроме того, вместо подсчёта среднего значения $\sin(\phi)$ вы можете аппроксимировать его значением $\sin(\pi/2 - \theta)$. Вы можете использовать дискретное число точек для подсчёта среднего значения.}

\QText{В каждый момент времени ток $I_{ion}$ направлен перпендикулярно поверхности земли, а значит и вектору магнитного поля. Модуль силы Ампера, действующей на станцию:
$$|F_{amp}| = ILB.$$
Тогда, проектируя силу Ампера на ось, сонаправленную со скоростью станции, получим выражение для силы в момент, когда угол между скоростью станции и магнитным полем равен $\phi$:
$$F_{ind}(\phi) = |F_{amp}| \sin \phi= ILB \sin \phi.$$
Используя $\left \langle \sin \phi \right \rangle = \sin(\pi/2-\theta)$, получим:
$$\left \langle F_{ind}(\phi) \right \rangle = F_{ind} = ILB \cos \theta = e \cdot S^{3/2} \cdot n_{ion} \cdot B \cdot \cos \theta \cdot \sqrt{\frac{g_0 R_E}{1+z_h}}.$$}

\ABlock{$$F_{ind} = e \cdot S^{3/2} \cdot n_{ion} \cdot B \cdot \cos \theta \cdot \sqrt{\frac{g_0 R_E}{1+z_h}}$$}

\QBlock{E3}{0.80}{Найдите скорость снижения станции из-за её взаимодействия с магнитным полем Земли. Найдите также изменение высоты за один оборот $H_h^{ind}$ в этом случае.\textit{Подсказка}: используйте соотношение $h \ll R_E$.}

\QText{Подставляя $F_{ion}$ в выражения для $u_h$ и $H_h$, получаем:
$$u_h^{ind} = 2n_{ion} \frac{eBS^{3/2} R_E \cos \theta}{M_S} (1+z_h), \\ 
H_h^{ind} = \frac{4 \pi e B (SR_E)^{3/2} \cos \theta}{M_S \sqrt{g_0}} (1+z_h)^{5/2}.$$}

\ABlock{$$u_h^{ind} = 2n_{ion} \frac{eBS^{3/2} R_E \cos \theta}{M_S} (1+z_h) \\ 
H_h^{ind} = \frac{4 \pi e B (SR_E)^{3/2} \cos \theta}{M_S \sqrt{g_0}} (1+z_h)^{5/2}$$}

\QBlock{F1}{0.40}{Рассчитайте необходимые величины и заполните Таблицу 1 в листе ответов.  \begin{tabular}{|c|c|c|c|c|c|c|} \hline 
$h, км$ & $T_h^{air}, дней$ & $u_{air}, м/день$ & $u_{ion}, м/день$ & $u_{ind}, м/день$ & $\sum, м/день$ & $u_{ISS}, м/день$\\ 
 \hline 
350 &   &   &   &   &   &  \\ 
 \hline 
375 &   &   &   &   &   &  \\ 
 \hline 
400 &   &   &   &   &   &  \\ 
 \hline 
410 &   &   &   &   &   &  \\ 
 \hline 
\end{tabular} }

\QText{Значения столбцов 2-6 считаем по полученным нами ранее формулам, значения столбца 7 оцениваем из графиков, данных в начале задачи.}

\ABlock{\begin{tabular}{|c|c|c|c|c|c|c|} \hline 
$h, км$ & $T_h^{air}, дней$ & $u_{air}, м/день$ & $u_{ion}, м/день$ & $u_{ind}, м/день$ & $\sum, м/день$ & $u_{ISS}, м/день$\\ 
 \hline 
350 & 358 & 171 & 0.67 & 1.3 & 173 & $\sim 170$ [в 2008]\\ 
 \hline 
375 & 2688 & 28.7 & 0.67 & 1.3 & 30.7 & $-$\\ 
 \hline 
400 & 20181 & 4.9 & 0.67 & 1.3 & 6.9 & $\le 100$ [в 2021]\\ 
 \hline 
410 & 45205 & 2.4 & 0.67 & 1.3 & 4.4 & $\le 70$ [в 2022]\\ 
 \hline 
\end{tabular}}

\QBlock{F2}{0.40}{Рассчитайте необходимые величины и заполните Таблицу 2 в листе ответов.  \begin{tabular}{|c|c|c|c|} \hline 
$h, км$ & $H_h^{air}, м$ & $H_h^{ion}, м$ & $H_h^{ind}, м$\\ 
 \hline 
350 &   &   &  \\ 
 \hline 
375 &   &   &  \\ 
 \hline 
400 &   &   &  \\ 
 \hline 
410 &   &   &  \\ 
 \hline 
\end{tabular} }

\ABlock{\begin{tabular}{|c|c|c|c|} \hline 
$h, км$ & $H_h^{air}, м$ & $H_h^{ion}, м$ & $H_h^{ind}, м$\\ 
 \hline 
350 & 10.6 & 0.04 & 0.08\\ 
 \hline 
375 & 1.8 & 0.04 & 0.08\\ 
 \hline 
400 & 0.31 & 0.04 & 0.08\\ 
 \hline 
410 & 0.15 & 0.04 & 0.08\\ 
 \hline 
\end{tabular}}

\QBlock{F3}{0.20}{МКС обращается по орбите на высотах выше 380 км. Расположите три рассмотренных эффекта торможения станции в порядке убывания их влияния.}

\QText{На основе значений в таблицах делаем вывод.}

\ABlock{\begin{itemize} 
\item Сопротивление атмосферы
\item Сила Ампера 
\item Столкновения с ионизированными молекулами кислорода
\end{itemize}}

\end{document}