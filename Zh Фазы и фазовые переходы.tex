
%\documentstyle[12pt,russian,amsthm,amsmath,amssymb]{article}
\documentclass[a4paper,11pt,twoside]{article}
\usepackage[left=14mm, top=10mm, right=14mm, bottom=10mm, nohead, nofoot]{geometry}
\usepackage{amsmath, amsfonts, amssymb, amsthm} % стандартный набор AMS-пакетов для математ. текстов
\usepackage{mathtext}
\usepackage[utf8]{inputenc} % кодировка utf8
\usepackage[russian]{babel} % русский язык
\usepackage[pdftex]{graphicx} % графика (картинки)
\usepackage{tikz}
\usepackage{fancyhdr,pageslts} % настройка колонтитулов
\usepackage{enumitem} % работа со списками
\usepackage{multicol} % работа с таблицами
%\usepackage{pscyr} % красивый шрифт
\usepackage{pgfornament} % красивые рюшечки и вензеля
\usepackage{ltxgrid} % управление написанием текста в две колонки
\usepackage{lipsum} % стандартный текст
\usepackage{tcolorbox} % рамка вокруг текста
\tcbuselibrary{skins}
% ----------------------------------------

\newcommand\ProblemName{Фазы и фазовые переходы}

\newcommand\Source{Zh}

\newcommand\Type{Условие задачи}

\newcommand\MyTextLeft{Президентский ФМЛ 239, г.~Санкт-Петербург}
\newcommand\MyTextRight{Использованы материалы сайта pho.rs}
\newcommand\MyHeading{Учебно-тренировочные сборы по физике}
% ----------------------------------------

% настройки полей
\geometry{
	left=12mm,
	top=21mm,
	right=15mm,
	bottom=26mm,
	marginparsep=0mm,
	marginparwidth=0mm,
	headheight=22pt,
	headsep=2mm,
	footskip=7mm}
% ----------------------------------------

% настройки колонтитулов
\pagestyle{fancy}

\fancypagestyle{style}{
	\fancyhf{}
	\fancyhead[L]{{\Large{\FancyTitle}}\\\vskip -5pt \dotfill}
	\fancyhead[R]{{\Large{\textbf{\Type}}}\\\vskip -5pt \dotfill}
	\renewcommand{\headrulewidth}{0pt}
	\renewcommand{\footrulewidth}{0pt}
	\fancyfoot[C]{\pgfornament[width=2em,anchor=south]{72}\hspace{1mm}
		{Страница \textbf{\thepage} из \textbf{\pageref{VeryLastPage}}}\hspace{2mm}
		\pgfornament[width=2em,symmetry=v,anchor=south]{72}\\ \vskip2mm
		{\small{\textit{\MyTextLeft\hfill\MyTextRight}}}}
}

\fancypagestyle{plain}{
	\fancyhf{}
	\renewcommand{\headrulewidth}{0pt}
	\renewcommand{\footrulewidth}{0pt}
	\fancyhead[C]{{\Large{\textit{\MyHeading}}}\\\vskip -5pt \dotfill}
	\fancyfoot[C]{\pgfornament[width=2em,anchor=south]{72}\hspace{1mm}
		{Страница \textbf{\thepage} из \textbf{\pageref{VeryLastPage}}}\hspace{2mm}
		\pgfornament[width=2em,symmetry=v,anchor=south]{72}\\ \vskip2mm
		{\small{\textit{\MyTextLeft\hfill\MyTextRight}}}}
}
% ----------------------------------------

% другие настройки
\pagenumbering{arabic}
\setlist[enumerate,itemize]{leftmargin=0pt,itemindent=2.7em,itemsep=0cm}
% ----------------------------------------

% собственные команды
\newcommand{\FancyTitle}{\textbf{\Source} --- \ProblemName}
\newcommand{\Title}{\begin{center}{\huge{\textbf{\Source} --- \ProblemName}}\end{center}}
\newcommand{\Chapter}[1]{\vskip5pt{\Large{\textbf{#1}}}\vskip5pt}
\newcommand{\QText}[1]{#1}
\newcommand{\QBlock}[3]{
	\begin{tcolorbox}[left=2mm,top=2mm,bottom=1mm,right=2mm,colback=white]
		\begin{tcolorbox}[enhanced,colframe=ProcessBlue,colback=ProcessBlue!30!white,
			frame style={opacity=0.7},interior style={opacity=1.0},
			nobeforeafter,tcbox raise base,shrink tight,extrude by=1.7mm,width=1.5cm]
			\textbf{#1\textsuperscript{#2}}
		\end{tcolorbox}\hspace{3mm}#3
	\end{tcolorbox}
}
\newcommand{\QPicture}[4]{
	\begin{figure}[H]
		\centering
		\includegraphics[width=0.35\linewidth]{#1}
		\caption{#3}
	\end{figure}
	
	#4
}
\newcommand{\ABlock}[1]{
	\vskip2mm
	\begin{tcolorbox}[enhanced,colframe=Magenta,colback=Magenta!15!white,
		frame style={opacity=0.5},interior style={opacity=1.0},
		nobeforeafter,tcbox raise base,shrink tight,extrude by=1.7mm,width=1.6cm]
		\textbf{Ответ:}
	\end{tcolorbox}\hspace{3mm}#1
}
\newcommand{\MBlock}[2]{
	\begin{tcolorbox}[enhanced,colframe=Yellow,colback=Yellow!15!white,
		frame style={opacity=0.5},interior style={opacity=1.0},
		nobeforeafter,tcbox raise base,shrink tight,extrude by=1.7mm,width=1.1cm]
		\textbf{#1}
	\end{tcolorbox}\hspace{3mm}#2
}
\newcommand{\MMBlock}[3]{
	\begin{tcolorbox}[enhanced,colframe=Yellow,colback=Yellow!15!white,
		frame style={opacity=0.5},interior style={opacity=1.0},
		nobeforeafter,tcbox raise base,shrink tight,extrude by=1.7mm,width=1.1cm]
		\textbf{#1}
	\end{tcolorbox}\hspace{3mm}
	\begin{tcolorbox}[enhanced,colframe=Orange,colback=Orange!15!white,
		frame style={opacity=0.5},interior style={opacity=1.0},
		nobeforeafter,tcbox raise base,shrink tight,extrude by=1.7mm,width=0.8cm]
		\textbf{#2}
	\end{tcolorbox}\hspace{3mm}#3
}
% ----------------------------------------


\begin{document}
	
	% настройки
	\pagestyle{style}\thispagestyle{plain}
	\Title
	% ----------------------------------------
	
	%\vskip5mm
	%\centering{\pgfornament[width=5cm,anchor=south]{89}}
	
	% смысловая часть


\QText{При заданном давлении переход из одного агрегатного состояния вещества (фазы) в другое происходит всегда при строго определённой температуре, при этом сам переход называется фазовым. Например, лёд при атмосферном давлении плавиться при $0 {}^\circ\mathrm{C}$, так что при подводе тепла температура смеси из льда и воды остается неизменной вплоть до того момента, пока весь лёд не превратится в воду.
Во всех предлагаемых ниже задачах считайте, что удельный объём жидкой фазы пренебрежимо мал по сравнению с удельным объёмом насыщенного пара, который можно считать идеальным газом. Теплоёмкость жидкой воды считайте независящей от температуры.}

\Chapter{Справочные данные}

\QText{Газовая постоянная $R=8,31\ \mathrm{Дж}/(\mathrm{моль}\cdot \mathrm{К})$;
молярная масса воздуха ${\mu }_{air}=29.0\ \mathrm{г}/\mathrm{моль}$;
ускорение свободного падения $g=9,81\ \mathrm{м}/{\mathrm{с}}^2$.

Нормальные условия:
давление $P_0=1\ \mathrm{атм}=760\ \mathrm{мм. рт. ст.}=101325\ \mathrm{Па}$,
температура $T_0=273.15\ \mathrm{К}=0\ {}^\circ\mathrm{C}$.

Свойства воды (Н${}_{2}$O)
Молярная масса ${\mu }_w=18.0\ \mathrm{г}/\mathrm{моль}$;
плотность воды ${\rho }_w=1.00\ \mathrm{г}/{\mathrm{см}}^3$;
плотность льда ${\rho }_i\ =\ 0.920\ \mathrm{г}/{\mathrm{см}}^3$;
температура плавления льда при нормальном давлении $t_m=0.00\ {}^\circ\mathrm{C}$;
температура кипения воды при нормальном давлении $t_b=100.0\ {}^\circ\mathrm{C}$;
удельная теплоёмкость воды $c_w=4.20\ \mathrm{Дж}/(\mathrm{г}\cdot \mathrm{К})$;
удельная теплота плавления льда $q_i=334\ \mathrm{Дж}/\mathrm{г}$;
удельная теплота парообразования воды (при $100\ {}^\circ\mathrm{C}$) $r_w=2259\ \mathrm{Дж}/\mathrm{г}$;
показатель адиабаты Пуассона для водяных паров $\gamma =C_P/C_V=4/3$.}

\Chapter{Теплота фазового перехода}

\QText{Если переход вещества из одной фазы в другую связан с выделением или поглощением некоторого количества теплоты, называемой теплотой перехода, то такой переход называется фазовым переходом первого рода. При этом теплота перехода $q$ для единичной массы называется удельной теплотой фазового перехода (плавления, испарения, возгонки). 
Поскольку фазовый переход происходит при постоянном давлении, то по первому началу термодинамики теплота $q$ расходуется на изменение внутренней энергии $u$ и на работу $A$ против постоянного внешнего давления:
\[q=u_2 -u_1+A,\] 
где $u_1, u_2$ - удельные внутренние энергии соответственно первой и второй фаз соответственно.
При плавлении (кристаллизации) из-за малого различия плотностей жидкой и твёрдой фаз изменение объёма в результате фазового перехода невелико, поэтому работой $A$ можно пренебречь по сравнению с изменением внутренней энергии.}

\QBlock{1}{1.00}{Рассчитайте, какая часть теплоты испарения воды при $t_b=100 {}^{\circ}\text{C}$ расходуется на измерение внутренней энергии. Ответ выразите в \%.}

\QBlock{2}{1.00}{Вычислите удельную теплоту парообразования воды при комнатной температуре $t=20.0 {}^{\circ}\text{C}$.}

\QText{В дальнейшем удельную теплоту испарения всех жидкостей считайте не зависящими от температуры.}

\Chapter{Формула Клапейрона - Клаузиуса}

\QText{При изменении давления температура фазового перехода первого рода меняется, то есть фазовый переход имеет место при строго определённой зависимости $P\left(T\right)$ между давлением $P$ и температурой $T$ вещества. Эта зависимость, изображённая на координатной $TP$-плоскости, называется фазовой $T-P$ диаграммой, а сама линия $TP$ - линией фазового равновесия. Формула Клапейрона-Клаузиуса дает наклон линии фазового равновесия $P(T)$ в виде:
\[\frac{dP}{dT}=\ \frac{q}{T\left(v_2-{\ v}_1\right)},\] 
где $q\ $- удельная теплота перехода из фазы 1 с удельным объёмом $v_1$ в фазу 2 с удельным объёмом $v_2$.}

\QBlock{3}{0.40}{Считая известным давление насыщенного пара воды при температуре $t_b=100 {}^{\circ}\text{C}$, получите явную зависимость давления насыщенных паров воды от темпетаруры $P(T)$.}

\QBlock{4}{1.00}{Вычислите температуру кипения воды на самой высокой вершине Казахстана - пике Хан-Тенгри. Высота пика Хан-Тенгри над уровнем моря $h\approx7000 \text{м}$. Температуру воздуха на высоте считать постоянной и равной $t_0=0 {}^{\circ}\text{C}$.}

\QBlock{5}{0.60}{При каком давлении (в атмосферах) лед будет плавиться при температруру $t=-1.00 {}^{\circ}\text{C}$?}

\QBlock{6}{0.60}{Известно, что кристаллики льда начинают разрушаться, если вдоль какого-либо направления кристалла приложить силу, создающую давление $P>P_{cr}\approx 1000 \text{атм}$. Поэтому снег в морозную погоду хрустит при ходьбе. Оцените максимальную температуру воздуха $t_{max}$, при которой сне все еще хрустит при ходьбе.}

\QBlock{7}{1.00}{В сосуде находится один моль насыщенного пара при температуре $t_b=100 {}^{\circ}\text{C}$. Пар нагревается и одновременно меняется его объем так, что он все время остается насыщенным. Найдите молярную теплоемкость пара в таком процессе.}

\Chapter{Пограничное кипение}

\QText{Пограничное кипение - это кипение на границе раздела двух несмешивающихся жидкостей. Температура пограничного кипения может существенно отличаться от температур объёмного кипения каждой из жидкостей. 

Тетрахлорметан или четырёххлористый водород представляет собой тяжёлую (плотность $\rho=1.60 \mathrm{г}/{\mathrm{см}}^3$) прозрачную жидкость с молярной массой $\mu=153.8\mathrm{г}/\mathrm{моль}$. При нормальном атмосферном давлении тетрахлорметан кипит при температуре $t=76.65 {}^\circ\mathrm{C}$, при этом он практически не растворяется в воде. Сосуд объемом $V=100 \mathrm{мл}$ наполовину наполняют тетрахлорметаном, а поверх заливают такое же (по объёму) количество воды. При этом образуется четкая граница вода-тетрахлорметан.  При равномерном нагревании сосуда на водяной бане кипение на границе раздела жидкостей начинается при температуре $t^*=66.0 {}^\circ\mathrm{C}$, что значительно ниже температуры объёмного кипения каждой из компонент в отдельности.}

\QBlock{8}{1.20}{Рассчитайте по этим данным удельную теплоту $r$ испарения тетрахлорметана, если известно, что давление насыщенных паров воды при температуре пограничного кипения $P_w(t^*)=196 \text{мм. рт. ст.}$}

\QBlock{9}{1.00}{Найдите массу остающейся в сосуде жидкости к моменту полного выкипания другой жидкости при таком пограничном кипении.}

\QText{Рассмотрим ещё одну пару несмешивающихся жидкостей, воду и фторкетон.  Жидкость фторкетон, иногда называемая «сухой водой», используется при тушении пожаров в библиотеках, музеях, офисах, поскольку не смачивает бумагу. Это тяжелая (плотность $\rho=1.72 \mathrm{г}/{\mathrm{см}}^3$) прозрачная жидкость с молярной массой $\mu=316 \mathrm{г}/\mathrm{моль}$, которая в воде практически не растворяется. Температура кипения фторкетона при атмосферном давлении $t_f=49.2 {}^\circ\mathrm{C}$, удельная теплота парообразования $r=95.0 \mathrm{Дж}/\mathrm{г}$. Если поверх фторкетона в сосуд налить воду, то также образуется чёткая граница вода-фторкетон.}

\QBlock{10}{2.20}{Оцените температуру $t_x$ закипания жидкостей на границе вода-фторкетон, если известно давление насыщенных паров воды при температуре объемного кипения фторкетона $P_w(t_f)=89.0 \text{мм. рт. ст.}$}

\end{document}