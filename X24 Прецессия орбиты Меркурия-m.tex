
%This file is part of Get pho.rs!

%Get pho.rs! is free software: you can redistribute it and/or modify it under the terms of the GNU General Public License as published by the Free Software Foundation, either version 3 of the License, or (at your option) any later version.

%Get pho.rs! is distributed in the hope that it will be useful, but WITHOUT ANY WARRANTY; without even the implied warranty of MERCHANTABILITY or FITNESS FOR A PARTICULAR PURPOSE. See the GNU General Public License for more details.

%You should have received a copy of the GNU General Public License along with Foobar. If not, see <https://www.gnu.org/licenses/>.

%\documentstyle[12pt,russian,amsthm,amsmath,amssymb]{article}
\documentclass[a4paper,11pt,twoside]{article}
\usepackage[left=14mm, top=10mm, right=14mm, bottom=10mm, nohead, nofoot]{geometry}
\usepackage{amsmath, amsfonts, amssymb, amsthm} % стандартный набор AMS-пакетов для математ. текстов
\usepackage{mathtext}
\usepackage[utf8]{inputenc} % кодировка utf8
\usepackage[russian]{babel} % русский язык
\usepackage[pdftex,dvipsnames]{xcolor} % работа с цветами
\usepackage[pdftex]{graphicx} % графика (картинки)
\usepackage{tikz} % рисунки
\usepackage{fancyhdr,pageslts} % настройка колонтитулов
\usepackage{enumitem} % работа со списками
\usepackage{multicol} % работа с таблицами
%\usepackage{pscyr} % красивый шрифт
\usepackage{pgfornament} % красивые рюшечки и вензеля
\usepackage{ltxgrid} % управление написанием текста в две колонки
\usepackage{lipsum} % стандартный текст
\usepackage{tcolorbox} % рамка вокруг текста
\usepackage{float} % для корректного размещения картинок
\tcbuselibrary{skins}
% ----------------------------------------

\newcommand\ProblemName{Прецессия орбиты Меркурия}

\newcommand\Source{X24}

\newcommand\Type{Разбалловка}

% настройки полей
\geometry{
	left=12mm,
	top=21mm,
	right=15mm,
	bottom=26mm,
	marginparsep=0mm,
	marginparwidth=0mm,
	headheight=22pt,
	headsep=2mm,
	footskip=7mm}
% ----------------------------------------

% настройки колонтитулов
\pagestyle{fancy}

\fancypagestyle{style}{
	\fancyhf{}
	\fancyhead[L]{{\Large{\FancyTitle}}\\\vskip -5pt \dotfill}
	\fancyhead[R]{{\Large{\textbf{\Type}}}\\\vskip -5pt \dotfill}
	\renewcommand{\headrulewidth}{0pt}
	\renewcommand{\footrulewidth}{0pt}
	\fancyfoot[C]{\pgfornament[width=2em,anchor=south]{72}\hspace{1mm}
		{Страница \textbf{\thepage} из \textbf{\pageref{VeryLastPage}}}\hspace{2mm}
		\pgfornament[width=2em,symmetry=v,anchor=south]{72}\\ \vskip2mm
		{\small{\textit{Условие собрано и подготовлено в Президентском ФМЛ №239 г.~Санкт-Петербурга}}}}
}

\fancypagestyle{plain}{
	\fancyhf{}
	\renewcommand{\headrulewidth}{0pt}
	\renewcommand{\footrulewidth}{0pt}
	\fancyhead[C]{{\Large{\textit{Учебно-тренировочные сборы к X23}}}\\\vskip -5pt \dotfill}
	\fancyfoot[C]{\pgfornament[width=2em,anchor=south]{72}\hspace{1mm}
		{Страница \textbf{\thepage} из \textbf{\pageref{VeryLastPage}}}\hspace{2mm}
		\pgfornament[width=2em,symmetry=v,anchor=south]{72}\\ \vskip2mm
		{\small{\textit{Условие собрано и подготовлено в Президентском ФМЛ №239 г.~Санкт-Петербурга}}}}
}
% ----------------------------------------

% другие настройки
\pagenumbering{arabic}
\setlist[enumerate,itemize]{leftmargin=0pt,itemindent=2.7em,itemsep=0cm}
% ----------------------------------------

% собственные команды
\newcommand{\FancyTitle}{\textbf{\Source} --- \ProblemName}
\newcommand{\Title}{\begin{center}{\huge{\textbf{\Source} --- \ProblemName}}\end{center}}
\newcommand{\Chapter}[1]{\vskip5pt{\Large{\textbf{#1}}}\vskip5pt}
\newcommand{\QText}[1]{#1}
\newcommand{\QBlock}[3]{
	\begin{tcolorbox}[left=4mm,top=3mm,bottom=2mm,right=4mm,colback=white]
		\begin{tcolorbox}[enhanced,colframe=blue,colback=blue!10!white,
			frame style={opacity=0.3},interior style={opacity=1.0},
			nobeforeafter,tcbox raise base,shrink tight,extrude by=1.7mm,width=1.5cm]
			\textbf{#1\textsuperscript{#2}}
		\end{tcolorbox}\hspace{3mm}#3
	\end{tcolorbox}
}
\newcommand{\QPicture}[4]{\QText{#4}  \includegraphics{#1}}
\newcommand{\ABlock}[1]{#1}
\newcommand{\MBlock}[2]{#1 #2}
\newcommand{\MMBlock}[3]{#1 #2 #3}
% ----------------------------------------


\begin{document}
	
	% настройки
	\pagestyle{style}\thispagestyle{plain}
	\Title
	% ----------------------------------------
	
	%\vskip5mm
	%\centering{\pgfornament[width=5cm,anchor=south]{89}}

\QBlock{A1}{0.40}{В этом пункте рассмотрим движение Меркурия без учета других планет. Орбиту считайте круговой. Найдите период его обращения вокруг Солнца $T$ (формулу и численное значение в годах) и величину момента импульса  $L$ (только формулу). Выразите ответ через $G$, $a$, $M_S$, $m$.}

\MBlock{0.10}{Закон Ньютона для движения по окружности}

\MBlock{0.10}{$
T  = 2\pi \sqrt{\dfrac{a^3}{G M_S}},
$}

\MBlock{0.10}{$T = 0.24  \text{года}$}

\MBlock{0.10}{$
L= m v a = m \sqrt{G M_S a}.
$}

\QBlock{A2}{0.20}{Найдите проекцию ускорения падения $g_z$, создаваемого кольцом на его оси симметрии на расстоянии $z \ll R$ от центра. Выразите ответ через $G$, $M$, $R$, $z$.}

\MBlock{0.10}{$
g_z = -  \dfrac{G M z }{R^3}
$}

\MBlock{0.10}{Правильный знак (если модуль верен)}

\QBlock{A3}{0.50}{Найдите радиальную проекцию ускорения свободного падения $g_r$, создаваемого кольцом в точке в плоскости кольца на расстоянии $ r \ll R$ от его центра. Используйте теорему Гаусса для гравитационного поля. Запишите ответ с точностью до слагаемых порядка $r$. Выразите ответ через $G$, $M$, $R$, $r$. Положительным считается направление от оси кольца.}

\MBlock{0.20}{Выбрана разумная поверхность для применения теоремы Гаусса}

\MBlock{0.10}{Записан поток $\vec{g}$}

\MBlock{0.10}{$g_r  = +\dfrac{G M r }{ 2 R^3}$}

\MBlock{0.10}{Верный знак (если модуль верен)}

\QBlock{A4}{0.30}{Найдите потенциальную энергию $V(r)$ Меркурия, если он находится на расстоянии $r$ от  Солнца. Выразите ответ через $r$, $R$, $G$, $M_S$, $M$, $m$. Потенциальная энергия взаимодействия с кольцом равна нулю, когда Меркурий находится в его центре, а потенциальная энергия взаимодействия с Солнцем - когда Меркурий находится на бесконечности.}

\MBlock{0.20}{Потенциальная энергия взаимодействия с кольцом $ - \dfrac{GM m}{4 R^3}r^2.
$}

\MBlock{0.10}{Потенциальная энергия взаимодействия с Солнцем $- GM_S m/r$}

\MBlock{-0.10}{Ошибка в знаке или в численном коэффициенте}

\QBlock{B1}{0.40}{Пусть момент импульса Меркурия равен $L$. Выразите производную расстояния до Солнца по времени $\dot{r} = dr/dt$ через производную $u$ по углу $u' = du/d\theta$, а также через $L$, $u$, $m$.}

\MBlock{0.10}{$
L = m r^2 \dfrac{d\theta}{dt}.
$}

\MBlock{0.20}{$
\dfrac{dr}{dt} = - \dfrac{L}{m} u'
$}

\MBlock{0.10}{Верные знак (только при наличии верного ответа)}

\QBlock{B2}{0.40}{Запишите выражение для полной механической энергии Меркурия. Выразите ответ через $u$, $u'$, $m$, $L$, $V(r)$.}

\MBlock{0.15}{Записан вклад радиального движения в кинетическую энергию $\dfrac{L^2}{2 m} u^{\prime 2} $}

\MBlock{0.15}{Записан вклад углового движения в кинетическую энергию $\dfrac{L^2}{2 m} u^2 $}

\MBlock{0.10}{Записана потенциальная энергия}

\QBlock{B3}{0.60}{Продифференцировав закон сохранения (например по углу), получите выражение для $u''(\theta)$. Выразите ответ через $u$, $G$, $m$, $M_S$, $M$, $R$, $L$.}

\MBlock{0.20}{Корректно продифференцирована кинетическая энергия}

\MBlock{0.20}{Корректно продифференцирована потенциальная энергия}

\MBlock{0.20}{$$
u'' = - u + \frac{G M_S m^2 }{L^2} - \dfrac{GM m^2}{2 R^3 L^2} \frac{1}{u^3}
$$}

\QBlock{C1}{0.30}{Используя уравнение из $B3$, получите уравнение, из которого можно найти радиус круговой орбиты $a$ (решать его не нужно). В уравнение могут входить $m$, $M_s$, $M$,  $R$, $G$, $L$, $a$.}

\MBlock{0.10}{Использовано $u'' = 0$}

\MBlock{0.10}{Подставлено $u = 1/a$}

\MBlock{0.10}{$$ 
- \frac{1}{a} + \frac{G M_S m^2 }{L^2}- \frac{GM m^2}{2 R^3 L^2} a^3 = 0
$$}

\QBlock{C2}{0.60}{Получите линеаризованное уравнение движения для отклонения $\delta u$ формы орбиты от круговой, то есть выражение для $\delta u''$ в первом порядке по $\delta u$. В ответ могут входить величины, использованные в C1.}

\MBlock{0.20}{Сократились не зависящие от $\delta u$ слагаемые}

\MBlock{0.40}{$$
\delta u'' = - \left( 1 - \frac{3 G M m^2}{2 R^3 L^2} a^4\right) \delta u 
$$}

\MBlock{-0.20}{Ошибка в знаке или численном коэффициенте}

\QBlock{C3}{0.50}{Пренебрежем слагаемыми, описывающими поле кольца. Запишите в таком приближении решение для $\delta u(\theta)$. Считайте, что $\theta = 0$ отвечает максимальному значению $u$. Выразите ответ через радиус круговой орбиты $a$ и эксцентриситет $e \ll 1$. Покажите, что решение описывает замкнутую орбиту.}

\MBlock{0.10}{Решение вида $u = A \cos (\theta - \theta_0)$}

\MBlock{0.10}{Правильная начальная фаза: $u = A \cos \theta$}

\MBlock{0.10}{$A = e/a$}

\MBlock{0.20}{Явно указано, что функция периодичная с периодом $2\pi$}

\QBlock{C4}{0.50}{Найдите решение уравнения для $\delta u(\theta)$ из $C2$  с теми же начальными условиями, что и в $C3$. Теперь учитывайте вклад кольца. Выразите ответ через радиус орбиты $a$, ее эксцентриситет $e\ll 1$ и $M$, $M_S$, $R$. Значение момента импульса равно найденному в A1.}

\MBlock{0.10}{Подставлено правильное значение $L$}

\MBlock{0.30}{$$
\delta u = \frac{e}{a} \cos \left( \theta\sqrt{1 - \frac{3  M a^3 }{2 M_S R^3 } }\right)
$$}

\MBlock{0.10}{Правильная начальная фаза ($\theta_0 = 0$)}

\QBlock{C5}{0.50}{Из найденного решения следует, что за один оборот орбиты положение перигелия Меркурия меняется на некоторую величину $\delta \theta$. Получите точное выражение для $\delta  \theta$, следующее из решения в $C4$, а также приближенное выражение для $\delta  \theta$ в первом порядке по $M$. Укажите направление смещения перигелия (по направлению вращения Меркурия или против). Выразите ответ через $M$, $M_S$, $R$, $a$.}

\MBlock{0.10}{Использовано выражение для максимумов вида (или его приближенный аналог)$$
\theta = \frac{2\pi n}{\sqrt{1 - \dfrac{3  M a^3 }{2 M_S R^3 }} }.
$$}

\MBlock{0.20}{$$
\delta \theta  = \frac{3 \pi }{2} \frac{M a^3}{M_S R^3} 
$$}

\MBlock{-0.10}{Ошибка в знаке или в численном коэффициенте}

\MBlock{0.20}{Указано, что направление смещения совпадает с направлением движения Меркурия}

\QBlock{С6}{0.70}{В таблице указаны радиусы орбит (считаем их приближенно равным большим полуосям) и массы для планет Солнечной системы. Для каждой из них вычислите сдвиг перигелия Меркурия за один оборот вокруг Солнца. Во всех случаях можете считать, что радиус орбиты планеты много больше радиуса орбиты Меркурия. Выразите ответ в угловых секундах. Укажите две планеты, дающие наибольший вклад в прецессию. \textit{Примечание: }Угловая секунда – единица измерения углов, которая составляет 1/3600 градуса.}

\MBlock{<figure class="table"><table style='margin-left: auto; margin-right: auto;'><tbody><tr><td>Планета</td><td>Вклад в $\delta \theta$,   ‘’</td}{<figure class="table"><table style='margin-left: auto; margin-right: auto;'><tbody><tr><td>Планета}

\MBlock{-0.10}{Если значения не в угловых секундах}

\MBlock{0.21}{Указаны Венера и Юпитер (засчитывается, только если обе планеты выбраны правильно)}

\QBlock{C7}{0.30}{Найдите угол, на который перигелий Меркурия смещается за одно столетие, с учетом вкладов всех планет, приведенных в таблице. Выразите ответ в угловых секундах.}

\MBlock{0.30}{$\Delta \theta = 391'' \text{ за столетие}$ (оценивается только правильное значение, независимо от ошибок в предыдущих пунктах)}

\QBlock{D1}{0.80}{Пусть Меркурий движется в центральном поле,  действующая на него сила равна $\vec{F}  = F_r \vec{e}_r$. Покажите, что производная равна
$$
\frac{d}{dt} \left( \vec{v} \times \vec{L}\right) = B \frac{d}{d t} \vec{e}_r.
$$
Найдите коэффициент $B$, выразите его через $r$, $m$, $F_r$.}

\MBlock{0.10}{$$
\frac{d}{dt} \left( \vec{v} \times \vec{L}\right) = \vec{a} \times \vec{L} 
$$}

\MBlock{0.10}{Второй закон Ньютона}

\MBlock{0.20}{Использована векторная формула $
\vec{L}  = m r^2 \vec{\omega}.
$ или преобразование двойного векторного произведения или продифференцировано $\vec{e}_r$}

\MBlock{0.20}{$B = - r^2 F_r$}

\MBlock{0.20}{Правильный знак (если ответ правильный)}

\QBlock{D2}{0.50}{Найдите производную по времени вектора Лапласа  для Меркурия, который движется в поле Солнца и кольца. Выразите ответ через $m$, $M$, $R$,  $r$, $G$, $\dfrac{d \vec{e}_r}{dt}$.}

\MBlock{0.10}{Использовано выражение для $
F_r = - \dfrac{GM_S m}{r^2 } + \dfrac{GMm}{2 R^3} r.
$}

\MBlock{0.20}{$$
\frac{d \vec{A}}{dt} = - \frac{GMm}{2 R^3} r^3\frac{d \vec{e}_r}{dt} 
$$}

\MBlock{0.20}{Правильный знак (если ответ правильный)}

\QBlock{D3}{1.00}{Пусть $\theta$ - полярный угол между радиус-вектором Меркурия $\vec{r}$ и направлением оси $x$. Угол $\theta$ возрастает при движении Меркурия. Вычислите производные компонент вектора Лапласа по $\theta$, $A_x'$ и $A_y'$. Выразите ответ через $m$, $M$, $R$,  $r$, $G$, $\theta$.}

\MBlock{0.30}{Переход от производных по времени к производным по углу}

\MBlock{0.10}{Записаны проекции $\vec{e}_r$ на оси}

\MBlock{0.10}{Производные проекций $\vec{e}_r$ по углам}

\MBlock{2 $\times$ 0.15}{$$
\frac{d A_x}{d\theta} =  \frac{GMm}{2 R^3} r^3 \sin \theta, \quad \frac{d A_y}{d\theta} =   - \frac{GMm}{2 R^3} r^3 \cos \theta
$$}

\MBlock{2 $\times$ 0.10}{Правильные знаки}

\QBlock{D4}{1.00}{Найдите изменение вектора Лапласа $\Delta \vec{A}$ за период. Укажите проекции этого изменения на оси, указанные в предыдущем пункте. При вычислении считайте, что Меркурий движется по эллиптической орбите. Выразите ответ через большую полуось орбиты $a$, эксцентриситет $e$, $G$, $m$, $M$, $M_S$.}

\MBlock{0.20}{Использовано уравнение эллиптической орбиты $r = p/(1 + e \cos \theta)$}

\MBlock{0.10}{Выражение для параметра $p = a(1 - e^2)$}

\MBlock{2 $\times$ 0.10}{Записаны интегралы для $\Delta A_x$, $\Delta A_y$}

\MBlock{0.20}{Обосновано, что $\Delta A_x = 0$.}

\MBlock{0.20}{$$ \Delta A_y = \frac{3 \pi GM m a^3}{2 R^3}  e(1- e^2)^{1/2}$$}

\MBlock{0.10}{Правильный знак $\Delta A_y$ (если модуль верен)}

\QBlock{D5}{0.50}{Найдите поворот направления на перигелий Меркурия за период $\Delta \theta$. Выразите ответ через $G$, $M_S$, $M$, $R$, $a$, $e$. На сколько процентов изменится результат пункта C7 для смещения перигелия за счет учета эксцентриситета орбиты Меркурия?}

\MBlock{0.10}{$\Delta \theta = \Delta A_y/A$}

\MBlock{0.30}{$$
\Delta \theta =  \frac{3\pi }{2} \frac{M a^3}{M_S R^3} (1 - e^2)^{1/2}.
$$}

\MBlock{0.10}{Смещение перигелия меньше на $2 \text{\%}$}

\end{document}