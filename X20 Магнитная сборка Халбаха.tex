
%\documentstyle[12pt,russian,amsthm,amsmath,amssymb]{article}
\documentclass[a4paper,11pt,twoside]{article}
\usepackage[left=14mm, top=10mm, right=14mm, bottom=10mm, nohead, nofoot]{geometry}
\usepackage{amsmath, amsfonts, amssymb, amsthm} % стандартный набор AMS-пакетов для математ. текстов
\usepackage{mathtext}
\usepackage[utf8]{inputenc} % кодировка utf8
\usepackage[russian]{babel} % русский язык
\usepackage[pdftex]{graphicx} % графика (картинки)
\usepackage{tikz}
\usepackage{fancyhdr,pageslts} % настройка колонтитулов
\usepackage{enumitem} % работа со списками
\usepackage{multicol} % работа с таблицами
%\usepackage{pscyr} % красивый шрифт
\usepackage{pgfornament} % красивые рюшечки и вензеля
\usepackage{ltxgrid} % управление написанием текста в две колонки
\usepackage{lipsum} % стандартный текст
\usepackage{tcolorbox} % рамка вокруг текста
\tcbuselibrary{skins}
% ----------------------------------------

\newcommand\ProblemName{Магнитная сборка Халбаха}

\newcommand\Source{X20}

\newcommand\Type{Условие задачи}

\newcommand\MyTextLeft{Президентский ФМЛ 239, г.~Санкт-Петербург}
\newcommand\MyTextRight{Использованы материалы сайта pho.rs}
\newcommand\MyHeading{Учебно-тренировочные сборы по физике}
% ----------------------------------------

% настройки полей
\geometry{
	left=12mm,
	top=21mm,
	right=15mm,
	bottom=26mm,
	marginparsep=0mm,
	marginparwidth=0mm,
	headheight=22pt,
	headsep=2mm,
	footskip=7mm}
% ----------------------------------------

% настройки колонтитулов
\pagestyle{fancy}

\fancypagestyle{style}{
	\fancyhf{}
	\fancyhead[L]{{\Large{\FancyTitle}}\\\vskip -5pt \dotfill}
	\fancyhead[R]{{\Large{\textbf{\Type}}}\\\vskip -5pt \dotfill}
	\renewcommand{\headrulewidth}{0pt}
	\renewcommand{\footrulewidth}{0pt}
	\fancyfoot[C]{\pgfornament[width=2em,anchor=south]{72}\hspace{1mm}
		{Страница \textbf{\thepage} из \textbf{\pageref{VeryLastPage}}}\hspace{2mm}
		\pgfornament[width=2em,symmetry=v,anchor=south]{72}\\ \vskip2mm
		{\small{\textit{\MyTextLeft\hfill\MyTextRight}}}}
}

\fancypagestyle{plain}{
	\fancyhf{}
	\renewcommand{\headrulewidth}{0pt}
	\renewcommand{\footrulewidth}{0pt}
	\fancyhead[C]{{\Large{\textit{\MyHeading}}}\\\vskip -5pt \dotfill}
	\fancyfoot[C]{\pgfornament[width=2em,anchor=south]{72}\hspace{1mm}
		{Страница \textbf{\thepage} из \textbf{\pageref{VeryLastPage}}}\hspace{2mm}
		\pgfornament[width=2em,symmetry=v,anchor=south]{72}\\ \vskip2mm
		{\small{\textit{\MyTextLeft\hfill\MyTextRight}}}}
}
% ----------------------------------------

% другие настройки
\pagenumbering{arabic}
\setlist[enumerate,itemize]{leftmargin=0pt,itemindent=2.7em,itemsep=0cm}
% ----------------------------------------

% собственные команды
\newcommand{\FancyTitle}{\textbf{\Source} --- \ProblemName}
\newcommand{\Title}{\begin{center}{\huge{\textbf{\Source} --- \ProblemName}}\end{center}}
\newcommand{\Chapter}[1]{\vskip5pt{\Large{\textbf{#1}}}\vskip5pt}
\newcommand{\QText}[1]{#1}
\newcommand{\QBlock}[3]{
	\begin{tcolorbox}[left=2mm,top=2mm,bottom=1mm,right=2mm,colback=white]
		\begin{tcolorbox}[enhanced,colframe=ProcessBlue,colback=ProcessBlue!30!white,
			frame style={opacity=0.7},interior style={opacity=1.0},
			nobeforeafter,tcbox raise base,shrink tight,extrude by=1.7mm,width=1.5cm]
			\textbf{#1\textsuperscript{#2}}
		\end{tcolorbox}\hspace{3mm}#3
	\end{tcolorbox}
}
\newcommand{\QPicture}[4]{
	\begin{figure}[H]
		\centering
		\includegraphics[width=0.35\linewidth]{#1}
		\caption{#3}
	\end{figure}
	
	#4
}
\newcommand{\ABlock}[1]{
	\vskip2mm
	\begin{tcolorbox}[enhanced,colframe=Magenta,colback=Magenta!15!white,
		frame style={opacity=0.5},interior style={opacity=1.0},
		nobeforeafter,tcbox raise base,shrink tight,extrude by=1.7mm,width=1.6cm]
		\textbf{Ответ:}
	\end{tcolorbox}\hspace{3mm}#1
}
\newcommand{\MBlock}[2]{
	\begin{tcolorbox}[enhanced,colframe=Yellow,colback=Yellow!15!white,
		frame style={opacity=0.5},interior style={opacity=1.0},
		nobeforeafter,tcbox raise base,shrink tight,extrude by=1.7mm,width=1.1cm]
		\textbf{#1}
	\end{tcolorbox}\hspace{3mm}#2
}
\newcommand{\MMBlock}[3]{
	\begin{tcolorbox}[enhanced,colframe=Yellow,colback=Yellow!15!white,
		frame style={opacity=0.5},interior style={opacity=1.0},
		nobeforeafter,tcbox raise base,shrink tight,extrude by=1.7mm,width=1.1cm]
		\textbf{#1}
	\end{tcolorbox}\hspace{3mm}
	\begin{tcolorbox}[enhanced,colframe=Orange,colback=Orange!15!white,
		frame style={opacity=0.5},interior style={opacity=1.0},
		nobeforeafter,tcbox raise base,shrink tight,extrude by=1.7mm,width=0.8cm]
		\textbf{#2}
	\end{tcolorbox}\hspace{3mm}#3
}
% ----------------------------------------


\begin{document}
	
	% настройки
	\pagestyle{style}\thispagestyle{plain}
	\Title
	% ----------------------------------------
	
	%\vskip5mm
	%\centering{\pgfornament[width=5cm,anchor=south]{89}}
	
	% смысловая часть


\QText{Магнитная сборка Халбаха - особая конфигурация постоянных магнитов, характеризующаяся тем, что магнитное поле с одной из её сторон практически полностью отсутствует благодаря особому расположению элементов сборки.
В этой задаче мы исследуем это явление.}

\Chapter{Магнитные диполи}

\QText{Диполем называется точечный магнитный элемент (например, маленькая петля с током. Ее дипольный момент $I\vec S$, где $I$ - это ток, бегущий по петле, a $\vec S$ - ориентированная площадь). Поле, создаваемое магнитным диполем, описывается следующей формулой:
 $$\vec B (\vec r) = \frac {\mu_0}{4\pi} \left(\frac{3\vec r (\vec m \cdot \vec r )} {r^5} - \frac { \vec m} {r^3}\right),
$$
где $m$ -  дипольный момент, $\vec r$ - радиус-вектор точки в пространстве относительно диполя. $\mu_0=4\pi\cdot 10^{-7} \text{Гн/м}$. На рисунке изображены диполь, вектор и угол между ними.}

\QPicture{X20 Магнитная сборка Халбаха_files/162.jpeg}{"max-width:500px;display:block;margin-left: auto;margin-right: auto;"}{}{}

\QBlock{A1}{0.50}{Магнитное поле ослабляется с увеличением расстояния, для заданного угла $\theta$ найдите зависимость магнитного поля $B$ на расстоянии $r$ от диполя.}

\Chapter{Магнитная шайба}

\QText{Рассмотрим лёгкий магнит, представляющий собой плоский цилиндр радиуса $R$ и толщиной $h \ll R$ c поверхностной плотностью магнитного момента $\vec\sigma$. Вектор поверхностной плотности ориентирован вдоль оси диска.}

\QPicture{X20 Магнитная сборка Халбаха_files/166.jpeg}{"max-width:500px;display:block;margin-left: auto;margin-right: auto;"}{}{}

\QBlock{B1}{1.50}{Выразите магнитное поле $B(y)$ вдоль оси, перпендикулярной магниту, на расстоянии $y$ от центра.}

\QText{Если приблизить маленький круглый магнит к металлической двери холодильника, магнит притягивается к ней с силой, зависящей от размеров и типа материала магнита, а также толщины двери. В этом пункте считайте толщину двери много большей линейных размеров магнита.

Рассмотрим цилиндрический магнит с объемной плотностью дипольного момента $\rho = 1.05 \cdot 10^6 \frac {\text{ Тл} \cdot \text{м}}{\text{Гн}}$ толщиной $t=2 \text{мм}$ и диаметром $D=20 \text{мм}$.}

\QBlock{B2}{0.50}{Оцените величину магнитного поля  вблизи поверхности магнита. Ответ выразите через величины $t, D, \rho, \mu_0$.}

\QPicture{X20 Магнитная сборка Халбаха_files/170.jpeg}{"max-width:500px;display:block;margin-left: auto;margin-right: auto;"}{}{}

\QText{Чтобы определить силу взаимодействия магнита и двери холодильника, необходимо воспользоваться законом сохранения энергии. Когда магнит отрывают от двери, между ними возникает поле, приблизительно равное полю с другой стороны магнита. Остальная часть поля (включая и то, что внутри двери) почти не меняется.  Выражение для объёмной плотности энергии магнитного поля в воздухе: $w=\frac{B^2}{2\mu_0}$.}

\QBlock{B3}{0.50}{Получите выражение и численное значение силы взаимодействия $F_0$ между дверью и прижатым к ней магнитом, также вычислите давление $P_0$ магнита на дверь.}

\Chapter{Магнитная сборка Халбаха}

\QText{Для того чтобы получить выражение магнитного поля в сборке Халбаха, найдем поле длинного ряда магнитов, как показано на рисунке.
Линейная плотность дипольного момента ряда $\rho_L$, направление - вдоль оси $y$.}

\QPicture{X20 Магнитная сборка Халбаха_files/175.jpeg}{"max-width:500px;display:block;margin-left: auto;margin-right: auto;"}{}{}

\QBlock{C1}{2.00}{Запишите выражение для поля $\vec B(\vec r_0, y)$ которое создает ряд магнитов. (Для удобства поле выражается и через $\vec r_0$, и через $y$, хотя технически $y = (\vec r_0)_y$.)}

\QText{В плоской магнитной сборке Халбаха направление поляризации маленького элемента площади непрерывно вращается. Его положение меняется в соответствии с формулой:
$$\beta(x,z)=\beta_0+k\cdot x, k=2\pi/\lambda,  t\ll\lambda$$

Здесь $\beta$ - это угол между направлением диполя и перпендикуляром к плоскости, этот угол вращается в плоскости $xy$. Длина $\lambda$ - это шаг сборки, а $t$ - толщина сборки.}

\QPicture{X20 Магнитная сборка Халбаха_files/178.jpeg}{"max-width:500px;display:block;margin-left: auto;margin-right: auto;"}{}{}

\QBlock{C2}{1.00}{Найдите магнитного поля с двух сторон от сборки. Ответ дать в виде некоторого интеграла.}

\QText{Один из этих интегралов нужен для решения этого пункта:

$$
    \int_{-\pi/2}^{\pi/2} dx\cdot\cos(2x)\cdot\cos(c\cdot\tan x)=\frac{c \cdot \pi} {e^c}=\int^{\pi/2}_{-\pi/2}dx\cdot\sin(2x)\cdot\sin(c\cdot\tan x)   
$$
$$
    \frac{\pi}{e^d} = \int_{-\infty}^{\infty} dx\cdot\frac{\cos x}{d^2+x^2}
$$
$$
    \frac{\pi}{a\cdot e^a}=
    \int^{\infty}_{-\infty} dx\cdot
    \frac {2x\cdot\sin(2x)}
    {(a^2+x^2)^2}
$$
$$
    \frac {(b+1)\pi}{e^b}=
    \int^{\infty}_{-\infty}dx
    \cdot \frac{2b^3x\cdot \cos(x)}{(b^2+x^2)^2}
$$}

\QBlock{C3}{1.00}{Покажите, что с одной стороны идеальной сборки магнитное поле стремится к нулю.}

\QBlock{C4}{1.00}{Запишите выражение для поля с другой стороны.}

\QBlock{C5}{1.50}{На основании выражения поля найдите среднее давление $Р$ такого магнита на дверь холодильника. Возьмите следующие параметры: толщина $t=0.5 \text{мм}$, объемная плотность магнитного диполя $\rho = 2 \cdot 10^5 \frac {\text{Тл} \cdot \text{м}} {\text{Гн}}$, шаг сборки $\lambda=5 \text{мм}$.}

\QBlock{C6}{0.50}{Найдите соотношение между давлением, которое создает магнитная сборка Халбаха и давлением, которое создает обычный магнит из того же материала, с теми же радиусом и толщиной. Здесь тоже следует пренебречь эффектами на периметре кружка и и толщиной магнита.}

\QText{Вы могли подумать, что ни в чем более инновационном, чем  магнитики для холодильника, сборка Халбаха не применяется. Однако компания Hendo разработала на ее основе разработала целый ховерборд (тот самый, что был в фильме "Назад в будущее").  Используя кольцевую сборку Халбаха из электромагнитов, Hendo добились подъемной силы достаточной, чтобы заставить человека средней массы подняться над землей! Но только над медью.}

\QPicture{X20 Магнитная сборка Халбаха_files/224.jpeg}{"max-width:500px;display:block;margin-left: auto;margin-right: auto;"}{}{}

\end{document}