
%This file is part of Get pho.rs!

%Get pho.rs! is free software: you can redistribute it and/or modify it under the terms of the GNU General Public License as published by the Free Software Foundation, either version 3 of the License, or (at your option) any later version.

%Get pho.rs! is distributed in the hope that it will be useful, but WITHOUT ANY WARRANTY; without even the implied warranty of MERCHANTABILITY or FITNESS FOR A PARTICULAR PURPOSE. See the GNU General Public License for more details.

%You should have received a copy of the GNU General Public License along with Foobar. If not, see <https://www.gnu.org/licenses/>.

%\documentstyle[12pt,russian,amsthm,amsmath,amssymb]{article}
\documentclass[a4paper,11pt,twoside]{article}
\usepackage[left=14mm, top=10mm, right=14mm, bottom=10mm, nohead, nofoot]{geometry}
\usepackage{amsmath, amsfonts, amssymb, amsthm} % стандартный набор AMS-пакетов для математ. текстов
\usepackage{mathtext}
\usepackage[utf8]{inputenc} % кодировка utf8
\usepackage[russian]{babel} % русский язык
\usepackage[pdftex,dvipsnames]{xcolor} % работа с цветами
\usepackage[pdftex]{graphicx} % графика (картинки)
\usepackage{tikz} % рисунки
\usepackage{fancyhdr,pageslts} % настройка колонтитулов
\usepackage{enumitem} % работа со списками
\usepackage{multicol} % работа с таблицами
%\usepackage{pscyr} % красивый шрифт
\usepackage{pgfornament} % красивые рюшечки и вензеля
\usepackage{ltxgrid} % управление написанием текста в две колонки
\usepackage{lipsum} % стандартный текст
\usepackage{tcolorbox} % рамка вокруг текста
\usepackage{float} % для корректного размещения картинок
\tcbuselibrary{skins}
% ----------------------------------------

\newcommand\ProblemName{Физика дождевых капель}

\newcommand\Source{X24}

\newcommand\Type{Разбалловка}

% настройки полей
\geometry{
	left=12mm,
	top=21mm,
	right=15mm,
	bottom=26mm,
	marginparsep=0mm,
	marginparwidth=0mm,
	headheight=22pt,
	headsep=2mm,
	footskip=7mm}
% ----------------------------------------

% настройки колонтитулов
\pagestyle{fancy}

\fancypagestyle{style}{
	\fancyhf{}
	\fancyhead[L]{{\Large{\FancyTitle}}\\\vskip -5pt \dotfill}
	\fancyhead[R]{{\Large{\textbf{\Type}}}\\\vskip -5pt \dotfill}
	\renewcommand{\headrulewidth}{0pt}
	\renewcommand{\footrulewidth}{0pt}
	\fancyfoot[C]{\pgfornament[width=2em,anchor=south]{72}\hspace{1mm}
		{Страница \textbf{\thepage} из \textbf{\pageref{VeryLastPage}}}\hspace{2mm}
		\pgfornament[width=2em,symmetry=v,anchor=south]{72}\\ \vskip2mm
		{\small{\textit{Условие собрано и подготовлено в Президентском ФМЛ №239 г.~Санкт-Петербурга}}}}
}

\fancypagestyle{plain}{
	\fancyhf{}
	\renewcommand{\headrulewidth}{0pt}
	\renewcommand{\footrulewidth}{0pt}
	\fancyhead[C]{{\Large{\textit{Учебно-тренировочные сборы к X23}}}\\\vskip -5pt \dotfill}
	\fancyfoot[C]{\pgfornament[width=2em,anchor=south]{72}\hspace{1mm}
		{Страница \textbf{\thepage} из \textbf{\pageref{VeryLastPage}}}\hspace{2mm}
		\pgfornament[width=2em,symmetry=v,anchor=south]{72}\\ \vskip2mm
		{\small{\textit{Условие собрано и подготовлено в Президентском ФМЛ №239 г.~Санкт-Петербурга}}}}
}
% ----------------------------------------

% другие настройки
\pagenumbering{arabic}
\setlist[enumerate,itemize]{leftmargin=0pt,itemindent=2.7em,itemsep=0cm}
% ----------------------------------------

% собственные команды
\newcommand{\FancyTitle}{\textbf{\Source} --- \ProblemName}
\newcommand{\Title}{\begin{center}{\huge{\textbf{\Source} --- \ProblemName}}\end{center}}
\newcommand{\Chapter}[1]{\vskip5pt{\Large{\textbf{#1}}}\vskip5pt}
\newcommand{\QText}[1]{#1}
\newcommand{\QBlock}[3]{
	\begin{tcolorbox}[left=4mm,top=3mm,bottom=2mm,right=4mm,colback=white]
		\begin{tcolorbox}[enhanced,colframe=blue,colback=blue!10!white,
			frame style={opacity=0.3},interior style={opacity=1.0},
			nobeforeafter,tcbox raise base,shrink tight,extrude by=1.7mm,width=1.5cm]
			\textbf{#1\textsuperscript{#2}}
		\end{tcolorbox}\hspace{3mm}#3
	\end{tcolorbox}
}
\newcommand{\QPicture}[4]{\QText{#4}  \includegraphics{#1}}
\newcommand{\ABlock}[1]{#1}
\newcommand{\MBlock}[2]{#1 #2}
\newcommand{\MMBlock}[3]{#1 #2 #3}
% ----------------------------------------


\begin{document}
	
	% настройки
	\pagestyle{style}\thispagestyle{plain}
	\Title
	% ----------------------------------------
	
	%\vskip5mm
	%\centering{\pgfornament[width=5cm,anchor=south]{89}}

\QBlock{A1}{1.00}{Найдите изменение свободной энергии водяного пара, если из него образовать каплю радиуса $r$. Выразите ответ через $r$, $\sigma$, $\varphi$, $R$, $T$, $\rho_L$, $\mu$.}

\MBlock{0.30}{Записан поверхностный вклад в свободную энергию $
\Delta G_{surf}= 4 \pi \sigma r^2. 
$}

\MBlock{0.20}{Найдено количество вещества в капле $
\nu = \frac{4\pi \rho_L r^3}{3 \mu}.$}

\MBlock{0.30}{Объемный вклад в свободную энергию $- \frac{4\pi \rho_L }{3 \mu}r^3 R T \ln \varphi$}

\MBlock{0.20}{Правильные знаки}

\QBlock{A2}{0.80}{Найдите критическое значение радиуса капли $r_c$, при котором $\Delta G$ максимально, а также соответствующее значение $\Delta G_с$. Выразите ответ через  $\sigma$, $\varphi$, $R$, $T$, $\rho_L$, $\mu$. Найдите численное значение $r_c$ при $\varphi = 1.01$.}

\MBlock{0.20}{Вычислена производная $\partial \Delta G/\partial r$}

\MBlock{0.20}{$
r_c = \frac{2 \sigma \mu}{\rho_L R T \ln \varphi}.
$}

\MBlock{0.10}{$r_c = 1.15 \cdot 10^{-7} \text{м}$}

\MBlock{0.30}{$$\Delta G_c = \frac{16 \pi}{3} \frac{\sigma^3 \mu^2}{\rho_L^2 R^2 T^2 \ln^2 \varphi}$$}

\MBlock{-0.10}{Ошибка в безразмерном численном коэффициенте в $\Delta G_c$}

\QBlock{A3}{0.70}{Рассмотрим каплю критического радиуса $r_c$. Определите время $\tau$, за которое количество молекул в ней увеличится на $g$. Выразите ответ через $r_c$, $g$, $p_s$, $m$, $k$, $T$, $\varphi$. Считайте, что в процессе роста радиус капли не меняется, испарением молекул из капли можно пренебречь.
Известно, что на площадь $dS$ поверхности за время $dt$ попадает
$$
dN = dt dS  \frac{p_v}{\sqrt{2\pi m k T}}
$$
молекул. Здесь $p_v$ - давление пара, $m$ - масса молекул, $T$ - температура газа.}

\MBlock{0.10}{$p_v = p_s \varphi$}

\MBlock{0.30}{Записан полный поток молекул в каплю}

\MBlock{0.30}{$$
\tau = \frac{g \sqrt{2\pi m kT}}{4\pi r_c^2 p_s \varphi}.
$$}

\MBlock{-0.20}{Ошибка в численном коэффициенте}

\QBlock{A4}{0.60}{Найдите количество капель $J$, которые образуются в единицу времени в единице объема перенасыщенного водяного пара. Выразите ответ через $\sigma$, $\varphi$, $p_s$, $r_c$, $T$, $g$.}

\MBlock{0.10}{Использована формула $J = n_c/ \tau$}

\MBlock{0.40}{$$
J =  \frac{4\pi r_c^2 }{ \sqrt{2\pi m kT}} \frac{p_s^2 \varphi^2}{k T} \frac{1}{g}\exp\left( -  \frac{16 \pi}{3 kT} \frac{\sigma^3 \mu^2}{\rho_L^2 R^2 T^2 \ln^2 \varphi}\right) = 
\frac{4\pi r_c^2 }{ \sqrt{2\pi m kT}} \frac{p_s^2 \varphi^2}{k T} \frac{1}{g}\exp\left( -  \frac{4\pi r_c^2 \sigma}{3 k T}\right).
$$}

\MBlock{0.10}{Концентрация выражена через давление $p_s$}

\MBlock{-0.20}{Ошибка в численном коэффициенте или в ответе остались не приведенные в условии величины}

\QBlock{A5}{0.90}{Из результатов предыдущего пункта следует, что скорость образования капель очень сильно зависит от коэффициента перенасыщения пара. Определите численно значение коэффициента перенасыщения пара $\varphi$, при котором при температуре $T = 283 \text{К}$ в $1 \text{см}^3$ воздуха рождается одна капля в секунду. Считайте, что $g = 100$. Остальные численные данные приведены в начале задачи.}

\MBlock{2 $\times$ 0.20}{Найдены численные значения коэффициента перед экспонентой ($J_0$) и постоянной в экспоненте $A$, или аналогичные им}

\MBlock{0.50}{Численный ответ $\varphi \in [3.8, 3.9]$}

\QBlock{B1}{0.80}{Для насыщенного пара, находящегося в равновесии с жидкостью, выразите производную давления по температуре $dp_s/dT$ через $p_s$, $L$, $R$, $T$, $\mu$. Используя полученный результат, найдите относительное изменение плотности насыщенного водяного пара $\Delta \rho_s/\rho_s$ при малом изменении температуры $\Delta T$. Выразите ответ через $\Delta T$, $T$, $L$, $\mu$, $R$. Вы можете использовать связь малых изменений давления, плотности и температуры идеального газа
$$
\frac{\Delta p_s}{p_s} = \frac{\Delta \rho_s}{\rho_s} +\frac {\Delta T}{T}.
$$}

\MBlock{0.30}{Использовано или получено уравнение Клапейрона-Клаузиуса в любом виде}

\MBlock{0.20}{$$
\frac{dp_s}{dT} = \frac{L \mu p_s}{R T^2}. 
$$}

\MBlock{0.30}{$$
\frac{\Delta \rho_s}{\rho_s} = \frac{\Delta T}{T} \left( \frac{\mu L}{R T} - 1\right).
$$}

\QBlock{B2}{0.20}{Выразите $dQ/dt$ через $dM/dt$ и $L$.}

\MBlock{0.20}{$$
\frac{dQ}{dt} = L \frac{dM}{dt}
$$}

\QBlock{B3}{0.30}{Используя результат предыдущего пункта и уравнение теплопроводности, выразите разность температур капли и атмосферы, $T_r- T$, через $dM/dt$, а также $r$, $L$, $K$.}

\MBlock{0.10}{$$
T_r -  T = \frac{1}{4 \pi r K} \frac{dQ}{dt} 
$$}

\MBlock{0.20}{$$
T_r -  T = \frac{L}{4 \pi r K} \frac{dM}{dt}.
$$}

\QBlock{B4}{0.30}{Будем считать, что вблизи поверхности капли плотность водяного пара равна плотности насыщенного пара при температуре капли. Считая разности температур и плотностей малыми и используя результаты $B1$, $B3$ выразите отношение $(\rho_r - \rho_s)/\rho_s$ ($\rho_r$ - давление пара вблизи поверхности капли) через $L$, $r$, $K$, $\mu$, $R$, $T$ и $dM/dt$.}

\MBlock{0.30}{$$
\frac{\rho_r - \rho_s}{\rho_s} =  \left( \frac{\mu L}{R T} - 1\right)\frac{L}{4 \pi r K T} \frac{dM}{dt}.
$$}

\QBlock{B5}{0.30}{Используя уравнение диффузии, выразите отношение $(\rho_r - \rho_v)/\rho_s$ через $dM/dt$, $r$, $D$, $\rho_s$.}

\MBlock{0.30}{$$
\frac{\rho_r - \rho_v}{\rho_s} = - \frac{1}{4 \pi r \rho_s D} \frac{dM}{dt}
$$}

\MBlock{-0.10}{Ошибка в знаке}

\QBlock{B6}{0.60}{Исключив из ответов в двух предыдущих пунктах плотность пара вблизи поверхности капли $\rho_r$, получите выражение для $dM/dt$. Выразите ответ через $\varphi$,  $\mu$, $R$, $T$, $D$, $p_s$, $L$, $K$, $r$.}

\MBlock{0.30}{Получено корректное соотношение, не содержащее $\rho_r$}

\MBlock{0.30}{$$
\frac{dM}{dt} = \frac{4 \pi r (\varphi - 1)}{\left( \frac{\mu L}{R T} - 1\right)\frac{L}{ K T} +\frac{R T}{\mu p_s D} } 
$$}

\MBlock{-0.10}{Не подставлено значение $\rho_s$}

\QBlock{B7}{0.50}{Скорость увеличения радиуса капли имеет вид
$$
\frac{dr}{dt} = \frac{\xi}{r^k}.
$$
Определите $k$ и $\xi$, выразите ответ через $\varphi $, $\rho_L$, $\mu$, $R$, $T$, $D$, $p_s$, $L$, $K$.}

\MBlock{0.20}{$dr/dt$ выражено через $dM/dt$}

\MBlock{0.10}{$k = 1$}

\MBlock{0.20}{$$
\quad \xi = \frac{\varphi - 1}{\left( \frac{\mu L}{R T} - 1\right)\frac{L}{ K T} +\frac{R T}{\mu p_s D} } \frac{1}{\rho_L}.
$$}

\QBlock{B8}{0.50}{Найдите зависимость радиуса капли от времени. Начальный радиус капли равен $r_0$. Выразите ответ через $r_0$, $\xi$, $t$.}

\MBlock{0.20}{Уравнение корректно проинтегрировано}

\MBlock{0.30}{$$
r(t) = \sqrt{r_0^2 + 2 \xi t}.
$$}

\MBlock{-0.10}{Ошибка в численном коэффициенте}

\QBlock{B9}{0.50}{Пусть начальный радиус капли равен $r_0 = 0.7 \text{мкм}$. Найдите численное значение времени, за которое она вырастет до размера $r_1 = 10 \text{мкм}$ при коэффициенте перенасыщения $\varphi = 1.1$. Остальные численные значения приведены в начале этой части.}

\MBlock{0.30}{$$
t = \frac{r_1^2 - r_0^2}{2 \xi} 
$$}

\MBlock{0.20}{$$
t = 5.50 \text{с}.
$$}

\end{document}