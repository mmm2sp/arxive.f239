
%\documentstyle[12pt,russian,amsthm,amsmath,amssymb]{article}
\documentclass[a4paper,11pt,twoside]{article}
\usepackage[left=14mm, top=10mm, right=14mm, bottom=10mm, nohead, nofoot]{geometry}
\usepackage{amsmath, amsfonts, amssymb, amsthm} % стандартный набор AMS-пакетов для математ. текстов
\usepackage{mathtext}
\usepackage[utf8]{inputenc} % кодировка utf8
\usepackage[russian]{babel} % русский язык
\usepackage[pdftex]{graphicx} % графика (картинки)
\usepackage{tikz}
\usepackage{fancyhdr,pageslts} % настройка колонтитулов
\usepackage{enumitem} % работа со списками
\usepackage{multicol} % работа с таблицами
%\usepackage{pscyr} % красивый шрифт
\usepackage{pgfornament} % красивые рюшечки и вензеля
\usepackage{ltxgrid} % управление написанием текста в две колонки
\usepackage{lipsum} % стандартный текст
\usepackage{tcolorbox} % рамка вокруг текста
\tcbuselibrary{skins}
% ----------------------------------------

\newcommand\ProblemName{Колебания в заряженном цилиндре}

\newcommand\Source{X22-T6}

\newcommand\Type{Решение}

\newcommand\MyTextLeft{Президентский ФМЛ 239, г.~Санкт-Петербург}
\newcommand\MyTextRight{Использованы материалы сайта pho.rs}
\newcommand\MyHeading{Учебно-тренировочные сборы по физике}
% ----------------------------------------

% настройки полей
\geometry{
	left=12mm,
	top=21mm,
	right=15mm,
	bottom=26mm,
	marginparsep=0mm,
	marginparwidth=0mm,
	headheight=22pt,
	headsep=2mm,
	footskip=7mm}
% ----------------------------------------

% настройки колонтитулов
\pagestyle{fancy}

\fancypagestyle{style}{
	\fancyhf{}
	\fancyhead[L]{{\Large{\FancyTitle}}\\\vskip -5pt \dotfill}
	\fancyhead[R]{{\Large{\textbf{\Type}}}\\\vskip -5pt \dotfill}
	\renewcommand{\headrulewidth}{0pt}
	\renewcommand{\footrulewidth}{0pt}
	\fancyfoot[C]{\pgfornament[width=2em,anchor=south]{72}\hspace{1mm}
		{Страница \textbf{\thepage} из \textbf{\pageref{VeryLastPage}}}\hspace{2mm}
		\pgfornament[width=2em,symmetry=v,anchor=south]{72}\\ \vskip2mm
		{\small{\textit{\MyTextLeft\hfill\MyTextRight}}}}
}

\fancypagestyle{plain}{
	\fancyhf{}
	\renewcommand{\headrulewidth}{0pt}
	\renewcommand{\footrulewidth}{0pt}
	\fancyhead[C]{{\Large{\textit{\MyHeading}}}\\\vskip -5pt \dotfill}
	\fancyfoot[C]{\pgfornament[width=2em,anchor=south]{72}\hspace{1mm}
		{Страница \textbf{\thepage} из \textbf{\pageref{VeryLastPage}}}\hspace{2mm}
		\pgfornament[width=2em,symmetry=v,anchor=south]{72}\\ \vskip2mm
		{\small{\textit{\MyTextLeft\hfill\MyTextRight}}}}
}
% ----------------------------------------

% другие настройки
\pagenumbering{arabic}
\setlist[enumerate,itemize]{leftmargin=0pt,itemindent=2.7em,itemsep=0cm}
% ----------------------------------------

% собственные команды
\newcommand{\FancyTitle}{\textbf{\Source} --- \ProblemName}
\newcommand{\Title}{\begin{center}{\huge{\textbf{\Source} --- \ProblemName}}\end{center}}
\newcommand{\Chapter}[1]{\vskip5pt{\Large{\textbf{#1}}}\vskip5pt}
\newcommand{\QText}[1]{#1}
\newcommand{\QBlock}[3]{
	\begin{tcolorbox}[left=2mm,top=2mm,bottom=1mm,right=2mm,colback=white]
		\begin{tcolorbox}[enhanced,colframe=ProcessBlue,colback=ProcessBlue!30!white,
			frame style={opacity=0.7},interior style={opacity=1.0},
			nobeforeafter,tcbox raise base,shrink tight,extrude by=1.7mm,width=1.5cm]
			\textbf{#1\textsuperscript{#2}}
		\end{tcolorbox}\hspace{3mm}#3
	\end{tcolorbox}
}
\newcommand{\QPicture}[4]{
	\begin{figure}[H]
		\centering
		\includegraphics[width=0.35\linewidth]{#1}
		\caption{#3}
	\end{figure}
	
	#4
}
\newcommand{\ABlock}[1]{
	\vskip2mm
	\begin{tcolorbox}[enhanced,colframe=Magenta,colback=Magenta!15!white,
		frame style={opacity=0.5},interior style={opacity=1.0},
		nobeforeafter,tcbox raise base,shrink tight,extrude by=1.7mm,width=1.6cm]
		\textbf{Ответ:}
	\end{tcolorbox}\hspace{3mm}#1
}
\newcommand{\MBlock}[2]{
	\begin{tcolorbox}[enhanced,colframe=Yellow,colback=Yellow!15!white,
		frame style={opacity=0.5},interior style={opacity=1.0},
		nobeforeafter,tcbox raise base,shrink tight,extrude by=1.7mm,width=1.1cm]
		\textbf{#1}
	\end{tcolorbox}\hspace{3mm}#2
}
\newcommand{\MMBlock}[3]{
	\begin{tcolorbox}[enhanced,colframe=Yellow,colback=Yellow!15!white,
		frame style={opacity=0.5},interior style={opacity=1.0},
		nobeforeafter,tcbox raise base,shrink tight,extrude by=1.7mm,width=1.1cm]
		\textbf{#1}
	\end{tcolorbox}\hspace{3mm}
	\begin{tcolorbox}[enhanced,colframe=Orange,colback=Orange!15!white,
		frame style={opacity=0.5},interior style={opacity=1.0},
		nobeforeafter,tcbox raise base,shrink tight,extrude by=1.7mm,width=0.8cm]
		\textbf{#2}
	\end{tcolorbox}\hspace{3mm}#3
}
% ----------------------------------------


\begin{document}
	
	% настройки
	\pagestyle{style}\thispagestyle{plain}
	\Title
	% ----------------------------------------
	
	%\vskip5mm
	%\centering{\pgfornament[width=5cm,anchor=south]{89}}
	
	% смысловая часть


\QBlock{A1}{0.50}{Диск радиусом $R$ заряжен поверхностной плотностью заряда $\sigma_R$. Определите потенциал $\varphi(y)$ в точке на оси на расстоянии $y$ от центра диска. Потенциал равен нулю на бесконечности.}

\QText{Потенциал от кольца шириной $dr$ радиусом $r$
$$d \varphi = k \frac{\sigma_R \cdot 2 \pi r dr}{\sqrt{y^2 + r^2} }$$

Потенциал от всего диска находим интегрированием:
$$\varphi(y)= \pi k \sigma_R \int^R_0 \frac{d(r^2)}{\sqrt{y^2 + r^2} } =2\pi k \sigma_R \left( \sqrt{R^2+y^2}- |y| \right)$$}

\ABlock{$$\varphi(y)=2\pi k \sigma_R \left( \sqrt{R^2+y^2}- |y| \right)$$}

\QBlock{A2}{1.00}{Два таких диска радиусом $R$ заряжены поверхностной плотностью заряда $\sigma_R>0$ находятся параллельно друг другу. Расстояние между центрами дисков равно $2L$, центры находятся на оси дисков. В положении равновесия находятся заряд $q$ массой $m$, который может двигаться только вдоль оси дисков. Определите угловую частоту $\omega_1$ колебаний такого заряда. Какой знак заряда?}

\QText{При смещении заряда на $x$ вдоль оси на ближайшем диске оказывается внешнее кольцо шириной $\delta = 2 Rx/L$, которое создаёт возвращающую силу. Суммарное поле остальных зарядов равно нулю.

Поле этого кольца
$$E_x(x) = - k \frac{2\pi R \delta \sigma_R}{(L^2+R^2)} \frac{L}{\sqrt{L^2+R^2}}=-\frac{\sigma_R R^2}{\varepsilon_0 (L^2+R^2)^{3/2}}x.$$

Уравнение движения заряда
$$m\ddot{x}=qE(x).$$

Угловая частота при колебаниях:
$$\omega^2=\frac{q\sigma_R R^2}{m \varepsilon_0  (L^2+R^2)^{3/2}}.$$}

\ABlock{$$\omega^2=\frac{q\sigma_R R^2}{m \varepsilon_0  (L^2+R^2)^{3/2}},$$

$$q>0.$$}

\QBlock{A3}{1.00}{Теперь этот заряд может двигаться только в перпендикулярном направлении. Выразите угловую частоту $\omega_2$ колебаний в таком случае через $\omega_1$. Какой теперь знак заряда?}

\QText{Используем теорему Гаусса для маленького цилиндра между дисками, чтобы найти зависимость радиального поля $E_r(r)$ от смещения от оси $r$:

$$2\pi r \cdot 2x E_r(r) + 2\pi r^2 E_x(x) = 0$$

$$E_r(r)=-\frac{1}{2}E_x(x)\frac{r}{x}.$$

Заряд должен быть другого знака.

Видно, что коэффициент в линейной зависимости в 2 раза меньше, поэтому, 
$$\omega_2 = \frac{\omega_1}{\sqrt{2}}.$$}

\ABlock{$$\omega_2 = \frac{\omega_1}{\sqrt{2}},$$
$$q<0.$$}

\QBlock{B1}{1.00}{Боковая поверхность цилиндра радиусом $R$ и длиной $L$ заряжена поверхностной плотностью заряда $\sigma_L$. Определите потенциал в точке на оси на расстоянии $z$ от центра одного из оснований цилиндра. Потенциал равен нулю на бесконечности.}

\QText{Потенциал от кольца высотой $dl$ на расстоянии $l$ от нуля
$$d \varphi (y) = k \frac{\sigma_L \cdot 2 \pi R dl}{\sqrt{(z+l)^2 + R^2}}.$$

Потенциал от всей боковой поверхности цилиндра находим интегрированием:
$$\varphi (y) = 2 \pi k R \sigma_L  \int^{L+z}_z \frac{d (l+z)}{\sqrt{(l+z)^2 + R^2}} = 2\pi k R\sigma_L \left( \mathrm{arth} \frac{L+z}{\sqrt{(L+z)^2 + R^2}} - \mathrm{arth} \frac{z}{\sqrt{z^2 + R^2}}\right).$$}

\ABlock{$$\varphi (y) = 2\pi k R \sigma_L \left( \mathrm{arth} \frac{L+z}{\sqrt{(L+z)^2 + R^2}} - \mathrm{arth} \frac{z}{\sqrt{z^2 + R^2}}\right).$$}

\QBlock{B2}{1.00}{Два таких цилиндра (радиусом $R$ и длиной $L$, поверхность заряжена поверхностной плотностью заряда $\sigma_L>0$) поставлены рядом вплотную и имеют общую ось. В положении равновесия находятся заряд $q$ массой $m$, который может двигаться только вдоль оси цилиндров. Определите угловую частоту $\omega_3$ колебаний такого заряда. Какой знак заряда?}

\QText{При смещении заряда на $x$ вдоль оси на дальнем цилиндре оказывается внешнее кольцо шириной $\delta = 2x$, которое создаёт возвращающую силу. Суммарное поле остальных зарядов равно нулю.

Поле этого кольца
$$E_x(x) = k \frac{2\pi R \delta \sigma_L}{(L^2+R^2)} \frac{L}{\sqrt{L^2+R^2}}=\frac{\sigma_L R L}{\varepsilon_0 (L^2+R^2)^{3/2}}x.$$

Уравнение движения заряда
$$m\ddot{x}=qE_x(x),$$

видно, что заряд должен быть отрицательным.

Угловая частота при колебаниях:
$$\omega_3^2=\frac{|q|\sigma_L R L}{m \varepsilon_0  (L^2+R^2)^{3/2}}.$$}

\ABlock{$$\omega_3^2=\frac{|q|\sigma_L R L}{m \varepsilon_0  (L^2+R^2)^{3/2}},$$

$$q<0.$$}

\QBlock{B3}{0.50}{Теперь этот заряд может двигаться только в перпендикулярном направлении. Выразите угловую частоту $\omega_4$ колебаний в таком случае через $\omega_3$. Какой теперь знак заряда?}

\QText{Всё происходит аналогично A3, то есть 
$$E_r(r)=-\frac{1}{2}E_x(x)\frac{r}{x},$$
значит $\omega^2_4 = \omega^2_3/2$ и заряд тоже должен поменять знак.}

\ABlock{$$\omega_4 = \frac{\omega_3}{\sqrt{2}},$$
$$q>0.$$}

\QBlock{С1}{1.50}{Заряженный цилиндр радиусом $R$ высотой $L=40R/9$ состоит из боковой поверхности и одного основания. Поверхностная плотность заряда боковой поверхности $\sigma_L$, основания $\sigma_R$. Если поместить точечный заряд в центр противоположного основания, то он окажется в положении равновесия. Определите отношение $\sigma_L/\sigma_R$.}

\QText{Напряженность поля, создаваемого основанием цилиндра найдём как 
$$E_R = - \left. \frac{d \varphi}{d y} \right|_L.$$

Эта напряженность равна
$$E_R = 2\pi \sigma_R k \left( 1 - \frac{L}{\sqrt{R^2+L^2}} \right).$$

Напряженность поля, создаваемого боковыми стенками цилиндра найдём как 
$$E_L = - \left. \frac{d \varphi}{d z} \right|_0.$$

Эта напряженность равна
$$E_L = 2\pi \sigma_L k \left( 1 - \frac{R}{\sqrt{R^2+L^2}} \right).$$

Суммарное поле равно нулю, отсюда получаем:
$$\frac{\sigma_{L}}{\sigma_{R}}=-\frac{\sqrt{R^{2}+L^{2}}-L}{\sqrt{R^{2}+L^{2}}-R}=-\frac{41-40}{41-9}=-\frac{1}{32}$$}

\ABlock{$$\frac{\sigma_{L}}{\sigma_{R}}=-\frac{\sqrt{R^{2}+L^{2}}-L}{\sqrt{R^{2}+L^{2}}-R}=-\frac{1}{32}$$}

\QBlock{С2}{2.50}{Заряженный цилиндр радиусом $R=28b$ высотой $L=45b$ состоит из боковой поверхности и одного основания. Заряд боковой поверхности $\sigma_L=-8\sigma_0$, заряд основания $\sigma_R=25\sigma_0>0$. На оси этой системы помещают частицу c зарядом $q>0$. Оцените численно координаты $z$ (в единицах $b$) положений равновесия если частица может двигаться только вдоль оси.  Координата $z$ отсчитывается как на картинке.
Сделайте это максимально точно, однако, достаточно с точностью 1\%. Ответы попадающие в 1\% от правильного получат полный балл.}

\QText{Ищем такую точку, в которой суммарное поле равно нулю. Далее все длины измеряем в единицах $b$. Для точек с координатами $z>-45$ получаем уравнение


$$E_{основание}(z)+E_{бок}(z) = 0,$$
$$25\left(\frac{45+z}{\sqrt{28^2+(45+z)^2}} -1 \right) - 8 \left(\frac{28}{\sqrt{28^2+(45+z)^2}} - \frac{28}{\sqrt{28^2+z^2}} \right) =0 .$$

Чтобы решить уравнение на калькуляторе, представим его в виде $z=f(z)$:
$$f(z) = - \frac{901}{25} + \sqrt{28^2+(45+z)^2} \left(1- \frac{224}{25 \sqrt{28^2+z^2}}  \right).$$

Получается ответ $z=0$. Можно его подставить в $f(z)$ и в этом явно убедиться.

Для того, чтобы учесть точки с координатами $z<-45$, составим следующее уравнение (меняется поле диска):
$$25\left(\frac{45+z}{\sqrt{28^2+(45+z)^2}} +1 \right) - 8 \left(\frac{28}{\sqrt{28^2+(45+z)^2}} - \frac{28}{\sqrt{28^2+z^2}} \right) =0 .$$

Чтобы решить уравнение на калькуляторе, представим его в виде $z=f(z)$:

$$f(z) = - \frac{901}{25} - \sqrt{28^2+(45+z)^2} \left(1+ \frac{224}{25 \sqrt{28^2+z^2}}  \right).$$

Данное уравнение сходится очень медленно. Можно найти его корень методом бинарного поиска: подставляя некоторое число и проводя итерации наблюдать, увеличивается или уменьшается величина $z$.

Правильный ответ $z = -1632.60163579414519440461610908047948157856626649004333963591...$}

\ABlock{$$\frac{z}{b}=0.$$

$$\frac{z}{b} = -1632.60163579414519440461610908047948157856626649004333963591...$$}

\QBlock{С3}{1.00}{В условиях предыдущего пункта частицу поместили в ближайшее к цилиндру положение равновесия, её масса $m$. Определите угловую частоту $\omega$ малых колебаний частицы.}

\QText{Выбираем точку с координатой $z=0$. Раскладываем напряженность поля, получаем, что в этой точке $E(z) = - \alpha z$, $\alpha=2\pi \sigma_0 k \cdot \frac{560}{2809} \frac{z}{b}$.

Из уравнения колебаний 
$$m \ddot{z} = - q \alpha z$$
получаем ответ.}

\ABlock{$$\omega^2 = \frac{560 \sigma_0}{5618 \varepsilon_0 b m}$$}

\end{document}