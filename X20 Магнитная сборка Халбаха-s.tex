
%This file is part of Get pho.rs!

%Get pho.rs! is free software: you can redistribute it and/or modify it under the terms of the GNU General Public License as published by the Free Software Foundation, either version 3 of the License, or (at your option) any later version.

%Get pho.rs! is distributed in the hope that it will be useful, but WITHOUT ANY WARRANTY; without even the implied warranty of MERCHANTABILITY or FITNESS FOR A PARTICULAR PURPOSE. See the GNU General Public License for more details.

%You should have received a copy of the GNU General Public License along with Foobar. If not, see <https://www.gnu.org/licenses/>.

%\documentstyle[12pt,russian,amsthm,amsmath,amssymb]{article}
\documentclass[a4paper,11pt,twoside]{article}
\usepackage[left=14mm, top=10mm, right=14mm, bottom=10mm, nohead, nofoot]{geometry}
\usepackage{amsmath, amsfonts, amssymb, amsthm} % стандартный набор AMS-пакетов для математ. текстов
\usepackage{mathtext}
\usepackage[utf8]{inputenc} % кодировка utf8
\usepackage[russian]{babel} % русский язык
\usepackage[pdftex,dvipsnames]{xcolor} % работа с цветами
\usepackage[pdftex]{graphicx} % графика (картинки)
\usepackage{tikz} % рисунки
\usepackage{fancyhdr,pageslts} % настройка колонтитулов
\usepackage{enumitem} % работа со списками
\usepackage{multicol} % работа с таблицами
%\usepackage{pscyr} % красивый шрифт
\usepackage{pgfornament} % красивые рюшечки и вензеля
\usepackage{ltxgrid} % управление написанием текста в две колонки
\usepackage{lipsum} % стандартный текст
\usepackage{tcolorbox} % рамка вокруг текста
\usepackage{float} % для корректного размещения картинок
\tcbuselibrary{skins}
% ----------------------------------------

\newcommand\ProblemName{Магнитная сборка Халбаха}

\newcommand\Source{X20}

\newcommand\Type{Решение}

% настройки полей
\geometry{
	left=12mm,
	top=21mm,
	right=15mm,
	bottom=26mm,
	marginparsep=0mm,
	marginparwidth=0mm,
	headheight=22pt,
	headsep=2mm,
	footskip=7mm}
% ----------------------------------------

% настройки колонтитулов
\pagestyle{fancy}

\fancypagestyle{style}{
	\fancyhf{}
	\fancyhead[L]{{\Large{\FancyTitle}}\\\vskip -5pt \dotfill}
	\fancyhead[R]{{\Large{\textbf{\Type}}}\\\vskip -5pt \dotfill}
	\renewcommand{\headrulewidth}{0pt}
	\renewcommand{\footrulewidth}{0pt}
	\fancyfoot[C]{\pgfornament[width=2em,anchor=south]{72}\hspace{1mm}
		{Страница \textbf{\thepage} из \textbf{\pageref{VeryLastPage}}}\hspace{2mm}
		\pgfornament[width=2em,symmetry=v,anchor=south]{72}\\ \vskip2mm
		{\small{\textit{Условие собрано и подготовлено в Президентском ФМЛ №239 г.~Санкт-Петербурга}}}}
}

\fancypagestyle{plain}{
	\fancyhf{}
	\renewcommand{\headrulewidth}{0pt}
	\renewcommand{\footrulewidth}{0pt}
	\fancyhead[C]{{\Large{\textit{Учебно-тренировочные сборы к X23}}}\\\vskip -5pt \dotfill}
	\fancyfoot[C]{\pgfornament[width=2em,anchor=south]{72}\hspace{1mm}
		{Страница \textbf{\thepage} из \textbf{\pageref{VeryLastPage}}}\hspace{2mm}
		\pgfornament[width=2em,symmetry=v,anchor=south]{72}\\ \vskip2mm
		{\small{\textit{Условие собрано и подготовлено в Президентском ФМЛ №239 г.~Санкт-Петербурга}}}}
}
% ----------------------------------------

% другие настройки
\pagenumbering{arabic}
\setlist[enumerate,itemize]{leftmargin=0pt,itemindent=2.7em,itemsep=0cm}
% ----------------------------------------

% собственные команды
\newcommand{\FancyTitle}{\textbf{\Source} --- \ProblemName}
\newcommand{\Title}{\begin{center}{\huge{\textbf{\Source} --- \ProblemName}}\end{center}}
\newcommand{\Chapter}[1]{\vskip5pt{\Large{\textbf{#1}}}\vskip5pt}
\newcommand{\QText}[1]{#1}
\newcommand{\QBlock}[3]{
	\begin{tcolorbox}[left=4mm,top=3mm,bottom=2mm,right=4mm,colback=white]
		\begin{tcolorbox}[enhanced,colframe=blue,colback=blue!10!white,
			frame style={opacity=0.3},interior style={opacity=1.0},
			nobeforeafter,tcbox raise base,shrink tight,extrude by=1.7mm,width=1.5cm]
			\textbf{#1\textsuperscript{#2}}
		\end{tcolorbox}\hspace{3mm}#3
	\end{tcolorbox}
}
\newcommand{\QPicture}[4]{\QText{#4}  \includegraphics{#1}}
\newcommand{\ABlock}[1]{#1}
\newcommand{\MBlock}[2]{#1 #2}
\newcommand{\MMBlock}[3]{#1 #2 #3}
% ----------------------------------------


\begin{document}
	
	% настройки
	\pagestyle{style}\thispagestyle{plain}
	\Title
	% ----------------------------------------
	
	%\vskip5mm
	%\centering{\pgfornament[width=5cm,anchor=south]{89}}

\QBlock{A1}{0.50}{Магнитное поле ослабляется с увеличением расстояния, для заданного угла $\theta$ найдите зависимость магнитного поля $B$ на расстоянии $r$ от диполя.}

\QText{Зависимость магнитного поля диполя от $\vec r$:

$\vec B (\vec r) = \frac {\mu_0}{4\pi} 
(\frac{3\vec r (\vec m \cdot \vec r )} {r^5} - 
\frac { \vec m} {r^3}),  
\vec m \cdot \vec r = mr \cos \theta \Rightarrow$

$\vec B (\vec r) = \frac {m \mu_0}{4\pi r^3}(3\cos \theta \hat{r} - \hat{y}) $

Находить модуль этого вектора можно по-разному, например, 
удобно ввести дополнительный единичный вектор:

$\hat{\theta} = - \sin \theta \cdot \hat{y} + \cos \theta \cdot \hat{x}, \hat{r}=\cos \theta \cdot \hat{y} +\sin \theta \cdot \hat{x}$

Или иначе:

$\hat{y}=\cos\theta\cdot \hat{r} - \sin \theta \cdot \hat{\theta}, \hat{x}= \sin \theta \cdot \hat{r} + \cos \theta \cdot \hat{\theta}$}

\QText{$3 \cos \theta \hat{r} - \hat{y} = 3 \cos \theta \hat{r} - (\cos \theta \cdot \hat{r} - \sin \theta \cdot \hat{\theta}) \Rightarrow$

$\vec B (\vec r) = \frac {\mu_0 m} {4 \pi r^3} ((2 \cos \theta) \hat{r} + (\sin \theta) \hat{\theta}) \Rightarrow B = \frac {\mu_0 m}{4\pi r^3} \sqrt{(2\cos \theta)^2+(\sin \theta)^2)} $}

\QBlock{B1}{1.50}{Выразите магнитное поле $B(y)$ вдоль оси, перпендикулярной магниту, на расстоянии $y$ от центра.}

\QText{Для нахождения ответа в этом пункте, разобьём магнит на бесконечно узкие кольца и проинтегрируем:}

\QText{Получаем слеующее выражение:

$\vec B (y) = \frac {\mu_0} {4\pi} \cdot 
\int ^R_0 2\pi x \cdot dx \cdot \sigma 
\cdot \frac {1} {(x^2+y^2)^{\frac{3}{2}}}
\cdot (2 \cos \theta \cdot \cos \theta
\cdot \hat{y} + \sin \theta \cdot \sin \theta
\cdot (-\hat{y}))$

Для удобства перейдём к интегрированию по углу:

$R = y \cdot \tan \theta_{max}, x=y\cdot 
\tan \theta, dx=y\cdot \frac {d\theta}{\cos^2\theta}, \frac{y}{\cos\theta}=\sqrt{x^2+y^2}$

Тогда

$B=\frac {\mu_0 \sigma}{2} \cdot \int
^{\theta_{max}}_0 y\cdot \tan \theta \cdot 
y \cdot \frac {d\theta} {\cos^2\theta}
\cdot \frac{\cos^3 \theta}{y^3}\cdot \hat{y} \cdot
(2\cos^2\theta-\sin^2\theta)$

$=\frac{\mu_0\sigma \cdot \hat{y}}{2y} \cdot 
\int^{\theta_{max}}_0 d\theta \cdot 
(2 \cos^2\theta \sin \theta - \sin^3 \theta)$

$=\frac{\mu_0\sigma \cdot \hat{y}}{2y}
\cdot \int^{\theta_{max}}_0 d
\theta \cdot (3\cos^2 \theta \sin
\theta - \sin \theta)$

$=\frac{\mu_0\sigma\cdot \hat{y}}{2y}
\cdot [-\cos^3\theta+\cos\theta]|^{
\theta_{max}}_0=\frac{\mu_0\sigma
\cdot \hat{y}}{2y}\cdot[\cos \theta_{max}-
\cos^3 \theta_{max}]$

$\cos \theta_{max} = \frac {y}
{\sqrt{R^2+y^2}}$

Упрощая, получаем итоговый результат:

$B(y)=\frac{\mu_0\sigma}{2}\cdot
\frac{R^2}{(R^2+y^2)^{\frac{3}{2}}}$}

\QBlock{B2}{0.50}{Оцените величину магнитного поля  вблизи поверхности магнита. Ответ выразите через величины $t, D, \rho, \mu_0$.}

\QText{Поскольку выполняется соотношение $t \ll D$, магнит можно считать плоским и пренебрегать его толщиной при нахождении поля у поверхности:

$\sigma=\rho\cdot t, B(y)=\frac{
\mu_0\sigma}{2}\cdot \frac {R^2}
{(R^2+y^2)^{\frac{3}{2}}}, y=0 \Rightarrow B_0 = \frac {\mu_0\rho t} {D}$

$\frac{\mu_0\rho t}{D}=0.13 \text{Тл}$}

\QBlock{B3}{0.50}{Получите выражение и численное значение силы взаимодействия $F_0$ между дверью и прижатым к ней магнитом, также вычислите давление $P_0$ магнита на дверь.}

\QText{B3) По условию, объемная плотность энергии магнитного поля составляет:

$\frac{E}{V}=\frac {1}{2\mu_0}
\cdot B^2$

При отрыве магнита от двери на небольшое расстояние $y$:

$\Delta E = \pi (D/2)^2\cdot y\cdot 
\frac{1}{2\mu_0} \cdot B^2_0 $

$F=\frac {\Delta E}{y} = \pi (D/2)^2
\cdot \frac {1}{2\mu_0}\cdot B^2_0=
2.2 H$

Зная силу и площадь соприкосновения, легко выразить давление:

$P=\frac{F}{S}=\frac{B^2_0}{2\mu_0}
=6.9\cdot 10^3 \text{Па}$}

\QBlock{C1}{2.00}{Запишите выражение для поля $\vec B(\vec r_0, y)$ которое создает ряд магнитов. (Для удобства поле выражается и через $\vec r_0$, и через $y$, хотя технически $y = (\vec r_0)_y$.)}

\QText{По схеме ниже:}

\QText{$r_0=\frac{y}{\cos \alpha}, x=y\tan \alpha, \vec{r_0} = y\hat{y} + x\hat{x} \Rightarrow
r_0 = \hat{y}\cdot \cos \alpha +\hat{x} \cdot \sin \alpha$

$\frac{r_0}{r}=\cos \theta, z=r_0
\tan \theta \Rightarrow dz =\frac{r_0d\theta}
{cos^2\theta}$

$\vec r=x\hat{x}+y\hat{y}+(-z)\hat{z}$

$\vec B(\vec{r_0},y)=\frac{\mu_0}{4
\pi}\cdot \int^\infty_{-\infty}dz
\cdot \rho_L\cdot[\frac{3}{r^4}
(\frac{y\cdot \cos \theta}{r_0}) \cdot (\vec{r_0}-z\hat{z})-\frac{\hat{y}}{r^3}]=
$

$\frac{\mu_0\cdot\rho_L}{4\pi}\cdot
\int^{\pi/2}_{-\pi/2}d\theta \cdot
\frac{r_0}{\cos^2\theta}\cdot[
\frac{3y}{r_0^5}\cdot \cos^5\theta
\cdot \vec{r_0} - \frac {\hat{y}}{r_0^3}
\cdot \cos^3\theta]=$

$\frac{\mu_0\cdot\rho_L}{4\pi r_0^2}\cdot
\int^{\pi/2}_{-\pi/2}d\theta\cdot[\frac{3y\cdot\vec{r_0}}{r_0^2}
\cdot\cos^3\theta-\hat{y}\cdot\cos\theta]=$

$\frac{\mu_0\cdot\rho_L}{4\pi r_0^2}\cdot
[\frac{3y\cdot\vec{r_0}}{r_0^2}
\cdot(\sin\theta-\frac{1}{3}\sin^3\theta)-
\hat{y}\cdot\sin\theta]|_{-\pi/2}^{\pi/2}=$

$\frac {\mu_0\cdot\rho_L}
{4\pi r_0^2}\cdot[\frac{4y\cdot\vec{r_0}}{r_0^2}-2\hat{y}]$}

\QBlock{C2}{1.00}{Найдите магнитного поля с двух сторон от сборки. Ответ дать в виде некоторого интеграла.}

\QText{По схеме ниже:}

\QText{$r_0=\frac{y_0}{\cos \alpha}=\frac{y}{\cos(
\beta+\alpha)} \Rightarrow  y=y_0\cdot \frac
{\cos(\beta+\alpha)}{\cos\alpha}$

$\hat{y}=\cos \beta \cdot \hat{y_0} +\sin \beta \cdot \hat{x}, \hat{r_0}=\cos \alpha \cdot \hat{y_0} - \sin \alpha \cdot \hat{x}$

$\vec B = \int_{-\infty}^{\infty} dx \cdot
\frac {\mu_0\sigma}{4\pi y_0^2} \cdot \cos^2 \alpha \cdot [4\cos(\beta+\alpha)[\cos\alpha
\cdot \hat{y_0}-\sin\alpha\cdot\hat{x}]-
2[\cos \beta \cdot \hat{y_0}+\sin \beta \cdot
\hat{x}] ]$

Путём несложных математических преобразований получаем:

$\frac{\mu_0\sigma}{2\pi y_0}\cdot \int 
_{-\pi/2}^{\pi/2} d\alpha\cdot [\hat{y_0}\cdot
\cos(\beta+2\alpha)-\hat{x}\cdot\sin(\beta+2
\alpha)]$

Подставим зависимость для $\beta$:

$\beta=\beta_0+kx_0+ky_0\cdot\tan \alpha$

Получаем:

$\frac{\mu_0\sigma}{2\pi y_0}\cdot
\int_{-\pi/2}^{\pi/2}d\alpha\cdot
[\hat{y_0}\cdot\cos(\beta_0+kx_0+ky_0\cdot\tan\alpha
+2\alpha)-\hat{x}\cdot\sin(\beta_0+kx_0+ky_0\cdot\tan\alpha+2\alpha)]$}

\QBlock{C3}{1.00}{Покажите, что с одной стороны идеальной сборки магнитное поле стремится к нулю.}

\QText{Посмотрим внимательно на полученное выражение:

$\cos(\beta_0+kx_0+ky_0\cdot\tan\alpha+2\alpha)$

$=\cos(ky_0\cdot\tan\alpha+2\alpha)-\sin(\beta_0+kx_0)
\sin(ky_0\cdot\tan\alpha+2\alpha)$

$\cos(ky_0\cdot\tan\alpha+2\alpha)=\cos(ky_0\cdot\tan
\alpha)\cos(2\alpha)-\sin(ky_0\cdot\tan\alpha)\sin
(2\alpha)$

По условию,

$\int_{-\pi/2}^{\pi/2} dx\cdot\cos(2x)\cos(c\cdot
\tan x)=\frac{c\cdot\pi}{e^c}=
\int^{\pi/2}_{-\pi/2}dx\cdot\sin(2x)\cdot
\sin(c\cdot\tan x)$

Несложно заметить, что это и есть наш интеграл, если взять в качестве $c$ величину $ky_0$.}

\QBlock{C4}{1.00}{Запишите выражение для поля с другой стороны.}

\QText{При переходе к другой стороне сборки, некоторые знаки в уравнении меняются 
на противоположные:

$\vec{B}=-\frac{\mu_0\sigma}{2\pi y_0}
\cdot \int^{\pi/2}_{-\pi/2} d\alpha
\cdot [\hat{y_0}\cdot [\cos(\beta_0+kx_0)\cdot
\cos(-ky_0\cdot\tan\alpha+2\alpha)-\sin"..."sin"..."]
-\hat{x}\cdot[\sin(\beta_0+kx_0)\cos(-ky_0\cdot
\tan\alpha+2\alpha)+\cos"..."sin"..."] ]$

Интегралы от нечётных функций будут зануляться из-за
соображений симметрии, чётные функции остаются.

$\vec B=- \frac{\mu_0\sigma}{2\pi y_0} \cdot \int^{\pi/2}
_{-\pi/2}d\alpha\cdot[\hat{y_0}\cdot[\cos(\beta_0+kx_0-ky_0
\cdot\tan\alpha+2\alpha)]-\hat{x}\cdot[\sin(\beta_0+kx_0)
\cos(-ky_0\cdot\tan\alpha+2\alpha)+\cos"..."\sin"..."] ]$

$=-\frac{\mu_0\sigma}{2\pi y_0}\cdot\int
^{\pi/2}_{-\pi/2}d\alpha\cdot[\hat{y_0}\cdot
\cos(\beta_0+kx_0)\cdot(cos(2\alpha)\cos(ky_0
\cdot\tan\alpha)+\sin"..."\sin"...")]-
\hat{x}\cdot[\sin(\beta_0+kx_0)\cdot"..."] ]$

$=-\frac{\mu_0\sigma}{2\pi y_0}\cdot
[\hat{y_0}\cdot\cos(\beta_0+kx_0)-\hat{x}\cdot
\sin(\beta_0+kx_0)]\cdot 2 \cdot\frac{ky_0\pi}
{e^{ky_0}}$

$=-\mu_0\sigma k\cdot e^{-ky_0}\cdot[\hat{y_0}
\cdot\cos(\beta_0+kx_0)-\hat{x}\cdot \sin(\beta_0+kx_0)]$}

\QBlock{C5}{1.50}{На основании выражения поля найдите среднее давление $Р$ такого магнита на дверь холодильника. Возьмите следующие параметры: толщина $t=0.5 \text{мм}$, объемная плотность магнитного диполя $\rho = 2 \cdot 10^5 \frac {\text{Тл} \cdot \text{м}} {\text{Гн}}$, шаг сборки $\lambda=5 \text{мм}$.}

\QText{Сила и давление находятся так же, как и в пункте B3:

$y_0 \rightarrow 0 \Rightarrow \vec B = 
-\mu_0\sigma k \cdot[\hat{y_0}\cdot\cos(\beta_0+kx_0)
-\hat{x}\cdot\sin(\beta_0+kx_0)] \Rightarrow$

$B^2=(\mu_0\sigma k)^2, P=\frac{1}{2\mu_0}\cdot B^2,
\sigma=\rho t \Rightarrow$

$B=(1.257\cdot 10^{-6} \text{Гн/м})
\cdot(2\cdot 10^5  \text{Тл}\cdot\text{м}/\text{Гн})
\cdot(5\cdot10^{-4} \text{м})
\cdot \frac{2\cdot 3.14}{5\cdot 10^{-3}\text{ м}}
=0.16 \text{ Тл}$

$P=\frac{1}{2\mu_0}\cdot B^2=\frac{1}
{2\cdot 1.257 \cdot 10^{-6} \text{ Гн/м}}
\cdot (0.16\text{ Тл}^2)=10\text{ кПа}$}

\QBlock{C6}{0.50}{Найдите соотношение между давлением, которое создает магнитная сборка Халбаха и давлением, которое создает обычный магнит из того же материала, с теми же радиусом и толщиной. Здесь тоже следует пренебречь эффектами на периметре кружка и и толщиной магнита.}

\QText{По формулам, полученным ранее: 

$\eta=\frac{P}{P_0}=(\frac{B}{B_0})^2
(\frac{\mu_0\rho tk}{\mu_0\rho t/2R})^2=
(\frac{4\pi R}{\lambda})^2$}

\end{document}