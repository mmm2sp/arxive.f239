
%This file is part of Get pho.rs!

%Get pho.rs! is free software: you can redistribute it and/or modify it under the terms of the GNU General Public License as published by the Free Software Foundation, either version 3 of the License, or (at your option) any later version.

%Get pho.rs! is distributed in the hope that it will be useful, but WITHOUT ANY WARRANTY; without even the implied warranty of MERCHANTABILITY or FITNESS FOR A PARTICULAR PURPOSE. See the GNU General Public License for more details.

%You should have received a copy of the GNU General Public License along with Foobar. If not, see <https://www.gnu.org/licenses/>.

%\documentstyle[12pt,russian,amsthm,amsmath,amssymb]{article}
\documentclass[a4paper,11pt,twoside]{article}
\usepackage[left=14mm, top=10mm, right=14mm, bottom=10mm, nohead, nofoot]{geometry}
\usepackage{amsmath, amsfonts, amssymb, amsthm} % стандартный набор AMS-пакетов для математ. текстов
\usepackage{mathtext}
\usepackage[utf8]{inputenc} % кодировка utf8
\usepackage[russian]{babel} % русский язык
\usepackage[pdftex,dvipsnames]{xcolor} % работа с цветами
\usepackage[pdftex]{graphicx} % графика (картинки)
\usepackage{tikz} % рисунки
\usepackage{fancyhdr,pageslts} % настройка колонтитулов
\usepackage{enumitem} % работа со списками
\usepackage{multicol} % работа с таблицами
%\usepackage{pscyr} % красивый шрифт
\usepackage{pgfornament} % красивые рюшечки и вензеля
\usepackage{ltxgrid} % управление написанием текста в две колонки
\usepackage{lipsum} % стандартный текст
\usepackage{tcolorbox} % рамка вокруг текста
\usepackage{float} % для корректного размещения картинок
\tcbuselibrary{skins}
% ----------------------------------------

\newcommand\ProblemName{Нецентральные движения шара}

\newcommand\Source{X24}

\newcommand\Type{Условие задачи}

% настройки полей
\geometry{
	left=12mm,
	top=21mm,
	right=15mm,
	bottom=26mm,
	marginparsep=0mm,
	marginparwidth=0mm,
	headheight=22pt,
	headsep=2mm,
	footskip=7mm}
% ----------------------------------------

% настройки колонтитулов
\pagestyle{fancy}

\fancypagestyle{style}{
	\fancyhf{}
	\fancyhead[L]{{\Large{\FancyTitle}}\\\vskip -5pt \dotfill}
	\fancyhead[R]{{\Large{\textbf{\Type}}}\\\vskip -5pt \dotfill}
	\renewcommand{\headrulewidth}{0pt}
	\renewcommand{\footrulewidth}{0pt}
	\fancyfoot[C]{\pgfornament[width=2em,anchor=south]{72}\hspace{1mm}
		{Страница \textbf{\thepage} из \textbf{\pageref{VeryLastPage}}}\hspace{2mm}
		\pgfornament[width=2em,symmetry=v,anchor=south]{72}\\ \vskip2mm
		{\small{\textit{Условие собрано и подготовлено в Президентском ФМЛ №239 г.~Санкт-Петербурга}}}}
}

\fancypagestyle{plain}{
	\fancyhf{}
	\renewcommand{\headrulewidth}{0pt}
	\renewcommand{\footrulewidth}{0pt}
	\fancyhead[C]{{\Large{\textit{Учебно-тренировочные сборы к X23}}}\\\vskip -5pt \dotfill}
	\fancyfoot[C]{\pgfornament[width=2em,anchor=south]{72}\hspace{1mm}
		{Страница \textbf{\thepage} из \textbf{\pageref{VeryLastPage}}}\hspace{2mm}
		\pgfornament[width=2em,symmetry=v,anchor=south]{72}\\ \vskip2mm
		{\small{\textit{Условие собрано и подготовлено в Президентском ФМЛ №239 г.~Санкт-Петербурга}}}}
}
% ----------------------------------------

% другие настройки
\pagenumbering{arabic}
\setlist[enumerate,itemize]{leftmargin=0pt,itemindent=2.7em,itemsep=0cm}
% ----------------------------------------

% собственные команды
\newcommand{\FancyTitle}{\textbf{\Source} --- \ProblemName}
\newcommand{\Title}{\begin{center}{\huge{\textbf{\Source} --- \ProblemName}}\end{center}}
\newcommand{\Chapter}[1]{\vskip5pt{\Large{\textbf{#1}}}\vskip5pt}
\newcommand{\QText}[1]{#1}
\newcommand{\QBlock}[3]{
	\begin{tcolorbox}[left=4mm,top=3mm,bottom=2mm,right=4mm,colback=white]
		\begin{tcolorbox}[enhanced,colframe=blue,colback=blue!10!white,
			frame style={opacity=0.3},interior style={opacity=1.0},
			nobeforeafter,tcbox raise base,shrink tight,extrude by=1.7mm,width=1.5cm]
			\textbf{#1\textsuperscript{#2}}
		\end{tcolorbox}\hspace{3mm}#3
	\end{tcolorbox}
}
\newcommand{\QPicture}[4]{\QText{#4}  \includegraphics{#1}}
\newcommand{\ABlock}[1]{#1}
\newcommand{\MBlock}[2]{#1 #2}
\newcommand{\MMBlock}[3]{#1 #2 #3}
% ----------------------------------------


\begin{document}
	
	% настройки
	\pagestyle{style}\thispagestyle{plain}
	\Title
	% ----------------------------------------
	
	%\vskip5mm
	%\centering{\pgfornament[width=5cm,anchor=south]{89}}

\QText{В обычной жизни вы наверняка сталкивались с такими физическими ситуациями, как соударение катящегося шара с вертикальной стенкой, а также падение шара с края горизонтального стола. Также вами наверняка решались задачи, связанные с этими ситуациями, однако вы ограничивались случаями, когда шар катится в направлении, перпендикулярном плоскости стены либо краю стола.
В рамках данной задачи вам предлагается получить обобщение результатов на случай, когда скорость центра шара направлена не перпендикулярно плоскости стены либо краю стола.}

\QText{Во всех пунктах задачи считайте известным следующее:}

\Chapter{Часть A. Соударение шара с вертикальной стеной (4.5 балла).}

\QPicture{X24 Нецентральные движения шара_files/28292.jpeg}{"width:100\%; max-width: 100\%;display:block;margin-left: auto;margin-right: auto; cursor:pointer;" data-toggle="modal" data-target="#kt_modal_28292"}{}{Данная часть задачи посвящена изучению столкновения шара с вертикальной стенкой. Шар катится по горизонтальному столу без проскальзывания, а его центр при этом движется со скоростью $v$ в направлении, образующем угол $\alpha$ с нормалью к стенке. Шар не вращается вокруг вертикальной оси $z$. 
В некоторый момент шар упруго сталкивается со стенкой. Коэффициент трения между шаром и стенкой равен $\mu$.}

\QText{Введём прямоугольную систему координат $xyz$ с началом в центре шара в момент соударения так, как показано на рисунке.}

\QPicture{X24 Нецентральные движения шара_files/28722.jpeg}{"max-width:500px;px;display:block;margin-left: auto;margin-right: auto;"}{}{}

\QText{Будем использовать следующие обозначения:}

\QBlock{A1}{0.40}{Выразите компоненту скорости $\vec{u}_{A}$ точки $A$ через компоненту скорости $\vec{u}_C$ центра шара, его угловую скорость $\vec{\omega}$, а также радиус-вектор $\vec{r}$ в произвольный момент.
Получите также производную по времени $\dot{\vec{u}}_A$ вектора $\vec{u}_A$. Ответ выразите через $\dot{\vec{u}}_C$, $\dot{\vec{\omega}}$ и $\vec{r}$.}

\QBlock{A2}{0.60}{Определите силу трения $\vec{F}_0$, действующую на шар в начальный момент контакта со стеной. Ответ выразите через $\vec{e}_x$, $\vec{e}_z$, $\alpha$, $\mu$ и силу нормальной реакции стены $N_0$ в начальный момент.}

\QBlock{A3}{1.00}{Докажите, что производная по времени $\dot{\vec{u}}_A$ компоненты скорости $\vec{u}_A$ связана с силой трения $\vec{F}$ соотношением:
$$\dot{\vec{u}}_A=\cfrac{7\vec{F}}{2m}{.}
$$
Данный факт можно использовать далее, даже если вы не смогли его доказать.}

\QBlock{A4}{0.50}{Определите компоненту скорости $\vec{u}_{A\text{к}}$ сразу после соударения, считая, что шар проскальзывает по стенке в течение всего времени соударения. Ответ выразите через $v$, $\alpha$, $\mu$, $\vec{e}_x$ и $\vec{e}_z$. 
При каком максимальном значении коэффициента трения $\mu_{max}$ проскальзывание не прекращается в течение всего времени соударения? Ответ выразите через $\alpha$.}

\QBlock{A5}{0.60}{При $\mu{<}\mu_{max}$ определите скорость центра шара $\vec{v}_{C\text{к}}$, а также под каким углом $\beta$ к горизонту она направлена  сразу после соударения. Ответы выразите через $v$, $\alpha$, $\mu$, $\vec{e}_x$, $\vec{e}_y$ и $\vec{e}_z$.}

\QBlock{A6}{0.40}{При $\mu{<}\mu_{max}$ определите координаты $x_C$, $y_C$ центра шара в момент его падения на стол. Ответы выразите через $v$, $g$, $\mu$ и $\alpha$.}

\QBlock{A7}{1.00}{При произвольных значениях $\mu$ определите количество теплоты $Q$, выделившееся в процессе соударения шара со стенкой. Ответ выразите через $m$, $v$, $\mu$ и $\alpha$. }

\QPicture{X24 Нецентральные движения шара_files/28562.jpeg}{"width:100\%; max-width: 100\%;display:block;margin-left: auto;margin-right: auto; cursor:pointer;" data-toggle="modal" data-target="#kt_modal_28562"}{}{Далее в рамках данной задачи вам предлагается изучить динамику падения однородного шара с прямолинейного края горизонтального стола. Перед тем, как попасть на край, центр шара двигался по столу со скоростью $v$ под углом $\alpha$ к перпендикуляру, проведённому к краю стола в его плоскости. До попадания на край стола шар не вращался вокруг вертикальной оси. При дальнейшем решении задачи считайте, что шар никогда не проскальзывает по столу.}

\QPicture{X24 Нецентральные движения шара_files/28300.jpeg}{"width:100\%; max-width: 100\%;display:block;margin-left: auto;margin-right: auto; cursor:pointer;" data-toggle="modal" data-target="#kt_modal_28300"}{}{Решение задачи наиболее удобно провести в цилиндрической системе координат $(r{,}\varphi{,}z)$. Ось $z$ совпадает с краем стола. 
На рисунке приведены единичные орты $\vec{e}_r$, $\vec{e}_\varphi$ и $\vec{e}_z$ цилиндрической системы координат. Радиус $r$ шара является расстоянием от его центра до оси $z$, а угол $\varphi$ является углом поворота линии, соединяющей центр шара с точкой его контакта со столом и отсчитывается от положения, в котором эта линия вертикальна.}

\QText{Произвольный вектор в цилиндрической системе координат можно представить в следующей форме:
$$\vec{A}=A_r\vec{e}_r+A_\varphi\vec{e}_\varphi+A_z\vec{e}_z{.}
$$
При дифференцировании вектора, заданного компонентами в цилиндрической системе координат, необходимо учитывать, что единичные орты цилиндрической системы координат являются переменными. Для их производных по времени можно записать:
$$\cfrac{d\vec{e}_i}{dt}=\bigl[\vec{\Omega}\times\vec{e}_i\bigr]{,}
$$
где $\vec{\Omega}=\dot{\varphi}\vec{e}_z$ - угловая скорость вращения цилиндрической системы координат. 

Таким образом, проекции производной по времени вектора $\vec{A}$ записываются следующим образом:
$$\bigl(\dot{\vec{A}}\bigr)_r=\dot{A}_r-\dot{\varphi}A_\varphi\qquad \bigl(\dot{\vec{A}}\bigr)_\varphi=\dot{A}_\varphi+\dot{\varphi}A_r\qquad \bigl(\dot{\vec{A}}\bigr)_z=\dot{A}_z
$$
Данные соотношения могут оказаться полезными в процессе дальнейшего решения задачи.}

\Chapter{Часть B. Уравнения кинематических связей (0.9 балла).}

\QText{Данная часть посвящена получению основных кинематических уравнений, описывающих движение шара.}

\QBlock{B1}{0.20}{Определите компоненты вектора скорости центра шара $v_\varphi$ и $v_z$ в цилиндрической системе координат. Ответы выразите через $r$, $\dot\varphi$ и $\dot{z}$.}

\QBlock{B2}{0.30}{Определите компоненты вектора ускорения центра шара $a_r$, $a_\varphi$ и $a_z$ в цилиндрической системе координат. Ответы выразите через $r$, $v_\varphi$, $\dot{v}_\varphi$ и $\dot{v}_z$.}

\QBlock{B3}{0.40}{Из условия отсутствия проскальзывания определите компоненты угловой скорости шара $\omega_\varphi$ и $\omega_z$ в цилиндрической системе координат. Ответы выразите через $r$, $v_\varphi$ и $v_z$.}

\Chapter{Часть C. Движение в плоскости, перпендикулярной краю стола (2.0 балла).}

\QText{В плоскости, перпендикулярной краю стола, шар движется по окружности, что очень упрощает анализ данной части его движения.}

\QBlock{C1}{0.80}{Определите компоненту силу трения $F_\varphi(\varphi)$, действующую на шар, а также компоненту ускорения $a_\varphi(\varphi)$ его центра. Ответы выразите через массу шара $m$, $g$ и $\varphi$.}

\QBlock{C2}{0.50}{Получите зависимость $v_\varphi(\varphi)$. Ответ выразите через $v$, $g$, $r$, $\alpha$ и $\varphi$.}

\QBlock{C3}{0.20}{При каком условии шар не отрывается от стола в момент, когда нижняя точка шара достигает его края? Запишите это условие через $v$, $g$, $r$ и $\alpha$. Во всех дальнейших пунктах считайте, что это условие выполняется.}

\QBlock{C4}{0.50}{Определите угол $\varphi_1$ в момент отрыва шара от стола. Ответ выразите через $v$, $g$, $r$ и $\alpha$.}

\Chapter{Часть D. Движение шара вдоль оси z (3.6 балла)}

\QText{В данной части задачи вам предлагается проанализировать зависимости от угла $\varphi$ компоненты скорости центра шара $v_z$, а также его угловой скорость верчения $\omega_r$.}

\QBlock{D1}{0.50}{Выразите кинетическую энергию шара $E_k$ через $m$, $v_\varphi$, $v_z$, $\omega_r$ и $r$.}

\QBlock{D2}{0.60}{Запишите для шара закон сохранения механической энергии. Комбинируя его с результатом пункта $\mathrm{C2}$, покажите, что величины $\omega_r$ и $v_z$ связаны соотношением:
$$1=\cfrac{\omega^2_r}{A^2}+\cfrac{v^2_z}{B^2}{,}
$$
где $A{,}B>0$ - постоянные коэффициенты.
Определите $A$ и $B$. Ответы выразите через $v$, $r$ и $\alpha$.}

\QText{Решение данной задачи осложняется тем, что компонента угловой скорости $\omega_r$ не может быть получена исключительно из уравнения кинематической связи, однако можно получить выражение для её производной по времени $\dot{\omega}_r$.}

\QBlock{D3}{0.50}{Вектор углового ускорения $\vec{\varepsilon}$ шара может быть представлен в виде:
$$\vec{\varepsilon}=\varepsilon_r\vec{e}_r+\varepsilon_\varphi\vec{e}_\varphi+\varepsilon_z\vec{e}_z{.}
$$
Используя уравнение динамики вращательного движения относительно центра шара, покажите, что $\varepsilon_r=0$. Используя полученное равенство, выразите $\dot{\omega}_r$ через $\dot{\varphi}$, $v_z$ и $r$.}

\QBlock{D4}{1.20}{Комбинируя результаты пунктов $\mathrm{D2}$ и $\mathrm{D3}$, получите зависимости $\omega_r(\varphi)$ и $v_z(\varphi)$. Ответы выразите через $v$, $\alpha$, $r$ и $\varphi$.}

\QBlock{D5}{0.80}{Рассмотрим предельный переход, когда угол $\alpha\to\pi/2$, т.е движение шара до контакта с краем стола происходит практически параллельно ему.
Определите проекцию скорости $v_z$ центра шара, а также проекцию его угловой скорости $\omega_y$ на ось $y$, направленную вертикально вниз, в момент отрыва шара от стола. Ответы выразите через $v$ и $r$. Все численные коэффициенты в ответе должны быть аналитическими, а не приближёнными!}

\end{document}