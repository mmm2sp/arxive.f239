
%This file is part of Get pho.rs!

%Get pho.rs! is free software: you can redistribute it and/or modify it under the terms of the GNU General Public License as published by the Free Software Foundation, either version 3 of the License, or (at your option) any later version.

%Get pho.rs! is distributed in the hope that it will be useful, but WITHOUT ANY WARRANTY; without even the implied warranty of MERCHANTABILITY or FITNESS FOR A PARTICULAR PURPOSE. See the GNU General Public License for more details.

%You should have received a copy of the GNU General Public License along with Foobar. If not, see <https://www.gnu.org/licenses/>.

%\documentstyle[12pt,russian,amsthm,amsmath,amssymb]{article}
\documentclass[a4paper,11pt,twoside]{article}
\usepackage[left=14mm, top=10mm, right=14mm, bottom=10mm, nohead, nofoot]{geometry}
\usepackage{amsmath, amsfonts, amssymb, amsthm} % стандартный набор AMS-пакетов для математ. текстов
\usepackage{mathtext}
\usepackage[utf8]{inputenc} % кодировка utf8
\usepackage[russian]{babel} % русский язык
\usepackage[pdftex,dvipsnames]{xcolor} % работа с цветами
\usepackage[pdftex]{graphicx} % графика (картинки)
\usepackage{tikz} % рисунки
\usepackage{fancyhdr,pageslts} % настройка колонтитулов
\usepackage{enumitem} % работа со списками
\usepackage{multicol} % работа с таблицами
%\usepackage{pscyr} % красивый шрифт
\usepackage{pgfornament} % красивые рюшечки и вензеля
\usepackage{ltxgrid} % управление написанием текста в две колонки
\usepackage{lipsum} % стандартный текст
\usepackage{tcolorbox} % рамка вокруг текста
\usepackage{float} % для корректного размещения картинок
\tcbuselibrary{skins}
% ----------------------------------------

\newcommand\ProblemName{Физика дождевых капель}

\newcommand\Source{X24}

\newcommand\Type{Условие задачи}

% настройки полей
\geometry{
	left=12mm,
	top=21mm,
	right=15mm,
	bottom=26mm,
	marginparsep=0mm,
	marginparwidth=0mm,
	headheight=22pt,
	headsep=2mm,
	footskip=7mm}
% ----------------------------------------

% настройки колонтитулов
\pagestyle{fancy}

\fancypagestyle{style}{
	\fancyhf{}
	\fancyhead[L]{{\Large{\FancyTitle}}\\\vskip -5pt \dotfill}
	\fancyhead[R]{{\Large{\textbf{\Type}}}\\\vskip -5pt \dotfill}
	\renewcommand{\headrulewidth}{0pt}
	\renewcommand{\footrulewidth}{0pt}
	\fancyfoot[C]{\pgfornament[width=2em,anchor=south]{72}\hspace{1mm}
		{Страница \textbf{\thepage} из \textbf{\pageref{VeryLastPage}}}\hspace{2mm}
		\pgfornament[width=2em,symmetry=v,anchor=south]{72}\\ \vskip2mm
		{\small{\textit{Условие собрано и подготовлено в Президентском ФМЛ №239 г.~Санкт-Петербурга}}}}
}

\fancypagestyle{plain}{
	\fancyhf{}
	\renewcommand{\headrulewidth}{0pt}
	\renewcommand{\footrulewidth}{0pt}
	\fancyhead[C]{{\Large{\textit{Учебно-тренировочные сборы к X23}}}\\\vskip -5pt \dotfill}
	\fancyfoot[C]{\pgfornament[width=2em,anchor=south]{72}\hspace{1mm}
		{Страница \textbf{\thepage} из \textbf{\pageref{VeryLastPage}}}\hspace{2mm}
		\pgfornament[width=2em,symmetry=v,anchor=south]{72}\\ \vskip2mm
		{\small{\textit{Условие собрано и подготовлено в Президентском ФМЛ №239 г.~Санкт-Петербурга}}}}
}
% ----------------------------------------

% другие настройки
\pagenumbering{arabic}
\setlist[enumerate,itemize]{leftmargin=0pt,itemindent=2.7em,itemsep=0cm}
% ----------------------------------------

% собственные команды
\newcommand{\FancyTitle}{\textbf{\Source} --- \ProblemName}
\newcommand{\Title}{\begin{center}{\huge{\textbf{\Source} --- \ProblemName}}\end{center}}
\newcommand{\Chapter}[1]{\vskip5pt{\Large{\textbf{#1}}}\vskip5pt}
\newcommand{\QText}[1]{#1}
\newcommand{\QBlock}[3]{
	\begin{tcolorbox}[left=4mm,top=3mm,bottom=2mm,right=4mm,colback=white]
		\begin{tcolorbox}[enhanced,colframe=blue,colback=blue!10!white,
			frame style={opacity=0.3},interior style={opacity=1.0},
			nobeforeafter,tcbox raise base,shrink tight,extrude by=1.7mm,width=1.5cm]
			\textbf{#1\textsuperscript{#2}}
		\end{tcolorbox}\hspace{3mm}#3
	\end{tcolorbox}
}
\newcommand{\QPicture}[4]{\QText{#4}  \includegraphics{#1}}
\newcommand{\ABlock}[1]{#1}
\newcommand{\MBlock}[2]{#1 #2}
\newcommand{\MMBlock}[3]{#1 #2 #3}
% ----------------------------------------


\begin{document}
	
	% настройки
	\pagestyle{style}\thispagestyle{plain}
	\Title
	% ----------------------------------------
	
	%\vskip5mm
	%\centering{\pgfornament[width=5cm,anchor=south]{89}}

\QText{Механизм образования облаков можно в общих чертах представить следующим образом. Влажный воздух поднимается вверх и охлаждается с высотой, из-за чего водяной пар становится перенасыщенным, то есть его  парциальное давление становится больше давления насыщенного пара при соответствующей температуре. При этом водяной пар начинает конденсироваться и образовывать капли воды. Пока эти капли достаточно малы, они падают достаточно медленно и остаются в облаке. Если же размер капель достаточно велик, чтобы они могли долетать до Земли не испарившись, начинается дождь. В этой задаче рассматривается механизм формирования капель в перенасыщенном водяном паре и их дальнейшего роста за счет диффузии.}

\QText{В задаче используются следующие обозначения и численные значения}

\QText{\begin{itemize} 
\item $\sigma = 7.5 \cdot 10^{-2}  \text{Н}/\text{м}$ – коэффициент поверхностного натяжения воды;
\item $T = 283 \text{К}$ – температура атмосферы;
\item $p_s = 1.23 \text{кПа}$ – давление насыщенного водяного пара при рассматриваемой температуре атмосферы;
\item $\rho_s$ – плотность насыщенного водяного пара при данной температуре;
\item $p_v$, $\rho_v$ – давление и плотность водяного пара в атмосфере;
\item $\varphi = \rho_v/\rho_s = p_v/p_s$ – коэффициент перенасыщения водяного пара;
\item $\rho_L = 1.0 \cdot 10^3  \text{кг}/\text{м}^3$ – плотность воды;
\item $\mu = 0.018  \text{кг}/\text{моль}$ – молярная масса воды;
\item $m$ – масса молекул воды;
\item $k = 1.38 \cdot 10^{-23} \text{Дж}/\text{К}$ – постоянная Больцмана;
\item $L = 2.48  \text{МДж}/\text{К}$ – удельная теплота парообразования воды;
\item $R = 8.31 \text{Дж}/(\text{моль} \cdot \text{К})$ – универсальная газовая постоянная;
\item $N_A = 6.02 \cdot 10^{23} \text{моль}^{-1}$ – постоянная Авогадро.
\end{itemize}}

\QText{Водяной пар во всех частях задачи можно считать идеальным газом.}

\Chapter{Часть A. Капли в однородной атмосфере (4 балла)}

\QText{Пусть атмосфера состоит только из воздуха и водяного пара без примесей. При образовании капли нужно затратить дополнительную энергию на создание поверхности воды. Поэтому даже в случае перенасыщенного водяного пара образование капель затруднено тем, что при малом размере относительная величина поверхностной энергии велика.}

\QText{ Из термодинамики известно, что для процесса при постоянной энергии его возможность определяется значением свободной энергии Гиббса $G$. Чем больше необходимая свободная энергия, тем менее вероятен процесс. Исследуем, как зависит свободная энергия, необходимая для формирования капли, от ее радиуса $r$. Для этого вам потребуются следующие факты:}

\QText{\begin{itemize} 
\item Свободная энергия поверхности жидкости площади $A$ равна $\Delta G_{surf} = \sigma A$, $\sigma$ – коэффициент поверхностного натяжения.
\item Свободная энергия одного моля  насыщенного водяного пара равна свободной энергии моля воды (без учета поверхностной энергии) при той же температуре и давлении.
\item Разность свободной энергии одного моля перенасыщенного водяного пара и насыщенного водяного пара при той же температуре равна $\Delta G_v= R T \ln \varphi$ ($\varphi = \rho_v/\rho_s > 1$ – коэффициент перенасыщения пара). Свободная энергия водяного пара пропорциональна его количеству вещества.
\end{itemize}}

\QBlock{A1}{1.00}{Найдите изменение свободной энергии водяного пара, если из него образовать каплю радиуса $r$. Выразите ответ через $r$, $\sigma$, $\varphi$, $R$, $T$, $\rho_L$, $\mu$.}

\QBlock{A2}{0.80}{Найдите критическое значение радиуса капли $r_c$, при котором $\Delta G$ максимально, а также соответствующее значение $\Delta G_с$. Выразите ответ через  $\sigma$, $\varphi$, $R$, $T$, $\rho_L$, $\mu$. Найдите численное значение $r_c$ при $\varphi = 1.01$.}

\QText{Пока капля не достигла радиуса $r_c$, ее рост сопровождается увеличением свободной энергии, поэтому маловероятен. Как только радиус превысит критический, дальнейший рост будет происходить без затруднений с уменьшением свободной энергии. Поэтому при исследовании количества возникающих капель можно сосредоточиться на каплях критического радиуса. Капля может сформироваться вокруг любой из молекул воды, однако вероятность такого процесса мала и определяется необходимой свободной энергией. Из статистической механики следует, что концентрация центров, вокруг которых фактически может произойти конденсация, равна 
$$
n_c = n e^{- \Delta G_c/ kT},
$$
где $n$ - концентрация молекул воды в паре, $\Delta G_c$ было найдено в предыдущем пункте.}

\QBlock{A3}{0.70}{Рассмотрим каплю критического радиуса $r_c$. Определите время $\tau$, за которое количество молекул в ней увеличится на $g$. Выразите ответ через $r_c$, $g$, $p_s$, $m$, $k$, $T$, $\varphi$. Считайте, что в процессе роста радиус капли не меняется, испарением молекул из капли можно пренебречь.
Известно, что на площадь $dS$ поверхности за время $dt$ попадает
$$
dN = dt dS  \frac{p_v}{\sqrt{2\pi m k T}}
$$
молекул. Здесь $p_v$ - давление пара, $m$ - масса молекул, $T$ - температура газа.}

\QText{Будем считать время $\tau$ характерным временем роста капель из зародыша. За время $\tau$ все имеющиеся в системе зародыши превращаются в капли критического радиуса, а на их месте появляются новые зародыши в таком же количестве.}

\QBlock{A4}{0.60}{Найдите количество капель $J$, которые образуются в единицу времени в единице объема перенасыщенного водяного пара. Выразите ответ через $\sigma$, $\varphi$, $p_s$, $r_c$, $T$, $g$.}

\QBlock{A5}{0.90}{Из результатов предыдущего пункта следует, что скорость образования капель очень сильно зависит от коэффициента перенасыщения пара. Определите численно значение коэффициента перенасыщения пара $\varphi$, при котором при температуре $T = 283 \text{К}$ в $1 \text{см}^3$ воздуха рождается одна капля в секунду. Считайте, что $g = 100$. Остальные численные данные приведены в начале задачи.}

\Chapter{Часть B. Диффузионный рост капель (4 балла)}

\QText{В этой части будем использовать следующие обозначения (в дополнение к приведенным в начале задачи):}

\QText{\begin{itemize} 
\item $\rho_v$ – плотность водяного пара на большом расстоянии от капли;
\item $\rho_r$ – плотность водяного пара вблизи поверхности капли; 
\item $\rho_s$ – плотность насыщенного водяного пара при температуре атмосферы на большом расстоянии до капли;
\item $T = 283 \text{К}$ – температура атмосферы на большом расстоянии до капли;
\item $T_r$ – температура капли;
\item $K =2.40 \cdot 10^{-2}  \text{Дж}/ (\text{м} \cdot \text{с} \cdot \text{К})$ – коэффициент теплопроводности воздуха;
\item $D = 2.36 \cdot 10^{-5} \text{м}^2/\text{с}$ – коэффициент диффузии водяного пара в воздухе;
\item $r$ – радиус капли;
\item $M$ – масса капли.
\end{itemize}}

\QText{Капля растет за счет диффузии. Скорость изменения массы капли и скорость отвода тепла задается соотношениями
$$
\frac{dM}{dt} = 4 \pi r D (\rho_v - \rho_r); \quad \frac{dQ}{dt} = 4\pi Kr (T_r - T).
$$
Будем считать, что температура капли в процессе ее роста остается постоянной, а все тепло выделяется только за  счет конденсации воды.}

\QBlock{B1}{0.80}{Для насыщенного пара, находящегося в равновесии с жидкостью, выразите производную давления по температуре $dp_s/dT$ через $p_s$, $L$, $R$, $T$, $\mu$. Используя полученный результат, найдите относительное изменение плотности насыщенного водяного пара $\Delta \rho_s/\rho_s$ при малом изменении температуры $\Delta T$. Выразите ответ через $\Delta T$, $T$, $L$, $\mu$, $R$. Вы можете использовать связь малых изменений давления, плотности и температуры идеального газа
$$
\frac{\Delta p_s}{p_s} = \frac{\Delta \rho_s}{\rho_s} +\frac {\Delta T}{T}.
$$}

\QBlock{B2}{0.20}{Выразите $dQ/dt$ через $dM/dt$ и $L$.}

\QBlock{B3}{0.30}{Используя результат предыдущего пункта и уравнение теплопроводности, выразите разность температур капли и атмосферы, $T_r- T$, через $dM/dt$, а также $r$, $L$, $K$.}

\QBlock{B4}{0.30}{Будем считать, что вблизи поверхности капли плотность водяного пара равна плотности насыщенного пара при температуре капли. Считая разности температур и плотностей малыми и используя результаты $B1$, $B3$ выразите отношение $(\rho_r - \rho_s)/\rho_s$ ($\rho_r$ - давление пара вблизи поверхности капли) через $L$, $r$, $K$, $\mu$, $R$, $T$ и $dM/dt$.}

\QBlock{B5}{0.30}{Используя уравнение диффузии, выразите отношение $(\rho_r - \rho_v)/\rho_s$ через $dM/dt$, $r$, $D$, $\rho_s$.}

\QBlock{B6}{0.60}{Исключив из ответов в двух предыдущих пунктах плотность пара вблизи поверхности капли $\rho_r$, получите выражение для $dM/dt$. Выразите ответ через $\varphi$,  $\mu$, $R$, $T$, $D$, $p_s$, $L$, $K$, $r$.}

\QBlock{B7}{0.50}{Скорость увеличения радиуса капли имеет вид
$$
\frac{dr}{dt} = \frac{\xi}{r^k}.
$$
Определите $k$ и $\xi$, выразите ответ через $\varphi $, $\rho_L$, $\mu$, $R$, $T$, $D$, $p_s$, $L$, $K$.}

\QBlock{B8}{0.50}{Найдите зависимость радиуса капли от времени. Начальный радиус капли равен $r_0$. Выразите ответ через $r_0$, $\xi$, $t$.}

\QBlock{B9}{0.50}{Пусть начальный радиус капли равен $r_0 = 0.7 \text{мкм}$. Найдите численное значение времени, за которое она вырастет до размера $r_1 = 10 \text{мкм}$ при коэффициенте перенасыщения $\varphi = 1.1$. Остальные численные значения приведены в начале этой части.}

\end{document}