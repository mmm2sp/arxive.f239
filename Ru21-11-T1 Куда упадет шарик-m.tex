
%\documentstyle[12pt,russian,amsthm,amsmath,amssymb]{article}
\documentclass[a4paper,11pt,twoside]{article}
\usepackage[left=14mm, top=10mm, right=14mm, bottom=10mm, nohead, nofoot]{geometry}
\usepackage{amsmath, amsfonts, amssymb, amsthm} % стандартный набор AMS-пакетов для математ. текстов
\usepackage{mathtext}
\usepackage[utf8]{inputenc} % кодировка utf8
\usepackage[russian]{babel} % русский язык
\usepackage[pdftex]{graphicx} % графика (картинки)
\usepackage{tikz}
\usepackage{fancyhdr,pageslts} % настройка колонтитулов
\usepackage{enumitem} % работа со списками
\usepackage{multicol} % работа с таблицами
%\usepackage{pscyr} % красивый шрифт
\usepackage{pgfornament} % красивые рюшечки и вензеля
\usepackage{ltxgrid} % управление написанием текста в две колонки
\usepackage{lipsum} % стандартный текст
\usepackage{tcolorbox} % рамка вокруг текста
\tcbuselibrary{skins}
% ----------------------------------------

\newcommand\ProblemName{Куда упадет шарик}

\newcommand\Source{Ru21-11-T1}

\newcommand\Type{Разбалловка}

\newcommand\MyTextLeft{Президентский ФМЛ 239, г.~Санкт-Петербург}
\newcommand\MyTextRight{Использованы материалы сайта pho.rs}
\newcommand\MyHeading{Учебно-тренировочные сборы по физике}
% ----------------------------------------

% настройки полей
\geometry{
	left=12mm,
	top=21mm,
	right=15mm,
	bottom=26mm,
	marginparsep=0mm,
	marginparwidth=0mm,
	headheight=22pt,
	headsep=2mm,
	footskip=7mm}
% ----------------------------------------

% настройки колонтитулов
\pagestyle{fancy}

\fancypagestyle{style}{
	\fancyhf{}
	\fancyhead[L]{{\Large{\FancyTitle}}\\\vskip -5pt \dotfill}
	\fancyhead[R]{{\Large{\textbf{\Type}}}\\\vskip -5pt \dotfill}
	\renewcommand{\headrulewidth}{0pt}
	\renewcommand{\footrulewidth}{0pt}
	\fancyfoot[C]{\pgfornament[width=2em,anchor=south]{72}\hspace{1mm}
		{Страница \textbf{\thepage} из \textbf{\pageref{VeryLastPage}}}\hspace{2mm}
		\pgfornament[width=2em,symmetry=v,anchor=south]{72}\\ \vskip2mm
		{\small{\textit{\MyTextLeft\hfill\MyTextRight}}}}
}

\fancypagestyle{plain}{
	\fancyhf{}
	\renewcommand{\headrulewidth}{0pt}
	\renewcommand{\footrulewidth}{0pt}
	\fancyhead[C]{{\Large{\textit{\MyHeading}}}\\\vskip -5pt \dotfill}
	\fancyfoot[C]{\pgfornament[width=2em,anchor=south]{72}\hspace{1mm}
		{Страница \textbf{\thepage} из \textbf{\pageref{VeryLastPage}}}\hspace{2mm}
		\pgfornament[width=2em,symmetry=v,anchor=south]{72}\\ \vskip2mm
		{\small{\textit{\MyTextLeft\hfill\MyTextRight}}}}
}
% ----------------------------------------

% другие настройки
\pagenumbering{arabic}
\setlist[enumerate,itemize]{leftmargin=0pt,itemindent=2.7em,itemsep=0cm}
% ----------------------------------------

% собственные команды
\newcommand{\FancyTitle}{\textbf{\Source} --- \ProblemName}
\newcommand{\Title}{\begin{center}{\huge{\textbf{\Source} --- \ProblemName}}\end{center}}
\newcommand{\Chapter}[1]{\vskip5pt{\Large{\textbf{#1}}}\vskip5pt}
\newcommand{\QText}[1]{#1}
\newcommand{\QBlock}[3]{
	\begin{tcolorbox}[left=2mm,top=2mm,bottom=1mm,right=2mm,colback=white]
		\begin{tcolorbox}[enhanced,colframe=ProcessBlue,colback=ProcessBlue!30!white,
			frame style={opacity=0.7},interior style={opacity=1.0},
			nobeforeafter,tcbox raise base,shrink tight,extrude by=1.7mm,width=1.5cm]
			\textbf{#1\textsuperscript{#2}}
		\end{tcolorbox}\hspace{3mm}#3
	\end{tcolorbox}
}
\newcommand{\QPicture}[4]{
	\begin{figure}[H]
		\centering
		\includegraphics[width=0.35\linewidth]{#1}
		\caption{#3}
	\end{figure}
	
	#4
}
\newcommand{\ABlock}[1]{
	\vskip2mm
	\begin{tcolorbox}[enhanced,colframe=Magenta,colback=Magenta!15!white,
		frame style={opacity=0.5},interior style={opacity=1.0},
		nobeforeafter,tcbox raise base,shrink tight,extrude by=1.7mm,width=1.6cm]
		\textbf{Ответ:}
	\end{tcolorbox}\hspace{3mm}#1
}
\newcommand{\MBlock}[2]{
	\begin{tcolorbox}[enhanced,colframe=Yellow,colback=Yellow!15!white,
		frame style={opacity=0.5},interior style={opacity=1.0},
		nobeforeafter,tcbox raise base,shrink tight,extrude by=1.7mm,width=1.1cm]
		\textbf{#1}
	\end{tcolorbox}\hspace{3mm}#2
}
\newcommand{\MMBlock}[3]{
	\begin{tcolorbox}[enhanced,colframe=Yellow,colback=Yellow!15!white,
		frame style={opacity=0.5},interior style={opacity=1.0},
		nobeforeafter,tcbox raise base,shrink tight,extrude by=1.7mm,width=1.1cm]
		\textbf{#1}
	\end{tcolorbox}\hspace{3mm}
	\begin{tcolorbox}[enhanced,colframe=Orange,colback=Orange!15!white,
		frame style={opacity=0.5},interior style={opacity=1.0},
		nobeforeafter,tcbox raise base,shrink tight,extrude by=1.7mm,width=0.8cm]
		\textbf{#2}
	\end{tcolorbox}\hspace{3mm}#3
}
% ----------------------------------------


\begin{document}
	
	% настройки
	\pagestyle{style}\thispagestyle{plain}
	\Title
	% ----------------------------------------
	
	%\vskip5mm
	%\centering{\pgfornament[width=5cm,anchor=south]{89}}
	
	% смысловая часть


\QBlock{1}{??}{Чему равно расстояние до точки падения на дно для лодки, движущейся в озере той же глубины, что и река?}

\MBlock{0.50}{Скорость движения лодки $v$ относительно воды (или в СО воды) постоянна}

\MBlock{0.75}{Время движения шарика от момента броска и до момента падения на дно $\tau$ во всех трех случаях одинаково.
Балл ставится только в случае корректного доказательства данного утверждения. 
Не влияет на оценку последующих пунктов.}

\MBlock{0.75}{Перемещение шарика в горизонтальной плоскости относительно воды (или в СО воды)  $s$ одинаково по модулю  во всех трех случаях.
Балл ставится только в случае корректного доказательства данного утверждения.
Не влияет на оценку последующих пунктов.}

\MBlock{0.50}{Модуль перемещения воды $u \tau$ относительно берега за время движения шарика одинаков для всех трёх случаев}

\MBlock{1.00}{Правильно записано выражение для связи модулей перемещений шарика в первом случае:
$$
l_1 = u \tau + s
$$}

\MBlock{1.00}{Правильно записано выражение для связи модулей перемещений шарика во втором случае:
$$
l_2 = u \tau - s
$$
При неверных знаках в правой части выражения за данный пункт ставится $0$, но в последующих пунктах баллы не снимаются}

\MBlock{1.00}{Получено  выражение для перемещения шарика в горизонтальной плоскости при движении в озере (или для всех случаев в СО воды):
$$
s = \frac{l_1 – l_2}{2}
$$}

\QBlock{2}{??}{Во сколько раз скорость лодки больше скорости течения?}

\MBlock{1.00}{Найден модуль перемещения воды относительно берега:
$$
u \tau = \frac{l_1 + l_2}{2}
$$}

\MBlock{1.00}{Правильно нарисована связь перемещений для третьего случая (или пояснена в тексте решения)}

\MBlock{1.50}{Правильно записана теорема косинусов или аналогичные выражения для прямоугольных треугольников в соответствие с рисунком
$$
l_3^2 = s^2+(u\tau)^2 - 2su\tau\cdot\cos\varphi
$$}

\MBlock{1.00}{Соотношение скоростей записано как тригонометрическая функция соответствующего угла (синус, косинус или тангенс)
$$
\cos \varphi = u/v
$$}

\MBlock{0.50}{Угол между направлениями скоростей показан на рисунке или есть его словесное определение}

\MBlock{1.50}{Получен верный ответ для соотношения скоростей:
$$
\frac{v}{u} = \frac{l_1^2 - l_2^2}{l_1^2+l_2^2 - 2 l_3^2}
$$}

\end{document}