
%This file is part of Get pho.rs!

%Get pho.rs! is free software: you can redistribute it and/or modify it under the terms of the GNU General Public License as published by the Free Software Foundation, either version 3 of the License, or (at your option) any later version.

%Get pho.rs! is distributed in the hope that it will be useful, but WITHOUT ANY WARRANTY; without even the implied warranty of MERCHANTABILITY or FITNESS FOR A PARTICULAR PURPOSE. See the GNU General Public License for more details.

%You should have received a copy of the GNU General Public License along with Foobar. If not, see <https://www.gnu.org/licenses/>.

%\documentstyle[12pt,russian,amsthm,amsmath,amssymb]{article}
\documentclass[a4paper,11pt,twoside]{article}
\usepackage[left=14mm, top=10mm, right=14mm, bottom=10mm, nohead, nofoot]{geometry}
\usepackage{amsmath, amsfonts, amssymb, amsthm} % стандартный набор AMS-пакетов для математ. текстов
\usepackage{mathtext}
\usepackage[utf8]{inputenc} % кодировка utf8
\usepackage[russian]{babel} % русский язык
\usepackage[pdftex,dvipsnames]{xcolor} % работа с цветами
\usepackage[pdftex]{graphicx} % графика (картинки)
\usepackage{tikz} % рисунки
\usepackage{fancyhdr,pageslts} % настройка колонтитулов
\usepackage{enumitem} % работа со списками
\usepackage{multicol} % работа с таблицами
%\usepackage{pscyr} % красивый шрифт
\usepackage{pgfornament} % красивые рюшечки и вензеля
\usepackage{ltxgrid} % управление написанием текста в две колонки
\usepackage{lipsum} % стандартный текст
\usepackage{tcolorbox} % рамка вокруг текста
\usepackage{float} % для корректного размещения картинок
\tcbuselibrary{skins}
% ----------------------------------------

\newcommand\ProblemName{Колебания в заряженном цилиндре}

\newcommand\Source{X22-T6}

\newcommand\Type{Разбалловка}

% настройки полей
\geometry{
	left=12mm,
	top=21mm,
	right=15mm,
	bottom=26mm,
	marginparsep=0mm,
	marginparwidth=0mm,
	headheight=22pt,
	headsep=2mm,
	footskip=7mm}
% ----------------------------------------

% настройки колонтитулов
\pagestyle{fancy}

\fancypagestyle{style}{
	\fancyhf{}
	\fancyhead[L]{{\Large{\FancyTitle}}\\\vskip -5pt \dotfill}
	\fancyhead[R]{{\Large{\textbf{\Type}}}\\\vskip -5pt \dotfill}
	\renewcommand{\headrulewidth}{0pt}
	\renewcommand{\footrulewidth}{0pt}
	\fancyfoot[C]{\pgfornament[width=2em,anchor=south]{72}\hspace{1mm}
		{Страница \textbf{\thepage} из \textbf{\pageref{VeryLastPage}}}\hspace{2mm}
		\pgfornament[width=2em,symmetry=v,anchor=south]{72}\\ \vskip2mm
		{\small{\textit{Условие собрано и подготовлено в Президентском ФМЛ №239 г.~Санкт-Петербурга}}}}
}

\fancypagestyle{plain}{
	\fancyhf{}
	\renewcommand{\headrulewidth}{0pt}
	\renewcommand{\footrulewidth}{0pt}
	\fancyhead[C]{{\Large{\textit{Учебно-тренировочные сборы к X23}}}\\\vskip -5pt \dotfill}
	\fancyfoot[C]{\pgfornament[width=2em,anchor=south]{72}\hspace{1mm}
		{Страница \textbf{\thepage} из \textbf{\pageref{VeryLastPage}}}\hspace{2mm}
		\pgfornament[width=2em,symmetry=v,anchor=south]{72}\\ \vskip2mm
		{\small{\textit{Условие собрано и подготовлено в Президентском ФМЛ №239 г.~Санкт-Петербурга}}}}
}
% ----------------------------------------

% другие настройки
\pagenumbering{arabic}
\setlist[enumerate,itemize]{leftmargin=0pt,itemindent=2.7em,itemsep=0cm}
% ----------------------------------------

% собственные команды
\newcommand{\FancyTitle}{\textbf{\Source} --- \ProblemName}
\newcommand{\Title}{\begin{center}{\huge{\textbf{\Source} --- \ProblemName}}\end{center}}
\newcommand{\Chapter}[1]{\vskip5pt{\Large{\textbf{#1}}}\vskip5pt}
\newcommand{\QText}[1]{#1}
\newcommand{\QBlock}[3]{
	\begin{tcolorbox}[left=4mm,top=3mm,bottom=2mm,right=4mm,colback=white]
		\begin{tcolorbox}[enhanced,colframe=blue,colback=blue!10!white,
			frame style={opacity=0.3},interior style={opacity=1.0},
			nobeforeafter,tcbox raise base,shrink tight,extrude by=1.7mm,width=1.5cm]
			\textbf{#1\textsuperscript{#2}}
		\end{tcolorbox}\hspace{3mm}#3
	\end{tcolorbox}
}
\newcommand{\QPicture}[4]{\QText{#4}  \includegraphics{#1}}
\newcommand{\ABlock}[1]{#1}
\newcommand{\MBlock}[2]{#1 #2}
\newcommand{\MMBlock}[3]{#1 #2 #3}
% ----------------------------------------


\begin{document}
	
	% настройки
	\pagestyle{style}\thispagestyle{plain}
	\Title
	% ----------------------------------------
	
	%\vskip5mm
	%\centering{\pgfornament[width=5cm,anchor=south]{89}}

\QBlock{A1}{0.50}{Диск радиусом $R$ заряжен поверхностной плотностью заряда $\sigma_R$. Определите потенциал $\varphi(y)$ в точке на оси на расстоянии $y$ от центра диска. Потенциал равен нулю на бесконечности.}

\MBlock{0.20}{Потенциал от кольца шириной $dr$ радиусом $r$
$$d \varphi = k \frac{\sigma_R \cdot 2 \pi r dr}{\sqrt{y^2 + r^2} }$$}

\MBlock{0.30}{Ответ. Потенциал диска:
$$\varphi(y)=2\pi k \sigma_R \left( \sqrt{R^2+y^2}-|y| \right)$$}

\QBlock{A2}{1.00}{Два таких диска радиусом $R$ заряжены поверхностной плотностью заряда $\sigma_R>0$ находятся параллельно друг другу. Расстояние между центрами дисков равно $2L$, центры находятся на оси дисков. В положении равновесия находятся заряд $q$ массой $m$, который может двигаться только вдоль оси дисков. Определите угловую частоту $\omega_1$ колебаний такого заряда. Какой знак заряда?}

\MBlock{0.10}{Идейно правильный способ найти $E(x)$. Например, посчитать поле кольца или $E_x = -\frac{d\varphi}{dx}$.}

\MBlock{0.50}{Напряженность электрического поля
$$E_x(x) = - k \frac{2\pi R \delta \sigma_R}{(L^2+R^2)} \frac{L}{\sqrt{L^2+R^2}}=-\frac{\sigma_R R^2}{\varepsilon_0 (L^2+R^2)^{3/2}}x.$$

ИЛИ найден коэффициент перед $x^2$ в разложении для потенциала.}

\MBlock{0.30}{Ответ. Угловая частота при колебаниях:
$$\omega_1^2=\frac{q\sigma_R R^2}{m \varepsilon_0  (L^2+R^2)^{3/2}}.$$}

\MBlock{0.10}{Ответ. Знак заряда:
$$q>0.$$}

\QBlock{A3}{1.00}{Теперь этот заряд может двигаться только в перпендикулярном направлении. Выразите угловую частоту $\omega_2$ колебаний в таком случае через $\omega_1$. Какой теперь знак заряда?}

\MBlock{0.10}{Идея использовать теорему Гаусса}

\MBlock{0.50}{Запись теоремы Гаусса
$$2\pi r \cdot 2x E_r(r) + 2\pi r^2 E_x(x) = 0$$}

\MBlock{0.20}{Найдена $\omega_2$ через заданные в условии величины  (это находить необязательно):
$$\omega_2^2=\frac{|q|\sigma_R R^2}{2m \varepsilon_0  (L^2+R^2)^{3/2}},$$}

\MBlock{0.10}{Ответ. Угловая частота:
$$\omega_2 = \frac{\omega_1}{\sqrt{2}}.$$}

\MBlock{0.10}{Ответ. Знак заряда:
$$q<0.$$}

\QBlock{B1}{1.00}{Боковая поверхность цилиндра радиусом $R$ и длиной $L$ заряжена поверхностной плотностью заряда $\sigma_L$. Определите потенциал в точке на оси на расстоянии $z$ от центра одного из оснований цилиндра. Потенциал равен нулю на бесконечности.}

\MBlock{0.30}{Потенциал от кольца высотой $dl$ на расстоянии $l$ от нуля
$$d \varphi (y) = k \frac{\sigma_L \cdot 2 \pi R dl}{\sqrt{(z+l)^2 + R^2}}.$$}

\MBlock{0.20}{Правильные пределы интегрирования (от $z$ до $z+L$)}

\MBlock{0.50}{Ответ. Потенциал боковых стенок цилиндра.
$$\varphi (y) = 2\pi k R\sigma_L \left( \mathrm{arth} \frac{L+z}{\sqrt{(L+z)^2 + R^2}} - \mathrm{arth} \frac{z}{\sqrt{z^2 + R^2}}\right).$$}

\QBlock{B2}{1.00}{Два таких цилиндра (радиусом $R$ и длиной $L$, поверхность заряжена поверхностной плотностью заряда $\sigma_L>0$) поставлены рядом вплотную и имеют общую ось. В положении равновесия находятся заряд $q$ массой $m$, который может двигаться только вдоль оси цилиндров. Определите угловую частоту $\omega_3$ колебаний такого заряда. Какой знак заряда?}

\MBlock{0.10}{Идейно правильный способ найти $E(x)$. Например, посчитать поле кольца или $E_x = -\frac{d\varphi}{dx}$.}

\MBlock{0.50}{Напряженность электрического поля
$$E_x(x) = k \frac{2\pi R \delta \sigma_L}{(L^2+R^2)} \frac{L}{\sqrt{L^2+R^2}}=\frac{\sigma_L R L}{\varepsilon_0 (L^2+R^2)^{3/2}}x.$$

ИЛИ найден коэффициент перед $x^2$ в разложении для потенциала.}

\MBlock{0.30}{Ответ. Угловая частота:
$$\omega_3^2=\frac{|q|\sigma_L R L}{m \varepsilon_0  (L^2+R^2)^{3/2}}.$$}

\MBlock{0.10}{Знак точечного заряда.
$$q<0.$$}

\QBlock{B3}{0.50}{Теперь этот заряд может двигаться только в перпендикулярном направлении. Выразите угловую частоту $\omega_4$ колебаний в таком случае через $\omega_3$. Какой теперь знак заряда?}

\MBlock{0.20}{Связь двух напряженностей
$$E_r(r)=-\frac{1}{2}E_x(x)\frac{r}{x},$$}

\MBlock{0.10}{Угловая частота (это находить необязательно):
$$\omega_4^2=\frac{|q|\sigma_L R L}{2m \varepsilon_0  (L^2+R^2)^{3/2}}.$$}

\MBlock{0.10}{Ответ. Угловая частота 
$$\omega_4 = \frac{\omega_3}{\sqrt{2}},$$}

\MBlock{0.10}{Ответ. Знак заряда.
$$q>0.$$}

\QBlock{С1}{1.50}{Заряженный цилиндр радиусом $R$ высотой $L=40R/9$ состоит из боковой поверхности и одного основания. Поверхностная плотность заряда боковой поверхности $\sigma_L$, основания $\sigma_R$. Если поместить точечный заряд в центр противоположного основания, то он окажется в положении равновесия. Определите отношение $\sigma_L/\sigma_R$.}

\MBlock{0.10}{Указано, что сумма двух действующих сил равна нулю.}

\MBlock{0.50}{Напряженность поля от основания цилиндра
$$E_R = 2\pi \sigma_R k \left( 1 - \frac{L}{\sqrt{R^2+L^2}} \right).$$}

\MBlock{0.50}{Напряженность поля от боковых стенок цилиндра
$$E_L = 2\pi \sigma_L k \left( 1 - \frac{R}{\sqrt{R^2+L^2}} \right).$$}

\MBlock{0.30}{Ответ в общем виде.
$$\frac{\sigma_{L}}{\sigma_{R}}=-\frac{\sqrt{R^{2}+L^{2}}-L}{\sqrt{R^{2}+L^{2}}-R}$$}

\MBlock{0.10}{Ответ численно.
$$\frac{\sigma_{L}}{\sigma_{R}}=-\frac{1}{32}$$}

\QBlock{С2}{2.50}{Заряженный цилиндр радиусом $R=28b$ высотой $L=45b$ состоит из боковой поверхности и одного основания. Заряд боковой поверхности $\sigma_L=-8\sigma_0$, заряд основания $\sigma_R=25\sigma_0>0$. На оси этой системы помещают частицу c зарядом $q>0$. Оцените численно координаты $z$ (в единицах $b$) положений равновесия если частица может двигаться только вдоль оси.  Координата $z$ отсчитывается как на картинке.
Сделайте это максимально точно, однако, достаточно с точностью 1\%. Ответы попадающие в 1\% от правильного получат полный балл.}

\MBlock{0.30}{Указано уравнение, решение которого позволяет найти положение равновесия $z=0$, например:
$$25\left(\frac{45+z}{\sqrt{28^2+(45+z)^2}} -1 \right) - 8 \left(\frac{28}{\sqrt{28^2+(45+z)^2}} - \frac{28}{\sqrt{28^2+z^2}} \right) =0 .$$}

\MBlock{0.30}{Указан способ решить уравнение}

\MBlock{0.40}{Ответ:
$z=0$}

\MBlock{0.30}{Понимание того, что нужно записать другое уравнение для точек с другой стороны цилиндра.}

\MBlock{0.30}{Указано уравнение, решение которого позволяет найти положение равновесия $z=-1632.6$, например:
$$25\left(\frac{45+z}{\sqrt{28^2+(45+z)^2}} +1 \right) - 8 \left(\frac{28}{\sqrt{28^2+(45+z)^2}} - \frac{28}{\sqrt{28^2+z^2}} \right) =0 .$$}

\MBlock{0.30}{Указан способ решить уравнение c достаточной точностью}

\MBlock{0.10}{Найден корень уравнения с точностью $10\%$}

\MBlock{0.20}{Найден корень уравнения с точностью $1\%$}

\MBlock{0.30}{Ответ:
$$\frac{z}{b}=-1632.6...$$}

\QBlock{С3}{1.00}{В условиях предыдущего пункта частицу поместили в ближайшее к цилиндру положение равновесия, её масса $m$. Определите угловую частоту $\omega$ малых колебаний частицы.}

\MBlock{0.30}{Выбрана точка с координатой 
$$z=0$$}

\MBlock{0.50}{Напряженность поля
$$E(z) = - \alpha z,$$
$$\alpha=2\pi \sigma_0 k \cdot \frac{560}{2809} \frac{z}{b}$$}

\MBlock{0.20}{Ответ:
$$\omega^2 = \frac{280 \sigma_0 q}{2809 \varepsilon_0 b m}$$}

\end{document}