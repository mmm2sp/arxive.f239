
%This file is part of Get pho.rs!

%Get pho.rs! is free software: you can redistribute it and/or modify it under the terms of the GNU General Public License as published by the Free Software Foundation, either version 3 of the License, or (at your option) any later version.

%Get pho.rs! is distributed in the hope that it will be useful, but WITHOUT ANY WARRANTY; without even the implied warranty of MERCHANTABILITY or FITNESS FOR A PARTICULAR PURPOSE. See the GNU General Public License for more details.

%You should have received a copy of the GNU General Public License along with Foobar. If not, see <https://www.gnu.org/licenses/>.

%\documentstyle[12pt,russian,amsthm,amsmath,amssymb]{article}
\documentclass[a4paper,11pt,twoside]{article}
\usepackage[left=14mm, top=10mm, right=14mm, bottom=10mm, nohead, nofoot]{geometry}
\usepackage{amsmath, amsfonts, amssymb, amsthm} % стандартный набор AMS-пакетов для математ. текстов
\usepackage{mathtext}
\usepackage[utf8]{inputenc} % кодировка utf8
\usepackage[russian]{babel} % русский язык
\usepackage[pdftex,dvipsnames]{xcolor} % работа с цветами
\usepackage[pdftex]{graphicx} % графика (картинки)
\usepackage{tikz} % рисунки
\usepackage{fancyhdr,pageslts} % настройка колонтитулов
\usepackage{enumitem} % работа со списками
\usepackage{multicol} % работа с таблицами
%\usepackage{pscyr} % красивый шрифт
\usepackage{pgfornament} % красивые рюшечки и вензеля
\usepackage{ltxgrid} % управление написанием текста в две колонки
\usepackage{lipsum} % стандартный текст
\usepackage{tcolorbox} % рамка вокруг текста
\usepackage{float} % для корректного размещения картинок
\tcbuselibrary{skins}
% ----------------------------------------

\newcommand\ProblemName{Проводники в магнитном поле}

\newcommand\Source{X24}

\newcommand\Type{Условие задачи}

% настройки полей
\geometry{
	left=12mm,
	top=21mm,
	right=15mm,
	bottom=26mm,
	marginparsep=0mm,
	marginparwidth=0mm,
	headheight=22pt,
	headsep=2mm,
	footskip=7mm}
% ----------------------------------------

% настройки колонтитулов
\pagestyle{fancy}

\fancypagestyle{style}{
	\fancyhf{}
	\fancyhead[L]{{\Large{\FancyTitle}}\\\vskip -5pt \dotfill}
	\fancyhead[R]{{\Large{\textbf{\Type}}}\\\vskip -5pt \dotfill}
	\renewcommand{\headrulewidth}{0pt}
	\renewcommand{\footrulewidth}{0pt}
	\fancyfoot[C]{\pgfornament[width=2em,anchor=south]{72}\hspace{1mm}
		{Страница \textbf{\thepage} из \textbf{\pageref{VeryLastPage}}}\hspace{2mm}
		\pgfornament[width=2em,symmetry=v,anchor=south]{72}\\ \vskip2mm
		{\small{\textit{Условие собрано и подготовлено в Президентском ФМЛ №239 г.~Санкт-Петербурга}}}}
}

\fancypagestyle{plain}{
	\fancyhf{}
	\renewcommand{\headrulewidth}{0pt}
	\renewcommand{\footrulewidth}{0pt}
	\fancyhead[C]{{\Large{\textit{Учебно-тренировочные сборы к X23}}}\\\vskip -5pt \dotfill}
	\fancyfoot[C]{\pgfornament[width=2em,anchor=south]{72}\hspace{1mm}
		{Страница \textbf{\thepage} из \textbf{\pageref{VeryLastPage}}}\hspace{2mm}
		\pgfornament[width=2em,symmetry=v,anchor=south]{72}\\ \vskip2mm
		{\small{\textit{Условие собрано и подготовлено в Президентском ФМЛ №239 г.~Санкт-Петербурга}}}}
}
% ----------------------------------------

% другие настройки
\pagenumbering{arabic}
\setlist[enumerate,itemize]{leftmargin=0pt,itemindent=2.7em,itemsep=0cm}
% ----------------------------------------

% собственные команды
\newcommand{\FancyTitle}{\textbf{\Source} --- \ProblemName}
\newcommand{\Title}{\begin{center}{\huge{\textbf{\Source} --- \ProblemName}}\end{center}}
\newcommand{\Chapter}[1]{\vskip5pt{\Large{\textbf{#1}}}\vskip5pt}
\newcommand{\QText}[1]{#1}
\newcommand{\QBlock}[3]{
	\begin{tcolorbox}[left=4mm,top=3mm,bottom=2mm,right=4mm,colback=white]
		\begin{tcolorbox}[enhanced,colframe=blue,colback=blue!10!white,
			frame style={opacity=0.3},interior style={opacity=1.0},
			nobeforeafter,tcbox raise base,shrink tight,extrude by=1.7mm,width=1.5cm]
			\textbf{#1\textsuperscript{#2}}
		\end{tcolorbox}\hspace{3mm}#3
	\end{tcolorbox}
}
\newcommand{\QPicture}[4]{\QText{#4}  \includegraphics{#1}}
\newcommand{\ABlock}[1]{#1}
\newcommand{\MBlock}[2]{#1 #2}
\newcommand{\MMBlock}[3]{#1 #2 #3}
% ----------------------------------------


\begin{document}
	
	% настройки
	\pagestyle{style}\thispagestyle{plain}
	\Title
	% ----------------------------------------
	
	%\vskip5mm
	%\centering{\pgfornament[width=5cm,anchor=south]{89}}

\QText{Если проводник находится в переменном магнитном поле, либо движется в неоднородном магнитном поле – в нём возникает электрический ток. Возникновение электрического тока в проводниках является следствием явления электромагнитной индукции и достаточно хорошо изучено. В частности, эффект возникновения электрического тока в проводниках приводит к эффекту так называемого ‘’магнитного торможения'', который многим из вас наверняка приходилось наблюдать вживую. }

\QText{В данной задаче изучаются колебания проводящих тел с учётом влияния магнитного поля. Мы изучим два предельных перехода, соответствующие проводящим телам:}

\Chapter{Часть A. Свободные колебания с затуханием (1.2 балла).}

\QPicture{X24 Проводники в магнитном поле_files/27287.jpeg}{"width:100\%; max-width: 100\%;display:block;margin-left: auto;margin-right: auto; cursor:pointer;" data-toggle="modal" data-target="#kt_modal_27287"}{}{Рассмотрим вертикальный пружинный маятник, состоящий из невесомой пружины с коэффициентом жёсткости $k$, один конец которой закреплён, а к другому концу прикреплён груз массой $m$. При движении со скоростью $\vec{v}$ на него действует сила сопротивления $\vec{F}_\text{с}=-\beta\vec{v}$, где $\beta$ - известная постоянная величина.
Введём ось $x$, направленную вдоль пружины так, что при увеличении координаты $x$ груза длина пружины уменьшается, а в положении равновесия координата груза $x_0=0$. Во всех пунктах частей $\mathrm{A}-\mathrm{C}$ шар перемещается только по вертикали.
Также введём обозначения:
$$\gamma=\cfrac{\beta}{2m}\qquad \omega_0=\sqrt{\cfrac{k}{m}}\quad \gamma{<}\omega_0{.}
$$}

\QBlock{A1}{0.60}{Пусть момент времени $t_0=0$ груз находится в начале координат, а проекция его скорости на ось $x$ равна $v_0$. Определите зависимости координаты $x(t)$ и скорости $v_x(t)$ груза от времени $t$. Ответ выразите через $v_0$, $\gamma$, $\omega_0$ и $t$.}

\QText{Пусть $E_0$ - кинетическая энергия груза при прохождении начала координат, а $E_1$ - кинетическая энергия груза при последующем прохождении начала координат с тем же направлением скорости. Определим добротность $Q$ колебательной системы следующим образом:
$$Q=\cfrac{2\pi E_0}{E_0-E_1}{.}
$$}

\QBlock{A2}{0.40}{Получите точное выражение для $Q$. Ответ выразите через $\omega_0$ и $\gamma$.}

\QBlock{A3}{0.20}{Получите приближённое выражение для добротности $Q$ при слабом затухании ($\gamma\ll\omega_0$).
Ответ выразите через $m$, $k$ и $\beta$.}

\Chapter{Часть B. Вынужденные колебания (1.2 балла).}

\QPicture{X24 Проводники в магнитном поле_files/27293.jpeg}{"width:100\%; max-width: 100\%;display:block;margin-left: auto;margin-right: auto; cursor:pointer;" data-toggle="modal" data-target="#kt_modal_27293"}{}{Рассмотрим пружинный маятник из части $\mathrm{A}$ задачи. Координата $x_1$ второго конца пружины (к которому не прикреплён груза) изменяется по следующему закону:
$$x_1(t)=x_{1(0)}+A_0\sin\Omega t{,}
$$
где $A_0{>}0$, а $x_{1(0)}$ соответствует состоянию покоя груза.
Далее рассматривайте только установившийся режим движения под действием вынуждающей силы. Используйте введённые ранее величины $\omega_0$ и $\gamma$.}

\QBlock{B1}{0.60}{Отклонение $x$ груза от положения зависит от времени $t$ следующим образом:
$$x(t)=A\sin\left(\Omega t+\varphi_0\right)
$$
Найдите $A$ и $\varphi_0$. Ответы выразите через $A_0$, $\Omega$, $\omega_0$ и $\gamma$.}

\QText{Будем называть резонансной такую циклическую частоту колебаний $\Omega_\text{рез}$, при которой амплитуда колебаний системы максимальна и обозначается как $A_\text{рез}$.}

\QBlock{B2}{0.30}{Получите точные выражения для резонансной циклической частоты $\Omega_\text{рез}$ и соответствующей ей амплитуды колебаний $A_\text{рез}$. Ответы выразите через $\omega_0$, $\gamma$ и $A_0$. Считайте, что $\gamma\sqrt{2}{<}\omega_0$.}

\QText{Шириной резонансной кривой $\Delta{\omega}$ называется разность максимальной и минимальной циклических частот $\Omega_{max}$ и $\Omega_{min}$ соответственно, при которых амплитуда колебаний меньше резонансной в $\sqrt{2}$ раз.}

\QBlock{B3}{0.30}{Получите приближённые выражения для $\Omega_\text{рез}$, $A_\text{рез}$ и $\Delta{\omega}$ при слабом затухании ($\gamma\ll{\omega_0}$).
Ответы выразите через $A_0$, $\omega_0$ и $\gamma$.}

\Chapter{Часть C. Влияние поля кольца на движение шара (4.2 балла).}

\QText{Сплошной однородный шар массой $m$ и радиусом $R_0$ изготовлен из материала с большим удельным сопротивлением $\rho$. Его центр может перемещаться вдоль оси вращения кольца радиусом $R$, плоскость которого горизонтальна.
Силу тока в кольце медленно увеличивают до $I$ и далее поддерживают постоянной.
Для определения положения центра шара введём ось $x$ с началом в центре кольца, направленную вверх.  Считайте, что диэлектрическая и магнитная проницаемости шара $\varepsilon$ и $\mu$ соответственно равны единице, а радиус шара $R_0$ удовлетворяет условиям:
$$R_0\ll{R{,}x}{.}
$$

Из общефизических соображений ясно, что при движении шара со скоростью $\vec{v}$ вдоль оси вращения кольца действующая на него со стороны кольца сила имеет вид:
$$\vec{F}=-\beta(x)\vec{v}{,}
$$
где $\beta(x)$ - коэффициент пропорциональности, зависящий от координаты $x$ центра шара.}

\QText{Начнём с изучения магнитного поля кольца.}

\QBlock{C1}{0.30}{Найдите индукцию $B_x$ магнитного поля кольца на его оси в точке с координатой $x$.
Ответ выразите через $x$, $R$, $I$ и магнитную постоянную $\mu_0$.}

\QText{Рассмотрим исходный шар радиусом $R_0$ с удельным сопротивлением $\rho$, находящийся в однородном магнитном поле $\vec{B}=\vec{e}_xB$. Не изменяя направления, величину магнитного поля изменяют со скоростью $dB/dt=\dot{B}$.
Выделим в шаре диск радиусом $r_0$ и толщиной $h\ll{r_0}$, основания которого перпендикулярны оси $x$.}

\QBlock{C2}{1.00}{Определите магнитный момент $\vec{m}$ диска.
Ответ выразите через $\vec{e}_x$, $r_0$, $h$, $\rho$ и $\dot{B}$.}

\QBlock{C3}{0.50}{Определите магнитный момент $\vec{m}$ шара.
Ответ выразите через $\vec{e}_x$, $R_0$, $\rho$ и $\dot{B}$.}

\QText{Теперь рассмотрим движение шара вдоль оси кольца со скоростью $v_x$.}

\QBlock{C4}{0.40}{Получите производную по времени индукции магнитного поля кольца в центре шара $dB_x/dt$, эквивалентную величине $\dot{B}$.
Ответ выразите через $v$, $I$, $R$, $x$ и магнитную постоянную $\mu_0$.}

\QBlock{C5}{0.50}{Найдите коэффициент пропорциональности $\beta(x)$.
Ответ выразите через $I$, $R$, $x$, $R_0$, $\rho$ и магнитную постоянную $\mu_0$.}

\QText{Воспользуемся полученным результатом для $\beta(x)$ при определении удельного сопротивления шара по свободным колебаниям, а также по амплитудно-частотной характеристике вынужденных колебаний.
Шар закрепили на одном из концов невесомой непроводящей пружины с коэффициентом жёсткости $k$. Другой конец пружины закреплён в точке с координатой $x_{1(0)}$.
В положении равновесия центр шара расположен на высоте $H\gg{R_0}$ над центром кольца.
Отклонение центра шара $\Delta{x}$ от положения равновесия всегда удовлетворяет условию:
$$\Delta{x}\ll{R{,}H}{;}
$$}

\QPicture{X24 Проводники в магнитном поле_files/16094.jpeg}{"width:100\%; max-width: 100\%;display:block;margin-left: auto;margin-right: auto; cursor:pointer;" data-toggle="modal" data-target="#kt_modal_16094"}{}{На первом графике представлена зависимость отклонения шара от положения равновесия при собственных колебаниях в некоторых условных единицах. Второй конец пружины при этом неподвижен.
На втором графике представлена зависимость амплитуды вынужденных колебаний $A$ от частоты $\Omega$ в некоторых условных единицах. Координата второго конца пружины начинает изменяется по закону:
$$x_1(t)=x_{1(0)}+A_0\sin\Omega t{.}
$$}

\QPicture{X24 Проводники в магнитном поле_files/27305.jpeg}{"max-width:500px;px;display:block;margin-left: auto;margin-right: auto;"}{}{}

\QPicture{X24 Проводники в магнитном поле_files/29614.jpeg}{"max-width:500px;px;display:block;margin-left: auto;margin-right: auto;"}{}{}

\QBlock{C6}{0.80}{Определите удельное сопротивление $\rho$ шара, используемого в первом эксперименте.
Ответ выразите через $m$, $k$, $R_0$, $R$, $H$, $I$ и магнитную постоянную $\mu_0$.}

\QBlock{C7}{0.70}{Определите удельное сопротивление $\rho$ шара, используемого во втором эксперименте.
Ответ выразите через $m$, $k$, $R_0$, $R$, $H$, $I$ и магнитную постоянную $\mu_0$.}

\Chapter{Часть D. "Вмороженность" магнитного поля (3.4 балла).}

\QText{Данная часть задачи посвящена изучению магнитного поля, возникающего в результате перемещения очень хороших проводников в них.}

\QPicture{X24 Проводники в магнитном поле_files/28478.jpeg}{"width:100\%; max-width: 100\%;display:block;margin-left: auto;margin-right: auto; cursor:pointer;" data-toggle="modal" data-target="#kt_modal_28478"}{}{Рассмотрим следующую конструкцию: Соосно полубесконечному круговому соленоиду радиусом $R$ с плотностью намотки витков $n$ и силой тока $I$ в них расположен очень длинный хорошо проводящий цилиндр массой $m$ радиусом $r\ll{R}$, концы которого расположены по разные стороны от основания соленоида и удалены от него на расстояния, во много раз превышающие его радиус. В изначальном положении цилиндра токи в нём отсутствуют.
Из-за высокой проводимости вещества силовые линии индукции магнитного поля оказываются в него вморожены. Это означает, что при перемещении вещества силовые линии индукции магнитного поля будут перемещаться вместе с ним. В данном случае, соответствующем твёрдому телу, это приводит к тому, что индукция магнитного поля в каждой точке цилиндра будет сохраняться при его перемещении, что обусловлено возникновением в стержне круговых токов Фуко.}

\QText{Решайте задачу в следующих приближениях:}

\QText{\begin{itemize} 
\item Возникающие в цилиндре токи текут только по его поверхности;
\item Вне цилиндра индукция магнитного поля равна индукции магнитного поля соленоида;
\item Цилиндр отклоняется от изначального положения на величину $x\ll{R}$;
\item Взаимодействием цилиндра с подводящими проводами можно пренебречь;
\item За времена, рассматриваемые в данной задаче, затуханием токов в цилиндре можно пренебречь.
\end{itemize}}

\QText{Индукцию магнитного поля соленоида будем характеризовать осью $z$, направленную наружу соленоида вдоль его оси. Начало оси $z$ совпадает с центром основания соленоида.}

\QPicture{X24 Проводники в магнитном поле_files/29612.jpeg}{"max-width:500px;px;display:block;margin-left: auto;margin-right: auto;"}{}{}

\QBlock{D1}{0.60}{Определите индукцию $B_z$ магнитного поля соленоида, а также её производную $dB_z/dz$ в точке с координатой $z$. Ответ выразите через $\mu_0$, $n$, $I$, $R$ и $z$.}

\QText{Пусть цилиндр отклоняют на величину $x\ll{R}$ вдоль оси $z$ от изначального положения.}

\QBlock{D2}{1.00}{Определите линейную плотность тока $i$ на поверхности цилиндра в точке с координатой $z$. Ответ выразите через $\mu_0$, $x$ и $dB_z(z)/dz$.}

\QBlock{D3}{1.50}{Определите силу $F_x$, действующую на цилиндр со стороны магнитного поля соленоида. Ответ выразите через $\mu_0$, $r$, $R$, $n$, $I$ и $x$.}

\QText{Пусть в изначальном положении цилиндру сообщили скорость $v_0$, направленную вдоль оси $z$.}

\QBlock{D4}{0.30}{Получите зависимость перемещения стержня $x$ от времени $t$. Ответ выразите через $\mu_0$, $r$, $R$, $n$, $I$ и $m$.}

\end{document}