
%This file is part of Get pho.rs!

%Get pho.rs! is free software: you can redistribute it and/or modify it under the terms of the GNU General Public License as published by the Free Software Foundation, either version 3 of the License, or (at your option) any later version.

%Get pho.rs! is distributed in the hope that it will be useful, but WITHOUT ANY WARRANTY; without even the implied warranty of MERCHANTABILITY or FITNESS FOR A PARTICULAR PURPOSE. See the GNU General Public License for more details.

%You should have received a copy of the GNU General Public License along with Foobar. If not, see <https://www.gnu.org/licenses/>.

%\documentstyle[12pt,russian,amsthm,amsmath,amssymb]{article}
\documentclass[a4paper,11pt,twoside]{article}
\usepackage[left=14mm, top=10mm, right=14mm, bottom=10mm, nohead, nofoot]{geometry}
\usepackage{amsmath, amsfonts, amssymb, amsthm} % стандартный набор AMS-пакетов для математ. текстов
\usepackage{mathtext}
\usepackage[utf8]{inputenc} % кодировка utf8
\usepackage[russian]{babel} % русский язык
\usepackage[pdftex,dvipsnames]{xcolor} % работа с цветами
\usepackage[pdftex]{graphicx} % графика (картинки)
\usepackage{tikz} % рисунки
\usepackage{fancyhdr,pageslts} % настройка колонтитулов
\usepackage{enumitem} % работа со списками
\usepackage{multicol} % работа с таблицами
%\usepackage{pscyr} % красивый шрифт
\usepackage{pgfornament} % красивые рюшечки и вензеля
\usepackage{ltxgrid} % управление написанием текста в две колонки
\usepackage{lipsum} % стандартный текст
\usepackage{tcolorbox} % рамка вокруг текста
\usepackage{float} % для корректного размещения картинок
\tcbuselibrary{skins}
% ----------------------------------------

\newcommand\ProblemName{Керлинг}

\newcommand\Source{Y21-T6}

\newcommand\Type{Решение}

% настройки полей
\geometry{
	left=12mm,
	top=21mm,
	right=15mm,
	bottom=26mm,
	marginparsep=0mm,
	marginparwidth=0mm,
	headheight=22pt,
	headsep=2mm,
	footskip=7mm}
% ----------------------------------------

% настройки колонтитулов
\pagestyle{fancy}

\fancypagestyle{style}{
	\fancyhf{}
	\fancyhead[L]{{\Large{\FancyTitle}}\\\vskip -5pt \dotfill}
	\fancyhead[R]{{\Large{\textbf{\Type}}}\\\vskip -5pt \dotfill}
	\renewcommand{\headrulewidth}{0pt}
	\renewcommand{\footrulewidth}{0pt}
	\fancyfoot[C]{\pgfornament[width=2em,anchor=south]{72}\hspace{1mm}
		{Страница \textbf{\thepage} из \textbf{\pageref{VeryLastPage}}}\hspace{2mm}
		\pgfornament[width=2em,symmetry=v,anchor=south]{72}\\ \vskip2mm
		{\small{\textit{Условие собрано и подготовлено в Президентском ФМЛ №239 г.~Санкт-Петербурга}}}}
}

\fancypagestyle{plain}{
	\fancyhf{}
	\renewcommand{\headrulewidth}{0pt}
	\renewcommand{\footrulewidth}{0pt}
	\fancyhead[C]{{\Large{\textit{Учебно-тренировочные сборы к X23}}}\\\vskip -5pt \dotfill}
	\fancyfoot[C]{\pgfornament[width=2em,anchor=south]{72}\hspace{1mm}
		{Страница \textbf{\thepage} из \textbf{\pageref{VeryLastPage}}}\hspace{2mm}
		\pgfornament[width=2em,symmetry=v,anchor=south]{72}\\ \vskip2mm
		{\small{\textit{Условие собрано и подготовлено в Президентском ФМЛ №239 г.~Санкт-Петербурга}}}}
}
% ----------------------------------------

% другие настройки
\pagenumbering{arabic}
\setlist[enumerate,itemize]{leftmargin=0pt,itemindent=2.7em,itemsep=0cm}
% ----------------------------------------

% собственные команды
\newcommand{\FancyTitle}{\textbf{\Source} --- \ProblemName}
\newcommand{\Title}{\begin{center}{\huge{\textbf{\Source} --- \ProblemName}}\end{center}}
\newcommand{\Chapter}[1]{\vskip5pt{\Large{\textbf{#1}}}\vskip5pt}
\newcommand{\QText}[1]{#1}
\newcommand{\QBlock}[3]{
	\begin{tcolorbox}[left=4mm,top=3mm,bottom=2mm,right=4mm,colback=white]
		\begin{tcolorbox}[enhanced,colframe=blue,colback=blue!10!white,
			frame style={opacity=0.3},interior style={opacity=1.0},
			nobeforeafter,tcbox raise base,shrink tight,extrude by=1.7mm,width=1.5cm]
			\textbf{#1\textsuperscript{#2}}
		\end{tcolorbox}\hspace{3mm}#3
	\end{tcolorbox}
}
\newcommand{\QPicture}[4]{\QText{#4}  \includegraphics{#1}}
\newcommand{\ABlock}[1]{#1}
\newcommand{\MBlock}[2]{#1 #2}
\newcommand{\MMBlock}[3]{#1 #2 #3}
% ----------------------------------------


\begin{document}
	
	% настройки
	\pagestyle{style}\thispagestyle{plain}
	\Title
	% ----------------------------------------
	
	%\vskip5mm
	%\centering{\pgfornament[width=5cm,anchor=south]{89}}

\QBlock{A1}{0.50}{Покажите, что суммарная сила, действующая на кольцо, определяется выражением:
$$
\vec F_{tot} = - \mu mg \cdot f \left( \frac{v(t)}{\omega (t)\ r} \right) \hat x,
$$
где 
$$
f(a) = \frac{1}{2 \pi} \int \limits_0^{2 \pi} \frac{a - \sin \theta}{\sqrt{ 1 + a^2 - 2 a \sin \theta}} d \theta
$$}

\QText{Обозначим скорость кусочка,  видимого из центра кольца под углом $d \theta$ как $\vec{u}$. Так как тело движется, а коэффициент трения не зависит от направления движения, то сила трения, действующая на выбранный кусочек,  направлена противоположно его скорости. Тогда её можно записать в таком виде:
$$
d \vec{F}_{fric} = - \frac{\vec{u}}{u} \mu  \: dN = - \frac{(v-\omega r \sin \theta) \hat x + (\omega r \cos \theta) \hat y}{\sqrt{v^2 + (\omega r)^2 - 2 v \omega r \sin \theta}} \mu mg \: \frac{d \theta}{2 \pi}
$$
Полная сила трения получается при интегрировании выражения вдоль всего кольца, т. е. в диапазоне углов от $0$ до $2 \pi$. 

Несложно заменить, что $F_{tot \: y} = 0$, так как всё выражение меняет знак при замене $\theta \rightarrow \pi - \theta$. Физически это соответствует тому, что силы, действующие на кусочки, симметричные относительно оси OX, проходящей через центр кольца, компенсируют y-составляющие друг друга.

Разделив числитель и знаменатель на $\omega r$, приходим к искомому выражению.}

\QBlock{A2}{0.50}{Покажите, что суммарный момент, действующий на кольцо, равен:
$$
\tau_{tot} = - \mu mg r f \left(\frac{\omega(t) r}{v(t)} \right)
$$}

\QText{$$
d \vec{\tau} = \left[ \vec r \times d \vec{F}_{fric} \right] = - \frac{\left[\vec r \times \vec u \right]}{u} \mu g \: dm
$$
В векторное произведение входит только компонента $\vec u$, перпендикулярная радиусу, т. е. $u_\tau = \omega r - v \sin \theta$. подставляя всё в итоговое выражение, получим:
$$
\vec \tau = - \mu m g r\int \limits_0^{2 \pi} \frac{\omega r-v \sin \theta }{\sqrt{v^2 + (\omega r)^2 - 2 v \omega r \sin \theta}} \: \frac{d \theta}{2 \pi},
$$
откуда после сокращения на $v$ получится искомая формула.}

\QBlock{A3}{0.10}{Докажите, что уравнения движения имеют вид:
$$
\dot v = - \mu g\cdot  f \left( \frac{v}{\omega r} \right) \\
\dot \omega r = - \mu g \cdot f \left( \frac{ \omega r}{v} \right)
$$}

\QText{Векторная сумма сил всегда сонаправлена скорости, поэтому $| \dot{\vec{v}} | =  \dot{|v|} $, то есть тангенциальное ускорение равно нулю. Тогда уравнения движения примут вид:
$$
m \dot v = - \mu mg f \left( \frac{v}{\omega r} \right) \\
m \dot \omega r^2 = - \mu m g r f \left( \frac{\omega r}{v} \right), 
$$
откуда после сокращения на $m$ и $mr$ соответственно получатся искомые равенства.}

\QBlock{B1}{0.50}{Докажите:
a) $ f(0) = 0, \: f(1) = \dfrac{2}{\pi}, \: f(\infty) = 1$
b) $ f(a) $ строго возрастает при $a \geqslant 0 $}

\QText{a)
$$
f(0) = \frac{1}{2 \pi} \int \limits_0^{2 \pi} -\sin \theta \: d \theta = 0 \\
f(1) =  \frac{1}{2 \pi} \int \limits_0^{2 \pi} \frac{1-\sin \theta}{\sqrt{2 - 2 \sin \theta}} \: d \theta =  \frac{1}{2 \pi}\int \limits_0^{2 \pi} \sqrt{\frac{1-\sin \theta}{2}} \: d \theta = \\
= \frac{1}{2 \pi}\int \limits_0^{2 \pi} \left|\sin \left(\frac{\theta}{2} - \frac{\pi}{4} \right) \right| d \theta = \frac{1}{\pi}\int \limits_0^{\pi} \sin \left( \frac{\theta}{2} \right) \: d \left( \frac{\theta}{2} \right) = \frac{2}{\pi} \\
f(\infty) = \frac{1}{2 \pi} \int \limits_0^{2 \pi} \lim \limits_{a \rightarrow \infty}  \frac{a - \sin \theta}{\sqrt{1 +a^2 - 2a \sin \theta}} \: d \theta = \frac{1}{2 \pi} \int \limits_0^{2 \pi} d \theta = 1
$$}

\QBlock{B2}{0.30}{Рассмотрим поведение параметра $a(t)  = \dfrac{v(t)}{\omega (t) r} $. Покажите, что происходит с  $a(t)$ (растёт/уменьшается/остаётся неизменным) в каждом из следующих случаев:

a) в некоторый момент $a(t) = 1$

b) в некоторый момент $a(t) < 1$

c) в некоторый момент $a(t) > 1$}

\QText{$$
\dot a = \frac{d}{dt} \left( \frac{v}{\omega r}\right) = \frac{\dot v \: \omega r- v \: \dot \omega r}{(\omega r)^2} = - \frac{\mu m g}{\omega r} \bigg( f(a) - a f \left(\tfrac{1}{a} \right) \bigg)
$$
Анализируя данное выражение, получаем: 
$$
f(a) - a f \left(\tfrac{1}{a} \right) = \frac{1}{2 \pi} \int \limits_0^{2 \pi} \frac{a - \sin \theta - a( 1- a \sin \theta)}{\sqrt{1 +a^2 - 2a \sin \theta}} d \theta= \\
\\ = \frac{a^2-1}{2 \pi} \int \limits_0^{2 \pi} \frac{\sin \theta}{\sqrt{1 +a^2 - 2a \sin \theta}} d\theta
$$
В области отрицательных значений подынтегральной функции знаменатель больше, чем в области положительных, поэтому интеграл всегда положителен.
Таким образом:
$$
sign( \dot a) = sign(1 - a); \\ \dot a=0 \Rightarrow a =1.
$$
Заметим, что ответ можно было бы получить и из более простых логических соображений}

\ABlock{a) $a=1 \quad  \Rightarrow \quad a \; -$ постоянна;
b) $a<1 \quad \Rightarrow \quad a \; - $ возрастает;
c) $a>1 \quad \Rightarrow \quad a \; - $ убывает.}

\QBlock{B3}{0.60}{Нарисуйте качественно на графике, осями которого являются $v$ и $\omega r$, траектории, отображающие разное движение кольца, то есть при заданных  $v_0$  и $\omega_0 r$ нарисуйте, как они будут изменяться с течением времени. 

Необходимо нарисовать хотя бы одну траекторию на каждый пункт предыдущего задания. Кроме того, нарисуйте траекторию, проходящую через точку $(v_0, 0)$ и еще одну, начинающуюся в точке $(0, \omega_0 r)$

Подпишите оси графика и укажите направления движения системы для каждой нарисованной траектории}

\QText{Как можно было заметить из предыдущего пункта, все траектории асимптотически стремятся к единице. При этом, $\dot v < 0$ и $\dot \omega r <0$. Приведём искомый график:}

\QPicture{Y21-T6 Керлинг-s_files/1314.jpeg}{"max-width:500px;display:block;margin-left: auto;margin-right: auto;"}{Семейство траекторий движения системы в плоскости $(v, \omega r)$ }{}

\ABlock{            <div class="kt-section__info" style="text-align: center;">Семейство траекторий движения системы в плоскости $(v, \omega r)$ </div>}

\QBlock{B4}{0.10}{Вычислите мгновенную мощность, которая расходуется, когда есть только угловая скорость $\omega$ $(v = 0)$, и отдельно,  когда присутствует только линейная $v$ $(\omega = 0)$.}

\QText{В случае отсутствия вращательного движения сила трения, действующая на каждый кусочек, направлена против оси X. Тогда:
$$
P_v = - F_{tot} v = - \mu  mg \:  v
$$
В случае отсутствия поступательного движения
$$
P_\omega = - \tau_{tot} \omega = - \mu mg \: \omega r
$$
Более строгие рассуждения будут приведены в пункте B5.}

\ABlock{$$
P_v = - \mu  mg \:  v \\
P_\omega =  - \mu mg \: \omega r
$$}

\QBlock{B5}{0.60}{Для заданных $v$ и $\omega$ вычислите мгновенную мощность $P$, которая расходуется на трение в данный момент времени. Дайте ответ в виде интеграла с безразмерной переменной.}

\QText{$$
P = \int \vec u \: \cdot \: d \vec F_{fric} = - \mu mg \: \int \limits_0^{2 \pi} \vec{u} \: \cdot \: \frac{\vec u}{u} \:\frac{d \theta }{2 \pi} =  - \mu mg \: \omega r \int \limits_0^{2 \pi} \frac{u}{\omega r} \:\frac{d \theta }{2 \pi}
$$
Подставляя выражение для скорости кусочка $\vec u$:}

\ABlock{$$
P = - \mu mg \: \omega r \int \limits_0^{2 \pi} \sqrt{1+\left(\frac{v}{\omega r} \right)^2 - 2 \left(\frac{v}{\omega r} \right) \sin (\theta)} \: \:\frac{d \theta }{2 \pi}
$$}

\QBlock{B6}{1.20}{Предположим, что кольцу придали определённую начальную кинетическую энергию $E_0$. Каково должно быть соотношение $a_0 = \dfrac{v_0}{\omega_0 r}$, при котором кольцо будет двигаться максимальное время?

Подсказка: Постарайтесь дать ответ на предыдущий пункт при помощи только $E_0$ и $a_0$ (и других данных из этого пункта), исключив из уравнения $v$ и $\omega$}

\QText{Воспользуемся подсказкой :)
$$
E = \frac{m}{2} \Big[ v^2 + (\omega r)^2 \Big] = \frac{m}{2} (\omega r)^2 \:( a^2 + 1) \\
m \: \omega r = \sqrt{2 mE} \: \cdot \: \frac{1}{\sqrt{1 + a^2}} \\
P(a, E) = - \mu g \sqrt{2 mE} \; \cdot \int \limits_0^{2 \pi} \sqrt{1 - \frac{2a}{1 + a^2} \sin (\theta)}  \: \: \frac{d \theta }{2 \pi}
$$
Анализировать последнее выражение можно несколькими способами. Например, взяв частную производную по a
$$
\frac{\partial}{\partial a} \Big( P(a, E) \Big) = \mu g \sqrt{2 mE}\; \: \frac{1 - a^2}{1+a^2} \; \cdot \int \limits_0^{2 \pi} \frac{\sin \theta}{\sqrt{1 +a^2 - 2a \sin \theta}} \frac{d\theta}{2 \pi}
$$
где интеграл в правой части совпадает с интегралом в $B2$ и имеет положительные значения при любых $a$. Так как при $a<1$ функция $P\:$ убывает, а при $a>1 - $ возрастает, то её минимум реализуется в $a_0=1$.
Из того факта, что мощность $P$ всегда отрицательна и стремится к $P \rightarrow 0$ только при $E \rightarrow 0$ очевидно, что движение может закончиться только при $E = 0$. Также, следует заметить, что движение с параметром $a = 1$ устойчиво, а значит, при $a_0 = 1$ всё движение будет происходит с минимально возможной мощностью, т. е. пройдет максимально возможное время при заданной начальной энергии $E = E_0$}

\ABlock{$$ a_0 =1 $$}

\QBlock{B7}{0.50}{Каково максимальное время движения при начальной энергии $E_0$?}

\QText{Как мы выяснили в предыдущем пункте, движение, продолжающееся максимальное время происходит при $a= 1$. Тогда время можно найти из уравнения:
$$
\dot E = P(1, E) = - \mu g \sqrt{2 mE} \; \cdot \int \limits_0^{2 \pi} \sqrt{1 - \sin (\theta)}  \: \: \frac{d \theta }{2 \pi}
$$
Интеграл вычисляется аналогично пункту $B1$ с помощью тригонометрической замены, указанной в условии.
$$
\int \limits_0^{2 \pi} \sqrt{1 - \sin (\theta)}  \: \: \frac{d \theta }{2 \pi} = \frac{\sqrt 2}{\pi} \int \limits_0^\pi \sin \left( \frac{\theta}{2} \right) d \left( \frac{\theta}{2} \right) = \frac{2 \sqrt{2}}{\pi}
$$
Разделяя переменные, получим
$$
\int \limits_{E_0}^0 \frac{dE}{\sqrt{E}} = - \mu g \sqrt{m} \cdot \frac{4}{\pi} \int \limits_{0}^\tau dt
$$
Из чего несложно выразить ответ.}

\ABlock{$$
\tau = \frac{\pi}{2 \mu g} \sqrt{\frac{E_0}{m}}
$$}

\QBlock{C1}{0.60}{Напишите заново уравнения движения из пункта А3 таким образом, чтобы они подходили под новое условие.}

\QText{Вывод уравнения для величины силы трения и момента силы трения остаётся тем же, однако, теперь полная реакция опоры равна $N = mg \cos \alpha$. Таким образом, суммарный момент, действующий на кольцо, просто умножится на $\cos \alpha$.
С уравнениями для проекций сил, лежащих в плоскости движения немного сложнее: суммарная сила теперь не сонаправлена со скоростью кольца.  Обозначив $\cos \varphi = (\widehat{\vec v , \: \vec g_{\tau}})$, где $\vec g_\tau \: -$ составляющая вектора $\vec g$, лежащая вдоль плоскости, получим уравнения:}

\ABlock{$$
\dot v = -\mu g \: f\left( \frac{v}{\omega r}\right) \cos{\alpha} + g \sin{\alpha} \cos{\varphi} \\
\dot \omega r = - \mu g \: f \left(\frac{\omega r}{v} \right) \cos \alpha
$$}

\QBlock{C2}{2.00}{При заданных начальных $\omega_0$ и $v_0 = 0$ нарисуйте все возможные семейства траекторий движения кольца  в координатах $(v, \omega r)$ (для каждого типа кривых нарисуйте свой график).  Укажите следующие составляющие:

a) cоответствующие значения параметров;

b) конечные точки (в которые траектории приходят за конечное или бесконечное время) в плоскости $(v, \omega r)$ . Здесь достаточно написать для каждой составляющей, что она стремится к нулю/ стремится к бесконечности/ равна или стремится к какой-то положительной величине.

Подпишите оси графика и укажите направления движения системы для каждой нарисованной траектории}

\QText{Так как при начале движения кольцо имеет нулевую скорость $v_0 = 0$, то в предыдущем уравнении на протяжении всего движения $\cos \varphi = 0$.  $\dot \omega < 0$, т.е. $\omega \rightarrow 0 $. В зависимости от величины $\tan \alpha$ возможны различные случаи: $v \rightarrow 0$, $v \rightarrow const \:$ и $v \rightarrow \infty \:$. Выбор одного из этих случаев определяется знаком $\dot v$ при $\dfrac{v}{\omega r} \rightarrow \infty$. Искомые траектории приведены на рисунках.}

\QPicture{Y21-T6 Керлинг-s_files/1340.jpeg}{"max-width:500px;display:block;margin-left: auto;margin-right: auto;"}{ Траектории при $ \mu &gt; \tan \alpha $}{}

\QPicture{Y21-T6 Керлинг-s_files/1341.jpeg}{"max-width:500px;display:block;margin-left: auto;margin-right: auto;"}{ Траектории при $\tan \alpha = \mu $}{}

\QPicture{Y21-T6 Керлинг-s_files/1342.jpeg}{"max-width:500px;display:block;margin-left: auto;margin-right: auto;"}{ Траектории при $\tan \alpha &gt; \mu $}{}

\end{document}