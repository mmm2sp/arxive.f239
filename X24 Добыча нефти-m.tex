
%\documentstyle[12pt,russian,amsthm,amsmath,amssymb]{article}
\documentclass[a4paper,11pt,twoside]{article}
\usepackage[left=14mm, top=10mm, right=14mm, bottom=10mm, nohead, nofoot]{geometry}
\usepackage{amsmath, amsfonts, amssymb, amsthm} % стандартный набор AMS-пакетов для математ. текстов
\usepackage{mathtext}
\usepackage[utf8]{inputenc} % кодировка utf8
\usepackage[russian]{babel} % русский язык
\usepackage[pdftex]{graphicx} % графика (картинки)
\usepackage{tikz}
\usepackage{fancyhdr,pageslts} % настройка колонтитулов
\usepackage{enumitem} % работа со списками
\usepackage{multicol} % работа с таблицами
%\usepackage{pscyr} % красивый шрифт
\usepackage{pgfornament} % красивые рюшечки и вензеля
\usepackage{ltxgrid} % управление написанием текста в две колонки
\usepackage{lipsum} % стандартный текст
\usepackage{tcolorbox} % рамка вокруг текста
\tcbuselibrary{skins}
% ----------------------------------------

\newcommand\ProblemName{Добыча нефти}

\newcommand\Source{X24}

\newcommand\Type{Разбалловка}

\newcommand\MyTextLeft{Президентский ФМЛ 239, г.~Санкт-Петербург}
\newcommand\MyTextRight{Использованы материалы сайта pho.rs}
\newcommand\MyHeading{Учебно-тренировочные сборы по физике}
% ----------------------------------------

% настройки полей
\geometry{
	left=12mm,
	top=21mm,
	right=15mm,
	bottom=26mm,
	marginparsep=0mm,
	marginparwidth=0mm,
	headheight=22pt,
	headsep=2mm,
	footskip=7mm}
% ----------------------------------------

% настройки колонтитулов
\pagestyle{fancy}

\fancypagestyle{style}{
	\fancyhf{}
	\fancyhead[L]{{\Large{\FancyTitle}}\\\vskip -5pt \dotfill}
	\fancyhead[R]{{\Large{\textbf{\Type}}}\\\vskip -5pt \dotfill}
	\renewcommand{\headrulewidth}{0pt}
	\renewcommand{\footrulewidth}{0pt}
	\fancyfoot[C]{\pgfornament[width=2em,anchor=south]{72}\hspace{1mm}
		{Страница \textbf{\thepage} из \textbf{\pageref{VeryLastPage}}}\hspace{2mm}
		\pgfornament[width=2em,symmetry=v,anchor=south]{72}\\ \vskip2mm
		{\small{\textit{\MyTextLeft\hfill\MyTextRight}}}}
}

\fancypagestyle{plain}{
	\fancyhf{}
	\renewcommand{\headrulewidth}{0pt}
	\renewcommand{\footrulewidth}{0pt}
	\fancyhead[C]{{\Large{\textit{\MyHeading}}}\\\vskip -5pt \dotfill}
	\fancyfoot[C]{\pgfornament[width=2em,anchor=south]{72}\hspace{1mm}
		{Страница \textbf{\thepage} из \textbf{\pageref{VeryLastPage}}}\hspace{2mm}
		\pgfornament[width=2em,symmetry=v,anchor=south]{72}\\ \vskip2mm
		{\small{\textit{\MyTextLeft\hfill\MyTextRight}}}}
}
% ----------------------------------------

% другие настройки
\pagenumbering{arabic}
\setlist[enumerate,itemize]{leftmargin=0pt,itemindent=2.7em,itemsep=0cm}
% ----------------------------------------

% собственные команды
\newcommand{\FancyTitle}{\textbf{\Source} --- \ProblemName}
\newcommand{\Title}{\begin{center}{\huge{\textbf{\Source} --- \ProblemName}}\end{center}}
\newcommand{\Chapter}[1]{\vskip5pt{\Large{\textbf{#1}}}\vskip5pt}
\newcommand{\QText}[1]{#1}
\newcommand{\QBlock}[3]{
	\begin{tcolorbox}[left=2mm,top=2mm,bottom=1mm,right=2mm,colback=white]
		\begin{tcolorbox}[enhanced,colframe=ProcessBlue,colback=ProcessBlue!30!white,
			frame style={opacity=0.7},interior style={opacity=1.0},
			nobeforeafter,tcbox raise base,shrink tight,extrude by=1.7mm,width=1.5cm]
			\textbf{#1\textsuperscript{#2}}
		\end{tcolorbox}\hspace{3mm}#3
	\end{tcolorbox}
}
\newcommand{\QPicture}[4]{
	\begin{figure}[H]
		\centering
		\includegraphics[width=0.35\linewidth]{#1}
		\caption{#3}
	\end{figure}
	
	#4
}
\newcommand{\ABlock}[1]{
	\vskip2mm
	\begin{tcolorbox}[enhanced,colframe=Magenta,colback=Magenta!15!white,
		frame style={opacity=0.5},interior style={opacity=1.0},
		nobeforeafter,tcbox raise base,shrink tight,extrude by=1.7mm,width=1.6cm]
		\textbf{Ответ:}
	\end{tcolorbox}\hspace{3mm}#1
}
\newcommand{\MBlock}[2]{
	\begin{tcolorbox}[enhanced,colframe=Yellow,colback=Yellow!15!white,
		frame style={opacity=0.5},interior style={opacity=1.0},
		nobeforeafter,tcbox raise base,shrink tight,extrude by=1.7mm,width=1.1cm]
		\textbf{#1}
	\end{tcolorbox}\hspace{3mm}#2
}
\newcommand{\MMBlock}[3]{
	\begin{tcolorbox}[enhanced,colframe=Yellow,colback=Yellow!15!white,
		frame style={opacity=0.5},interior style={opacity=1.0},
		nobeforeafter,tcbox raise base,shrink tight,extrude by=1.7mm,width=1.1cm]
		\textbf{#1}
	\end{tcolorbox}\hspace{3mm}
	\begin{tcolorbox}[enhanced,colframe=Orange,colback=Orange!15!white,
		frame style={opacity=0.5},interior style={opacity=1.0},
		nobeforeafter,tcbox raise base,shrink tight,extrude by=1.7mm,width=0.8cm]
		\textbf{#2}
	\end{tcolorbox}\hspace{3mm}#3
}
% ----------------------------------------


\begin{document}
	
	% настройки
	\pagestyle{style}\thispagestyle{plain}
	\Title
	% ----------------------------------------
	
	%\vskip5mm
	%\centering{\pgfornament[width=5cm,anchor=south]{89}}
	
	% смысловая часть


\QBlock{A1}{0.30}{Пусть залежь нефти представляет собой участок древних речных отложений песчаника в форме параллелепипеда высотой $h = 10 \text{м}$, шириной $b = 100 \text{м}$ и длинной $L = 2000 \text{м}$. Пористость породы $\varphi = 0.1$. Оцените запасы нефти $m_\text{н}$ в данном месторождении. Выразите ответ через $L$, $b$, $h$, $\rho$ и $\varphi$, а также приведите его численное значение в тоннах. Считайте, что нефтяной флюид целиком заполняет объём пор.}

\MBlock{0.10}{Для объёма нефти получено:
$$V_\text{н}=\varphi bhL{.}
$$}

\MBlock{2 $\times$ 0.10}{Получен правильный ответ (по $0{.}1$ балла за выражение и численное значение):
$$m_\text{н}=\rho\varphi bhL=160\cdot 10^3 \text{тонн}{.}
$$}

\QBlock{A2}{0.30}{Пусть пластовое давление нефти на дне залежей составляет $p_\text{пл}=250 \text{атм}$. Найдите, при какой максимальной глубине залегания $H_{max}$ месторождение будет фонтанирующим, т.е. нефть будет вытекать на поверхность под действием собственного давления. Выразите ответ через $\rho$, $g$ и $p_{\text{пл}}$, а также приведите его численное значение. Сжимаемостью нефти можно пренебречь.}

\MBlock{0.10}{Получено выражение для $H_{max}$:
$$H_{max}=\cfrac{p_\text{пл}}{\rho g}{.}
$$}

\MBlock{0.20}{Рассчитана величина $H_{max}$:
$$H_{max}\approx 3{.}125 \text{км}
$$}

\QBlock{A3}{0.60}{Оцените максимально возможный КИН $\alpha_{max}$ в режиме фонтанирования при пластовом давлении $p_\text{пл}=250 \text{атм}$, если сжимаемость нефти $\beta = 5\cdot 10^{-10} \text{ Па}$. Выразите ответ через $\beta$ и $p_\text{пл}$, а также приведите его численное значение. Считайте, что отложения русла рек изолированы непроницаемыми глинами с малой пористостью. Глубина залежей $H$ может быть выбрана произвольным образом.}

\MBlock{0.30}{Указано или используется, что максимальная доля запасов добывается в случае очень малых глубин залежей.}

\MBlock{0.20}{Получено выражение для $\alpha_{max}$:
$$\alpha_{max}=\beta p_\text{пл}{.}
$$}

\MBlock{0.10}{Определено численное значение для $\alpha_{max}$:
$$\alpha_{max}\approx 1{.}25\text{\%}{.}
$$}

\QBlock{A4}{0.30}{При тех же самых данных оцените максимально возможный КИН $\alpha_{max}$ в режиме фонтанирования, если снизу в пластовых отложениях находится вода объемом $kV_0$ ($k = 9$) при начальных запасах нефти $V_0$. Сжимаемость воды считайте равной сжимаемости нефти. Выразите ответ через $\beta$ и $p_\text{пл}$, а также приведите его численное значение. Считайте что забор жидкости происходит сверху, т.е. забирается только нефть. Глубина залежей $H$ может быть выбрана произвольным образом.}

\MBlock{0.10}{Проявлено понимание, что из-за наличия в системе воды объём добываемой нефти увеличивается на величину, равную изменению объёма воды.}

\MBlock{2 $\times$ 0.10}{Получен правильный ответ (по $0{.}1$ балла за выражение и численное значение):
$$\alpha_{max}=10\beta p_\text{пл}\approx 12{.}5\text{\%}{.}
$$}

\QBlock{B1}{1.00}{Рассмотрим горизонтальное течение жидкости вдоль оси $x$ между двумя параллельными плоскостями высотой $h$. Расстояние между плоскостями $w\ll{h}$. Определите объёмный расход (далее во всех пунктах задачи - поток) жидкости $Q$ через поперечное сечение $wh$. Ответ выразите через $\eta$, $w$, $h$ и градиент давления $dp(x)/dx$.}

\MBlock{0.40}{Из условия постоянства импульса прямоугольного параллелепипеда получено:
$$\cfrac{d^2v}{dz^2}=\cfrac{1}{\eta}\cfrac{dp}{dx}{.}
$$
При неправильном знаке в последующих выкладках применяется PEP везде, кроме ответов.}

\MBlock{0.10}{Получено выражение для $dv(z)/dz$:
$$\cfrac{dv}{dz}=\cfrac{z}{\eta}\cfrac{dp}{dx}{.}
$$}

\MBlock{0.20}{Получено выражение для $v(z)$:
$$v(z)=-\cfrac{1}{2\eta}\cfrac{dp}{dx}\left(\cfrac{w^2}{4}-z^2\right){.}
$$}

\MBlock{0.10}{Для потока $Q$ записано:
$$Q=\int\limits_{-w/2}^{w/2}v(z)\cdot hdz{.}
$$}

\MBlock{2 $\times$ 0.10}{Получено выражение для $Q$ (по $0{.}1$ балла за величину и знак):
$$Q=-\cfrac{w^3h}{12\eta}\cfrac{dp}{dx}{.}
$$}

\QBlock{B2}{1.00}{В центре щели создается избыточное давление $\Delta p$. Найдите зависимость избыточного давления $p'$ в щели от координаты $x$. Ответ выразите через $\Delta{p}$, $Q$, $E$, $h$, $\eta$ и $x$.}

\MBlock{0.10}{Учтено, что в каждой половине трещины поток жидкости равен $Q/2$ и записано:
$$\cfrac{Q}{2}=-\cfrac{w^3h}{12\eta}\cfrac{dp'}{dx}{.}
$$
Если вместо $Q/2$ записано $Q$, то в последующих выкладках применяется PEP везде, кроме ответов.}

\MBlock{0.20}{Записано выражение для объёма с подстановкой эмпирической формулы для $w(x)$:
$$\cfrac{Q}{2}=-\cfrac{h^4p'^3}{12\eta E^3}\cfrac{dp'}{dx}{.}
$$}

\MBlock{0.50}{Получено  уравнение для определения $p'$ соотношение:
$$\int\limits_{\Delta{p}}^{p'(x)}p'^3dp'=\cfrac{p'^4(x)-\Delta{p}^4}{4}=-\cfrac{6Q\eta E^3x}{h^4}{.}
$$}

\MBlock{0.20}{Получено правильная зависимость $p'(x)$ (по $0{.}1$ балла за правильные величины обоих слагаемых и знак):
$$p'(x)=\sqrt[4]{\Delta{p}^4-\cfrac{24Q\eta E^3x}{h^4}}{.}
$$}

\QBlock{B3}{0.20}{Трещина заканчивается в положении, соответствующем равному нулю избыточному давлению. Определите длину трещины $L$. Ответ выразите через $\Delta{p}$, $E$, $h$, $\eta$ и $Q$.}

\MBlock{0.10}{Для своей формулы правильно выражена полудлина трещины:
$$\cfrac{L}{2}=\cfrac{\Delta{p}^4h^4}{24Q\eta E^3}{.}
$$}

\MBlock{0.10}{Получено выражение для $L$:
$$L=\cfrac{\Delta{p}^4h^4}{12Q\eta E^3}{.}
$$}

\QBlock{B4}{0.70}{Определите объем трещины $V$. Ответ выразите через $\Delta{p}$, $h$, $\eta$, $Q$ и $E$.}

\MBlock{0.10}{Записано выражение для объёма $V$:
$$V=2\int\limits_{0}^{L}hw(x)dx{.}
$$}

\MBlock{0.20}{Для своей формулы получено выражение для объёма как интеграл функции от $x$:
$$V=\cfrac{2h^2\Delta{p}}{E}\int\limits_{0}^{L}\sqrt[4]{1-\cfrac{24Q\eta E^3x}{h^4\Delta{p}^4}}dx{.}
$$}

\MBlock{0.20}{Верно вычислен интеграл:
$$\int\limits_{0}^{1}\sqrt[4]{1-z}dz=\cfrac{4}{5}{.}
$$}

\MBlock{0.20}{Получено выражение для $V$:
$$V=\cfrac{h^6\Delta{p}^5}{15Q\eta E^4}{.}
$$}

\QBlock{B5}{0.30}{Рассчитайте максимально возможные значения длины трещины $L_{max}$ и её объёма $V_{max}$.}

\MBlock{0.10}{Определено численное значение $L_{max}$:
$$L_{max}\approx 83 \text{м}{.}
$$
Оценивается только правильное число.}

\MBlock{0.20}{Определено численное значение $V_{max}$:
$$V_{max}=6{.}7 \text{м}{.}
$$
Оценивается только правильное число.}

\QBlock{С1}{1.00}{Определите скорость $v$ движения границы жидкостей при перемещении фронта на величину $S$. Ответ выразите через $p_1$, $p_2$, $L$, $\eta_1$, $\eta_2$, $k_1$ и $k_2$.}

\MBlock{2 $\times$ 0.20}{Получены градиенты давлений в жидкостях $1$ и $2$:
$$\cfrac{\partial p_\text{н}}{\partial x}=-\cfrac{\eta_1v}{k_1}\qquad \cfrac{\partial p_\text{в}}{\partial x}=-\cfrac{\eta_2v}{k_2}{.}
$$}

\MBlock{0.40}{Получена связь разности давлений со скоростью $v$:
$$p_2-p_1=\cfrac{\eta_2xv}{k_2}+\cfrac{\eta_1(L-x)v}{k_1}{.}
$$}

\MBlock{0.20}{Получено выражение для $v$:
$$v=\cfrac{p_2-p_1}{\cfrac{\eta_1L}{k_1}+\left(\cfrac{\eta_2}{k_2}-\cfrac{\eta_1}{k_1}\right)S}{.}
$$}

\QBlock{C2}{0.90}{Определите зависимость перемещения $S$ фронта от времени $t$. Ответ выразите через $p_1$, $p_2$, $L$, $\eta_1$, $\eta_2$, $k$ и $t$}

\MBlock{0.20}{Выражено время $dt$, за которое фронт перемещается на величину $dS$:}

\MBlock{0.20}{Получена зависимость времени $t$ от перемещения $S$:
$$t=\cfrac{1}{k(p_2-p_1)}\int\limits_{0}^S(\eta_1L+(\eta_2-\eta_1)x)dx=\cfrac{1}{k(p_2-p_1)}\left(\eta_1LS+\cfrac{(\eta_2-\eta_1)S^2}{2}\right){.}
$$}

\MBlock{0.10}{Составлено квадратное уравнение относительно $S$:
$$S^2-\cfrac{2\eta_1LS}{\eta_1-\eta_2}+\cfrac{2k(p_2-p_1)t}{\eta_1-\eta_2}=0{.}
$$}

\MBlock{0.20}{Решено квадратное уравнение относительно $S$:
$$S(t)=\cfrac{\eta_1L}{\eta_1-\eta_2}\pm\sqrt{\left(\cfrac{\eta_1L}{\eta_1-\eta_2}\right)^2-\cfrac{2k(p_2-p_1)t}{\eta_1-\eta_2}}{.}
$$}

\MBlock{0.20}{Выбран нужный корень и получена правильная зависимость $S(t)$:
$$S(t)=\cfrac{\eta_1L}{\eta_1-\eta_2}-\sqrt{\left(\cfrac{\eta_1L}{\eta_1-\eta_2}\right)^2-\cfrac{2k(p_2-p_1)t}{\eta_1-\eta_2}}{.}
$$}

\QBlock{C3}{0.50}{Определите полное время $\tau$ вытеснения нефти из месторождения. Выразите ответ через $p_1$, $p_2$, $L$, $\eta_1$, $\eta_2$ и $k$ и рассчитайте его.}

\MBlock{0.10}{Указано или следует из решения, что $\tau=t(L)$.}

\MBlock{2 $\times$ 0.20}{Получен правильный ответ для $\tau$ (по $0{.}2$ балла за выражение и численное значение):
$$\tau=\cfrac{(\eta_1+\eta_2)L^2}{2k(p_2-p_1)}\approx 26 \text{лет}{.}
$$}

\QBlock{С4}{0.80}{При каком условии на параметры системы движение границы будет устойчивым, то есть при малом отклонении формы границы от плоской это отклонение не будет возрастать? Запишите условие устойчивости через $\eta_1$, $\eta_2$, $k_1$ и $k_2$.
Устойчиво ли течение жидкости, рассмотренное в пунктах $\mathrm{C2}$ и $\mathrm{C3}$?}

\MBlock{0.30}{Указано или следует из решения, что отклонение не будет возрастать, если $v(x+dx){<}v(x)$.}

\MBlock{0.40}{Сделан вывод, что критерием устойчивости является следующее неравенство:
$$\cfrac{\eta_2}{k_2}{>}\cfrac{\eta_1}{k_1}{.}
$$}

\MBlock{0.10}{Сделан вывод, что движение рассматриваемого течения является неустойчивым.}

\QBlock{D1}{0.80}{Найдите зависимость скорости течения жидкости в такой трубе от расстояния до оси трубы $v(r)$, максимальное значение скорости $v_{max}$ и полный поток $Q$ жидкости через сечение цилиндра. Ответы выразите через $\Delta{p}$, $\eta$, $L$, $R$ и $r$.}

\MBlock{0.30}{Из условия постоянства импульса цилиндра радиусом $r$ получено:
$$\cfrac{\partial v}{\partial r}=-\cfrac{r}{2}\cfrac{\Delta{p}}{\eta L}{.}
$$}

\MBlock{0.20}{Получен правильная зависимость $v(r)$:
$$v(r)=\cfrac{\Delta{p}(R^2-r^2)}{4\eta L}{.}
$$}

\MBlock{0.10}{Для потока $Q$ записано:
$$Q=\int\limits_{0}^R v(r)\cdot 2\pi rdr{.}
$$}

\MBlock{0.20}{Получено выражение для потока $Q$:
$$Q=\cfrac{\pi\Delta{p}R^4}{8\eta L}{.}
$$}

\QBlock{D2}{0.20}{Выразите распределение скорости течения жидкости $v(r)$ через полный поток $Q$, $R$ и $r$.}

\MBlock{0.20}{Получена правильная зависимость $v(r)$:
$$v(r)=\cfrac{2Q}{\pi R^2}\left(1-\cfrac{r^2}{R^2}\right){.}
$$}

\QBlock{D3}{0.20}{Найдите поток $Q$ в сечении забоя на расстоянии $h$ от его нижнего края и соответствующее выражение для вертикальной скорости $v(r{,}h)$ в зависимости от расстояния до оси $r$ и высоты $h$. Ответы выразите через $Q_0$, $H$, $R$, $r$ и $h$.}

\MBlock{0.10}{Получена правильная зависимость $Q(h)$:
$$Q(h)=\cfrac{Q_0h}{H}{.}
$$}

\MBlock{0.10}{Получена правильная зависимость $v(r{,}h)$:
$$v(r{,}h)=\cfrac{2Q_0h}{\pi R^2H}\left(1-\cfrac{r^2}{R^2}\right){.}
$$}

\QBlock{D4}{0.30}{Рассмотрим кольцо высотой $dh$ с внутренним и внешним радиусами $r$ и $r+dr$ соответственно. Используя тот факт, что жидкость несжимаема, покажите, что из условия постоянства объёма жидкости внутри выделенного кольца следует соотношение:
$$\cfrac{\partial v}{\partial h}=-\cfrac{1}{r}\cfrac{\partial (u_rr)}{\partial r}{.}
$$
Вы можете использовать это соотношение, даже если не смогли его доказать.}

\MBlock{0.10}{Правильно записан поток вектора скорости через основания кольца:
$$q_\text{осн}=q=2\pi rdrv(r{,}h+dh)-2\pi rdrv(r{,}h){.}
$$}

\MBlock{0.10}{Правильно записан поток вектора скорости через боковую поверхность кольца:
$$q_\text{бок}=2\pi dh(r+dr)u_r(r+dr{,}h)-2\pi dh ru_r(r{,}h){.}
$$}

\MBlock{0.10}{Из условия $q=q_\text{осн}+q_\text{бок}=0$ показано требуемое.}

\QBlock{D5}{0.50}{Найдите радиальную скорость течения жидкости $u_r(r{,}h)$ в зависимости от расстояния до оси $r$ и высоты $h$, а также максимальную величину её модуля $u_{r(max)}$. Ответы выразите через $Q_0$, $R$, $H$, $h$ и $r$.}

\MBlock{0.10}{Правильно выполнено интегрирование выражения, полученного в $\mathrm{D4}$:
$$ru_r(r{,}h)=-\cfrac{2Q_0}{\pi R^2H}\int\limits_{0}^r\left(1-\cfrac{z^2}{R^2}\right)zdz{.}
$$}

\MBlock{2 $\times$ 0.10}{Получена правильная зависимость $u_r(r{,}h)$ (по $0{.}1$ балла за величину и знак):
$$u_r(r{,}h)=-\cfrac{Q_0}{\pi R^2H}\left(r-\cfrac{r^3}{2R^2}\right){.}
$$}

\MBlock{0.10}{Определено расстояние $r_{max}$, соответствующее $u_{r(max)}$:
$$r_{max}=\sqrt{\cfrac{2}{3}}R{.}
$$}

\MBlock{0.10}{Получен правильный ответ для $u_{r(max)}$:
$$u_{max}=\left(\cfrac{2}{3}\right)^{3/2}\cfrac{Q_0}{\pi RH}{.}
$$}

\QBlock{D6}{0.10}{Чему равно отношение $u_{r(max)}/v_{max}$? Ответ выразите через $R$ и $H$.}

\MBlock{0.10}{Получен правильный ответ:
$$\cfrac{u_{r(max)}}{v_{max}}=\cfrac{\sqrt{2}R}{3\sqrt{3}H}
$$}

\end{document}