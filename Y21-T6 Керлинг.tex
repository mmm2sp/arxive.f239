
%This file is part of Get pho.rs!

%Get pho.rs! is free software: you can redistribute it and/or modify it under the terms of the GNU General Public License as published by the Free Software Foundation, either version 3 of the License, or (at your option) any later version.

%Get pho.rs! is distributed in the hope that it will be useful, but WITHOUT ANY WARRANTY; without even the implied warranty of MERCHANTABILITY or FITNESS FOR A PARTICULAR PURPOSE. See the GNU General Public License for more details.

%You should have received a copy of the GNU General Public License along with Foobar. If not, see <https://www.gnu.org/licenses/>.

%\documentstyle[12pt,russian,amsthm,amsmath,amssymb]{article}
\documentclass[a4paper,11pt,twoside]{article}
\usepackage[left=14mm, top=10mm, right=14mm, bottom=10mm, nohead, nofoot]{geometry}
\usepackage{amsmath, amsfonts, amssymb, amsthm} % стандартный набор AMS-пакетов для математ. текстов
\usepackage{mathtext}
\usepackage[utf8]{inputenc} % кодировка utf8
\usepackage[russian]{babel} % русский язык
\usepackage[pdftex,dvipsnames]{xcolor} % работа с цветами
\usepackage[pdftex]{graphicx} % графика (картинки)
\usepackage{tikz} % рисунки
\usepackage{fancyhdr,pageslts} % настройка колонтитулов
\usepackage{enumitem} % работа со списками
\usepackage{multicol} % работа с таблицами
%\usepackage{pscyr} % красивый шрифт
\usepackage{pgfornament} % красивые рюшечки и вензеля
\usepackage{ltxgrid} % управление написанием текста в две колонки
\usepackage{lipsum} % стандартный текст
\usepackage{tcolorbox} % рамка вокруг текста
\usepackage{float} % для корректного размещения картинок
\tcbuselibrary{skins}
% ----------------------------------------

\newcommand\ProblemName{Керлинг}

\newcommand\Source{Y21-T6}

\newcommand\Type{Условие задачи}

% настройки полей
\geometry{
	left=12mm,
	top=21mm,
	right=15mm,
	bottom=26mm,
	marginparsep=0mm,
	marginparwidth=0mm,
	headheight=22pt,
	headsep=2mm,
	footskip=7mm}
% ----------------------------------------

% настройки колонтитулов
\pagestyle{fancy}

\fancypagestyle{style}{
	\fancyhf{}
	\fancyhead[L]{{\Large{\FancyTitle}}\\\vskip -5pt \dotfill}
	\fancyhead[R]{{\Large{\textbf{\Type}}}\\\vskip -5pt \dotfill}
	\renewcommand{\headrulewidth}{0pt}
	\renewcommand{\footrulewidth}{0pt}
	\fancyfoot[C]{\pgfornament[width=2em,anchor=south]{72}\hspace{1mm}
		{Страница \textbf{\thepage} из \textbf{\pageref{VeryLastPage}}}\hspace{2mm}
		\pgfornament[width=2em,symmetry=v,anchor=south]{72}\\ \vskip2mm
		{\small{\textit{Условие собрано и подготовлено в Президентском ФМЛ №239 г.~Санкт-Петербурга}}}}
}

\fancypagestyle{plain}{
	\fancyhf{}
	\renewcommand{\headrulewidth}{0pt}
	\renewcommand{\footrulewidth}{0pt}
	\fancyhead[C]{{\Large{\textit{Учебно-тренировочные сборы к X23}}}\\\vskip -5pt \dotfill}
	\fancyfoot[C]{\pgfornament[width=2em,anchor=south]{72}\hspace{1mm}
		{Страница \textbf{\thepage} из \textbf{\pageref{VeryLastPage}}}\hspace{2mm}
		\pgfornament[width=2em,symmetry=v,anchor=south]{72}\\ \vskip2mm
		{\small{\textit{Условие собрано и подготовлено в Президентском ФМЛ №239 г.~Санкт-Петербурга}}}}
}
% ----------------------------------------

% другие настройки
\pagenumbering{arabic}
\setlist[enumerate,itemize]{leftmargin=0pt,itemindent=2.7em,itemsep=0cm}
% ----------------------------------------

% собственные команды
\newcommand{\FancyTitle}{\textbf{\Source} --- \ProblemName}
\newcommand{\Title}{\begin{center}{\huge{\textbf{\Source} --- \ProblemName}}\end{center}}
\newcommand{\Chapter}[1]{\vskip5pt{\Large{\textbf{#1}}}\vskip5pt}
\newcommand{\QText}[1]{#1}
\newcommand{\QBlock}[3]{
	\begin{tcolorbox}[left=4mm,top=3mm,bottom=2mm,right=4mm,colback=white]
		\begin{tcolorbox}[enhanced,colframe=blue,colback=blue!10!white,
			frame style={opacity=0.3},interior style={opacity=1.0},
			nobeforeafter,tcbox raise base,shrink tight,extrude by=1.7mm,width=1.5cm]
			\textbf{#1\textsuperscript{#2}}
		\end{tcolorbox}\hspace{3mm}#3
	\end{tcolorbox}
}
\newcommand{\QPicture}[4]{\QText{#4}  \includegraphics{#1}}
\newcommand{\ABlock}[1]{#1}
\newcommand{\MBlock}[2]{#1 #2}
\newcommand{\MMBlock}[3]{#1 #2 #3}
% ----------------------------------------


\begin{document}
	
	% настройки
	\pagestyle{style}\thispagestyle{plain}
	\Title
	% ----------------------------------------
	
	%\vskip5mm
	%\centering{\pgfornament[width=5cm,anchor=south]{89}}

\QText{Тонкое кольцо массой $m$ и радиусом $r$ находится  на столе с коэффициентом трения $\mu$ и гравитационным ускорением $g$. Кольцу сообщили линейную скорость $v$  (в направлении $x$) и угловую скорость $\omega$  (против часовой стрелки). Кольцо движется по прямой, вращаясь вокруг собственной оси, и, в конце концов, останавливается.  В дальнейшем мы попытаемся найти величины $v(t)$, $\omega(t)$ и некоторую симметрию между ними. Во всех последующих пунктах предполагается, что действие реакции опоры на кольцо распределено равномерно.}

\QText{Вам может понадобиться следующая формулa:
$$
\sqrt{\frac{1 - \sin \theta}{2}} = \Bigg| \sin \left( {\frac{\theta}{2}} - \frac{\pi}{4} \right) \Bigg|
$$}

\Chapter{Уравнения движения}

\QBlock{A1}{0.50}{Покажите, что суммарная сила, действующая на кольцо, определяется выражением:
$$
\vec F_{tot} = - \mu mg \cdot f \left( \frac{v(t)}{\omega (t)\ r} \right) \hat x,
$$
где 
$$
f(a) = \frac{1}{2 \pi} \int \limits_0^{2 \pi} \frac{a - \sin \theta}{\sqrt{ 1 + a^2 - 2 a \sin \theta}} d \theta
$$}

\QBlock{A2}{0.50}{Покажите, что суммарный момент, действующий на кольцо, равен:
$$
\tau_{tot} = - \mu mg r f \left(\frac{\omega(t) r}{v(t)} \right)
$$}

\QBlock{A3}{0.10}{Докажите, что уравнения движения имеют вид:
$$
\dot v = - \mu g\cdot  f \left( \frac{v}{\omega r} \right) \\
\dot \omega r = - \mu g \cdot f \left( \frac{ \omega r}{v} \right)
$$}

\Chapter{Первичное исследование}

\QText{Мы получили систему дифференциальных уравнений, связывающие $v(t)$ и $\omega (t)$. В этой части хотим найти интересные физические аспекты этой ситуации, для чего надо произвести нестандартные математические вычисления. Начнём с рассмотрения свойств функции $f(a)$:}

\QBlock{B1}{0.50}{Докажите:
a) $ f(0) = 0, \: f(1) = \dfrac{2}{\pi}, \: f(\infty) = 1$
b) $ f(a) $ строго возрастает при $a \geqslant 0 $}

\QBlock{B2}{0.30}{Рассмотрим поведение параметра $a(t)  = \dfrac{v(t)}{\omega (t) r} $. Покажите, что происходит с  $a(t)$ (растёт/уменьшается/остаётся неизменным) в каждом из следующих случаев:

a) в некоторый момент $a(t) = 1$

b) в некоторый момент $a(t) < 1$

c) в некоторый момент $a(t) > 1$}

\QBlock{B3}{0.60}{Нарисуйте качественно на графике, осями которого являются $v$ и $\omega r$, траектории, отображающие разное движение кольца, то есть при заданных  $v_0$  и $\omega_0 r$ нарисуйте, как они будут изменяться с течением времени. 

Необходимо нарисовать хотя бы одну траекторию на каждый пункт предыдущего задания. Кроме того, нарисуйте траекторию, проходящую через точку $(v_0, 0)$ и еще одну, начинающуюся в точке $(0, \omega_0 r)$

Подпишите оси графика и укажите направления движения системы для каждой нарисованной траектории}

\QText{Рассмотрим мощность, которая расходуется во время движения кольца.}

\QBlock{B4}{0.10}{Вычислите мгновенную мощность, которая расходуется, когда есть только угловая скорость $\omega$ $(v = 0)$, и отдельно,  когда присутствует только линейная $v$ $(\omega = 0)$.}

\QBlock{B5}{0.60}{Для заданных $v$ и $\omega$ вычислите мгновенную мощность $P$, которая расходуется на трение в данный момент времени. Дайте ответ в виде интеграла с безразмерной переменной.}

\QBlock{B6}{1.20}{Предположим, что кольцу придали определённую начальную кинетическую энергию $E_0$. Каково должно быть соотношение $a_0 = \dfrac{v_0}{\omega_0 r}$, при котором кольцо будет двигаться максимальное время?

Подсказка: Постарайтесь дать ответ на предыдущий пункт при помощи только $E_0$ и $a_0$ (и других данных из этого пункта), исключив из уравнения $v$ и $\omega$}

\QBlock{B7}{0.50}{Каково максимальное время движения при начальной энергии $E_0$?}

\Chapter{Неоднородная система}

\QText{Система уравнений, которую мы получили в начале задачи называется однородной потому, что система, начинающая движение из точки $(0,0)$ останется в состоянии покоя,  то есть в уравнениях нет члена &quot;тока&quot;, создающего движение. Введем такой член в одно из уравнений рассмотрев похожую физическую задачу:

То же самое кольцо положим теперь на наклонную плоскость с углом $\alpha$ с тем же коэффициентом трения $\mu$.}

\QBlock{C1}{0.60}{Напишите заново уравнения движения из пункта А3 таким образом, чтобы они подходили под новое условие.}

\QBlock{C2}{2.00}{При заданных начальных $\omega_0$ и $v_0 = 0$ нарисуйте все возможные семейства траекторий движения кольца  в координатах $(v, \omega r)$ (для каждого типа кривых нарисуйте свой график).  Укажите следующие составляющие:

a) cоответствующие значения параметров;

b) конечные точки (в которые траектории приходят за конечное или бесконечное время) в плоскости $(v, \omega r)$ . Здесь достаточно написать для каждой составляющей, что она стремится к нулю/ стремится к бесконечности/ равна или стремится к какой-то положительной величине.

Подпишите оси графика и укажите направления движения системы для каждой нарисованной траектории}

\end{document}