
%This file is part of Get pho.rs!

%Get pho.rs! is free software: you can redistribute it and/or modify it under the terms of the GNU General Public License as published by the Free Software Foundation, either version 3 of the License, or (at your option) any later version.

%Get pho.rs! is distributed in the hope that it will be useful, but WITHOUT ANY WARRANTY; without even the implied warranty of MERCHANTABILITY or FITNESS FOR A PARTICULAR PURPOSE. See the GNU General Public License for more details.

%You should have received a copy of the GNU General Public License along with Foobar. If not, see <https://www.gnu.org/licenses/>.

%\documentstyle[12pt,russian,amsthm,amsmath,amssymb]{article}
\documentclass[a4paper,11pt,twoside]{article}
\usepackage[left=14mm, top=10mm, right=14mm, bottom=10mm, nohead, nofoot]{geometry}
\usepackage{amsmath, amsfonts, amssymb, amsthm} % стандартный набор AMS-пакетов для математ. текстов
\usepackage{mathtext}
\usepackage[utf8]{inputenc} % кодировка utf8
\usepackage[russian]{babel} % русский язык
\usepackage[pdftex,dvipsnames]{xcolor} % работа с цветами
\usepackage[pdftex]{graphicx} % графика (картинки)
\usepackage{tikz} % рисунки
\usepackage{fancyhdr,pageslts} % настройка колонтитулов
\usepackage{enumitem} % работа со списками
\usepackage{multicol} % работа с таблицами
%\usepackage{pscyr} % красивый шрифт
\usepackage{pgfornament} % красивые рюшечки и вензеля
\usepackage{ltxgrid} % управление написанием текста в две колонки
\usepackage{lipsum} % стандартный текст
\usepackage{tcolorbox} % рамка вокруг текста
\usepackage{float} % для корректного размещения картинок
\tcbuselibrary{skins}
% ----------------------------------------

\newcommand\ProblemName{Добыча нефти}

\newcommand\Source{X24}

\newcommand\Type{Решение}

% настройки полей
\geometry{
	left=12mm,
	top=21mm,
	right=15mm,
	bottom=26mm,
	marginparsep=0mm,
	marginparwidth=0mm,
	headheight=22pt,
	headsep=2mm,
	footskip=7mm}
% ----------------------------------------

% настройки колонтитулов
\pagestyle{fancy}

\fancypagestyle{style}{
	\fancyhf{}
	\fancyhead[L]{{\Large{\FancyTitle}}\\\vskip -5pt \dotfill}
	\fancyhead[R]{{\Large{\textbf{\Type}}}\\\vskip -5pt \dotfill}
	\renewcommand{\headrulewidth}{0pt}
	\renewcommand{\footrulewidth}{0pt}
	\fancyfoot[C]{\pgfornament[width=2em,anchor=south]{72}\hspace{1mm}
		{Страница \textbf{\thepage} из \textbf{\pageref{VeryLastPage}}}\hspace{2mm}
		\pgfornament[width=2em,symmetry=v,anchor=south]{72}\\ \vskip2mm
		{\small{\textit{Условие собрано и подготовлено в Президентском ФМЛ №239 г.~Санкт-Петербурга}}}}
}

\fancypagestyle{plain}{
	\fancyhf{}
	\renewcommand{\headrulewidth}{0pt}
	\renewcommand{\footrulewidth}{0pt}
	\fancyhead[C]{{\Large{\textit{Учебно-тренировочные сборы к X23}}}\\\vskip -5pt \dotfill}
	\fancyfoot[C]{\pgfornament[width=2em,anchor=south]{72}\hspace{1mm}
		{Страница \textbf{\thepage} из \textbf{\pageref{VeryLastPage}}}\hspace{2mm}
		\pgfornament[width=2em,symmetry=v,anchor=south]{72}\\ \vskip2mm
		{\small{\textit{Условие собрано и подготовлено в Президентском ФМЛ №239 г.~Санкт-Петербурга}}}}
}
% ----------------------------------------

% другие настройки
\pagenumbering{arabic}
\setlist[enumerate,itemize]{leftmargin=0pt,itemindent=2.7em,itemsep=0cm}
% ----------------------------------------

% собственные команды
\newcommand{\FancyTitle}{\textbf{\Source} --- \ProblemName}
\newcommand{\Title}{\begin{center}{\huge{\textbf{\Source} --- \ProblemName}}\end{center}}
\newcommand{\Chapter}[1]{\vskip5pt{\Large{\textbf{#1}}}\vskip5pt}
\newcommand{\QText}[1]{#1}
\newcommand{\QBlock}[3]{
	\begin{tcolorbox}[left=4mm,top=3mm,bottom=2mm,right=4mm,colback=white]
		\begin{tcolorbox}[enhanced,colframe=blue,colback=blue!10!white,
			frame style={opacity=0.3},interior style={opacity=1.0},
			nobeforeafter,tcbox raise base,shrink tight,extrude by=1.7mm,width=1.5cm]
			\textbf{#1\textsuperscript{#2}}
		\end{tcolorbox}\hspace{3mm}#3
	\end{tcolorbox}
}
\newcommand{\QPicture}[4]{\QText{#4}  \includegraphics{#1}}
\newcommand{\ABlock}[1]{#1}
\newcommand{\MBlock}[2]{#1 #2}
\newcommand{\MMBlock}[3]{#1 #2 #3}
% ----------------------------------------


\begin{document}
	
	% настройки
	\pagestyle{style}\thispagestyle{plain}
	\Title
	% ----------------------------------------
	
	%\vskip5mm
	%\centering{\pgfornament[width=5cm,anchor=south]{89}}

\QBlock{A1}{0.30}{Пусть залежь нефти представляет собой участок древних речных отложений песчаника в форме параллелепипеда высотой $h = 10 \text{м}$, шириной $b = 100 \text{м}$ и длинной $L = 2000 \text{м}$. Пористость породы $\varphi = 0.1$. Оцените запасы нефти $m_\text{н}$ в данном месторождении. Выразите ответ через $L$, $b$, $h$, $\rho$ и $\varphi$, а также приведите его численное значение в тоннах. Считайте, что нефтяной флюид целиком заполняет объём пор.}

\QText{Из определения пористости для объёма нефти находим:
$$V_\text{н}=\varphi\cdot bhL{.}
$$
Поскольку $m_\text{н}=\rho V_\text{н}$, имеем:}

\ABlock{$$m_\text{н}=\rho\varphi bhL=160\cdot 10^3 \text{тонн}{.}
$$}

\QBlock{A2}{0.30}{Пусть пластовое давление нефти на дне залежей составляет $p_\text{пл}=250 \text{атм}$. Найдите, при какой максимальной глубине залегания $H_{max}$ месторождение будет фонтанирующим, т.е. нефть будет вытекать на поверхность под действием собственного давления. Выразите ответ через $\rho$, $g$ и $p_{\text{пл}}$, а также приведите его численное значение. Сжимаемостью нефти можно пренебречь.}

\QText{Жидкость будет вытекать под действием пластового давления, если оно превышает давление гидростатического столба нефти:
$$p_\text{пл}\geq \rho gh{.}
$$
Таким образом:}

\ABlock{$$H_{max}=\cfrac{p_\text{пл}}{\rho g}\approx 3{.}125 \text{км}
$$}

\QBlock{A3}{0.60}{Оцените максимально возможный КИН $\alpha_{max}$ в режиме фонтанирования при пластовом давлении $p_\text{пл}=250 \text{атм}$, если сжимаемость нефти $\beta = 5\cdot 10^{-10} \text{ Па}$. Выразите ответ через $\beta$ и $p_\text{пл}$, а также приведите его численное значение. Считайте, что отложения русла рек изолированы непроницаемыми глинами с малой пористостью. Глубина залежей $H$ может быть выбрана произвольным образом.}

\QText{Как уже говорилось условии - пластовое давление обусловлено сжатием флюида относительно нормальных условий. Если флюид находится под нормальным атмосферным давлением - его фиксированное количество вещества занимает наибольший возможный объём, а значит, наибольшая доля вытесненного флюида достигается, если в конечном состоянии он находится при как можно меньшем давлении.
Это можно реализовать, если устремить глубину залежей $H$ к нулю.
Имеем:
$$\alpha=\cfrac{dV}{V}\approx \beta\Delta{p}{.}
$$
Поскольку пластовое давление во много раз превышает атмосферное - находим:}

\ABlock{$$\alpha_{max}\approx \beta p_\text{пл}\approx 1{.}25\text{\%}{.}
$$}

\QBlock{A4}{0.30}{При тех же самых данных оцените максимально возможный КИН $\alpha_{max}$ в режиме фонтанирования, если снизу в пластовых отложениях находится вода объемом $kV_0$ ($k = 9$) при начальных запасах нефти $V_0$. Сжимаемость воды считайте равной сжимаемости нефти. Выразите ответ через $\beta$ и $p_\text{пл}$, а также приведите его численное значение. Считайте что забор жидкости происходит сверху, т.е. забирается только нефть. Глубина залежей $H$ может быть выбрана произвольным образом.}

\QText{Аналогично предыдущему пункту, глубина залежей $H$ должна стремиться к нулю. Однако в данном случае КИН серьёзно возрастает, поскольку разжимается не только флюид, но и вода, и, поскольку забирается только нефть, забираемый её объём $dV$ складывается из изменения объёмов нефти и воды::
$$\alpha=\cfrac{dV}{V_0}=\cfrac{dV_\text{н}+dV_\text{в}}{V_0}=(1+9)\beta\Delta{p}{.}
$$
Таким образом:}

\ABlock{$$\alpha_{max}=10\beta p_\text{пл}\approx 12{.}5\text{\%}{.}
$$}

\QBlock{B1}{1.00}{Рассмотрим горизонтальное течение жидкости вдоль оси $x$ между двумя параллельными плоскостями высотой $h$. Расстояние между плоскостями $w\ll{h}$. Определите объёмный расход (далее во всех пунктах задачи - поток) жидкости $Q$ через поперечное сечение $wh$. Ответ выразите через $\eta$, $w$, $h$ и градиент давления $dp(x)/dx$.}

\QText{Получим распределение скоростей жидкости вдоль координаты $z$, направленной перпендикулярно направлению потока от одной пластины к другой. Начало оси $z$ расположено посередине между пластинами.
Из условия сохранения импульса имеем:
$$-hdzdp+\eta hdx\left(\cfrac{dv(z+dz)}{dz}-\cfrac{dv(z)}{dz}\right)=0{,}
$$
откуда:
$$\cfrac{d^2v}{dz^2}=\cfrac{1}{\eta}\cfrac{dp}{dx}{.}
$$
Интегрируя один раз, находим:
$$\cfrac{dv}{dz}=\cfrac{z}{\eta}\cfrac{dp}{dx}+A{.}
$$
Поскольку скорость посередине между пластинами испытывает экстремум - $A=0$. Тогда повторно интегрируя, находим:
$$v(z)=C+\cfrac{z^2}{2\eta}\cfrac{dp}{dx}{.}
$$
При $z=\pm w/2$ скорость должна обращаться в ноль, откуда находим:
$$v(z)=-\cfrac{1}{2\eta}\cfrac{dp}{dx}\left(\cfrac{w^2}{4}-z^2\right){.}
$$
Для потока $Q$ имеем:
$$Q=\int\limits_{-w/2}^{w/2}v(z)hdz=-\cfrac{h}{2\eta}\cfrac{dp}{dx}\int\limits_{-w/2}^{w/2}\left(\cfrac{w^2}{4}-z^2\right)dz{.}
$$
Интегрируя, находим:}

\ABlock{$$Q=-\cfrac{w^3h}{12\eta}\cfrac{dp}{dx}{.}
$$}

\QBlock{B2}{1.00}{В центре щели создается избыточное давление $\Delta p$. Найдите зависимость избыточного давления $p'$ в щели от координаты $x$. Ответ выразите через $\Delta{p}$, $Q$, $E$, $h$, $\eta$ и $x$.}

\QText{Поскольку поток жидкости одинаково растекается в две разные стороны - в каждой из них он равен $Q/2$.
Пусть ось $x$ имеет начало в центре щели и направлена вдоль одного из потоков. Воспользуемся результатом пункта $\mathrm{B1}$:
$$\cfrac{Q}{2}=-\cfrac{w^3h}{12\eta}\cfrac{dp'}{dx}{.}
$$
Воспользовавшись эмпирическим соотношением для ширины щели, получим:
$$\cfrac{Q}{2}=-\cfrac{h^4p'^3}{12\eta E^3}\cfrac{dp'}{dx}{.}
$$
Проинтегрируем полученное выражение:
$$\int\limits_{\Delta{p}}^{p'(x)}p'^3dp'=\cfrac{p'^4(x)-\Delta{p}^4}{4}=-\cfrac{6Q\eta E^3x}{h^4}{.}
$$
Таким образом:}

\ABlock{$$p'(x)=\sqrt[4]{\Delta{p}^4-\cfrac{24Q\eta E^3x}{h^4}}{.}
$$}

\QBlock{B3}{0.20}{Трещина заканчивается в положении, соответствующем равному нулю избыточному давлению. Определите длину трещины $L$. Ответ выразите через $\Delta{p}$, $E$, $h$, $\eta$ и $Q$.}

\QText{При максимальной длине трещины давление $p'$ обращается в ноль, откуда полудлина трещины $L/2$ составляет:
$$\cfrac{L}{2}=\cfrac{\Delta{p}^4h^4}{24Q\eta E^3}{.}
$$
Таким образом:}

\ABlock{$$L=\cfrac{\Delta{p}^4h^4}{12Q\eta E^3}{.}
$$}

\QBlock{B4}{0.70}{Определите объем трещины $V$. Ответ выразите через $\Delta{p}$, $h$, $\eta$, $Q$ и $E$.}

\QText{Объём трещина равен удвоенному объёму её половины:
$$V=2\int\limits_{0}^{L}hw(x)dx=2\int\limits_{0}^{L}\cfrac{h^2p'}{E}dx{.}
$$
Воспользуемся выражением для $p'(x)$:
$$V=\cfrac{2h^2\Delta{p}}{E}\int\limits_{0}^{L}\sqrt[4]{1-\cfrac{24Q\eta E^3x}{h^4\Delta{p}^4}}dx=\cfrac{h^6\Delta{p}^5}{12Q\eta E^4}\int\limits_{0}^{1}\sqrt[4]{1-z}dz{.}
$$
После элементарного интегрирования находим:}

\ABlock{$$V=\cfrac{h^6\Delta{p}^5}{15Q\eta E^4}{.}
$$}

\QBlock{B5}{0.30}{Рассчитайте максимально возможные значения длины трещины $L_{max}$ и её объёма $V_{max}$.}

\QText{<div class="example-preview" style="margin-bottom: -1px;margin-top: 1rem;">}

\ABlock{$$L_{max}=\approx 83 \text{м}{.}
$$}

\ABlock{$$V_{max}=6{.}7 \text{м}{.}
$$}

\QBlock{С1}{1.00}{Определите скорость $v$ движения границы жидкостей при перемещении фронта на величину $S$. Ответ выразите через $p_1$, $p_2$, $L$, $\eta_1$, $\eta_2$, $k_1$ и $k_2$.}

\QText{Закон Дарси в рамках задачи можно переписать в следующей форме:
$$v=-\cfrac{k}{\eta}\cfrac{\partial p}{\partial x}{.}
$$
Скорость движения $v$ одинакова для обеих жидкостей и постоянна по всей длине $L$. Тогда для градиентов давления в нефти и в воде с учётом равенства их проницаемостей $k_1=k_2=k$ получим:
$$\cfrac{\partial p_\text{н}}{\partial x}=-\cfrac{\eta_1v}{k_1}\qquad \cfrac{\partial p_\text{в}}{\partial x}=-\cfrac{\eta_2v}{k_2}{.}
$$
Для полного изменения давления находим:
$$p_2-p_1=\cfrac{\eta_2Sv}{k_2}+\cfrac{\eta_1(L-S)v}{k_1}{.}
$$
Таким образом:}

\ABlock{$$v=\cfrac{p_2-p_1}{\cfrac{\eta_1L}{k_1}+\left(\cfrac{\eta_2}{k_2}-\cfrac{\eta_1}{k_1}\right)S}{.}
$$}

\QBlock{C2}{0.90}{Определите зависимость перемещения $S$ фронта от времени $t$. Ответ выразите через $p_1$, $p_2$, $L$, $\eta_1$, $\eta_2$, $k$ и $t$}

\QText{Поскольку $v=dx/dt$ и $k_1=k_2=k$, имеем:
$$dt=\cfrac{(\eta_1L+(\eta_2-\eta_1)S)dS}{k(p_2-p_1)}{.}
$$
Интегрируя, найдём:
$$t=\cfrac{1}{k(p_2-p_1)}\int\limits_{0}^S(\eta_1L+(\eta_2-\eta_1)x)dx=\cfrac{1}{k(p_2-p_1)}\left(\eta_1LS+\cfrac{(\eta_2-\eta_1)S^2}{2}\right){.}
$$
Приведём квадратное уравнение к классическому виду:
$$S^2-\cfrac{2\eta_1LS}{\eta_1-\eta_2}+\cfrac{2k(p_2-p_1)t}{\eta_1-\eta_2}=0{.}
$$
Решая, получим:
$$S(t)=\cfrac{\eta_1L}{\eta_1-\eta_2}\pm\sqrt{\left(\cfrac{\eta_1L}{\eta_1-\eta_2}\right)^2-\cfrac{2k(p_2-p_1)t}{\eta_1-\eta_2}}{.}
$$
Поскольку $x(0)=0$ - выберем корень со знаком минус.
Окончательно:}

\ABlock{$$S(t)=\cfrac{\eta_1L}{\eta_1-\eta_2}-\sqrt{\left(\cfrac{\eta_1L}{\eta_1-\eta_2}\right)^2-\cfrac{2k(p_2-p_1)t}{\eta_1-\eta_2}}{.}
$$}

\QBlock{C3}{0.50}{Определите полное время $\tau$ вытеснения нефти из месторождения. Выразите ответ через $p_1$, $p_2$, $L$, $\eta_1$, $\eta_2$ и $k$ и рассчитайте его.}

\QText{Полное время $\tau$ вытеснения нефти равняется $t(L)$.
Таким образом:}

\ABlock{$$\tau=\cfrac{(\eta_1+\eta_2)L^2}{2k(p_2-p_1)}\approx 26 \text{лет}{.}
$$}

\QBlock{С4}{0.80}{При каком условии на параметры системы движение границы будет устойчивым, то есть при малом отклонении формы границы от плоской это отклонение не будет возрастать? Запишите условие устойчивости через $\eta_1$, $\eta_2$, $k_1$ и $k_2$.
Устойчиво ли течение жидкости, рассмотренное в пунктах $\mathrm{C2}$ и $\mathrm{C3}$?}

\QText{Отклонение не будет возрастать, если $v(x+dx){<}v(x)$.
Обратим внимание, что числитель выражения для скорости постоянный, а знаменатель является линейной функцией $x$ с угловым коэффициентом $a=\eta_2/k_2-\eta_1/k_1$.
Величина скорости уменьшается с ростом $x$, если величина $a$ является положительной, т.е при условии:}

\ABlock{$$\cfrac{\eta_2}{k_2}{>}\cfrac{\eta_1}{k_1}{.}
$$}

\ABlock{Движение рассматриваемой конструкции неустойчиво.}

\QBlock{D1}{0.80}{Найдите зависимость скорости течения жидкости в такой трубе от расстояния до оси трубы $v(r)$, максимальное значение скорости $v_{max}$ и полный поток $Q$ жидкости через сечение цилиндра. Ответы выразите через $\Delta{p}$, $\eta$, $L$, $R$ и $r$.}

\QText{Выделим цилиндр радиусом $r$ длиной $L$. Запишем условие постоянства его импульса:
$$\pi r^2\Delta{p}+2\pi rL\eta\cfrac{\partial v}{\partial r}=0\Rightarrow \cfrac{\partial v}{\partial r}=-\cfrac{r}{2}\cfrac{\Delta{p}}{\eta L}{.}
$$
Интегрируя, получим:
$$v=C-\cfrac{\Delta{p}r^2}{4\eta L}{.}
$$
Поскольку $v(R)=0$, получим:}

\ABlock{$$v(r)=\cfrac{\Delta{p}(R^2-r^2)}{4\eta L}{.}
$$}

\QText{Для потока $Q$ имеем:
$$Q=\int\limits_0^Rv(r)\cdot 2\pi rdr=\cfrac{\pi\Delta{p}}{2\eta L}\int\limits_{0}^R (R^2-r^2)rdr{.}
$$
Интегрируя, находим:}

\ABlock{$$Q=\cfrac{\pi\Delta{p}R^4}{8\eta L}{.}
$$}

\QBlock{D2}{0.20}{Выразите распределение скорости течения жидкости $v(r)$ через полный поток $Q$, $R$ и $r$.}

\QText{Выразим комбинацию $\Delta{p}/(4\eta L)$ через полный поток $Q$:
$$\cfrac{\Delta{p}}{4\eta L}=\cfrac{2Q}{\pi R^4}{.}
$$
Подставляя в выражение для $v(r)$, находим:}

\ABlock{$$v(r)=\cfrac{2Q}{\pi R^2}\left(1-\cfrac{r^2}{R^2}\right){.}
$$}

\QBlock{D3}{0.20}{Найдите поток $Q$ в сечении забоя на расстоянии $h$ от его нижнего края и соответствующее выражение для вертикальной скорости $v(r{,}h)$ в зависимости от расстояния до оси $r$ и высоты $h$. Ответы выразите через $Q_0$, $H$, $R$, $r$ и $h$.}

\QText{Поскольку жидкость поступает в цилиндр равномерно по его боковой поверхности, имеем:
$$\cfrac{dQ}{dh}=\cfrac{Q_0}{H}{.}
$$
Поскольку $Q(0)=0$, находим:}

\ABlock{$$Q(h)=\cfrac{Q_0h}{H}{.}
$$}

\QText{Воспользуемся выражением, полученным в пункте $\mathrm{D2}$:
$$v(r{,}h)=\cfrac{2Q(h)}{\pi R^2}\left(1-\cfrac{r^2}{R^2}\right){.}
$$
Подставляя $Q(h)$, получим:}

\ABlock{$$v(r{,}h)=\cfrac{2Q_0h}{\pi R^2H}\left(1-\cfrac{r^2}{R^2}\right){.}
$$}

\QBlock{D4}{0.30}{Рассмотрим кольцо высотой $dh$ с внутренним и внешним радиусами $r$ и $r+dr$ соответственно. Используя тот факт, что жидкость несжимаема, покажите, что из условия постоянства объёма жидкости внутри выделенного кольца следует соотношение:
$$\cfrac{\partial v}{\partial h}=-\cfrac{1}{r}\cfrac{\partial (u_rr)}{\partial r}{.}
$$
Вы можете использовать это соотношение, даже если не смогли его доказать.}

\QText{Запишем выражение для полного потока жидкости $q$ в кольцо:
$$q=2\pi rdrv(r{,}h+dh)-2\pi rdrv(r{,}h)+2\pi dh(r+dr)u_r(r+dr{,}h)-2\pi dh ru_r(r{,}h)=0{.}
$$
Таким образом:
$$2\pi rdrdv+2\pi drd(ru_r)=0{,}
$$
или же:
$$\cfrac{\partial v}{\partial h}+\cfrac{1}{r}\cfrac{\partial (ru_r)}{\partial r}=0{.}
$$
Что и требовалось доказать.}

\QBlock{D5}{0.50}{Найдите радиальную скорость течения жидкости $u_r(r{,}h)$ в зависимости от расстояния до оси $r$ и высоты $h$, а также максимальную величину её модуля $u_{r(max)}$. Ответы выразите через $Q_0$, $R$, $H$, $h$ и $r$.}

\QText{Последовательно воспользуемся результатами пунктов $\mathrm{D4}$ и $\mathrm{D3}$:
$$\cfrac{\partial (ru_r)}{\partial r}=-r\cfrac{\partial v}{\partial h}=-\cfrac{2Q_0r}{\pi R^2H}\left(1-\cfrac{r^2}{R^2}\right){.}
$$
Обратим внимание, что при $r=0$ величина $ru_r$ также равна нулю. 
Временно переобозначим $r$ на $z$ в подынтегральной функции и получим:
$$ru_r(r{,}h)=-\cfrac{2Q_0}{\pi R^2H}\int\limits_{0}^r\left(1-\cfrac{z^2}{R^2}\right)zdz=-\cfrac{Q_0}{\pi R^2H}\left(r^2-\cfrac{r^4}{2R^2}\right){.}
$$
Таким образом:}

\ABlock{$$u_r(r{,}h)=-\cfrac{Q_0}{\pi R^2H}\left(r-\cfrac{r^3}{2R^2}\right){.}
$$}

\QText{Максимум величины $u_r$ достигается при равенстве нулю её производной по $r$:
$$\cfrac{du_r}{dr}=0\Rightarrow 1-\cfrac{3r^2}{2R^2}=0\Rightarrow r_{max}=\sqrt{\cfrac{2}{3}}R{.}
$$
Тогда для максимального значения радиальной компоненты скорости $u_{max}$ находим:}

\ABlock{$$u_{max}=\left(\cfrac{2}{3}\right)^{3/2}\cfrac{Q_0}{\pi RH}{.}
$$}

\QBlock{D6}{0.10}{Чему равно отношение $u_{r(max)}/v_{max}$? Ответ выразите через $R$ и $H$.}

\QText{<div class="example-preview" style="margin-bottom: -1px;margin-top: 1rem;">}

\ABlock{$$\cfrac{u_{r(max)}}{v_{max}}=\cfrac{\sqrt{2}R}{3\sqrt{3}H}
$$}

\end{document}